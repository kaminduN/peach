%
% API Documentation for Peach - Computational Intelligence for Python
% Package peach.fuzzy
%
% Generated by epydoc 3.0beta1
% [Mon Dec 21 08:51:35 2009]
%

%%%%%%%%%%%%%%%%%%%%%%%%%%%%%%%%%%%%%%%%%%%%%%%%%%%%%%%%%%%%%%%%%%%%%%%%%%%
%%                          Module Description                           %%
%%%%%%%%%%%%%%%%%%%%%%%%%%%%%%%%%%%%%%%%%%%%%%%%%%%%%%%%%%%%%%%%%%%%%%%%%%%

    \index{peach \textit{(package)}!peach.fuzzy \textit{(package)}|(}
\section{Package peach.fuzzy}

    \label{peach:fuzzy}

This package implements fuzzy logic. Consult:
\begin{quote}
\begin{description}
%[visit_definition_list_item]
\item[{fuzzy}] %[visit_definition]

Basic definitions, classes and operations in fuzzy logic;

%[depart_definition]
%[depart_definition_list_item]
%[visit_definition_list_item]
\item[{mf}] %[visit_definition]

Membership functions;

%[depart_definition]
%[depart_definition_list_item]
%[visit_definition_list_item]
\item[{defuzzy}] %[visit_definition]

Defuzzification methods;

%[depart_definition]
%[depart_definition_list_item]
%[visit_definition_list_item]
\item[{control}] %[visit_definition]

Fuzzy controllers (FIS - Fuzzy Inference Systems), for Mamdani- and
Sugeno-type controllers and others;

%[depart_definition]
%[depart_definition_list_item]
\end{description}
\end{quote}

%%%%%%%%%%%%%%%%%%%%%%%%%%%%%%%%%%%%%%%%%%%%%%%%%%%%%%%%%%%%%%%%%%%%%%%%%%%
%%                                Modules                                %%
%%%%%%%%%%%%%%%%%%%%%%%%%%%%%%%%%%%%%%%%%%%%%%%%%%%%%%%%%%%%%%%%%%%%%%%%%%%

\subsection{Modules}

\begin{itemize}
\setlength{\parskip}{0ex}
\item \textbf{control}: 
This package implements fuzzy controllers, of fuzzy inference systems.


  \textit{(Section \ref{peach:fuzzy:control}, p.~\pageref{peach:fuzzy:control})}

\item \textbf{defuzzy}: 
This package implements defuzzification methods for use with fuzzy controllers.


  \textit{(Section \ref{peach:fuzzy:defuzzy}, p.~\pageref{peach:fuzzy:defuzzy})}

\item \textbf{fuzzy}: 
This package implements basic definitions for fuzzy logic


  \textit{(Section \ref{peach:fuzzy:fuzzy}, p.~\pageref{peach:fuzzy:fuzzy})}

\item \textbf{mf}: 
Membership functions


  \textit{(Section \ref{peach:fuzzy:mf}, p.~\pageref{peach:fuzzy:mf})}

\item \textbf{norms}: 
This package implements operations of fuzzy logic.


  \textit{(Section \ref{peach:fuzzy:norms}, p.~\pageref{peach:fuzzy:norms})}

\end{itemize}


%%%%%%%%%%%%%%%%%%%%%%%%%%%%%%%%%%%%%%%%%%%%%%%%%%%%%%%%%%%%%%%%%%%%%%%%%%%
%%                               Variables                               %%
%%%%%%%%%%%%%%%%%%%%%%%%%%%%%%%%%%%%%%%%%%%%%%%%%%%%%%%%%%%%%%%%%%%%%%%%%%%

  \subsection{Variables}

\begin{longtable}{|p{.30\textwidth}|p{.62\textwidth}|l}
\cline{1-2}
\cline{1-2} \centering \textbf{Name} & \centering \textbf{Description}& \\
\cline{1-2}
\endhead\cline{1-2}\multicolumn{3}{r}{\small\textit{continued on next page}}\\\endfoot\cline{1-2}
\endlastfoot\raggedright \_\-\_\-d\-o\-c\-\_\-\_\- & \raggedright \textbf{Value:} 
{\tt \texttt{...}}&\\
\cline{1-2}
\end{longtable}

    \index{peach \textit{(package)}!peach.fuzzy \textit{(package)}|)}
