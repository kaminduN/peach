%
% API Documentation for Peach - Computational Intelligence for Python
% Module peach.ga.crossover
%
% Generated by epydoc 3.0beta1
% [Mon Dec 21 08:51:36 2009]
%

%%%%%%%%%%%%%%%%%%%%%%%%%%%%%%%%%%%%%%%%%%%%%%%%%%%%%%%%%%%%%%%%%%%%%%%%%%%
%%                          Module Description                           %%
%%%%%%%%%%%%%%%%%%%%%%%%%%%%%%%%%%%%%%%%%%%%%%%%%%%%%%%%%%%%%%%%%%%%%%%%%%%

    \index{peach \textit{(package)}!peach.ga \textit{(package)}!peach.ga.crossover \textit{(module)}|(}
\section{Module peach.ga.crossover}

    \label{peach:ga:crossover}

Basic definitions for crossover operations and base classes.

Crossover is a very basic and important operation in genetic algorithms. It is
by means of crossover among the chromosomes that population gains diversity,
thus exploring more completelly the solution space and giving better answers.
This sub-module provides definitions of the most common crossover operations,
and provides a class that can be subclassed to construct different types of
crossover for experimentation.

%%%%%%%%%%%%%%%%%%%%%%%%%%%%%%%%%%%%%%%%%%%%%%%%%%%%%%%%%%%%%%%%%%%%%%%%%%%
%%                               Variables                               %%
%%%%%%%%%%%%%%%%%%%%%%%%%%%%%%%%%%%%%%%%%%%%%%%%%%%%%%%%%%%%%%%%%%%%%%%%%%%

  \subsection{Variables}

\begin{longtable}{|p{.30\textwidth}|p{.62\textwidth}|l}
\cline{1-2}
\cline{1-2} \centering \textbf{Name} & \centering \textbf{Description}& \\
\cline{1-2}
\endhead\cline{1-2}\multicolumn{3}{r}{\small\textit{continued on next page}}\\\endfoot\cline{1-2}
\endlastfoot\raggedright \_\-\_\-d\-o\-c\-\_\-\_\- & \raggedright \textbf{Value:} 
{\tt \texttt{...}}&\\
\cline{1-2}
\end{longtable}


%%%%%%%%%%%%%%%%%%%%%%%%%%%%%%%%%%%%%%%%%%%%%%%%%%%%%%%%%%%%%%%%%%%%%%%%%%%
%%                           Class Description                           %%
%%%%%%%%%%%%%%%%%%%%%%%%%%%%%%%%%%%%%%%%%%%%%%%%%%%%%%%%%%%%%%%%%%%%%%%%%%%

    \index{peach \textit{(package)}!peach.ga \textit{(package)}!peach.ga.crossover \textit{(module)}!peach.ga.crossover.Crossover \textit{(class)}|(}
\subsection{Class Crossover}

    \label{peach:ga:crossover:Crossover}
\begin{tabular}{cccccc}
% Line for object, linespec=[False]
\multicolumn{2}{r}{\settowidth{\BCL}{object}\multirow{2}{\BCL}{object}}
&&
  \\\cline{3-3}
  &&\multicolumn{1}{c|}{}
&&
  \\
&&\multicolumn{2}{l}{\textbf{peach.ga.crossover.Crossover}}
\end{tabular}

\textbf{Known Subclasses:}
peach.ga.crossover.OnePoint,
    peach.ga.crossover.TwoPoint,
    peach.ga.crossover.Uniform


Base class for crossover operators.

This class should be subclassed if you want to create your own crossover
operator. The base class doesn't do much, it is only a prototype. As is done
with all the base classes within this library, use the \texttt{{\_}{\_}init{\_}{\_}} method
to configure your crossover behaviour -{}- if needed -{}- and the \texttt{{\_}{\_}call{\_}{\_}}
method to operate over a population.

A class derived from this one should implement at least 2 methods, defined
below:
\begin{quote}
\begin{description}
%[visit_definition_list_item]
\item[{{\_}{\_}init{\_}{\_}(self, {\color{red}\bfseries{}*}cnf, {\color{red}\bfseries{}**}kw)}] %[visit_definition]

Initializes the object. There is no mandatory arguments, but any
parameters can be used here to configure the operator. For example, a
class can define a crossover rate -{}- this should be defined here:
\begin{quote}{\ttfamily \raggedright \noindent
{\_}{\_}init{\_}{\_}(self,~rate=0.75)
}\end{quote}

A default value should always be offered, if possible.

%[depart_definition]
%[depart_definition_list_item]
%[visit_definition_list_item]
\item[{{\_}{\_}call{\_}{\_}(self, population)}] %[visit_definition]

The \texttt{{\_}{\_}call{\_}{\_}} implementation should receive a population and operate
over it. Please, consult the \texttt{ga} module to see more information on
populations. It should return the processed population. No recomendation
on the internals of the method is made. That being said, in general the
crossover operators pairs chromosomes and swap bits among them (but
there is nothing to say that you can't do it differently).

%[depart_definition]
%[depart_definition_list_item]
\end{description}
\end{quote}

Please, note that the GA implementations relies on this behaviour: it will
pass a population to your \texttt{{\_}{\_}call{\_}{\_}} method and expects to received the
result back.

%%%%%%%%%%%%%%%%%%%%%%%%%%%%%%%%%%%%%%%%%%%%%%%%%%%%%%%%%%%%%%%%%%%%%%%%%%%
%%                                Methods                                %%
%%%%%%%%%%%%%%%%%%%%%%%%%%%%%%%%%%%%%%%%%%%%%%%%%%%%%%%%%%%%%%%%%%%%%%%%%%%

  \subsubsection{Methods}

    \label{object:__delattr__}
    \index{object.\_\_delattr\_\_ \textit{(function)}}

    \vspace{0.5ex}

    \begin{boxedminipage}{\textwidth}

    \raggedright \textbf{\_\_delattr\_\_}(\textit{...})

    \vspace{-1.5ex}

    \rule{\textwidth}{0.5\fboxrule}

x.{\_}{\_}delattr{\_}{\_}('name') {\textless}=={\textgreater} del x.name
    \vspace{1ex}

    \end{boxedminipage}

    \label{object:__getattribute__}
    \index{object.\_\_getattribute\_\_ \textit{(function)}}

    \vspace{0.5ex}

    \begin{boxedminipage}{\textwidth}

    \raggedright \textbf{\_\_getattribute\_\_}(\textit{...})

    \vspace{-1.5ex}

    \rule{\textwidth}{0.5\fboxrule}

x.{\_}{\_}getattribute{\_}{\_}('name') {\textless}=={\textgreater} x.name
    \vspace{1ex}

    \end{boxedminipage}

    \label{object:__hash__}
    \index{object.\_\_hash\_\_ \textit{(function)}}

    \vspace{0.5ex}

    \begin{boxedminipage}{\textwidth}

    \raggedright \textbf{\_\_hash\_\_}(\textit{x})

    \vspace{-1.5ex}

    \rule{\textwidth}{0.5\fboxrule}

hash(x)
    \vspace{1ex}

    \end{boxedminipage}

    \label{object:__init__}
    \index{object.\_\_init\_\_ \textit{(function)}}

    \vspace{0.5ex}

    \begin{boxedminipage}{\textwidth}

    \raggedright \textbf{\_\_init\_\_}(\textit{...})

    \vspace{-1.5ex}

    \rule{\textwidth}{0.5\fboxrule}

x.{\_}{\_}init{\_}{\_}(...) initializes x; see x.{\_}{\_}class{\_}{\_}.{\_}{\_}doc{\_}{\_} for signature
    \vspace{1ex}

    \end{boxedminipage}

    \label{object:__new__}
    \index{object.\_\_new\_\_ \textit{(function)}}

    \vspace{0.5ex}

    \begin{boxedminipage}{\textwidth}

    \raggedright \textbf{\_\_new\_\_}(\textit{T}, \textit{S}, \textit{...})

      \textbf{Return Value}
      \begin{quote}
\begin{alltt}
a new object with type S, a subtype of T
\end{alltt}

      \end{quote}

    \vspace{1ex}

    \end{boxedminipage}

    \label{object:__reduce__}
    \index{object.\_\_reduce\_\_ \textit{(function)}}

    \vspace{0.5ex}

    \begin{boxedminipage}{\textwidth}

    \raggedright \textbf{\_\_reduce\_\_}(\textit{...})

    \vspace{-1.5ex}

    \rule{\textwidth}{0.5\fboxrule}

helper for pickle
    \vspace{1ex}

    \end{boxedminipage}

    \label{object:__reduce_ex__}
    \index{object.\_\_reduce\_ex\_\_ \textit{(function)}}

    \vspace{0.5ex}

    \begin{boxedminipage}{\textwidth}

    \raggedright \textbf{\_\_reduce\_ex\_\_}(\textit{...})

    \vspace{-1.5ex}

    \rule{\textwidth}{0.5\fboxrule}

helper for pickle
    \vspace{1ex}

    \end{boxedminipage}

    \label{object:__repr__}
    \index{object.\_\_repr\_\_ \textit{(function)}}

    \vspace{0.5ex}

    \begin{boxedminipage}{\textwidth}

    \raggedright \textbf{\_\_repr\_\_}(\textit{x})

    \vspace{-1.5ex}

    \rule{\textwidth}{0.5\fboxrule}

repr(x)
    \vspace{1ex}

    \end{boxedminipage}

    \label{object:__setattr__}
    \index{object.\_\_setattr\_\_ \textit{(function)}}

    \vspace{0.5ex}

    \begin{boxedminipage}{\textwidth}

    \raggedright \textbf{\_\_setattr\_\_}(\textit{...})

    \vspace{-1.5ex}

    \rule{\textwidth}{0.5\fboxrule}

x.{\_}{\_}setattr{\_}{\_}('name', value) {\textless}=={\textgreater} x.name = value
    \vspace{1ex}

    \end{boxedminipage}

    \label{object:__str__}
    \index{object.\_\_str\_\_ \textit{(function)}}

    \vspace{0.5ex}

    \begin{boxedminipage}{\textwidth}

    \raggedright \textbf{\_\_str\_\_}(\textit{x})

    \vspace{-1.5ex}

    \rule{\textwidth}{0.5\fboxrule}

str(x)
    \vspace{1ex}

    \end{boxedminipage}


%%%%%%%%%%%%%%%%%%%%%%%%%%%%%%%%%%%%%%%%%%%%%%%%%%%%%%%%%%%%%%%%%%%%%%%%%%%
%%                              Properties                               %%
%%%%%%%%%%%%%%%%%%%%%%%%%%%%%%%%%%%%%%%%%%%%%%%%%%%%%%%%%%%%%%%%%%%%%%%%%%%

  \subsubsection{Properties}

\begin{longtable}{|p{.30\textwidth}|p{.62\textwidth}|l}
\cline{1-2}
\cline{1-2} \centering \textbf{Name} & \centering \textbf{Description}& \\
\cline{1-2}
\endhead\cline{1-2}\multicolumn{3}{r}{\small\textit{continued on next page}}\\\endfoot\cline{1-2}
\endlastfoot\raggedright \_\-\_\-c\-l\-a\-s\-s\-\_\-\_\- & \raggedright \textbf{Value:} 
{\tt {\textless}attribute '\_\_class\_\_' of 'object' objects{\textgreater}}&\\
\cline{1-2}
\end{longtable}

    \index{peach \textit{(package)}!peach.ga \textit{(package)}!peach.ga.crossover \textit{(module)}!peach.ga.crossover.Crossover \textit{(class)}|)}

%%%%%%%%%%%%%%%%%%%%%%%%%%%%%%%%%%%%%%%%%%%%%%%%%%%%%%%%%%%%%%%%%%%%%%%%%%%
%%                           Class Description                           %%
%%%%%%%%%%%%%%%%%%%%%%%%%%%%%%%%%%%%%%%%%%%%%%%%%%%%%%%%%%%%%%%%%%%%%%%%%%%

    \index{peach \textit{(package)}!peach.ga \textit{(package)}!peach.ga.crossover \textit{(module)}!peach.ga.crossover.OnePoint \textit{(class)}|(}
\subsection{Class OnePoint}

    \label{peach:ga:crossover:OnePoint}
\begin{tabular}{cccccccc}
% Line for object, linespec=[False, False]
\multicolumn{2}{r}{\settowidth{\BCL}{object}\multirow{2}{\BCL}{object}}
&&
&&
  \\\cline{3-3}
  &&\multicolumn{1}{c|}{}
&&
&&
  \\
% Line for peach.ga.crossover.Crossover, linespec=[False]
\multicolumn{4}{r}{\settowidth{\BCL}{peach.ga.crossover.Crossover}\multirow{2}{\BCL}{peach.ga.crossover.Crossover}}
&&
  \\\cline{5-5}
  &&&&\multicolumn{1}{c|}{}
&&
  \\
&&&&\multicolumn{2}{l}{\textbf{peach.ga.crossover.OnePoint}}
\end{tabular}


A one-point crossover operator.

A one-point crossover randomly selects a single point in two chromosomes and
swaps the bits among them from that point until the end of the bit stream.
The crossover rate is the probability that two paired chromosomes will
exchange bits.

%%%%%%%%%%%%%%%%%%%%%%%%%%%%%%%%%%%%%%%%%%%%%%%%%%%%%%%%%%%%%%%%%%%%%%%%%%%
%%                                Methods                                %%
%%%%%%%%%%%%%%%%%%%%%%%%%%%%%%%%%%%%%%%%%%%%%%%%%%%%%%%%%%%%%%%%%%%%%%%%%%%

  \subsubsection{Methods}

    \vspace{0.5ex}

    \begin{boxedminipage}{\textwidth}

    \raggedright \textbf{\_\_init\_\_}(\textit{self}, \textit{rate}=\texttt{0.75})

    \vspace{-1.5ex}

    \rule{\textwidth}{0.5\fboxrule}

Initialize the crossover operator.
    \vspace{1ex}

      \textbf{Parameters}
      \begin{quote}
        \begin{Ventry}{xxxx}

          \item[rate]


Probability that two paired chromosomes will exchange bits.
        \end{Ventry}

      \end{quote}

    \vspace{1ex}

      Overrides: object.\_\_init\_\_

    \end{boxedminipage}

    \label{peach:ga:crossover:OnePoint:__call__}
    \index{peach \textit{(package)}!peach.ga \textit{(package)}!peach.ga.crossover \textit{(module)}!peach.ga.crossover.OnePoint \textit{(class)}!peach.ga.crossover.OnePoint.\_\_call\_\_ \textit{(method)}}

    \vspace{0.5ex}

    \begin{boxedminipage}{\textwidth}

    \raggedright \textbf{\_\_call\_\_}(\textit{self}, \textit{population})

    \vspace{-1.5ex}

    \rule{\textwidth}{0.5\fboxrule}

Proceeds the crossover over a population.

In one-point crossover, chromosomes from a population are randomly
paired. If a uniform random number is below the \texttt{rate} given in the
instantiation of the operator, then a random point is selected and bits
from that point until the end of the chromosomes are exchanged.
    \vspace{1ex}

      \textbf{Parameters}
      \begin{quote}
        \begin{Ventry}{xxxxxxxxxx}

          \item[population]


A list of \texttt{Chromosomes} containing the present population of the
algorithm. It is processed and the results of the exchange are
returned to the caller.
        \end{Ventry}

      \end{quote}

    \vspace{1ex}

      \textbf{Return Value}
      \begin{quote}

The processed population, a list of \texttt{Chromosomes}.
      \end{quote}

    \vspace{1ex}

    \end{boxedminipage}

    \label{object:__delattr__}
    \index{object.\_\_delattr\_\_ \textit{(function)}}

    \vspace{0.5ex}

    \begin{boxedminipage}{\textwidth}

    \raggedright \textbf{\_\_delattr\_\_}(\textit{...})

    \vspace{-1.5ex}

    \rule{\textwidth}{0.5\fboxrule}

x.{\_}{\_}delattr{\_}{\_}('name') {\textless}=={\textgreater} del x.name
    \vspace{1ex}

    \end{boxedminipage}

    \label{object:__getattribute__}
    \index{object.\_\_getattribute\_\_ \textit{(function)}}

    \vspace{0.5ex}

    \begin{boxedminipage}{\textwidth}

    \raggedright \textbf{\_\_getattribute\_\_}(\textit{...})

    \vspace{-1.5ex}

    \rule{\textwidth}{0.5\fboxrule}

x.{\_}{\_}getattribute{\_}{\_}('name') {\textless}=={\textgreater} x.name
    \vspace{1ex}

    \end{boxedminipage}

    \label{object:__hash__}
    \index{object.\_\_hash\_\_ \textit{(function)}}

    \vspace{0.5ex}

    \begin{boxedminipage}{\textwidth}

    \raggedright \textbf{\_\_hash\_\_}(\textit{x})

    \vspace{-1.5ex}

    \rule{\textwidth}{0.5\fboxrule}

hash(x)
    \vspace{1ex}

    \end{boxedminipage}

    \label{object:__new__}
    \index{object.\_\_new\_\_ \textit{(function)}}

    \vspace{0.5ex}

    \begin{boxedminipage}{\textwidth}

    \raggedright \textbf{\_\_new\_\_}(\textit{T}, \textit{S}, \textit{...})

      \textbf{Return Value}
      \begin{quote}
\begin{alltt}
a new object with type S, a subtype of T
\end{alltt}

      \end{quote}

    \vspace{1ex}

    \end{boxedminipage}

    \label{object:__reduce__}
    \index{object.\_\_reduce\_\_ \textit{(function)}}

    \vspace{0.5ex}

    \begin{boxedminipage}{\textwidth}

    \raggedright \textbf{\_\_reduce\_\_}(\textit{...})

    \vspace{-1.5ex}

    \rule{\textwidth}{0.5\fboxrule}

helper for pickle
    \vspace{1ex}

    \end{boxedminipage}

    \label{object:__reduce_ex__}
    \index{object.\_\_reduce\_ex\_\_ \textit{(function)}}

    \vspace{0.5ex}

    \begin{boxedminipage}{\textwidth}

    \raggedright \textbf{\_\_reduce\_ex\_\_}(\textit{...})

    \vspace{-1.5ex}

    \rule{\textwidth}{0.5\fboxrule}

helper for pickle
    \vspace{1ex}

    \end{boxedminipage}

    \label{object:__repr__}
    \index{object.\_\_repr\_\_ \textit{(function)}}

    \vspace{0.5ex}

    \begin{boxedminipage}{\textwidth}

    \raggedright \textbf{\_\_repr\_\_}(\textit{x})

    \vspace{-1.5ex}

    \rule{\textwidth}{0.5\fboxrule}

repr(x)
    \vspace{1ex}

    \end{boxedminipage}

    \label{object:__setattr__}
    \index{object.\_\_setattr\_\_ \textit{(function)}}

    \vspace{0.5ex}

    \begin{boxedminipage}{\textwidth}

    \raggedright \textbf{\_\_setattr\_\_}(\textit{...})

    \vspace{-1.5ex}

    \rule{\textwidth}{0.5\fboxrule}

x.{\_}{\_}setattr{\_}{\_}('name', value) {\textless}=={\textgreater} x.name = value
    \vspace{1ex}

    \end{boxedminipage}

    \label{object:__str__}
    \index{object.\_\_str\_\_ \textit{(function)}}

    \vspace{0.5ex}

    \begin{boxedminipage}{\textwidth}

    \raggedright \textbf{\_\_str\_\_}(\textit{x})

    \vspace{-1.5ex}

    \rule{\textwidth}{0.5\fboxrule}

str(x)
    \vspace{1ex}

    \end{boxedminipage}


%%%%%%%%%%%%%%%%%%%%%%%%%%%%%%%%%%%%%%%%%%%%%%%%%%%%%%%%%%%%%%%%%%%%%%%%%%%
%%                              Properties                               %%
%%%%%%%%%%%%%%%%%%%%%%%%%%%%%%%%%%%%%%%%%%%%%%%%%%%%%%%%%%%%%%%%%%%%%%%%%%%

  \subsubsection{Properties}

\begin{longtable}{|p{.30\textwidth}|p{.62\textwidth}|l}
\cline{1-2}
\cline{1-2} \centering \textbf{Name} & \centering \textbf{Description}& \\
\cline{1-2}
\endhead\cline{1-2}\multicolumn{3}{r}{\small\textit{continued on next page}}\\\endfoot\cline{1-2}
\endlastfoot\raggedright \_\-\_\-c\-l\-a\-s\-s\-\_\-\_\- & \raggedright \textbf{Value:} 
{\tt {\textless}attribute '\_\_class\_\_' of 'object' objects{\textgreater}}&\\
\cline{1-2}
\end{longtable}


%%%%%%%%%%%%%%%%%%%%%%%%%%%%%%%%%%%%%%%%%%%%%%%%%%%%%%%%%%%%%%%%%%%%%%%%%%%
%%                          Instance Variables                           %%
%%%%%%%%%%%%%%%%%%%%%%%%%%%%%%%%%%%%%%%%%%%%%%%%%%%%%%%%%%%%%%%%%%%%%%%%%%%

  \subsubsection{Instance Variables}

\begin{longtable}{|p{.30\textwidth}|p{.62\textwidth}|l}
\cline{1-2}
\cline{1-2} \centering \textbf{Name} & \centering \textbf{Description}& \\
\cline{1-2}
\endhead\cline{1-2}\multicolumn{3}{r}{\small\textit{continued on next page}}\\\endfoot\cline{1-2}
\endlastfoot\raggedright r\-a\-t\-e\- & Property that contains the crossover rate.&\\
\cline{1-2}
\end{longtable}

    \index{peach \textit{(package)}!peach.ga \textit{(package)}!peach.ga.crossover \textit{(module)}!peach.ga.crossover.OnePoint \textit{(class)}|)}

%%%%%%%%%%%%%%%%%%%%%%%%%%%%%%%%%%%%%%%%%%%%%%%%%%%%%%%%%%%%%%%%%%%%%%%%%%%
%%                           Class Description                           %%
%%%%%%%%%%%%%%%%%%%%%%%%%%%%%%%%%%%%%%%%%%%%%%%%%%%%%%%%%%%%%%%%%%%%%%%%%%%

    \index{peach \textit{(package)}!peach.ga \textit{(package)}!peach.ga.crossover \textit{(module)}!peach.ga.crossover.TwoPoint \textit{(class)}|(}
\subsection{Class TwoPoint}

    \label{peach:ga:crossover:TwoPoint}
\begin{tabular}{cccccccc}
% Line for object, linespec=[False, False]
\multicolumn{2}{r}{\settowidth{\BCL}{object}\multirow{2}{\BCL}{object}}
&&
&&
  \\\cline{3-3}
  &&\multicolumn{1}{c|}{}
&&
&&
  \\
% Line for peach.ga.crossover.Crossover, linespec=[False]
\multicolumn{4}{r}{\settowidth{\BCL}{peach.ga.crossover.Crossover}\multirow{2}{\BCL}{peach.ga.crossover.Crossover}}
&&
  \\\cline{5-5}
  &&&&\multicolumn{1}{c|}{}
&&
  \\
&&&&\multicolumn{2}{l}{\textbf{peach.ga.crossover.TwoPoint}}
\end{tabular}


A two-point crossover operator.

A two-point crossover randomly selects two points in two chromosomes and
swaps the bits among them between these points. The crossover rate is the
probability that two paired chromosomes will exchange bits.

%%%%%%%%%%%%%%%%%%%%%%%%%%%%%%%%%%%%%%%%%%%%%%%%%%%%%%%%%%%%%%%%%%%%%%%%%%%
%%                                Methods                                %%
%%%%%%%%%%%%%%%%%%%%%%%%%%%%%%%%%%%%%%%%%%%%%%%%%%%%%%%%%%%%%%%%%%%%%%%%%%%

  \subsubsection{Methods}

    \vspace{0.5ex}

    \begin{boxedminipage}{\textwidth}

    \raggedright \textbf{\_\_init\_\_}(\textit{self}, \textit{rate}=\texttt{0.75})

    \vspace{-1.5ex}

    \rule{\textwidth}{0.5\fboxrule}

Initialize the crossover operator.
    \vspace{1ex}

      \textbf{Parameters}
      \begin{quote}
        \begin{Ventry}{xxxx}

          \item[rate]


Probability that two paired chromosomes will exchange bits.
        \end{Ventry}

      \end{quote}

    \vspace{1ex}

      Overrides: object.\_\_init\_\_

    \end{boxedminipage}

    \label{peach:ga:crossover:TwoPoint:__call__}
    \index{peach \textit{(package)}!peach.ga \textit{(package)}!peach.ga.crossover \textit{(module)}!peach.ga.crossover.TwoPoint \textit{(class)}!peach.ga.crossover.TwoPoint.\_\_call\_\_ \textit{(method)}}

    \vspace{0.5ex}

    \begin{boxedminipage}{\textwidth}

    \raggedright \textbf{\_\_call\_\_}(\textit{self}, \textit{population})

    \vspace{-1.5ex}

    \rule{\textwidth}{0.5\fboxrule}

Proceeds the crossover over a population.

In two-point crossover, chromosomes from a population are randomly
paired. If a uniform random number is below the \texttt{rate} given in the
instantiation of the operator, then random points are selected and bits
between those points are exchanged.
    \vspace{1ex}

      \textbf{Parameters}
      \begin{quote}
        \begin{Ventry}{xxxxxxxxxx}

          \item[population]


A list of \texttt{Chromosomes} containing the present population of the
algorithm. It is processed and the results of the exchange are
returned to the caller.
        \end{Ventry}

      \end{quote}

    \vspace{1ex}

      \textbf{Return Value}
      \begin{quote}

The processed population, a list of \texttt{Chromosomes}.
      \end{quote}

    \vspace{1ex}

    \end{boxedminipage}

    \label{object:__delattr__}
    \index{object.\_\_delattr\_\_ \textit{(function)}}

    \vspace{0.5ex}

    \begin{boxedminipage}{\textwidth}

    \raggedright \textbf{\_\_delattr\_\_}(\textit{...})

    \vspace{-1.5ex}

    \rule{\textwidth}{0.5\fboxrule}

x.{\_}{\_}delattr{\_}{\_}('name') {\textless}=={\textgreater} del x.name
    \vspace{1ex}

    \end{boxedminipage}

    \label{object:__getattribute__}
    \index{object.\_\_getattribute\_\_ \textit{(function)}}

    \vspace{0.5ex}

    \begin{boxedminipage}{\textwidth}

    \raggedright \textbf{\_\_getattribute\_\_}(\textit{...})

    \vspace{-1.5ex}

    \rule{\textwidth}{0.5\fboxrule}

x.{\_}{\_}getattribute{\_}{\_}('name') {\textless}=={\textgreater} x.name
    \vspace{1ex}

    \end{boxedminipage}

    \label{object:__hash__}
    \index{object.\_\_hash\_\_ \textit{(function)}}

    \vspace{0.5ex}

    \begin{boxedminipage}{\textwidth}

    \raggedright \textbf{\_\_hash\_\_}(\textit{x})

    \vspace{-1.5ex}

    \rule{\textwidth}{0.5\fboxrule}

hash(x)
    \vspace{1ex}

    \end{boxedminipage}

    \label{object:__new__}
    \index{object.\_\_new\_\_ \textit{(function)}}

    \vspace{0.5ex}

    \begin{boxedminipage}{\textwidth}

    \raggedright \textbf{\_\_new\_\_}(\textit{T}, \textit{S}, \textit{...})

      \textbf{Return Value}
      \begin{quote}
\begin{alltt}
a new object with type S, a subtype of T
\end{alltt}

      \end{quote}

    \vspace{1ex}

    \end{boxedminipage}

    \label{object:__reduce__}
    \index{object.\_\_reduce\_\_ \textit{(function)}}

    \vspace{0.5ex}

    \begin{boxedminipage}{\textwidth}

    \raggedright \textbf{\_\_reduce\_\_}(\textit{...})

    \vspace{-1.5ex}

    \rule{\textwidth}{0.5\fboxrule}

helper for pickle
    \vspace{1ex}

    \end{boxedminipage}

    \label{object:__reduce_ex__}
    \index{object.\_\_reduce\_ex\_\_ \textit{(function)}}

    \vspace{0.5ex}

    \begin{boxedminipage}{\textwidth}

    \raggedright \textbf{\_\_reduce\_ex\_\_}(\textit{...})

    \vspace{-1.5ex}

    \rule{\textwidth}{0.5\fboxrule}

helper for pickle
    \vspace{1ex}

    \end{boxedminipage}

    \label{object:__repr__}
    \index{object.\_\_repr\_\_ \textit{(function)}}

    \vspace{0.5ex}

    \begin{boxedminipage}{\textwidth}

    \raggedright \textbf{\_\_repr\_\_}(\textit{x})

    \vspace{-1.5ex}

    \rule{\textwidth}{0.5\fboxrule}

repr(x)
    \vspace{1ex}

    \end{boxedminipage}

    \label{object:__setattr__}
    \index{object.\_\_setattr\_\_ \textit{(function)}}

    \vspace{0.5ex}

    \begin{boxedminipage}{\textwidth}

    \raggedright \textbf{\_\_setattr\_\_}(\textit{...})

    \vspace{-1.5ex}

    \rule{\textwidth}{0.5\fboxrule}

x.{\_}{\_}setattr{\_}{\_}('name', value) {\textless}=={\textgreater} x.name = value
    \vspace{1ex}

    \end{boxedminipage}

    \label{object:__str__}
    \index{object.\_\_str\_\_ \textit{(function)}}

    \vspace{0.5ex}

    \begin{boxedminipage}{\textwidth}

    \raggedright \textbf{\_\_str\_\_}(\textit{x})

    \vspace{-1.5ex}

    \rule{\textwidth}{0.5\fboxrule}

str(x)
    \vspace{1ex}

    \end{boxedminipage}


%%%%%%%%%%%%%%%%%%%%%%%%%%%%%%%%%%%%%%%%%%%%%%%%%%%%%%%%%%%%%%%%%%%%%%%%%%%
%%                              Properties                               %%
%%%%%%%%%%%%%%%%%%%%%%%%%%%%%%%%%%%%%%%%%%%%%%%%%%%%%%%%%%%%%%%%%%%%%%%%%%%

  \subsubsection{Properties}

\begin{longtable}{|p{.30\textwidth}|p{.62\textwidth}|l}
\cline{1-2}
\cline{1-2} \centering \textbf{Name} & \centering \textbf{Description}& \\
\cline{1-2}
\endhead\cline{1-2}\multicolumn{3}{r}{\small\textit{continued on next page}}\\\endfoot\cline{1-2}
\endlastfoot\raggedright \_\-\_\-c\-l\-a\-s\-s\-\_\-\_\- & \raggedright \textbf{Value:} 
{\tt {\textless}attribute '\_\_class\_\_' of 'object' objects{\textgreater}}&\\
\cline{1-2}
\end{longtable}


%%%%%%%%%%%%%%%%%%%%%%%%%%%%%%%%%%%%%%%%%%%%%%%%%%%%%%%%%%%%%%%%%%%%%%%%%%%
%%                          Instance Variables                           %%
%%%%%%%%%%%%%%%%%%%%%%%%%%%%%%%%%%%%%%%%%%%%%%%%%%%%%%%%%%%%%%%%%%%%%%%%%%%

  \subsubsection{Instance Variables}

\begin{longtable}{|p{.30\textwidth}|p{.62\textwidth}|l}
\cline{1-2}
\cline{1-2} \centering \textbf{Name} & \centering \textbf{Description}& \\
\cline{1-2}
\endhead\cline{1-2}\multicolumn{3}{r}{\small\textit{continued on next page}}\\\endfoot\cline{1-2}
\endlastfoot\raggedright r\-a\-t\-e\- & Property that contains the crossover rate.&\\
\cline{1-2}
\end{longtable}

    \index{peach \textit{(package)}!peach.ga \textit{(package)}!peach.ga.crossover \textit{(module)}!peach.ga.crossover.TwoPoint \textit{(class)}|)}

%%%%%%%%%%%%%%%%%%%%%%%%%%%%%%%%%%%%%%%%%%%%%%%%%%%%%%%%%%%%%%%%%%%%%%%%%%%
%%                           Class Description                           %%
%%%%%%%%%%%%%%%%%%%%%%%%%%%%%%%%%%%%%%%%%%%%%%%%%%%%%%%%%%%%%%%%%%%%%%%%%%%

    \index{peach \textit{(package)}!peach.ga \textit{(package)}!peach.ga.crossover \textit{(module)}!peach.ga.crossover.Uniform \textit{(class)}|(}
\subsection{Class Uniform}

    \label{peach:ga:crossover:Uniform}
\begin{tabular}{cccccccc}
% Line for object, linespec=[False, False]
\multicolumn{2}{r}{\settowidth{\BCL}{object}\multirow{2}{\BCL}{object}}
&&
&&
  \\\cline{3-3}
  &&\multicolumn{1}{c|}{}
&&
&&
  \\
% Line for peach.ga.crossover.Crossover, linespec=[False]
\multicolumn{4}{r}{\settowidth{\BCL}{peach.ga.crossover.Crossover}\multirow{2}{\BCL}{peach.ga.crossover.Crossover}}
&&
  \\\cline{5-5}
  &&&&\multicolumn{1}{c|}{}
&&
  \\
&&&&\multicolumn{2}{l}{\textbf{peach.ga.crossover.Uniform}}
\end{tabular}


A uniform crossover operator.

A uniform crossover scans two chromosomes in a bit-to-bit fashion. According
to a given crossover rate, the corresponding bits are exchanged. The
crossover rate is the probability that two bits will be exchanged.

%%%%%%%%%%%%%%%%%%%%%%%%%%%%%%%%%%%%%%%%%%%%%%%%%%%%%%%%%%%%%%%%%%%%%%%%%%%
%%                                Methods                                %%
%%%%%%%%%%%%%%%%%%%%%%%%%%%%%%%%%%%%%%%%%%%%%%%%%%%%%%%%%%%%%%%%%%%%%%%%%%%

  \subsubsection{Methods}

    \vspace{0.5ex}

    \begin{boxedminipage}{\textwidth}

    \raggedright \textbf{\_\_init\_\_}(\textit{self}, \textit{rate}=\texttt{0.75})

    \vspace{-1.5ex}

    \rule{\textwidth}{0.5\fboxrule}

Initialize the crossover operator.
    \vspace{1ex}

      \textbf{Parameters}
      \begin{quote}
        \begin{Ventry}{xxxx}

          \item[rate]


Probability that bits from two paired chromosomes will be exchanged.
        \end{Ventry}

      \end{quote}

    \vspace{1ex}

      Overrides: object.\_\_init\_\_

    \end{boxedminipage}

    \label{peach:ga:crossover:Uniform:__call__}
    \index{peach \textit{(package)}!peach.ga \textit{(package)}!peach.ga.crossover \textit{(module)}!peach.ga.crossover.Uniform \textit{(class)}!peach.ga.crossover.Uniform.\_\_call\_\_ \textit{(method)}}

    \vspace{0.5ex}

    \begin{boxedminipage}{\textwidth}

    \raggedright \textbf{\_\_call\_\_}(\textit{self}, \textit{population})

    \vspace{-1.5ex}

    \rule{\textwidth}{0.5\fboxrule}

Proceeds the crossover over a population.

In uniform crossover, chromosomes from a population are randomly paired,
and scaned in a bit-to-bit fashion. If a uniform random number is below
the \texttt{rate} given in the instantiation of the operator, then the bits
under scan will be exchanged in the chromosomes.
    \vspace{1ex}

      \textbf{Parameters}
      \begin{quote}
        \begin{Ventry}{xxxxxxxxxx}

          \item[population]


A list of \texttt{Chromosomes} containing the present population of the
algorithm. It is processed and the results of the exchange are
returned to the caller.
        \end{Ventry}

      \end{quote}

    \vspace{1ex}

      \textbf{Return Value}
      \begin{quote}

The processed population, a list of \texttt{Chromosomes}.
      \end{quote}

    \vspace{1ex}

    \end{boxedminipage}

    \label{object:__delattr__}
    \index{object.\_\_delattr\_\_ \textit{(function)}}

    \vspace{0.5ex}

    \begin{boxedminipage}{\textwidth}

    \raggedright \textbf{\_\_delattr\_\_}(\textit{...})

    \vspace{-1.5ex}

    \rule{\textwidth}{0.5\fboxrule}

x.{\_}{\_}delattr{\_}{\_}('name') {\textless}=={\textgreater} del x.name
    \vspace{1ex}

    \end{boxedminipage}

    \label{object:__getattribute__}
    \index{object.\_\_getattribute\_\_ \textit{(function)}}

    \vspace{0.5ex}

    \begin{boxedminipage}{\textwidth}

    \raggedright \textbf{\_\_getattribute\_\_}(\textit{...})

    \vspace{-1.5ex}

    \rule{\textwidth}{0.5\fboxrule}

x.{\_}{\_}getattribute{\_}{\_}('name') {\textless}=={\textgreater} x.name
    \vspace{1ex}

    \end{boxedminipage}

    \label{object:__hash__}
    \index{object.\_\_hash\_\_ \textit{(function)}}

    \vspace{0.5ex}

    \begin{boxedminipage}{\textwidth}

    \raggedright \textbf{\_\_hash\_\_}(\textit{x})

    \vspace{-1.5ex}

    \rule{\textwidth}{0.5\fboxrule}

hash(x)
    \vspace{1ex}

    \end{boxedminipage}

    \label{object:__new__}
    \index{object.\_\_new\_\_ \textit{(function)}}

    \vspace{0.5ex}

    \begin{boxedminipage}{\textwidth}

    \raggedright \textbf{\_\_new\_\_}(\textit{T}, \textit{S}, \textit{...})

      \textbf{Return Value}
      \begin{quote}
\begin{alltt}
a new object with type S, a subtype of T
\end{alltt}

      \end{quote}

    \vspace{1ex}

    \end{boxedminipage}

    \label{object:__reduce__}
    \index{object.\_\_reduce\_\_ \textit{(function)}}

    \vspace{0.5ex}

    \begin{boxedminipage}{\textwidth}

    \raggedright \textbf{\_\_reduce\_\_}(\textit{...})

    \vspace{-1.5ex}

    \rule{\textwidth}{0.5\fboxrule}

helper for pickle
    \vspace{1ex}

    \end{boxedminipage}

    \label{object:__reduce_ex__}
    \index{object.\_\_reduce\_ex\_\_ \textit{(function)}}

    \vspace{0.5ex}

    \begin{boxedminipage}{\textwidth}

    \raggedright \textbf{\_\_reduce\_ex\_\_}(\textit{...})

    \vspace{-1.5ex}

    \rule{\textwidth}{0.5\fboxrule}

helper for pickle
    \vspace{1ex}

    \end{boxedminipage}

    \label{object:__repr__}
    \index{object.\_\_repr\_\_ \textit{(function)}}

    \vspace{0.5ex}

    \begin{boxedminipage}{\textwidth}

    \raggedright \textbf{\_\_repr\_\_}(\textit{x})

    \vspace{-1.5ex}

    \rule{\textwidth}{0.5\fboxrule}

repr(x)
    \vspace{1ex}

    \end{boxedminipage}

    \label{object:__setattr__}
    \index{object.\_\_setattr\_\_ \textit{(function)}}

    \vspace{0.5ex}

    \begin{boxedminipage}{\textwidth}

    \raggedright \textbf{\_\_setattr\_\_}(\textit{...})

    \vspace{-1.5ex}

    \rule{\textwidth}{0.5\fboxrule}

x.{\_}{\_}setattr{\_}{\_}('name', value) {\textless}=={\textgreater} x.name = value
    \vspace{1ex}

    \end{boxedminipage}

    \label{object:__str__}
    \index{object.\_\_str\_\_ \textit{(function)}}

    \vspace{0.5ex}

    \begin{boxedminipage}{\textwidth}

    \raggedright \textbf{\_\_str\_\_}(\textit{x})

    \vspace{-1.5ex}

    \rule{\textwidth}{0.5\fboxrule}

str(x)
    \vspace{1ex}

    \end{boxedminipage}


%%%%%%%%%%%%%%%%%%%%%%%%%%%%%%%%%%%%%%%%%%%%%%%%%%%%%%%%%%%%%%%%%%%%%%%%%%%
%%                              Properties                               %%
%%%%%%%%%%%%%%%%%%%%%%%%%%%%%%%%%%%%%%%%%%%%%%%%%%%%%%%%%%%%%%%%%%%%%%%%%%%

  \subsubsection{Properties}

\begin{longtable}{|p{.30\textwidth}|p{.62\textwidth}|l}
\cline{1-2}
\cline{1-2} \centering \textbf{Name} & \centering \textbf{Description}& \\
\cline{1-2}
\endhead\cline{1-2}\multicolumn{3}{r}{\small\textit{continued on next page}}\\\endfoot\cline{1-2}
\endlastfoot\raggedright \_\-\_\-c\-l\-a\-s\-s\-\_\-\_\- & \raggedright \textbf{Value:} 
{\tt {\textless}attribute '\_\_class\_\_' of 'object' objects{\textgreater}}&\\
\cline{1-2}
\end{longtable}


%%%%%%%%%%%%%%%%%%%%%%%%%%%%%%%%%%%%%%%%%%%%%%%%%%%%%%%%%%%%%%%%%%%%%%%%%%%
%%                          Instance Variables                           %%
%%%%%%%%%%%%%%%%%%%%%%%%%%%%%%%%%%%%%%%%%%%%%%%%%%%%%%%%%%%%%%%%%%%%%%%%%%%

  \subsubsection{Instance Variables}

\begin{longtable}{|p{.30\textwidth}|p{.62\textwidth}|l}
\cline{1-2}
\cline{1-2} \centering \textbf{Name} & \centering \textbf{Description}& \\
\cline{1-2}
\endhead\cline{1-2}\multicolumn{3}{r}{\small\textit{continued on next page}}\\\endfoot\cline{1-2}
\endlastfoot\raggedright r\-a\-t\-e\- & Property that contains the crossover rate.&\\
\cline{1-2}
\end{longtable}

    \index{peach \textit{(package)}!peach.ga \textit{(package)}!peach.ga.crossover \textit{(module)}!peach.ga.crossover.Uniform \textit{(class)}|)}
    \index{peach \textit{(package)}!peach.ga \textit{(package)}!peach.ga.crossover \textit{(module)}|)}
