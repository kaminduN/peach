%
% API Documentation for Peach - Computational Intelligence for Python
% Module peach.nn.af
%
% Generated by epydoc 3.0beta1
% [Mon Dec 21 08:51:37 2009]
%

%%%%%%%%%%%%%%%%%%%%%%%%%%%%%%%%%%%%%%%%%%%%%%%%%%%%%%%%%%%%%%%%%%%%%%%%%%%
%%                          Module Description                           %%
%%%%%%%%%%%%%%%%%%%%%%%%%%%%%%%%%%%%%%%%%%%%%%%%%%%%%%%%%%%%%%%%%%%%%%%%%%%

    \index{peach \textit{(package)}!peach.nn \textit{(package)}!peach.nn.af \textit{(module)}|(}
\section{Module peach.nn.af}

    \label{peach:nn:af}

Base activation functions and base class

Activation functions define if a neuron is activated or not. There are a lot of
different definitions for activation functions in the literature, and this
sub-package implements some of them. An activation function is defined by its
response and its derivative. Being conveniently defined as classes, it is
possible to define a custom derivative method.

In this package, also, there is a base class that should be subclassed if you
want to define your own activation function. This class, however, can be
instantiated with a standard Python function as an initialization parameter, and
it is adjusted to work with the internals of the package.

If the base class is instantiated, then the function should take a real number
as input, and return a real number. The response of the function determines if
the neuron is activated or not.

%%%%%%%%%%%%%%%%%%%%%%%%%%%%%%%%%%%%%%%%%%%%%%%%%%%%%%%%%%%%%%%%%%%%%%%%%%%
%%                               Variables                               %%
%%%%%%%%%%%%%%%%%%%%%%%%%%%%%%%%%%%%%%%%%%%%%%%%%%%%%%%%%%%%%%%%%%%%%%%%%%%

  \subsection{Variables}

\begin{longtable}{|p{.30\textwidth}|p{.62\textwidth}|l}
\cline{1-2}
\cline{1-2} \centering \textbf{Name} & \centering \textbf{Description}& \\
\cline{1-2}
\endhead\cline{1-2}\multicolumn{3}{r}{\small\textit{continued on next page}}\\\endfoot\cline{1-2}
\endlastfoot\raggedright \_\-\_\-d\-o\-c\-\_\-\_\- & \raggedright \textbf{Value:} 
{\tt \texttt{...}}&\\
\cline{1-2}
\end{longtable}


%%%%%%%%%%%%%%%%%%%%%%%%%%%%%%%%%%%%%%%%%%%%%%%%%%%%%%%%%%%%%%%%%%%%%%%%%%%
%%                           Class Description                           %%
%%%%%%%%%%%%%%%%%%%%%%%%%%%%%%%%%%%%%%%%%%%%%%%%%%%%%%%%%%%%%%%%%%%%%%%%%%%

    \index{peach \textit{(package)}!peach.nn \textit{(package)}!peach.nn.af \textit{(module)}!peach.nn.af.Activation \textit{(class)}|(}
\subsection{Class Activation}

    \label{peach:nn:af:Activation}
\begin{tabular}{cccccc}
% Line for object, linespec=[False]
\multicolumn{2}{r}{\settowidth{\BCL}{object}\multirow{2}{\BCL}{object}}
&&
  \\\cline{3-3}
  &&\multicolumn{1}{c|}{}
&&
  \\
&&\multicolumn{2}{l}{\textbf{peach.nn.af.Activation}}
\end{tabular}

\textbf{Known Subclasses:}
peach.nn.af.ArcTan,
    peach.nn.af.Linear,
    peach.nn.af.Sigmoid,
    peach.nn.af.Ramp,
    peach.nn.af.Signum,
    peach.nn.af.Threshold,
    peach.nn.af.TanH


Base class for activation functions.

This class can be used as base for activation functions. A subclass should
have at least three methods, described below:
\begin{quote}
\begin{description}
%[visit_definition_list_item]
\item[{{\_}{\_}init{\_}{\_}}] %[visit_definition]

This method should be used to configure the function. In general, some
parameters to change the behaviour of a simple function is passed. In a
subclass, the \texttt{{\_}{\_}init{\_}{\_}} method should call the mother class
initialization procedure.

%[depart_definition]
%[depart_definition_list_item]
%[visit_definition_list_item]
\item[{{\_}{\_}call{\_}{\_}}] %[visit_definition]

The \texttt{{\_}{\_}call{\_}{\_}} interface is the function call. It should receive a
\emph{vector} of real numbers and return a \emph{vector} of real numbers. Using
the capabilities of the \texttt{numpy} module will help a lot. In case you
don't know how to use, maybe instantiating this class instead will work
better (see below).

%[depart_definition]
%[depart_definition_list_item]
%[visit_definition_list_item]
\item[{derivative}] %[visit_definition]

This method implements the derivative of the activation function. It is
used in the learning methods. If one is not provided (but remember to
call the superclass \texttt{{\_}{\_}init{\_}{\_}} so that it is created).

%[depart_definition]
%[depart_definition_list_item]
\end{description}
\end{quote}

%%%%%%%%%%%%%%%%%%%%%%%%%%%%%%%%%%%%%%%%%%%%%%%%%%%%%%%%%%%%%%%%%%%%%%%%%%%
%%                                Methods                                %%
%%%%%%%%%%%%%%%%%%%%%%%%%%%%%%%%%%%%%%%%%%%%%%%%%%%%%%%%%%%%%%%%%%%%%%%%%%%

  \subsubsection{Methods}

    \vspace{0.5ex}

    \begin{boxedminipage}{\textwidth}

    \raggedright \textbf{\_\_init\_\_}(\textit{self}, \textit{f}=\texttt{None}, \textit{df}=\texttt{None})

    \vspace{-1.5ex}

    \rule{\textwidth}{0.5\fboxrule}

Initializes the activation function.

Instantiating this class creates and adjusts a standard Python function
to work with layers of neurons.
    \vspace{1ex}

      \textbf{Parameters}
      \begin{quote}
        \begin{Ventry}{xx}

          \item[f]


The activation function. It can be created as a lambda function or
any other method, but it should take a real value, corresponding to
the activation potential of a neuron, and return a real value,
corresponding to its activation. Defaults to \texttt{None}, if none is
given, the identity function is used.
          \item[df]


The derivative of the above function. It can be defined as above, or
not given. If not given, an estimate is calculated based on the
given function. Defaults to \texttt{None}.
        \end{Ventry}

      \end{quote}

    \vspace{1ex}

      Overrides: object.\_\_init\_\_

    \end{boxedminipage}

    \label{peach:nn:af:Activation:__call__}
    \index{peach \textit{(package)}!peach.nn \textit{(package)}!peach.nn.af \textit{(module)}!peach.nn.af.Activation \textit{(class)}!peach.nn.af.Activation.\_\_call\_\_ \textit{(method)}}

    \vspace{0.5ex}

    \begin{boxedminipage}{\textwidth}

    \raggedright \textbf{\_\_call\_\_}(\textit{self}, \textit{x})

    \vspace{-1.5ex}

    \rule{\textwidth}{0.5\fboxrule}

Call interface to the object.

This method applies the activation function over a vector of activation
potentials, and returns the results.
    \vspace{1ex}

      \textbf{Parameters}
      \begin{quote}
        \begin{Ventry}{x}

          \item[x]


A real number or a vector of real numbers representing the
activation potential of a neuron or a layer of neurons.
        \end{Ventry}

      \end{quote}

    \vspace{1ex}

      \textbf{Return Value}
      \begin{quote}

The activation function applied over the input vector.
      \end{quote}

    \vspace{1ex}

    \end{boxedminipage}

    \label{peach:nn:af:Activation:derivative}
    \index{peach \textit{(package)}!peach.nn \textit{(package)}!peach.nn.af \textit{(module)}!peach.nn.af.Activation \textit{(class)}!peach.nn.af.Activation.derivative \textit{(method)}}

    \vspace{0.5ex}

    \begin{boxedminipage}{\textwidth}

    \raggedright \textbf{derivative}(\textit{self}, \textit{x}, \textit{dx}=\texttt{5e-05})

    \vspace{-1.5ex}

    \rule{\textwidth}{0.5\fboxrule}

An estimate of the derivative of the activation function.

This method estimates the derivative using difference equations. This is
a simple estimate, but efficient nonetheless.
    \vspace{1ex}

      \textbf{Parameters}
      \begin{quote}
        \begin{Ventry}{xx}

          \item[x]


A real number or vector of real numbers representing the point over
which the derivative is to be calculated.
          \item[dx]


The value of the interval of the estimate. The smaller this number
is, the better. However, if made too small, the precision is not
enough to avoid errors. This defaults to 5e-5, which is the values
that gives the best results.
        \end{Ventry}

      \end{quote}

    \vspace{1ex}

      \textbf{Return Value}
      \begin{quote}

The value of the derivative over the given point.
      \end{quote}

    \vspace{1ex}

    \end{boxedminipage}

    \label{object:__delattr__}
    \index{object.\_\_delattr\_\_ \textit{(function)}}

    \vspace{0.5ex}

    \begin{boxedminipage}{\textwidth}

    \raggedright \textbf{\_\_delattr\_\_}(\textit{...})

    \vspace{-1.5ex}

    \rule{\textwidth}{0.5\fboxrule}

x.{\_}{\_}delattr{\_}{\_}('name') {\textless}=={\textgreater} del x.name
    \vspace{1ex}

    \end{boxedminipage}

    \label{object:__getattribute__}
    \index{object.\_\_getattribute\_\_ \textit{(function)}}

    \vspace{0.5ex}

    \begin{boxedminipage}{\textwidth}

    \raggedright \textbf{\_\_getattribute\_\_}(\textit{...})

    \vspace{-1.5ex}

    \rule{\textwidth}{0.5\fboxrule}

x.{\_}{\_}getattribute{\_}{\_}('name') {\textless}=={\textgreater} x.name
    \vspace{1ex}

    \end{boxedminipage}

    \label{object:__hash__}
    \index{object.\_\_hash\_\_ \textit{(function)}}

    \vspace{0.5ex}

    \begin{boxedminipage}{\textwidth}

    \raggedright \textbf{\_\_hash\_\_}(\textit{x})

    \vspace{-1.5ex}

    \rule{\textwidth}{0.5\fboxrule}

hash(x)
    \vspace{1ex}

    \end{boxedminipage}

    \label{object:__new__}
    \index{object.\_\_new\_\_ \textit{(function)}}

    \vspace{0.5ex}

    \begin{boxedminipage}{\textwidth}

    \raggedright \textbf{\_\_new\_\_}(\textit{T}, \textit{S}, \textit{...})

      \textbf{Return Value}
      \begin{quote}
\begin{alltt}
a new object with type S, a subtype of T
\end{alltt}

      \end{quote}

    \vspace{1ex}

    \end{boxedminipage}

    \label{object:__reduce__}
    \index{object.\_\_reduce\_\_ \textit{(function)}}

    \vspace{0.5ex}

    \begin{boxedminipage}{\textwidth}

    \raggedright \textbf{\_\_reduce\_\_}(\textit{...})

    \vspace{-1.5ex}

    \rule{\textwidth}{0.5\fboxrule}

helper for pickle
    \vspace{1ex}

    \end{boxedminipage}

    \label{object:__reduce_ex__}
    \index{object.\_\_reduce\_ex\_\_ \textit{(function)}}

    \vspace{0.5ex}

    \begin{boxedminipage}{\textwidth}

    \raggedright \textbf{\_\_reduce\_ex\_\_}(\textit{...})

    \vspace{-1.5ex}

    \rule{\textwidth}{0.5\fboxrule}

helper for pickle
    \vspace{1ex}

    \end{boxedminipage}

    \label{object:__repr__}
    \index{object.\_\_repr\_\_ \textit{(function)}}

    \vspace{0.5ex}

    \begin{boxedminipage}{\textwidth}

    \raggedright \textbf{\_\_repr\_\_}(\textit{x})

    \vspace{-1.5ex}

    \rule{\textwidth}{0.5\fboxrule}

repr(x)
    \vspace{1ex}

    \end{boxedminipage}

    \label{object:__setattr__}
    \index{object.\_\_setattr\_\_ \textit{(function)}}

    \vspace{0.5ex}

    \begin{boxedminipage}{\textwidth}

    \raggedright \textbf{\_\_setattr\_\_}(\textit{...})

    \vspace{-1.5ex}

    \rule{\textwidth}{0.5\fboxrule}

x.{\_}{\_}setattr{\_}{\_}('name', value) {\textless}=={\textgreater} x.name = value
    \vspace{1ex}

    \end{boxedminipage}

    \label{object:__str__}
    \index{object.\_\_str\_\_ \textit{(function)}}

    \vspace{0.5ex}

    \begin{boxedminipage}{\textwidth}

    \raggedright \textbf{\_\_str\_\_}(\textit{x})

    \vspace{-1.5ex}

    \rule{\textwidth}{0.5\fboxrule}

str(x)
    \vspace{1ex}

    \end{boxedminipage}


%%%%%%%%%%%%%%%%%%%%%%%%%%%%%%%%%%%%%%%%%%%%%%%%%%%%%%%%%%%%%%%%%%%%%%%%%%%
%%                              Properties                               %%
%%%%%%%%%%%%%%%%%%%%%%%%%%%%%%%%%%%%%%%%%%%%%%%%%%%%%%%%%%%%%%%%%%%%%%%%%%%

  \subsubsection{Properties}

\begin{longtable}{|p{.30\textwidth}|p{.62\textwidth}|l}
\cline{1-2}
\cline{1-2} \centering \textbf{Name} & \centering \textbf{Description}& \\
\cline{1-2}
\endhead\cline{1-2}\multicolumn{3}{r}{\small\textit{continued on next page}}\\\endfoot\cline{1-2}
\endlastfoot\raggedright \_\-\_\-c\-l\-a\-s\-s\-\_\-\_\- & \raggedright \textbf{Value:} 
{\tt {\textless}attribute '\_\_class\_\_' of 'object' objects{\textgreater}}&\\
\cline{1-2}
\end{longtable}


%%%%%%%%%%%%%%%%%%%%%%%%%%%%%%%%%%%%%%%%%%%%%%%%%%%%%%%%%%%%%%%%%%%%%%%%%%%
%%                          Instance Variables                           %%
%%%%%%%%%%%%%%%%%%%%%%%%%%%%%%%%%%%%%%%%%%%%%%%%%%%%%%%%%%%%%%%%%%%%%%%%%%%

  \subsubsection{Instance Variables}

\begin{longtable}{|p{.30\textwidth}|p{.62\textwidth}|l}
\cline{1-2}
\cline{1-2} \centering \textbf{Name} & \centering \textbf{Description}& \\
\cline{1-2}
\endhead\cline{1-2}\multicolumn{3}{r}{\small\textit{continued on next page}}\\\endfoot\cline{1-2}
\endlastfoot\raggedright d\- & An alias to the derivative of the function.&\\
\cline{1-2}
\end{longtable}

    \index{peach \textit{(package)}!peach.nn \textit{(package)}!peach.nn.af \textit{(module)}!peach.nn.af.Activation \textit{(class)}|)}

%%%%%%%%%%%%%%%%%%%%%%%%%%%%%%%%%%%%%%%%%%%%%%%%%%%%%%%%%%%%%%%%%%%%%%%%%%%
%%                           Class Description                           %%
%%%%%%%%%%%%%%%%%%%%%%%%%%%%%%%%%%%%%%%%%%%%%%%%%%%%%%%%%%%%%%%%%%%%%%%%%%%

    \index{peach \textit{(package)}!peach.nn \textit{(package)}!peach.nn.af \textit{(module)}!peach.nn.af.Threshold \textit{(class)}|(}
\subsection{Class Threshold}

    \label{peach:nn:af:Threshold}
\begin{tabular}{cccccccc}
% Line for object, linespec=[False, False]
\multicolumn{2}{r}{\settowidth{\BCL}{object}\multirow{2}{\BCL}{object}}
&&
&&
  \\\cline{3-3}
  &&\multicolumn{1}{c|}{}
&&
&&
  \\
% Line for peach.nn.af.Activation, linespec=[False]
\multicolumn{4}{r}{\settowidth{\BCL}{peach.nn.af.Activation}\multirow{2}{\BCL}{peach.nn.af.Activation}}
&&
  \\\cline{5-5}
  &&&&\multicolumn{1}{c|}{}
&&
  \\
&&&&\multicolumn{2}{l}{\textbf{peach.nn.af.Threshold}}
\end{tabular}


Threshold activation function.

%%%%%%%%%%%%%%%%%%%%%%%%%%%%%%%%%%%%%%%%%%%%%%%%%%%%%%%%%%%%%%%%%%%%%%%%%%%
%%                                Methods                                %%
%%%%%%%%%%%%%%%%%%%%%%%%%%%%%%%%%%%%%%%%%%%%%%%%%%%%%%%%%%%%%%%%%%%%%%%%%%%

  \subsubsection{Methods}

    \vspace{0.5ex}

    \begin{boxedminipage}{\textwidth}

    \raggedright \textbf{\_\_init\_\_}(\textit{self}, \textit{threshold}=\texttt{0.0}, \textit{amplitude}=\texttt{1.0})

    \vspace{-1.5ex}

    \rule{\textwidth}{0.5\fboxrule}

Initializes the object.
    \vspace{1ex}

      \textbf{Parameters}
      \begin{quote}
        \begin{Ventry}{xxxxxxxxx}

          \item[threshold]


The threshold value. If the value of the input is lower than this,
the function is 0, otherwise, it is the given \texttt{amplitude}.
          \item[amplitude]


The maximum value of the function.
        \end{Ventry}

      \end{quote}

    \vspace{1ex}

      Overrides: peach.nn.af.Activation.\_\_init\_\_

    \end{boxedminipage}

    \vspace{0.5ex}

    \begin{boxedminipage}{\textwidth}

    \raggedright \textbf{\_\_call\_\_}(\textit{self}, \textit{x})

    \vspace{-1.5ex}

    \rule{\textwidth}{0.5\fboxrule}

Call interface to the object.

This method applies the activation function over a vector of activation
potentials, and returns the results.
    \vspace{1ex}

      \textbf{Parameters}
      \begin{quote}
        \begin{Ventry}{x}

          \item[x]


A real number or a vector of real numbers representing the
activation potential of a neuron or a layer of neurons.
        \end{Ventry}

      \end{quote}

    \vspace{1ex}

      \textbf{Return Value}
      \begin{quote}

The activation function applied over the input vector.
      \end{quote}

    \vspace{1ex}

      Overrides: peach.nn.af.Activation.\_\_call\_\_

    \end{boxedminipage}

    \vspace{0.5ex}

    \begin{boxedminipage}{\textwidth}

    \raggedright \textbf{derivative}(\textit{self}, \textit{x})

    \vspace{-1.5ex}

    \rule{\textwidth}{0.5\fboxrule}

The function derivative. Technically, this function doesn't have a
derivative, but making it equals to 1, this can be used in learning
algorithms.
    \vspace{1ex}

      \textbf{Parameters}
      \begin{quote}
        \begin{Ventry}{x}

          \item[x]


A real number or a vector of real numbers representing the
activation potential of a neuron or a layer of neurons.
        \end{Ventry}

      \end{quote}

    \vspace{1ex}

      \textbf{Return Value}
      \begin{quote}

The derivative of the activation function applied over the input
vector.
      \end{quote}

    \vspace{1ex}

      Overrides: peach.nn.af.Activation.derivative

    \end{boxedminipage}

    \label{object:__delattr__}
    \index{object.\_\_delattr\_\_ \textit{(function)}}

    \vspace{0.5ex}

    \begin{boxedminipage}{\textwidth}

    \raggedright \textbf{\_\_delattr\_\_}(\textit{...})

    \vspace{-1.5ex}

    \rule{\textwidth}{0.5\fboxrule}

x.{\_}{\_}delattr{\_}{\_}('name') {\textless}=={\textgreater} del x.name
    \vspace{1ex}

    \end{boxedminipage}

    \label{object:__getattribute__}
    \index{object.\_\_getattribute\_\_ \textit{(function)}}

    \vspace{0.5ex}

    \begin{boxedminipage}{\textwidth}

    \raggedright \textbf{\_\_getattribute\_\_}(\textit{...})

    \vspace{-1.5ex}

    \rule{\textwidth}{0.5\fboxrule}

x.{\_}{\_}getattribute{\_}{\_}('name') {\textless}=={\textgreater} x.name
    \vspace{1ex}

    \end{boxedminipage}

    \label{object:__hash__}
    \index{object.\_\_hash\_\_ \textit{(function)}}

    \vspace{0.5ex}

    \begin{boxedminipage}{\textwidth}

    \raggedright \textbf{\_\_hash\_\_}(\textit{x})

    \vspace{-1.5ex}

    \rule{\textwidth}{0.5\fboxrule}

hash(x)
    \vspace{1ex}

    \end{boxedminipage}

    \label{object:__new__}
    \index{object.\_\_new\_\_ \textit{(function)}}

    \vspace{0.5ex}

    \begin{boxedminipage}{\textwidth}

    \raggedright \textbf{\_\_new\_\_}(\textit{T}, \textit{S}, \textit{...})

      \textbf{Return Value}
      \begin{quote}
\begin{alltt}
a new object with type S, a subtype of T
\end{alltt}

      \end{quote}

    \vspace{1ex}

    \end{boxedminipage}

    \label{object:__reduce__}
    \index{object.\_\_reduce\_\_ \textit{(function)}}

    \vspace{0.5ex}

    \begin{boxedminipage}{\textwidth}

    \raggedright \textbf{\_\_reduce\_\_}(\textit{...})

    \vspace{-1.5ex}

    \rule{\textwidth}{0.5\fboxrule}

helper for pickle
    \vspace{1ex}

    \end{boxedminipage}

    \label{object:__reduce_ex__}
    \index{object.\_\_reduce\_ex\_\_ \textit{(function)}}

    \vspace{0.5ex}

    \begin{boxedminipage}{\textwidth}

    \raggedright \textbf{\_\_reduce\_ex\_\_}(\textit{...})

    \vspace{-1.5ex}

    \rule{\textwidth}{0.5\fboxrule}

helper for pickle
    \vspace{1ex}

    \end{boxedminipage}

    \label{object:__repr__}
    \index{object.\_\_repr\_\_ \textit{(function)}}

    \vspace{0.5ex}

    \begin{boxedminipage}{\textwidth}

    \raggedright \textbf{\_\_repr\_\_}(\textit{x})

    \vspace{-1.5ex}

    \rule{\textwidth}{0.5\fboxrule}

repr(x)
    \vspace{1ex}

    \end{boxedminipage}

    \label{object:__setattr__}
    \index{object.\_\_setattr\_\_ \textit{(function)}}

    \vspace{0.5ex}

    \begin{boxedminipage}{\textwidth}

    \raggedright \textbf{\_\_setattr\_\_}(\textit{...})

    \vspace{-1.5ex}

    \rule{\textwidth}{0.5\fboxrule}

x.{\_}{\_}setattr{\_}{\_}('name', value) {\textless}=={\textgreater} x.name = value
    \vspace{1ex}

    \end{boxedminipage}

    \label{object:__str__}
    \index{object.\_\_str\_\_ \textit{(function)}}

    \vspace{0.5ex}

    \begin{boxedminipage}{\textwidth}

    \raggedright \textbf{\_\_str\_\_}(\textit{x})

    \vspace{-1.5ex}

    \rule{\textwidth}{0.5\fboxrule}

str(x)
    \vspace{1ex}

    \end{boxedminipage}


%%%%%%%%%%%%%%%%%%%%%%%%%%%%%%%%%%%%%%%%%%%%%%%%%%%%%%%%%%%%%%%%%%%%%%%%%%%
%%                              Properties                               %%
%%%%%%%%%%%%%%%%%%%%%%%%%%%%%%%%%%%%%%%%%%%%%%%%%%%%%%%%%%%%%%%%%%%%%%%%%%%

  \subsubsection{Properties}

\begin{longtable}{|p{.30\textwidth}|p{.62\textwidth}|l}
\cline{1-2}
\cline{1-2} \centering \textbf{Name} & \centering \textbf{Description}& \\
\cline{1-2}
\endhead\cline{1-2}\multicolumn{3}{r}{\small\textit{continued on next page}}\\\endfoot\cline{1-2}
\endlastfoot\raggedright \_\-\_\-c\-l\-a\-s\-s\-\_\-\_\- & \raggedright \textbf{Value:} 
{\tt {\textless}attribute '\_\_class\_\_' of 'object' objects{\textgreater}}&\\
\cline{1-2}
\end{longtable}


%%%%%%%%%%%%%%%%%%%%%%%%%%%%%%%%%%%%%%%%%%%%%%%%%%%%%%%%%%%%%%%%%%%%%%%%%%%
%%                          Instance Variables                           %%
%%%%%%%%%%%%%%%%%%%%%%%%%%%%%%%%%%%%%%%%%%%%%%%%%%%%%%%%%%%%%%%%%%%%%%%%%%%

  \subsubsection{Instance Variables}

\begin{longtable}{|p{.30\textwidth}|p{.62\textwidth}|l}
\cline{1-2}
\cline{1-2} \centering \textbf{Name} & \centering \textbf{Description}& \\
\cline{1-2}
\endhead\cline{1-2}\multicolumn{3}{r}{\small\textit{continued on next page}}\\\endfoot\cline{1-2}
\endlastfoot\raggedright d\- & An alias to the derivative of the function.&\\
\cline{1-2}
\end{longtable}

    \index{peach \textit{(package)}!peach.nn \textit{(package)}!peach.nn.af \textit{(module)}!peach.nn.af.Threshold \textit{(class)}|)}

%%%%%%%%%%%%%%%%%%%%%%%%%%%%%%%%%%%%%%%%%%%%%%%%%%%%%%%%%%%%%%%%%%%%%%%%%%%
%%                           Class Description                           %%
%%%%%%%%%%%%%%%%%%%%%%%%%%%%%%%%%%%%%%%%%%%%%%%%%%%%%%%%%%%%%%%%%%%%%%%%%%%

    \index{peach \textit{(package)}!peach.nn \textit{(package)}!peach.nn.af \textit{(module)}!peach.nn.af.Threshold \textit{(class)}|(}
\subsection{Class Threshold}

    \label{peach:nn:af:Threshold}
\begin{tabular}{cccccccc}
% Line for object, linespec=[False, False]
\multicolumn{2}{r}{\settowidth{\BCL}{object}\multirow{2}{\BCL}{object}}
&&
&&
  \\\cline{3-3}
  &&\multicolumn{1}{c|}{}
&&
&&
  \\
% Line for peach.nn.af.Activation, linespec=[False]
\multicolumn{4}{r}{\settowidth{\BCL}{peach.nn.af.Activation}\multirow{2}{\BCL}{peach.nn.af.Activation}}
&&
  \\\cline{5-5}
  &&&&\multicolumn{1}{c|}{}
&&
  \\
&&&&\multicolumn{2}{l}{\textbf{peach.nn.af.Threshold}}
\end{tabular}


Threshold activation function.

%%%%%%%%%%%%%%%%%%%%%%%%%%%%%%%%%%%%%%%%%%%%%%%%%%%%%%%%%%%%%%%%%%%%%%%%%%%
%%                                Methods                                %%
%%%%%%%%%%%%%%%%%%%%%%%%%%%%%%%%%%%%%%%%%%%%%%%%%%%%%%%%%%%%%%%%%%%%%%%%%%%

  \subsubsection{Methods}

    \vspace{0.5ex}

    \begin{boxedminipage}{\textwidth}

    \raggedright \textbf{\_\_init\_\_}(\textit{self}, \textit{threshold}=\texttt{0.0}, \textit{amplitude}=\texttt{1.0})

    \vspace{-1.5ex}

    \rule{\textwidth}{0.5\fboxrule}

Initializes the object.
    \vspace{1ex}

      \textbf{Parameters}
      \begin{quote}
        \begin{Ventry}{xxxxxxxxx}

          \item[threshold]


The threshold value. If the value of the input is lower than this,
the function is 0, otherwise, it is the given \texttt{amplitude}.
          \item[amplitude]


The maximum value of the function.
        \end{Ventry}

      \end{quote}

    \vspace{1ex}

      Overrides: peach.nn.af.Activation.\_\_init\_\_

    \end{boxedminipage}

    \vspace{0.5ex}

    \begin{boxedminipage}{\textwidth}

    \raggedright \textbf{\_\_call\_\_}(\textit{self}, \textit{x})

    \vspace{-1.5ex}

    \rule{\textwidth}{0.5\fboxrule}

Call interface to the object.

This method applies the activation function over a vector of activation
potentials, and returns the results.
    \vspace{1ex}

      \textbf{Parameters}
      \begin{quote}
        \begin{Ventry}{x}

          \item[x]


A real number or a vector of real numbers representing the
activation potential of a neuron or a layer of neurons.
        \end{Ventry}

      \end{quote}

    \vspace{1ex}

      \textbf{Return Value}
      \begin{quote}

The activation function applied over the input vector.
      \end{quote}

    \vspace{1ex}

      Overrides: peach.nn.af.Activation.\_\_call\_\_

    \end{boxedminipage}

    \vspace{0.5ex}

    \begin{boxedminipage}{\textwidth}

    \raggedright \textbf{derivative}(\textit{self}, \textit{x})

    \vspace{-1.5ex}

    \rule{\textwidth}{0.5\fboxrule}

The function derivative. Technically, this function doesn't have a
derivative, but making it equals to 1, this can be used in learning
algorithms.
    \vspace{1ex}

      \textbf{Parameters}
      \begin{quote}
        \begin{Ventry}{x}

          \item[x]


A real number or a vector of real numbers representing the
activation potential of a neuron or a layer of neurons.
        \end{Ventry}

      \end{quote}

    \vspace{1ex}

      \textbf{Return Value}
      \begin{quote}

The derivative of the activation function applied over the input
vector.
      \end{quote}

    \vspace{1ex}

      Overrides: peach.nn.af.Activation.derivative

    \end{boxedminipage}

    \label{object:__delattr__}
    \index{object.\_\_delattr\_\_ \textit{(function)}}

    \vspace{0.5ex}

    \begin{boxedminipage}{\textwidth}

    \raggedright \textbf{\_\_delattr\_\_}(\textit{...})

    \vspace{-1.5ex}

    \rule{\textwidth}{0.5\fboxrule}

x.{\_}{\_}delattr{\_}{\_}('name') {\textless}=={\textgreater} del x.name
    \vspace{1ex}

    \end{boxedminipage}

    \label{object:__getattribute__}
    \index{object.\_\_getattribute\_\_ \textit{(function)}}

    \vspace{0.5ex}

    \begin{boxedminipage}{\textwidth}

    \raggedright \textbf{\_\_getattribute\_\_}(\textit{...})

    \vspace{-1.5ex}

    \rule{\textwidth}{0.5\fboxrule}

x.{\_}{\_}getattribute{\_}{\_}('name') {\textless}=={\textgreater} x.name
    \vspace{1ex}

    \end{boxedminipage}

    \label{object:__hash__}
    \index{object.\_\_hash\_\_ \textit{(function)}}

    \vspace{0.5ex}

    \begin{boxedminipage}{\textwidth}

    \raggedright \textbf{\_\_hash\_\_}(\textit{x})

    \vspace{-1.5ex}

    \rule{\textwidth}{0.5\fboxrule}

hash(x)
    \vspace{1ex}

    \end{boxedminipage}

    \label{object:__new__}
    \index{object.\_\_new\_\_ \textit{(function)}}

    \vspace{0.5ex}

    \begin{boxedminipage}{\textwidth}

    \raggedright \textbf{\_\_new\_\_}(\textit{T}, \textit{S}, \textit{...})

      \textbf{Return Value}
      \begin{quote}
\begin{alltt}
a new object with type S, a subtype of T
\end{alltt}

      \end{quote}

    \vspace{1ex}

    \end{boxedminipage}

    \label{object:__reduce__}
    \index{object.\_\_reduce\_\_ \textit{(function)}}

    \vspace{0.5ex}

    \begin{boxedminipage}{\textwidth}

    \raggedright \textbf{\_\_reduce\_\_}(\textit{...})

    \vspace{-1.5ex}

    \rule{\textwidth}{0.5\fboxrule}

helper for pickle
    \vspace{1ex}

    \end{boxedminipage}

    \label{object:__reduce_ex__}
    \index{object.\_\_reduce\_ex\_\_ \textit{(function)}}

    \vspace{0.5ex}

    \begin{boxedminipage}{\textwidth}

    \raggedright \textbf{\_\_reduce\_ex\_\_}(\textit{...})

    \vspace{-1.5ex}

    \rule{\textwidth}{0.5\fboxrule}

helper for pickle
    \vspace{1ex}

    \end{boxedminipage}

    \label{object:__repr__}
    \index{object.\_\_repr\_\_ \textit{(function)}}

    \vspace{0.5ex}

    \begin{boxedminipage}{\textwidth}

    \raggedright \textbf{\_\_repr\_\_}(\textit{x})

    \vspace{-1.5ex}

    \rule{\textwidth}{0.5\fboxrule}

repr(x)
    \vspace{1ex}

    \end{boxedminipage}

    \label{object:__setattr__}
    \index{object.\_\_setattr\_\_ \textit{(function)}}

    \vspace{0.5ex}

    \begin{boxedminipage}{\textwidth}

    \raggedright \textbf{\_\_setattr\_\_}(\textit{...})

    \vspace{-1.5ex}

    \rule{\textwidth}{0.5\fboxrule}

x.{\_}{\_}setattr{\_}{\_}('name', value) {\textless}=={\textgreater} x.name = value
    \vspace{1ex}

    \end{boxedminipage}

    \label{object:__str__}
    \index{object.\_\_str\_\_ \textit{(function)}}

    \vspace{0.5ex}

    \begin{boxedminipage}{\textwidth}

    \raggedright \textbf{\_\_str\_\_}(\textit{x})

    \vspace{-1.5ex}

    \rule{\textwidth}{0.5\fboxrule}

str(x)
    \vspace{1ex}

    \end{boxedminipage}


%%%%%%%%%%%%%%%%%%%%%%%%%%%%%%%%%%%%%%%%%%%%%%%%%%%%%%%%%%%%%%%%%%%%%%%%%%%
%%                              Properties                               %%
%%%%%%%%%%%%%%%%%%%%%%%%%%%%%%%%%%%%%%%%%%%%%%%%%%%%%%%%%%%%%%%%%%%%%%%%%%%

  \subsubsection{Properties}

\begin{longtable}{|p{.30\textwidth}|p{.62\textwidth}|l}
\cline{1-2}
\cline{1-2} \centering \textbf{Name} & \centering \textbf{Description}& \\
\cline{1-2}
\endhead\cline{1-2}\multicolumn{3}{r}{\small\textit{continued on next page}}\\\endfoot\cline{1-2}
\endlastfoot\raggedright \_\-\_\-c\-l\-a\-s\-s\-\_\-\_\- & \raggedright \textbf{Value:} 
{\tt {\textless}attribute '\_\_class\_\_' of 'object' objects{\textgreater}}&\\
\cline{1-2}
\end{longtable}


%%%%%%%%%%%%%%%%%%%%%%%%%%%%%%%%%%%%%%%%%%%%%%%%%%%%%%%%%%%%%%%%%%%%%%%%%%%
%%                          Instance Variables                           %%
%%%%%%%%%%%%%%%%%%%%%%%%%%%%%%%%%%%%%%%%%%%%%%%%%%%%%%%%%%%%%%%%%%%%%%%%%%%

  \subsubsection{Instance Variables}

\begin{longtable}{|p{.30\textwidth}|p{.62\textwidth}|l}
\cline{1-2}
\cline{1-2} \centering \textbf{Name} & \centering \textbf{Description}& \\
\cline{1-2}
\endhead\cline{1-2}\multicolumn{3}{r}{\small\textit{continued on next page}}\\\endfoot\cline{1-2}
\endlastfoot\raggedright d\- & An alias to the derivative of the function.&\\
\cline{1-2}
\end{longtable}

    \index{peach \textit{(package)}!peach.nn \textit{(package)}!peach.nn.af \textit{(module)}!peach.nn.af.Threshold \textit{(class)}|)}

%%%%%%%%%%%%%%%%%%%%%%%%%%%%%%%%%%%%%%%%%%%%%%%%%%%%%%%%%%%%%%%%%%%%%%%%%%%
%%                           Class Description                           %%
%%%%%%%%%%%%%%%%%%%%%%%%%%%%%%%%%%%%%%%%%%%%%%%%%%%%%%%%%%%%%%%%%%%%%%%%%%%

    \index{peach \textit{(package)}!peach.nn \textit{(package)}!peach.nn.af \textit{(module)}!peach.nn.af.Linear \textit{(class)}|(}
\subsection{Class Linear}

    \label{peach:nn:af:Linear}
\begin{tabular}{cccccccc}
% Line for object, linespec=[False, False]
\multicolumn{2}{r}{\settowidth{\BCL}{object}\multirow{2}{\BCL}{object}}
&&
&&
  \\\cline{3-3}
  &&\multicolumn{1}{c|}{}
&&
&&
  \\
% Line for peach.nn.af.Activation, linespec=[False]
\multicolumn{4}{r}{\settowidth{\BCL}{peach.nn.af.Activation}\multirow{2}{\BCL}{peach.nn.af.Activation}}
&&
  \\\cline{5-5}
  &&&&\multicolumn{1}{c|}{}
&&
  \\
&&&&\multicolumn{2}{l}{\textbf{peach.nn.af.Linear}}
\end{tabular}


Identity activation function

%%%%%%%%%%%%%%%%%%%%%%%%%%%%%%%%%%%%%%%%%%%%%%%%%%%%%%%%%%%%%%%%%%%%%%%%%%%
%%                                Methods                                %%
%%%%%%%%%%%%%%%%%%%%%%%%%%%%%%%%%%%%%%%%%%%%%%%%%%%%%%%%%%%%%%%%%%%%%%%%%%%

  \subsubsection{Methods}

    \vspace{0.5ex}

    \begin{boxedminipage}{\textwidth}

    \raggedright \textbf{\_\_init\_\_}(\textit{self})

    \vspace{-1.5ex}

    \rule{\textwidth}{0.5\fboxrule}

Initializes the function
    \vspace{1ex}

      Overrides: peach.nn.af.Activation.\_\_init\_\_

    \end{boxedminipage}

    \vspace{0.5ex}

    \begin{boxedminipage}{\textwidth}

    \raggedright \textbf{\_\_call\_\_}(\textit{self}, \textit{x})

    \vspace{-1.5ex}

    \rule{\textwidth}{0.5\fboxrule}

Call interface to the object.

This method applies the activation function over a vector of activation
potentials, and returns the results.
    \vspace{1ex}

      \textbf{Parameters}
      \begin{quote}
        \begin{Ventry}{x}

          \item[x]


A real number or a vector of real numbers representing the
activation potential of a neuron or a layer of neurons.
        \end{Ventry}

      \end{quote}

    \vspace{1ex}

      \textbf{Return Value}
      \begin{quote}

The activation function applied over the input vector.
      \end{quote}

    \vspace{1ex}

      Overrides: peach.nn.af.Activation.\_\_call\_\_

    \end{boxedminipage}

    \vspace{0.5ex}

    \begin{boxedminipage}{\textwidth}

    \raggedright \textbf{derivative}(\textit{self}, \textit{x})

    \vspace{-1.5ex}

    \rule{\textwidth}{0.5\fboxrule}

The function derivative.
    \vspace{1ex}

      \textbf{Parameters}
      \begin{quote}
        \begin{Ventry}{x}

          \item[x]


A real number or a vector of real numbers representing the
activation potential of a neuron or a layer of neurons.
        \end{Ventry}

      \end{quote}

    \vspace{1ex}

      \textbf{Return Value}
      \begin{quote}

The derivative of the activation function applied over the input
vector.
      \end{quote}

    \vspace{1ex}

      Overrides: peach.nn.af.Activation.derivative

    \end{boxedminipage}

    \label{object:__delattr__}
    \index{object.\_\_delattr\_\_ \textit{(function)}}

    \vspace{0.5ex}

    \begin{boxedminipage}{\textwidth}

    \raggedright \textbf{\_\_delattr\_\_}(\textit{...})

    \vspace{-1.5ex}

    \rule{\textwidth}{0.5\fboxrule}

x.{\_}{\_}delattr{\_}{\_}('name') {\textless}=={\textgreater} del x.name
    \vspace{1ex}

    \end{boxedminipage}

    \label{object:__getattribute__}
    \index{object.\_\_getattribute\_\_ \textit{(function)}}

    \vspace{0.5ex}

    \begin{boxedminipage}{\textwidth}

    \raggedright \textbf{\_\_getattribute\_\_}(\textit{...})

    \vspace{-1.5ex}

    \rule{\textwidth}{0.5\fboxrule}

x.{\_}{\_}getattribute{\_}{\_}('name') {\textless}=={\textgreater} x.name
    \vspace{1ex}

    \end{boxedminipage}

    \label{object:__hash__}
    \index{object.\_\_hash\_\_ \textit{(function)}}

    \vspace{0.5ex}

    \begin{boxedminipage}{\textwidth}

    \raggedright \textbf{\_\_hash\_\_}(\textit{x})

    \vspace{-1.5ex}

    \rule{\textwidth}{0.5\fboxrule}

hash(x)
    \vspace{1ex}

    \end{boxedminipage}

    \label{object:__new__}
    \index{object.\_\_new\_\_ \textit{(function)}}

    \vspace{0.5ex}

    \begin{boxedminipage}{\textwidth}

    \raggedright \textbf{\_\_new\_\_}(\textit{T}, \textit{S}, \textit{...})

      \textbf{Return Value}
      \begin{quote}
\begin{alltt}
a new object with type S, a subtype of T
\end{alltt}

      \end{quote}

    \vspace{1ex}

    \end{boxedminipage}

    \label{object:__reduce__}
    \index{object.\_\_reduce\_\_ \textit{(function)}}

    \vspace{0.5ex}

    \begin{boxedminipage}{\textwidth}

    \raggedright \textbf{\_\_reduce\_\_}(\textit{...})

    \vspace{-1.5ex}

    \rule{\textwidth}{0.5\fboxrule}

helper for pickle
    \vspace{1ex}

    \end{boxedminipage}

    \label{object:__reduce_ex__}
    \index{object.\_\_reduce\_ex\_\_ \textit{(function)}}

    \vspace{0.5ex}

    \begin{boxedminipage}{\textwidth}

    \raggedright \textbf{\_\_reduce\_ex\_\_}(\textit{...})

    \vspace{-1.5ex}

    \rule{\textwidth}{0.5\fboxrule}

helper for pickle
    \vspace{1ex}

    \end{boxedminipage}

    \label{object:__repr__}
    \index{object.\_\_repr\_\_ \textit{(function)}}

    \vspace{0.5ex}

    \begin{boxedminipage}{\textwidth}

    \raggedright \textbf{\_\_repr\_\_}(\textit{x})

    \vspace{-1.5ex}

    \rule{\textwidth}{0.5\fboxrule}

repr(x)
    \vspace{1ex}

    \end{boxedminipage}

    \label{object:__setattr__}
    \index{object.\_\_setattr\_\_ \textit{(function)}}

    \vspace{0.5ex}

    \begin{boxedminipage}{\textwidth}

    \raggedright \textbf{\_\_setattr\_\_}(\textit{...})

    \vspace{-1.5ex}

    \rule{\textwidth}{0.5\fboxrule}

x.{\_}{\_}setattr{\_}{\_}('name', value) {\textless}=={\textgreater} x.name = value
    \vspace{1ex}

    \end{boxedminipage}

    \label{object:__str__}
    \index{object.\_\_str\_\_ \textit{(function)}}

    \vspace{0.5ex}

    \begin{boxedminipage}{\textwidth}

    \raggedright \textbf{\_\_str\_\_}(\textit{x})

    \vspace{-1.5ex}

    \rule{\textwidth}{0.5\fboxrule}

str(x)
    \vspace{1ex}

    \end{boxedminipage}


%%%%%%%%%%%%%%%%%%%%%%%%%%%%%%%%%%%%%%%%%%%%%%%%%%%%%%%%%%%%%%%%%%%%%%%%%%%
%%                              Properties                               %%
%%%%%%%%%%%%%%%%%%%%%%%%%%%%%%%%%%%%%%%%%%%%%%%%%%%%%%%%%%%%%%%%%%%%%%%%%%%

  \subsubsection{Properties}

\begin{longtable}{|p{.30\textwidth}|p{.62\textwidth}|l}
\cline{1-2}
\cline{1-2} \centering \textbf{Name} & \centering \textbf{Description}& \\
\cline{1-2}
\endhead\cline{1-2}\multicolumn{3}{r}{\small\textit{continued on next page}}\\\endfoot\cline{1-2}
\endlastfoot\raggedright \_\-\_\-c\-l\-a\-s\-s\-\_\-\_\- & \raggedright \textbf{Value:} 
{\tt {\textless}attribute '\_\_class\_\_' of 'object' objects{\textgreater}}&\\
\cline{1-2}
\end{longtable}


%%%%%%%%%%%%%%%%%%%%%%%%%%%%%%%%%%%%%%%%%%%%%%%%%%%%%%%%%%%%%%%%%%%%%%%%%%%
%%                          Instance Variables                           %%
%%%%%%%%%%%%%%%%%%%%%%%%%%%%%%%%%%%%%%%%%%%%%%%%%%%%%%%%%%%%%%%%%%%%%%%%%%%

  \subsubsection{Instance Variables}

\begin{longtable}{|p{.30\textwidth}|p{.62\textwidth}|l}
\cline{1-2}
\cline{1-2} \centering \textbf{Name} & \centering \textbf{Description}& \\
\cline{1-2}
\endhead\cline{1-2}\multicolumn{3}{r}{\small\textit{continued on next page}}\\\endfoot\cline{1-2}
\endlastfoot\raggedright d\- & An alias to the derivative of the function.&\\
\cline{1-2}
\end{longtable}

    \index{peach \textit{(package)}!peach.nn \textit{(package)}!peach.nn.af \textit{(module)}!peach.nn.af.Linear \textit{(class)}|)}

%%%%%%%%%%%%%%%%%%%%%%%%%%%%%%%%%%%%%%%%%%%%%%%%%%%%%%%%%%%%%%%%%%%%%%%%%%%
%%                           Class Description                           %%
%%%%%%%%%%%%%%%%%%%%%%%%%%%%%%%%%%%%%%%%%%%%%%%%%%%%%%%%%%%%%%%%%%%%%%%%%%%

    \index{peach \textit{(package)}!peach.nn \textit{(package)}!peach.nn.af \textit{(module)}!peach.nn.af.Linear \textit{(class)}|(}
\subsection{Class Linear}

    \label{peach:nn:af:Linear}
\begin{tabular}{cccccccc}
% Line for object, linespec=[False, False]
\multicolumn{2}{r}{\settowidth{\BCL}{object}\multirow{2}{\BCL}{object}}
&&
&&
  \\\cline{3-3}
  &&\multicolumn{1}{c|}{}
&&
&&
  \\
% Line for peach.nn.af.Activation, linespec=[False]
\multicolumn{4}{r}{\settowidth{\BCL}{peach.nn.af.Activation}\multirow{2}{\BCL}{peach.nn.af.Activation}}
&&
  \\\cline{5-5}
  &&&&\multicolumn{1}{c|}{}
&&
  \\
&&&&\multicolumn{2}{l}{\textbf{peach.nn.af.Linear}}
\end{tabular}


Identity activation function

%%%%%%%%%%%%%%%%%%%%%%%%%%%%%%%%%%%%%%%%%%%%%%%%%%%%%%%%%%%%%%%%%%%%%%%%%%%
%%                                Methods                                %%
%%%%%%%%%%%%%%%%%%%%%%%%%%%%%%%%%%%%%%%%%%%%%%%%%%%%%%%%%%%%%%%%%%%%%%%%%%%

  \subsubsection{Methods}

    \vspace{0.5ex}

    \begin{boxedminipage}{\textwidth}

    \raggedright \textbf{\_\_init\_\_}(\textit{self})

    \vspace{-1.5ex}

    \rule{\textwidth}{0.5\fboxrule}

Initializes the function
    \vspace{1ex}

      Overrides: peach.nn.af.Activation.\_\_init\_\_

    \end{boxedminipage}

    \vspace{0.5ex}

    \begin{boxedminipage}{\textwidth}

    \raggedright \textbf{\_\_call\_\_}(\textit{self}, \textit{x})

    \vspace{-1.5ex}

    \rule{\textwidth}{0.5\fboxrule}

Call interface to the object.

This method applies the activation function over a vector of activation
potentials, and returns the results.
    \vspace{1ex}

      \textbf{Parameters}
      \begin{quote}
        \begin{Ventry}{x}

          \item[x]


A real number or a vector of real numbers representing the
activation potential of a neuron or a layer of neurons.
        \end{Ventry}

      \end{quote}

    \vspace{1ex}

      \textbf{Return Value}
      \begin{quote}

The activation function applied over the input vector.
      \end{quote}

    \vspace{1ex}

      Overrides: peach.nn.af.Activation.\_\_call\_\_

    \end{boxedminipage}

    \vspace{0.5ex}

    \begin{boxedminipage}{\textwidth}

    \raggedright \textbf{derivative}(\textit{self}, \textit{x})

    \vspace{-1.5ex}

    \rule{\textwidth}{0.5\fboxrule}

The function derivative.
    \vspace{1ex}

      \textbf{Parameters}
      \begin{quote}
        \begin{Ventry}{x}

          \item[x]


A real number or a vector of real numbers representing the
activation potential of a neuron or a layer of neurons.
        \end{Ventry}

      \end{quote}

    \vspace{1ex}

      \textbf{Return Value}
      \begin{quote}

The derivative of the activation function applied over the input
vector.
      \end{quote}

    \vspace{1ex}

      Overrides: peach.nn.af.Activation.derivative

    \end{boxedminipage}

    \label{object:__delattr__}
    \index{object.\_\_delattr\_\_ \textit{(function)}}

    \vspace{0.5ex}

    \begin{boxedminipage}{\textwidth}

    \raggedright \textbf{\_\_delattr\_\_}(\textit{...})

    \vspace{-1.5ex}

    \rule{\textwidth}{0.5\fboxrule}

x.{\_}{\_}delattr{\_}{\_}('name') {\textless}=={\textgreater} del x.name
    \vspace{1ex}

    \end{boxedminipage}

    \label{object:__getattribute__}
    \index{object.\_\_getattribute\_\_ \textit{(function)}}

    \vspace{0.5ex}

    \begin{boxedminipage}{\textwidth}

    \raggedright \textbf{\_\_getattribute\_\_}(\textit{...})

    \vspace{-1.5ex}

    \rule{\textwidth}{0.5\fboxrule}

x.{\_}{\_}getattribute{\_}{\_}('name') {\textless}=={\textgreater} x.name
    \vspace{1ex}

    \end{boxedminipage}

    \label{object:__hash__}
    \index{object.\_\_hash\_\_ \textit{(function)}}

    \vspace{0.5ex}

    \begin{boxedminipage}{\textwidth}

    \raggedright \textbf{\_\_hash\_\_}(\textit{x})

    \vspace{-1.5ex}

    \rule{\textwidth}{0.5\fboxrule}

hash(x)
    \vspace{1ex}

    \end{boxedminipage}

    \label{object:__new__}
    \index{object.\_\_new\_\_ \textit{(function)}}

    \vspace{0.5ex}

    \begin{boxedminipage}{\textwidth}

    \raggedright \textbf{\_\_new\_\_}(\textit{T}, \textit{S}, \textit{...})

      \textbf{Return Value}
      \begin{quote}
\begin{alltt}
a new object with type S, a subtype of T
\end{alltt}

      \end{quote}

    \vspace{1ex}

    \end{boxedminipage}

    \label{object:__reduce__}
    \index{object.\_\_reduce\_\_ \textit{(function)}}

    \vspace{0.5ex}

    \begin{boxedminipage}{\textwidth}

    \raggedright \textbf{\_\_reduce\_\_}(\textit{...})

    \vspace{-1.5ex}

    \rule{\textwidth}{0.5\fboxrule}

helper for pickle
    \vspace{1ex}

    \end{boxedminipage}

    \label{object:__reduce_ex__}
    \index{object.\_\_reduce\_ex\_\_ \textit{(function)}}

    \vspace{0.5ex}

    \begin{boxedminipage}{\textwidth}

    \raggedright \textbf{\_\_reduce\_ex\_\_}(\textit{...})

    \vspace{-1.5ex}

    \rule{\textwidth}{0.5\fboxrule}

helper for pickle
    \vspace{1ex}

    \end{boxedminipage}

    \label{object:__repr__}
    \index{object.\_\_repr\_\_ \textit{(function)}}

    \vspace{0.5ex}

    \begin{boxedminipage}{\textwidth}

    \raggedright \textbf{\_\_repr\_\_}(\textit{x})

    \vspace{-1.5ex}

    \rule{\textwidth}{0.5\fboxrule}

repr(x)
    \vspace{1ex}

    \end{boxedminipage}

    \label{object:__setattr__}
    \index{object.\_\_setattr\_\_ \textit{(function)}}

    \vspace{0.5ex}

    \begin{boxedminipage}{\textwidth}

    \raggedright \textbf{\_\_setattr\_\_}(\textit{...})

    \vspace{-1.5ex}

    \rule{\textwidth}{0.5\fboxrule}

x.{\_}{\_}setattr{\_}{\_}('name', value) {\textless}=={\textgreater} x.name = value
    \vspace{1ex}

    \end{boxedminipage}

    \label{object:__str__}
    \index{object.\_\_str\_\_ \textit{(function)}}

    \vspace{0.5ex}

    \begin{boxedminipage}{\textwidth}

    \raggedright \textbf{\_\_str\_\_}(\textit{x})

    \vspace{-1.5ex}

    \rule{\textwidth}{0.5\fboxrule}

str(x)
    \vspace{1ex}

    \end{boxedminipage}


%%%%%%%%%%%%%%%%%%%%%%%%%%%%%%%%%%%%%%%%%%%%%%%%%%%%%%%%%%%%%%%%%%%%%%%%%%%
%%                              Properties                               %%
%%%%%%%%%%%%%%%%%%%%%%%%%%%%%%%%%%%%%%%%%%%%%%%%%%%%%%%%%%%%%%%%%%%%%%%%%%%

  \subsubsection{Properties}

\begin{longtable}{|p{.30\textwidth}|p{.62\textwidth}|l}
\cline{1-2}
\cline{1-2} \centering \textbf{Name} & \centering \textbf{Description}& \\
\cline{1-2}
\endhead\cline{1-2}\multicolumn{3}{r}{\small\textit{continued on next page}}\\\endfoot\cline{1-2}
\endlastfoot\raggedright \_\-\_\-c\-l\-a\-s\-s\-\_\-\_\- & \raggedright \textbf{Value:} 
{\tt {\textless}attribute '\_\_class\_\_' of 'object' objects{\textgreater}}&\\
\cline{1-2}
\end{longtable}


%%%%%%%%%%%%%%%%%%%%%%%%%%%%%%%%%%%%%%%%%%%%%%%%%%%%%%%%%%%%%%%%%%%%%%%%%%%
%%                          Instance Variables                           %%
%%%%%%%%%%%%%%%%%%%%%%%%%%%%%%%%%%%%%%%%%%%%%%%%%%%%%%%%%%%%%%%%%%%%%%%%%%%

  \subsubsection{Instance Variables}

\begin{longtable}{|p{.30\textwidth}|p{.62\textwidth}|l}
\cline{1-2}
\cline{1-2} \centering \textbf{Name} & \centering \textbf{Description}& \\
\cline{1-2}
\endhead\cline{1-2}\multicolumn{3}{r}{\small\textit{continued on next page}}\\\endfoot\cline{1-2}
\endlastfoot\raggedright d\- & An alias to the derivative of the function.&\\
\cline{1-2}
\end{longtable}

    \index{peach \textit{(package)}!peach.nn \textit{(package)}!peach.nn.af \textit{(module)}!peach.nn.af.Linear \textit{(class)}|)}

%%%%%%%%%%%%%%%%%%%%%%%%%%%%%%%%%%%%%%%%%%%%%%%%%%%%%%%%%%%%%%%%%%%%%%%%%%%
%%                           Class Description                           %%
%%%%%%%%%%%%%%%%%%%%%%%%%%%%%%%%%%%%%%%%%%%%%%%%%%%%%%%%%%%%%%%%%%%%%%%%%%%

    \index{peach \textit{(package)}!peach.nn \textit{(package)}!peach.nn.af \textit{(module)}!peach.nn.af.Ramp \textit{(class)}|(}
\subsection{Class Ramp}

    \label{peach:nn:af:Ramp}
\begin{tabular}{cccccccc}
% Line for object, linespec=[False, False]
\multicolumn{2}{r}{\settowidth{\BCL}{object}\multirow{2}{\BCL}{object}}
&&
&&
  \\\cline{3-3}
  &&\multicolumn{1}{c|}{}
&&
&&
  \\
% Line for peach.nn.af.Activation, linespec=[False]
\multicolumn{4}{r}{\settowidth{\BCL}{peach.nn.af.Activation}\multirow{2}{\BCL}{peach.nn.af.Activation}}
&&
  \\\cline{5-5}
  &&&&\multicolumn{1}{c|}{}
&&
  \\
&&&&\multicolumn{2}{l}{\textbf{peach.nn.af.Ramp}}
\end{tabular}


Ramp activation function

%%%%%%%%%%%%%%%%%%%%%%%%%%%%%%%%%%%%%%%%%%%%%%%%%%%%%%%%%%%%%%%%%%%%%%%%%%%
%%                                Methods                                %%
%%%%%%%%%%%%%%%%%%%%%%%%%%%%%%%%%%%%%%%%%%%%%%%%%%%%%%%%%%%%%%%%%%%%%%%%%%%

  \subsubsection{Methods}

    \vspace{0.5ex}

    \begin{boxedminipage}{\textwidth}

    \raggedright \textbf{\_\_init\_\_}(\textit{self}, \textit{p0}=\texttt{\texttt{(}-0.5\texttt{, }0.0\texttt{)}}, \textit{p1}=\texttt{\texttt{(}0.5\texttt{, }1.0\texttt{)}})

    \vspace{-1.5ex}

    \rule{\textwidth}{0.5\fboxrule}

Initializes the object.

Two points are needed to set this function. They are used to determine
where the ramp begins and where it ends.
    \vspace{1ex}

      \textbf{Parameters}
      \begin{quote}
        \begin{Ventry}{xx}

          \item[p0]


The starting point, given as a tuple \texttt{(x0, y0)}. For values of the
input below \texttt{x0}, the function returns \texttt{y0}. Defaults to
\texttt{(-0.5, 0.0)}.
          \item[p1]


The ending point, given as a tuple \texttt{(x1, y1)}. For values of the
input above \texttt{x1}, the function returns \texttt{y1}. Defaults to
\texttt{(0.5, 1.0)}.
        \end{Ventry}

      \end{quote}

    \vspace{1ex}

      Overrides: peach.nn.af.Activation.\_\_init\_\_

    \end{boxedminipage}

    \vspace{0.5ex}

    \begin{boxedminipage}{\textwidth}

    \raggedright \textbf{\_\_call\_\_}(\textit{self}, \textit{x})

    \vspace{-1.5ex}

    \rule{\textwidth}{0.5\fboxrule}

Call interface to the object.

This method applies the activation function over a vector of activation
potentials, and returns the results.
    \vspace{1ex}

      \textbf{Parameters}
      \begin{quote}
        \begin{Ventry}{x}

          \item[x]


A real number or a vector of real numbers representing the
activation potential of a neuron or a layer of neurons.
        \end{Ventry}

      \end{quote}

    \vspace{1ex}

      \textbf{Return Value}
      \begin{quote}

The activation function applied over the input vector.
      \end{quote}

    \vspace{1ex}

      Overrides: peach.nn.af.Activation.\_\_call\_\_

    \end{boxedminipage}

    \vspace{0.5ex}

    \begin{boxedminipage}{\textwidth}

    \raggedright \textbf{derivative}(\textit{self}, \textit{x})

    \vspace{-1.5ex}

    \rule{\textwidth}{0.5\fboxrule}

The function derivative.
    \vspace{1ex}

      \textbf{Parameters}
      \begin{quote}
        \begin{Ventry}{x}

          \item[x]


A real number or a vector of real numbers representing the
activation potential of a neuron or a layer of neurons.
        \end{Ventry}

      \end{quote}

    \vspace{1ex}

      \textbf{Return Value}
      \begin{quote}

The derivative of the activation function applied over the input
vector.
      \end{quote}

    \vspace{1ex}

      Overrides: peach.nn.af.Activation.derivative

    \end{boxedminipage}

    \label{object:__delattr__}
    \index{object.\_\_delattr\_\_ \textit{(function)}}

    \vspace{0.5ex}

    \begin{boxedminipage}{\textwidth}

    \raggedright \textbf{\_\_delattr\_\_}(\textit{...})

    \vspace{-1.5ex}

    \rule{\textwidth}{0.5\fboxrule}

x.{\_}{\_}delattr{\_}{\_}('name') {\textless}=={\textgreater} del x.name
    \vspace{1ex}

    \end{boxedminipage}

    \label{object:__getattribute__}
    \index{object.\_\_getattribute\_\_ \textit{(function)}}

    \vspace{0.5ex}

    \begin{boxedminipage}{\textwidth}

    \raggedright \textbf{\_\_getattribute\_\_}(\textit{...})

    \vspace{-1.5ex}

    \rule{\textwidth}{0.5\fboxrule}

x.{\_}{\_}getattribute{\_}{\_}('name') {\textless}=={\textgreater} x.name
    \vspace{1ex}

    \end{boxedminipage}

    \label{object:__hash__}
    \index{object.\_\_hash\_\_ \textit{(function)}}

    \vspace{0.5ex}

    \begin{boxedminipage}{\textwidth}

    \raggedright \textbf{\_\_hash\_\_}(\textit{x})

    \vspace{-1.5ex}

    \rule{\textwidth}{0.5\fboxrule}

hash(x)
    \vspace{1ex}

    \end{boxedminipage}

    \label{object:__new__}
    \index{object.\_\_new\_\_ \textit{(function)}}

    \vspace{0.5ex}

    \begin{boxedminipage}{\textwidth}

    \raggedright \textbf{\_\_new\_\_}(\textit{T}, \textit{S}, \textit{...})

      \textbf{Return Value}
      \begin{quote}
\begin{alltt}
a new object with type S, a subtype of T
\end{alltt}

      \end{quote}

    \vspace{1ex}

    \end{boxedminipage}

    \label{object:__reduce__}
    \index{object.\_\_reduce\_\_ \textit{(function)}}

    \vspace{0.5ex}

    \begin{boxedminipage}{\textwidth}

    \raggedright \textbf{\_\_reduce\_\_}(\textit{...})

    \vspace{-1.5ex}

    \rule{\textwidth}{0.5\fboxrule}

helper for pickle
    \vspace{1ex}

    \end{boxedminipage}

    \label{object:__reduce_ex__}
    \index{object.\_\_reduce\_ex\_\_ \textit{(function)}}

    \vspace{0.5ex}

    \begin{boxedminipage}{\textwidth}

    \raggedright \textbf{\_\_reduce\_ex\_\_}(\textit{...})

    \vspace{-1.5ex}

    \rule{\textwidth}{0.5\fboxrule}

helper for pickle
    \vspace{1ex}

    \end{boxedminipage}

    \label{object:__repr__}
    \index{object.\_\_repr\_\_ \textit{(function)}}

    \vspace{0.5ex}

    \begin{boxedminipage}{\textwidth}

    \raggedright \textbf{\_\_repr\_\_}(\textit{x})

    \vspace{-1.5ex}

    \rule{\textwidth}{0.5\fboxrule}

repr(x)
    \vspace{1ex}

    \end{boxedminipage}

    \label{object:__setattr__}
    \index{object.\_\_setattr\_\_ \textit{(function)}}

    \vspace{0.5ex}

    \begin{boxedminipage}{\textwidth}

    \raggedright \textbf{\_\_setattr\_\_}(\textit{...})

    \vspace{-1.5ex}

    \rule{\textwidth}{0.5\fboxrule}

x.{\_}{\_}setattr{\_}{\_}('name', value) {\textless}=={\textgreater} x.name = value
    \vspace{1ex}

    \end{boxedminipage}

    \label{object:__str__}
    \index{object.\_\_str\_\_ \textit{(function)}}

    \vspace{0.5ex}

    \begin{boxedminipage}{\textwidth}

    \raggedright \textbf{\_\_str\_\_}(\textit{x})

    \vspace{-1.5ex}

    \rule{\textwidth}{0.5\fboxrule}

str(x)
    \vspace{1ex}

    \end{boxedminipage}


%%%%%%%%%%%%%%%%%%%%%%%%%%%%%%%%%%%%%%%%%%%%%%%%%%%%%%%%%%%%%%%%%%%%%%%%%%%
%%                              Properties                               %%
%%%%%%%%%%%%%%%%%%%%%%%%%%%%%%%%%%%%%%%%%%%%%%%%%%%%%%%%%%%%%%%%%%%%%%%%%%%

  \subsubsection{Properties}

\begin{longtable}{|p{.30\textwidth}|p{.62\textwidth}|l}
\cline{1-2}
\cline{1-2} \centering \textbf{Name} & \centering \textbf{Description}& \\
\cline{1-2}
\endhead\cline{1-2}\multicolumn{3}{r}{\small\textit{continued on next page}}\\\endfoot\cline{1-2}
\endlastfoot\raggedright \_\-\_\-c\-l\-a\-s\-s\-\_\-\_\- & \raggedright \textbf{Value:} 
{\tt {\textless}attribute '\_\_class\_\_' of 'object' objects{\textgreater}}&\\
\cline{1-2}
\end{longtable}


%%%%%%%%%%%%%%%%%%%%%%%%%%%%%%%%%%%%%%%%%%%%%%%%%%%%%%%%%%%%%%%%%%%%%%%%%%%
%%                          Instance Variables                           %%
%%%%%%%%%%%%%%%%%%%%%%%%%%%%%%%%%%%%%%%%%%%%%%%%%%%%%%%%%%%%%%%%%%%%%%%%%%%

  \subsubsection{Instance Variables}

\begin{longtable}{|p{.30\textwidth}|p{.62\textwidth}|l}
\cline{1-2}
\cline{1-2} \centering \textbf{Name} & \centering \textbf{Description}& \\
\cline{1-2}
\endhead\cline{1-2}\multicolumn{3}{r}{\small\textit{continued on next page}}\\\endfoot\cline{1-2}
\endlastfoot\raggedright d\- & An alias to the derivative of the function.&\\
\cline{1-2}
\end{longtable}

    \index{peach \textit{(package)}!peach.nn \textit{(package)}!peach.nn.af \textit{(module)}!peach.nn.af.Ramp \textit{(class)}|)}

%%%%%%%%%%%%%%%%%%%%%%%%%%%%%%%%%%%%%%%%%%%%%%%%%%%%%%%%%%%%%%%%%%%%%%%%%%%
%%                           Class Description                           %%
%%%%%%%%%%%%%%%%%%%%%%%%%%%%%%%%%%%%%%%%%%%%%%%%%%%%%%%%%%%%%%%%%%%%%%%%%%%

    \index{peach \textit{(package)}!peach.nn \textit{(package)}!peach.nn.af \textit{(module)}!peach.nn.af.Sigmoid \textit{(class)}|(}
\subsection{Class Sigmoid}

    \label{peach:nn:af:Sigmoid}
\begin{tabular}{cccccccc}
% Line for object, linespec=[False, False]
\multicolumn{2}{r}{\settowidth{\BCL}{object}\multirow{2}{\BCL}{object}}
&&
&&
  \\\cline{3-3}
  &&\multicolumn{1}{c|}{}
&&
&&
  \\
% Line for peach.nn.af.Activation, linespec=[False]
\multicolumn{4}{r}{\settowidth{\BCL}{peach.nn.af.Activation}\multirow{2}{\BCL}{peach.nn.af.Activation}}
&&
  \\\cline{5-5}
  &&&&\multicolumn{1}{c|}{}
&&
  \\
&&&&\multicolumn{2}{l}{\textbf{peach.nn.af.Sigmoid}}
\end{tabular}


Sigmoid activation function

%%%%%%%%%%%%%%%%%%%%%%%%%%%%%%%%%%%%%%%%%%%%%%%%%%%%%%%%%%%%%%%%%%%%%%%%%%%
%%                                Methods                                %%
%%%%%%%%%%%%%%%%%%%%%%%%%%%%%%%%%%%%%%%%%%%%%%%%%%%%%%%%%%%%%%%%%%%%%%%%%%%

  \subsubsection{Methods}

    \vspace{0.5ex}

    \begin{boxedminipage}{\textwidth}

    \raggedright \textbf{\_\_init\_\_}(\textit{self}, \textit{a}=\texttt{1.0}, \textit{x0}=\texttt{0.0})

    \vspace{-1.5ex}

    \rule{\textwidth}{0.5\fboxrule}

Initializes the object.
    \vspace{1ex}

      \textbf{Parameters}
      \begin{quote}
        \begin{Ventry}{xx}

          \item[a]


The slope of the function in the center \texttt{x0}. Defaults to 1.0.
          \item[x0]


The center of the sigmoid. Defaults to 0.0.
        \end{Ventry}

      \end{quote}

    \vspace{1ex}

      Overrides: peach.nn.af.Activation.\_\_init\_\_

    \end{boxedminipage}

    \vspace{0.5ex}

    \begin{boxedminipage}{\textwidth}

    \raggedright \textbf{\_\_call\_\_}(\textit{self}, \textit{x})

    \vspace{-1.5ex}

    \rule{\textwidth}{0.5\fboxrule}

Call interface to the object.

This method applies the activation function over a vector of activation
potentials, and returns the results.
    \vspace{1ex}

      \textbf{Parameters}
      \begin{quote}
        \begin{Ventry}{x}

          \item[x]


A real number or a vector of real numbers representing the
activation potential of a neuron or a layer of neurons.
        \end{Ventry}

      \end{quote}

    \vspace{1ex}

      \textbf{Return Value}
      \begin{quote}

The activation function applied over the input vector.
      \end{quote}

    \vspace{1ex}

      Overrides: peach.nn.af.Activation.\_\_call\_\_

    \end{boxedminipage}

    \vspace{0.5ex}

    \begin{boxedminipage}{\textwidth}

    \raggedright \textbf{derivative}(\textit{self}, \textit{x})

    \vspace{-1.5ex}

    \rule{\textwidth}{0.5\fboxrule}

The function derivative.
    \vspace{1ex}

      \textbf{Parameters}
      \begin{quote}
        \begin{Ventry}{x}

          \item[x]


A real number or a vector of real numbers representing the
activation potential of a neuron or a layer of neurons.
        \end{Ventry}

      \end{quote}

    \vspace{1ex}

      \textbf{Return Value}
      \begin{quote}

The derivative of the activation function applied over the input
vector.
      \end{quote}

    \vspace{1ex}

      Overrides: peach.nn.af.Activation.derivative

    \end{boxedminipage}

    \label{object:__delattr__}
    \index{object.\_\_delattr\_\_ \textit{(function)}}

    \vspace{0.5ex}

    \begin{boxedminipage}{\textwidth}

    \raggedright \textbf{\_\_delattr\_\_}(\textit{...})

    \vspace{-1.5ex}

    \rule{\textwidth}{0.5\fboxrule}

x.{\_}{\_}delattr{\_}{\_}('name') {\textless}=={\textgreater} del x.name
    \vspace{1ex}

    \end{boxedminipage}

    \label{object:__getattribute__}
    \index{object.\_\_getattribute\_\_ \textit{(function)}}

    \vspace{0.5ex}

    \begin{boxedminipage}{\textwidth}

    \raggedright \textbf{\_\_getattribute\_\_}(\textit{...})

    \vspace{-1.5ex}

    \rule{\textwidth}{0.5\fboxrule}

x.{\_}{\_}getattribute{\_}{\_}('name') {\textless}=={\textgreater} x.name
    \vspace{1ex}

    \end{boxedminipage}

    \label{object:__hash__}
    \index{object.\_\_hash\_\_ \textit{(function)}}

    \vspace{0.5ex}

    \begin{boxedminipage}{\textwidth}

    \raggedright \textbf{\_\_hash\_\_}(\textit{x})

    \vspace{-1.5ex}

    \rule{\textwidth}{0.5\fboxrule}

hash(x)
    \vspace{1ex}

    \end{boxedminipage}

    \label{object:__new__}
    \index{object.\_\_new\_\_ \textit{(function)}}

    \vspace{0.5ex}

    \begin{boxedminipage}{\textwidth}

    \raggedright \textbf{\_\_new\_\_}(\textit{T}, \textit{S}, \textit{...})

      \textbf{Return Value}
      \begin{quote}
\begin{alltt}
a new object with type S, a subtype of T
\end{alltt}

      \end{quote}

    \vspace{1ex}

    \end{boxedminipage}

    \label{object:__reduce__}
    \index{object.\_\_reduce\_\_ \textit{(function)}}

    \vspace{0.5ex}

    \begin{boxedminipage}{\textwidth}

    \raggedright \textbf{\_\_reduce\_\_}(\textit{...})

    \vspace{-1.5ex}

    \rule{\textwidth}{0.5\fboxrule}

helper for pickle
    \vspace{1ex}

    \end{boxedminipage}

    \label{object:__reduce_ex__}
    \index{object.\_\_reduce\_ex\_\_ \textit{(function)}}

    \vspace{0.5ex}

    \begin{boxedminipage}{\textwidth}

    \raggedright \textbf{\_\_reduce\_ex\_\_}(\textit{...})

    \vspace{-1.5ex}

    \rule{\textwidth}{0.5\fboxrule}

helper for pickle
    \vspace{1ex}

    \end{boxedminipage}

    \label{object:__repr__}
    \index{object.\_\_repr\_\_ \textit{(function)}}

    \vspace{0.5ex}

    \begin{boxedminipage}{\textwidth}

    \raggedright \textbf{\_\_repr\_\_}(\textit{x})

    \vspace{-1.5ex}

    \rule{\textwidth}{0.5\fboxrule}

repr(x)
    \vspace{1ex}

    \end{boxedminipage}

    \label{object:__setattr__}
    \index{object.\_\_setattr\_\_ \textit{(function)}}

    \vspace{0.5ex}

    \begin{boxedminipage}{\textwidth}

    \raggedright \textbf{\_\_setattr\_\_}(\textit{...})

    \vspace{-1.5ex}

    \rule{\textwidth}{0.5\fboxrule}

x.{\_}{\_}setattr{\_}{\_}('name', value) {\textless}=={\textgreater} x.name = value
    \vspace{1ex}

    \end{boxedminipage}

    \label{object:__str__}
    \index{object.\_\_str\_\_ \textit{(function)}}

    \vspace{0.5ex}

    \begin{boxedminipage}{\textwidth}

    \raggedright \textbf{\_\_str\_\_}(\textit{x})

    \vspace{-1.5ex}

    \rule{\textwidth}{0.5\fboxrule}

str(x)
    \vspace{1ex}

    \end{boxedminipage}


%%%%%%%%%%%%%%%%%%%%%%%%%%%%%%%%%%%%%%%%%%%%%%%%%%%%%%%%%%%%%%%%%%%%%%%%%%%
%%                              Properties                               %%
%%%%%%%%%%%%%%%%%%%%%%%%%%%%%%%%%%%%%%%%%%%%%%%%%%%%%%%%%%%%%%%%%%%%%%%%%%%

  \subsubsection{Properties}

\begin{longtable}{|p{.30\textwidth}|p{.62\textwidth}|l}
\cline{1-2}
\cline{1-2} \centering \textbf{Name} & \centering \textbf{Description}& \\
\cline{1-2}
\endhead\cline{1-2}\multicolumn{3}{r}{\small\textit{continued on next page}}\\\endfoot\cline{1-2}
\endlastfoot\raggedright \_\-\_\-c\-l\-a\-s\-s\-\_\-\_\- & \raggedright \textbf{Value:} 
{\tt {\textless}attribute '\_\_class\_\_' of 'object' objects{\textgreater}}&\\
\cline{1-2}
\end{longtable}


%%%%%%%%%%%%%%%%%%%%%%%%%%%%%%%%%%%%%%%%%%%%%%%%%%%%%%%%%%%%%%%%%%%%%%%%%%%
%%                          Instance Variables                           %%
%%%%%%%%%%%%%%%%%%%%%%%%%%%%%%%%%%%%%%%%%%%%%%%%%%%%%%%%%%%%%%%%%%%%%%%%%%%

  \subsubsection{Instance Variables}

\begin{longtable}{|p{.30\textwidth}|p{.62\textwidth}|l}
\cline{1-2}
\cline{1-2} \centering \textbf{Name} & \centering \textbf{Description}& \\
\cline{1-2}
\endhead\cline{1-2}\multicolumn{3}{r}{\small\textit{continued on next page}}\\\endfoot\cline{1-2}
\endlastfoot\raggedright d\- & An alias to the derivative of the function.&\\
\cline{1-2}
\end{longtable}

    \index{peach \textit{(package)}!peach.nn \textit{(package)}!peach.nn.af \textit{(module)}!peach.nn.af.Sigmoid \textit{(class)}|)}

%%%%%%%%%%%%%%%%%%%%%%%%%%%%%%%%%%%%%%%%%%%%%%%%%%%%%%%%%%%%%%%%%%%%%%%%%%%
%%                           Class Description                           %%
%%%%%%%%%%%%%%%%%%%%%%%%%%%%%%%%%%%%%%%%%%%%%%%%%%%%%%%%%%%%%%%%%%%%%%%%%%%

    \index{peach \textit{(package)}!peach.nn \textit{(package)}!peach.nn.af \textit{(module)}!peach.nn.af.Sigmoid \textit{(class)}|(}
\subsection{Class Sigmoid}

    \label{peach:nn:af:Sigmoid}
\begin{tabular}{cccccccc}
% Line for object, linespec=[False, False]
\multicolumn{2}{r}{\settowidth{\BCL}{object}\multirow{2}{\BCL}{object}}
&&
&&
  \\\cline{3-3}
  &&\multicolumn{1}{c|}{}
&&
&&
  \\
% Line for peach.nn.af.Activation, linespec=[False]
\multicolumn{4}{r}{\settowidth{\BCL}{peach.nn.af.Activation}\multirow{2}{\BCL}{peach.nn.af.Activation}}
&&
  \\\cline{5-5}
  &&&&\multicolumn{1}{c|}{}
&&
  \\
&&&&\multicolumn{2}{l}{\textbf{peach.nn.af.Sigmoid}}
\end{tabular}


Sigmoid activation function

%%%%%%%%%%%%%%%%%%%%%%%%%%%%%%%%%%%%%%%%%%%%%%%%%%%%%%%%%%%%%%%%%%%%%%%%%%%
%%                                Methods                                %%
%%%%%%%%%%%%%%%%%%%%%%%%%%%%%%%%%%%%%%%%%%%%%%%%%%%%%%%%%%%%%%%%%%%%%%%%%%%

  \subsubsection{Methods}

    \vspace{0.5ex}

    \begin{boxedminipage}{\textwidth}

    \raggedright \textbf{\_\_init\_\_}(\textit{self}, \textit{a}=\texttt{1.0}, \textit{x0}=\texttt{0.0})

    \vspace{-1.5ex}

    \rule{\textwidth}{0.5\fboxrule}

Initializes the object.
    \vspace{1ex}

      \textbf{Parameters}
      \begin{quote}
        \begin{Ventry}{xx}

          \item[a]


The slope of the function in the center \texttt{x0}. Defaults to 1.0.
          \item[x0]


The center of the sigmoid. Defaults to 0.0.
        \end{Ventry}

      \end{quote}

    \vspace{1ex}

      Overrides: peach.nn.af.Activation.\_\_init\_\_

    \end{boxedminipage}

    \vspace{0.5ex}

    \begin{boxedminipage}{\textwidth}

    \raggedright \textbf{\_\_call\_\_}(\textit{self}, \textit{x})

    \vspace{-1.5ex}

    \rule{\textwidth}{0.5\fboxrule}

Call interface to the object.

This method applies the activation function over a vector of activation
potentials, and returns the results.
    \vspace{1ex}

      \textbf{Parameters}
      \begin{quote}
        \begin{Ventry}{x}

          \item[x]


A real number or a vector of real numbers representing the
activation potential of a neuron or a layer of neurons.
        \end{Ventry}

      \end{quote}

    \vspace{1ex}

      \textbf{Return Value}
      \begin{quote}

The activation function applied over the input vector.
      \end{quote}

    \vspace{1ex}

      Overrides: peach.nn.af.Activation.\_\_call\_\_

    \end{boxedminipage}

    \vspace{0.5ex}

    \begin{boxedminipage}{\textwidth}

    \raggedright \textbf{derivative}(\textit{self}, \textit{x})

    \vspace{-1.5ex}

    \rule{\textwidth}{0.5\fboxrule}

The function derivative.
    \vspace{1ex}

      \textbf{Parameters}
      \begin{quote}
        \begin{Ventry}{x}

          \item[x]


A real number or a vector of real numbers representing the
activation potential of a neuron or a layer of neurons.
        \end{Ventry}

      \end{quote}

    \vspace{1ex}

      \textbf{Return Value}
      \begin{quote}

The derivative of the activation function applied over the input
vector.
      \end{quote}

    \vspace{1ex}

      Overrides: peach.nn.af.Activation.derivative

    \end{boxedminipage}

    \label{object:__delattr__}
    \index{object.\_\_delattr\_\_ \textit{(function)}}

    \vspace{0.5ex}

    \begin{boxedminipage}{\textwidth}

    \raggedright \textbf{\_\_delattr\_\_}(\textit{...})

    \vspace{-1.5ex}

    \rule{\textwidth}{0.5\fboxrule}

x.{\_}{\_}delattr{\_}{\_}('name') {\textless}=={\textgreater} del x.name
    \vspace{1ex}

    \end{boxedminipage}

    \label{object:__getattribute__}
    \index{object.\_\_getattribute\_\_ \textit{(function)}}

    \vspace{0.5ex}

    \begin{boxedminipage}{\textwidth}

    \raggedright \textbf{\_\_getattribute\_\_}(\textit{...})

    \vspace{-1.5ex}

    \rule{\textwidth}{0.5\fboxrule}

x.{\_}{\_}getattribute{\_}{\_}('name') {\textless}=={\textgreater} x.name
    \vspace{1ex}

    \end{boxedminipage}

    \label{object:__hash__}
    \index{object.\_\_hash\_\_ \textit{(function)}}

    \vspace{0.5ex}

    \begin{boxedminipage}{\textwidth}

    \raggedright \textbf{\_\_hash\_\_}(\textit{x})

    \vspace{-1.5ex}

    \rule{\textwidth}{0.5\fboxrule}

hash(x)
    \vspace{1ex}

    \end{boxedminipage}

    \label{object:__new__}
    \index{object.\_\_new\_\_ \textit{(function)}}

    \vspace{0.5ex}

    \begin{boxedminipage}{\textwidth}

    \raggedright \textbf{\_\_new\_\_}(\textit{T}, \textit{S}, \textit{...})

      \textbf{Return Value}
      \begin{quote}
\begin{alltt}
a new object with type S, a subtype of T
\end{alltt}

      \end{quote}

    \vspace{1ex}

    \end{boxedminipage}

    \label{object:__reduce__}
    \index{object.\_\_reduce\_\_ \textit{(function)}}

    \vspace{0.5ex}

    \begin{boxedminipage}{\textwidth}

    \raggedright \textbf{\_\_reduce\_\_}(\textit{...})

    \vspace{-1.5ex}

    \rule{\textwidth}{0.5\fboxrule}

helper for pickle
    \vspace{1ex}

    \end{boxedminipage}

    \label{object:__reduce_ex__}
    \index{object.\_\_reduce\_ex\_\_ \textit{(function)}}

    \vspace{0.5ex}

    \begin{boxedminipage}{\textwidth}

    \raggedright \textbf{\_\_reduce\_ex\_\_}(\textit{...})

    \vspace{-1.5ex}

    \rule{\textwidth}{0.5\fboxrule}

helper for pickle
    \vspace{1ex}

    \end{boxedminipage}

    \label{object:__repr__}
    \index{object.\_\_repr\_\_ \textit{(function)}}

    \vspace{0.5ex}

    \begin{boxedminipage}{\textwidth}

    \raggedright \textbf{\_\_repr\_\_}(\textit{x})

    \vspace{-1.5ex}

    \rule{\textwidth}{0.5\fboxrule}

repr(x)
    \vspace{1ex}

    \end{boxedminipage}

    \label{object:__setattr__}
    \index{object.\_\_setattr\_\_ \textit{(function)}}

    \vspace{0.5ex}

    \begin{boxedminipage}{\textwidth}

    \raggedright \textbf{\_\_setattr\_\_}(\textit{...})

    \vspace{-1.5ex}

    \rule{\textwidth}{0.5\fboxrule}

x.{\_}{\_}setattr{\_}{\_}('name', value) {\textless}=={\textgreater} x.name = value
    \vspace{1ex}

    \end{boxedminipage}

    \label{object:__str__}
    \index{object.\_\_str\_\_ \textit{(function)}}

    \vspace{0.5ex}

    \begin{boxedminipage}{\textwidth}

    \raggedright \textbf{\_\_str\_\_}(\textit{x})

    \vspace{-1.5ex}

    \rule{\textwidth}{0.5\fboxrule}

str(x)
    \vspace{1ex}

    \end{boxedminipage}


%%%%%%%%%%%%%%%%%%%%%%%%%%%%%%%%%%%%%%%%%%%%%%%%%%%%%%%%%%%%%%%%%%%%%%%%%%%
%%                              Properties                               %%
%%%%%%%%%%%%%%%%%%%%%%%%%%%%%%%%%%%%%%%%%%%%%%%%%%%%%%%%%%%%%%%%%%%%%%%%%%%

  \subsubsection{Properties}

\begin{longtable}{|p{.30\textwidth}|p{.62\textwidth}|l}
\cline{1-2}
\cline{1-2} \centering \textbf{Name} & \centering \textbf{Description}& \\
\cline{1-2}
\endhead\cline{1-2}\multicolumn{3}{r}{\small\textit{continued on next page}}\\\endfoot\cline{1-2}
\endlastfoot\raggedright \_\-\_\-c\-l\-a\-s\-s\-\_\-\_\- & \raggedright \textbf{Value:} 
{\tt {\textless}attribute '\_\_class\_\_' of 'object' objects{\textgreater}}&\\
\cline{1-2}
\end{longtable}


%%%%%%%%%%%%%%%%%%%%%%%%%%%%%%%%%%%%%%%%%%%%%%%%%%%%%%%%%%%%%%%%%%%%%%%%%%%
%%                          Instance Variables                           %%
%%%%%%%%%%%%%%%%%%%%%%%%%%%%%%%%%%%%%%%%%%%%%%%%%%%%%%%%%%%%%%%%%%%%%%%%%%%

  \subsubsection{Instance Variables}

\begin{longtable}{|p{.30\textwidth}|p{.62\textwidth}|l}
\cline{1-2}
\cline{1-2} \centering \textbf{Name} & \centering \textbf{Description}& \\
\cline{1-2}
\endhead\cline{1-2}\multicolumn{3}{r}{\small\textit{continued on next page}}\\\endfoot\cline{1-2}
\endlastfoot\raggedright d\- & An alias to the derivative of the function.&\\
\cline{1-2}
\end{longtable}

    \index{peach \textit{(package)}!peach.nn \textit{(package)}!peach.nn.af \textit{(module)}!peach.nn.af.Sigmoid \textit{(class)}|)}

%%%%%%%%%%%%%%%%%%%%%%%%%%%%%%%%%%%%%%%%%%%%%%%%%%%%%%%%%%%%%%%%%%%%%%%%%%%
%%                           Class Description                           %%
%%%%%%%%%%%%%%%%%%%%%%%%%%%%%%%%%%%%%%%%%%%%%%%%%%%%%%%%%%%%%%%%%%%%%%%%%%%

    \index{peach \textit{(package)}!peach.nn \textit{(package)}!peach.nn.af \textit{(module)}!peach.nn.af.Signum \textit{(class)}|(}
\subsection{Class Signum}

    \label{peach:nn:af:Signum}
\begin{tabular}{cccccccc}
% Line for object, linespec=[False, False]
\multicolumn{2}{r}{\settowidth{\BCL}{object}\multirow{2}{\BCL}{object}}
&&
&&
  \\\cline{3-3}
  &&\multicolumn{1}{c|}{}
&&
&&
  \\
% Line for peach.nn.af.Activation, linespec=[False]
\multicolumn{4}{r}{\settowidth{\BCL}{peach.nn.af.Activation}\multirow{2}{\BCL}{peach.nn.af.Activation}}
&&
  \\\cline{5-5}
  &&&&\multicolumn{1}{c|}{}
&&
  \\
&&&&\multicolumn{2}{l}{\textbf{peach.nn.af.Signum}}
\end{tabular}


Signum activation function

%%%%%%%%%%%%%%%%%%%%%%%%%%%%%%%%%%%%%%%%%%%%%%%%%%%%%%%%%%%%%%%%%%%%%%%%%%%
%%                                Methods                                %%
%%%%%%%%%%%%%%%%%%%%%%%%%%%%%%%%%%%%%%%%%%%%%%%%%%%%%%%%%%%%%%%%%%%%%%%%%%%

  \subsubsection{Methods}

    \vspace{0.5ex}

    \begin{boxedminipage}{\textwidth}

    \raggedright \textbf{\_\_init\_\_}(\textit{self})

    \vspace{-1.5ex}

    \rule{\textwidth}{0.5\fboxrule}

Initializes the object.
    \vspace{1ex}

      Overrides: peach.nn.af.Activation.\_\_init\_\_

    \end{boxedminipage}

    \vspace{0.5ex}

    \begin{boxedminipage}{\textwidth}

    \raggedright \textbf{\_\_call\_\_}(\textit{self}, \textit{x})

    \vspace{-1.5ex}

    \rule{\textwidth}{0.5\fboxrule}

Call interface to the object.

This method applies the activation function over a vector of activation
potentials, and returns the results.
    \vspace{1ex}

      \textbf{Parameters}
      \begin{quote}
        \begin{Ventry}{x}

          \item[x]


A real number or a vector of real numbers representing the
activation potential of a neuron or a layer of neurons.
        \end{Ventry}

      \end{quote}

    \vspace{1ex}

      \textbf{Return Value}
      \begin{quote}

The activation function applied over the input vector.
      \end{quote}

    \vspace{1ex}

      Overrides: peach.nn.af.Activation.\_\_call\_\_

    \end{boxedminipage}

    \vspace{0.5ex}

    \begin{boxedminipage}{\textwidth}

    \raggedright \textbf{derivative}(\textit{self}, \textit{x})

    \vspace{-1.5ex}

    \rule{\textwidth}{0.5\fboxrule}

The function derivative. Technically, this function doesn't have a
derivative, but making it equals to 1, this can be used in learning
algorithms.
    \vspace{1ex}

      \textbf{Parameters}
      \begin{quote}
        \begin{Ventry}{x}

          \item[x]


A real number or a vector of real numbers representing the
activation potential of a neuron or a layer of neurons.
        \end{Ventry}

      \end{quote}

    \vspace{1ex}

      \textbf{Return Value}
      \begin{quote}

The derivative of the activation function applied over the input
vector.
      \end{quote}

    \vspace{1ex}

      Overrides: peach.nn.af.Activation.derivative

    \end{boxedminipage}

    \label{object:__delattr__}
    \index{object.\_\_delattr\_\_ \textit{(function)}}

    \vspace{0.5ex}

    \begin{boxedminipage}{\textwidth}

    \raggedright \textbf{\_\_delattr\_\_}(\textit{...})

    \vspace{-1.5ex}

    \rule{\textwidth}{0.5\fboxrule}

x.{\_}{\_}delattr{\_}{\_}('name') {\textless}=={\textgreater} del x.name
    \vspace{1ex}

    \end{boxedminipage}

    \label{object:__getattribute__}
    \index{object.\_\_getattribute\_\_ \textit{(function)}}

    \vspace{0.5ex}

    \begin{boxedminipage}{\textwidth}

    \raggedright \textbf{\_\_getattribute\_\_}(\textit{...})

    \vspace{-1.5ex}

    \rule{\textwidth}{0.5\fboxrule}

x.{\_}{\_}getattribute{\_}{\_}('name') {\textless}=={\textgreater} x.name
    \vspace{1ex}

    \end{boxedminipage}

    \label{object:__hash__}
    \index{object.\_\_hash\_\_ \textit{(function)}}

    \vspace{0.5ex}

    \begin{boxedminipage}{\textwidth}

    \raggedright \textbf{\_\_hash\_\_}(\textit{x})

    \vspace{-1.5ex}

    \rule{\textwidth}{0.5\fboxrule}

hash(x)
    \vspace{1ex}

    \end{boxedminipage}

    \label{object:__new__}
    \index{object.\_\_new\_\_ \textit{(function)}}

    \vspace{0.5ex}

    \begin{boxedminipage}{\textwidth}

    \raggedright \textbf{\_\_new\_\_}(\textit{T}, \textit{S}, \textit{...})

      \textbf{Return Value}
      \begin{quote}
\begin{alltt}
a new object with type S, a subtype of T
\end{alltt}

      \end{quote}

    \vspace{1ex}

    \end{boxedminipage}

    \label{object:__reduce__}
    \index{object.\_\_reduce\_\_ \textit{(function)}}

    \vspace{0.5ex}

    \begin{boxedminipage}{\textwidth}

    \raggedright \textbf{\_\_reduce\_\_}(\textit{...})

    \vspace{-1.5ex}

    \rule{\textwidth}{0.5\fboxrule}

helper for pickle
    \vspace{1ex}

    \end{boxedminipage}

    \label{object:__reduce_ex__}
    \index{object.\_\_reduce\_ex\_\_ \textit{(function)}}

    \vspace{0.5ex}

    \begin{boxedminipage}{\textwidth}

    \raggedright \textbf{\_\_reduce\_ex\_\_}(\textit{...})

    \vspace{-1.5ex}

    \rule{\textwidth}{0.5\fboxrule}

helper for pickle
    \vspace{1ex}

    \end{boxedminipage}

    \label{object:__repr__}
    \index{object.\_\_repr\_\_ \textit{(function)}}

    \vspace{0.5ex}

    \begin{boxedminipage}{\textwidth}

    \raggedright \textbf{\_\_repr\_\_}(\textit{x})

    \vspace{-1.5ex}

    \rule{\textwidth}{0.5\fboxrule}

repr(x)
    \vspace{1ex}

    \end{boxedminipage}

    \label{object:__setattr__}
    \index{object.\_\_setattr\_\_ \textit{(function)}}

    \vspace{0.5ex}

    \begin{boxedminipage}{\textwidth}

    \raggedright \textbf{\_\_setattr\_\_}(\textit{...})

    \vspace{-1.5ex}

    \rule{\textwidth}{0.5\fboxrule}

x.{\_}{\_}setattr{\_}{\_}('name', value) {\textless}=={\textgreater} x.name = value
    \vspace{1ex}

    \end{boxedminipage}

    \label{object:__str__}
    \index{object.\_\_str\_\_ \textit{(function)}}

    \vspace{0.5ex}

    \begin{boxedminipage}{\textwidth}

    \raggedright \textbf{\_\_str\_\_}(\textit{x})

    \vspace{-1.5ex}

    \rule{\textwidth}{0.5\fboxrule}

str(x)
    \vspace{1ex}

    \end{boxedminipage}


%%%%%%%%%%%%%%%%%%%%%%%%%%%%%%%%%%%%%%%%%%%%%%%%%%%%%%%%%%%%%%%%%%%%%%%%%%%
%%                              Properties                               %%
%%%%%%%%%%%%%%%%%%%%%%%%%%%%%%%%%%%%%%%%%%%%%%%%%%%%%%%%%%%%%%%%%%%%%%%%%%%

  \subsubsection{Properties}

\begin{longtable}{|p{.30\textwidth}|p{.62\textwidth}|l}
\cline{1-2}
\cline{1-2} \centering \textbf{Name} & \centering \textbf{Description}& \\
\cline{1-2}
\endhead\cline{1-2}\multicolumn{3}{r}{\small\textit{continued on next page}}\\\endfoot\cline{1-2}
\endlastfoot\raggedright \_\-\_\-c\-l\-a\-s\-s\-\_\-\_\- & \raggedright \textbf{Value:} 
{\tt {\textless}attribute '\_\_class\_\_' of 'object' objects{\textgreater}}&\\
\cline{1-2}
\end{longtable}


%%%%%%%%%%%%%%%%%%%%%%%%%%%%%%%%%%%%%%%%%%%%%%%%%%%%%%%%%%%%%%%%%%%%%%%%%%%
%%                          Instance Variables                           %%
%%%%%%%%%%%%%%%%%%%%%%%%%%%%%%%%%%%%%%%%%%%%%%%%%%%%%%%%%%%%%%%%%%%%%%%%%%%

  \subsubsection{Instance Variables}

\begin{longtable}{|p{.30\textwidth}|p{.62\textwidth}|l}
\cline{1-2}
\cline{1-2} \centering \textbf{Name} & \centering \textbf{Description}& \\
\cline{1-2}
\endhead\cline{1-2}\multicolumn{3}{r}{\small\textit{continued on next page}}\\\endfoot\cline{1-2}
\endlastfoot\raggedright d\- & An alias to the derivative of the function.&\\
\cline{1-2}
\end{longtable}

    \index{peach \textit{(package)}!peach.nn \textit{(package)}!peach.nn.af \textit{(module)}!peach.nn.af.Signum \textit{(class)}|)}

%%%%%%%%%%%%%%%%%%%%%%%%%%%%%%%%%%%%%%%%%%%%%%%%%%%%%%%%%%%%%%%%%%%%%%%%%%%
%%                           Class Description                           %%
%%%%%%%%%%%%%%%%%%%%%%%%%%%%%%%%%%%%%%%%%%%%%%%%%%%%%%%%%%%%%%%%%%%%%%%%%%%

    \index{peach \textit{(package)}!peach.nn \textit{(package)}!peach.nn.af \textit{(module)}!peach.nn.af.ArcTan \textit{(class)}|(}
\subsection{Class ArcTan}

    \label{peach:nn:af:ArcTan}
\begin{tabular}{cccccccc}
% Line for object, linespec=[False, False]
\multicolumn{2}{r}{\settowidth{\BCL}{object}\multirow{2}{\BCL}{object}}
&&
&&
  \\\cline{3-3}
  &&\multicolumn{1}{c|}{}
&&
&&
  \\
% Line for peach.nn.af.Activation, linespec=[False]
\multicolumn{4}{r}{\settowidth{\BCL}{peach.nn.af.Activation}\multirow{2}{\BCL}{peach.nn.af.Activation}}
&&
  \\\cline{5-5}
  &&&&\multicolumn{1}{c|}{}
&&
  \\
&&&&\multicolumn{2}{l}{\textbf{peach.nn.af.ArcTan}}
\end{tabular}


Inverse tangent activation function

%%%%%%%%%%%%%%%%%%%%%%%%%%%%%%%%%%%%%%%%%%%%%%%%%%%%%%%%%%%%%%%%%%%%%%%%%%%
%%                                Methods                                %%
%%%%%%%%%%%%%%%%%%%%%%%%%%%%%%%%%%%%%%%%%%%%%%%%%%%%%%%%%%%%%%%%%%%%%%%%%%%

  \subsubsection{Methods}

    \vspace{0.5ex}

    \begin{boxedminipage}{\textwidth}

    \raggedright \textbf{\_\_init\_\_}(\textit{self}, \textit{a}=\texttt{1.0}, \textit{x0}=\texttt{0.0})

    \vspace{-1.5ex}

    \rule{\textwidth}{0.5\fboxrule}

Initializes the object
    \vspace{1ex}

      \textbf{Parameters}
      \begin{quote}
        \begin{Ventry}{xx}

          \item[a]


The slope of the function in the center \texttt{x0}. Defaults to 1.0.
          \item[x0]


The center of the sigmoid. Defaults to 0.0.
        \end{Ventry}

      \end{quote}

    \vspace{1ex}

      Overrides: peach.nn.af.Activation.\_\_init\_\_

    \end{boxedminipage}

    \vspace{0.5ex}

    \begin{boxedminipage}{\textwidth}

    \raggedright \textbf{\_\_call\_\_}(\textit{self}, \textit{x})

    \vspace{-1.5ex}

    \rule{\textwidth}{0.5\fboxrule}

Call interface to the object.

This method applies the activation function over a vector of activation
potentials, and returns the results.
    \vspace{1ex}

      \textbf{Parameters}
      \begin{quote}
        \begin{Ventry}{x}

          \item[x]


A real number or a vector of real numbers representing the
activation potential of a neuron or a layer of neurons.
        \end{Ventry}

      \end{quote}

    \vspace{1ex}

      \textbf{Return Value}
      \begin{quote}

The activation function applied over the input vector.
      \end{quote}

    \vspace{1ex}

      Overrides: peach.nn.af.Activation.\_\_call\_\_

    \end{boxedminipage}

    \vspace{0.5ex}

    \begin{boxedminipage}{\textwidth}

    \raggedright \textbf{derivative}(\textit{self}, \textit{x})

    \vspace{-1.5ex}

    \rule{\textwidth}{0.5\fboxrule}

The function derivative.
    \vspace{1ex}

      \textbf{Parameters}
      \begin{quote}
        \begin{Ventry}{x}

          \item[x]


A real number or a vector of real numbers representing the
activation potential of a neuron or a layer of neurons.
        \end{Ventry}

      \end{quote}

    \vspace{1ex}

      \textbf{Return Value}
      \begin{quote}

The derivative of the activation function applied over the input
vector.
      \end{quote}

    \vspace{1ex}

      Overrides: peach.nn.af.Activation.derivative

    \end{boxedminipage}

    \label{object:__delattr__}
    \index{object.\_\_delattr\_\_ \textit{(function)}}

    \vspace{0.5ex}

    \begin{boxedminipage}{\textwidth}

    \raggedright \textbf{\_\_delattr\_\_}(\textit{...})

    \vspace{-1.5ex}

    \rule{\textwidth}{0.5\fboxrule}

x.{\_}{\_}delattr{\_}{\_}('name') {\textless}=={\textgreater} del x.name
    \vspace{1ex}

    \end{boxedminipage}

    \label{object:__getattribute__}
    \index{object.\_\_getattribute\_\_ \textit{(function)}}

    \vspace{0.5ex}

    \begin{boxedminipage}{\textwidth}

    \raggedright \textbf{\_\_getattribute\_\_}(\textit{...})

    \vspace{-1.5ex}

    \rule{\textwidth}{0.5\fboxrule}

x.{\_}{\_}getattribute{\_}{\_}('name') {\textless}=={\textgreater} x.name
    \vspace{1ex}

    \end{boxedminipage}

    \label{object:__hash__}
    \index{object.\_\_hash\_\_ \textit{(function)}}

    \vspace{0.5ex}

    \begin{boxedminipage}{\textwidth}

    \raggedright \textbf{\_\_hash\_\_}(\textit{x})

    \vspace{-1.5ex}

    \rule{\textwidth}{0.5\fboxrule}

hash(x)
    \vspace{1ex}

    \end{boxedminipage}

    \label{object:__new__}
    \index{object.\_\_new\_\_ \textit{(function)}}

    \vspace{0.5ex}

    \begin{boxedminipage}{\textwidth}

    \raggedright \textbf{\_\_new\_\_}(\textit{T}, \textit{S}, \textit{...})

      \textbf{Return Value}
      \begin{quote}
\begin{alltt}
a new object with type S, a subtype of T
\end{alltt}

      \end{quote}

    \vspace{1ex}

    \end{boxedminipage}

    \label{object:__reduce__}
    \index{object.\_\_reduce\_\_ \textit{(function)}}

    \vspace{0.5ex}

    \begin{boxedminipage}{\textwidth}

    \raggedright \textbf{\_\_reduce\_\_}(\textit{...})

    \vspace{-1.5ex}

    \rule{\textwidth}{0.5\fboxrule}

helper for pickle
    \vspace{1ex}

    \end{boxedminipage}

    \label{object:__reduce_ex__}
    \index{object.\_\_reduce\_ex\_\_ \textit{(function)}}

    \vspace{0.5ex}

    \begin{boxedminipage}{\textwidth}

    \raggedright \textbf{\_\_reduce\_ex\_\_}(\textit{...})

    \vspace{-1.5ex}

    \rule{\textwidth}{0.5\fboxrule}

helper for pickle
    \vspace{1ex}

    \end{boxedminipage}

    \label{object:__repr__}
    \index{object.\_\_repr\_\_ \textit{(function)}}

    \vspace{0.5ex}

    \begin{boxedminipage}{\textwidth}

    \raggedright \textbf{\_\_repr\_\_}(\textit{x})

    \vspace{-1.5ex}

    \rule{\textwidth}{0.5\fboxrule}

repr(x)
    \vspace{1ex}

    \end{boxedminipage}

    \label{object:__setattr__}
    \index{object.\_\_setattr\_\_ \textit{(function)}}

    \vspace{0.5ex}

    \begin{boxedminipage}{\textwidth}

    \raggedright \textbf{\_\_setattr\_\_}(\textit{...})

    \vspace{-1.5ex}

    \rule{\textwidth}{0.5\fboxrule}

x.{\_}{\_}setattr{\_}{\_}('name', value) {\textless}=={\textgreater} x.name = value
    \vspace{1ex}

    \end{boxedminipage}

    \label{object:__str__}
    \index{object.\_\_str\_\_ \textit{(function)}}

    \vspace{0.5ex}

    \begin{boxedminipage}{\textwidth}

    \raggedright \textbf{\_\_str\_\_}(\textit{x})

    \vspace{-1.5ex}

    \rule{\textwidth}{0.5\fboxrule}

str(x)
    \vspace{1ex}

    \end{boxedminipage}


%%%%%%%%%%%%%%%%%%%%%%%%%%%%%%%%%%%%%%%%%%%%%%%%%%%%%%%%%%%%%%%%%%%%%%%%%%%
%%                              Properties                               %%
%%%%%%%%%%%%%%%%%%%%%%%%%%%%%%%%%%%%%%%%%%%%%%%%%%%%%%%%%%%%%%%%%%%%%%%%%%%

  \subsubsection{Properties}

\begin{longtable}{|p{.30\textwidth}|p{.62\textwidth}|l}
\cline{1-2}
\cline{1-2} \centering \textbf{Name} & \centering \textbf{Description}& \\
\cline{1-2}
\endhead\cline{1-2}\multicolumn{3}{r}{\small\textit{continued on next page}}\\\endfoot\cline{1-2}
\endlastfoot\raggedright \_\-\_\-c\-l\-a\-s\-s\-\_\-\_\- & \raggedright \textbf{Value:} 
{\tt {\textless}attribute '\_\_class\_\_' of 'object' objects{\textgreater}}&\\
\cline{1-2}
\end{longtable}


%%%%%%%%%%%%%%%%%%%%%%%%%%%%%%%%%%%%%%%%%%%%%%%%%%%%%%%%%%%%%%%%%%%%%%%%%%%
%%                          Instance Variables                           %%
%%%%%%%%%%%%%%%%%%%%%%%%%%%%%%%%%%%%%%%%%%%%%%%%%%%%%%%%%%%%%%%%%%%%%%%%%%%

  \subsubsection{Instance Variables}

\begin{longtable}{|p{.30\textwidth}|p{.62\textwidth}|l}
\cline{1-2}
\cline{1-2} \centering \textbf{Name} & \centering \textbf{Description}& \\
\cline{1-2}
\endhead\cline{1-2}\multicolumn{3}{r}{\small\textit{continued on next page}}\\\endfoot\cline{1-2}
\endlastfoot\raggedright d\- & An alias to the derivative of the function.&\\
\cline{1-2}
\end{longtable}

    \index{peach \textit{(package)}!peach.nn \textit{(package)}!peach.nn.af \textit{(module)}!peach.nn.af.ArcTan \textit{(class)}|)}

%%%%%%%%%%%%%%%%%%%%%%%%%%%%%%%%%%%%%%%%%%%%%%%%%%%%%%%%%%%%%%%%%%%%%%%%%%%
%%                           Class Description                           %%
%%%%%%%%%%%%%%%%%%%%%%%%%%%%%%%%%%%%%%%%%%%%%%%%%%%%%%%%%%%%%%%%%%%%%%%%%%%

    \index{peach \textit{(package)}!peach.nn \textit{(package)}!peach.nn.af \textit{(module)}!peach.nn.af.TanH \textit{(class)}|(}
\subsection{Class TanH}

    \label{peach:nn:af:TanH}
\begin{tabular}{cccccccc}
% Line for object, linespec=[False, False]
\multicolumn{2}{r}{\settowidth{\BCL}{object}\multirow{2}{\BCL}{object}}
&&
&&
  \\\cline{3-3}
  &&\multicolumn{1}{c|}{}
&&
&&
  \\
% Line for peach.nn.af.Activation, linespec=[False]
\multicolumn{4}{r}{\settowidth{\BCL}{peach.nn.af.Activation}\multirow{2}{\BCL}{peach.nn.af.Activation}}
&&
  \\\cline{5-5}
  &&&&\multicolumn{1}{c|}{}
&&
  \\
&&&&\multicolumn{2}{l}{\textbf{peach.nn.af.TanH}}
\end{tabular}


Hyperbolic tangent activation function

%%%%%%%%%%%%%%%%%%%%%%%%%%%%%%%%%%%%%%%%%%%%%%%%%%%%%%%%%%%%%%%%%%%%%%%%%%%
%%                                Methods                                %%
%%%%%%%%%%%%%%%%%%%%%%%%%%%%%%%%%%%%%%%%%%%%%%%%%%%%%%%%%%%%%%%%%%%%%%%%%%%

  \subsubsection{Methods}

    \vspace{0.5ex}

    \begin{boxedminipage}{\textwidth}

    \raggedright \textbf{\_\_init\_\_}(\textit{self}, \textit{a}=\texttt{1.0}, \textit{x0}=\texttt{0.0})

    \vspace{-1.5ex}

    \rule{\textwidth}{0.5\fboxrule}

Initializes the object
    \vspace{1ex}

      \textbf{Parameters}
      \begin{quote}
        \begin{Ventry}{xx}

          \item[a]


The slope of the function in the center \texttt{x0}. Defaults to 1.0.
          \item[x0]


The center of the sigmoid. Defaults to 0.0.
        \end{Ventry}

      \end{quote}

    \vspace{1ex}

      Overrides: peach.nn.af.Activation.\_\_init\_\_

    \end{boxedminipage}

    \vspace{0.5ex}

    \begin{boxedminipage}{\textwidth}

    \raggedright \textbf{\_\_call\_\_}(\textit{self}, \textit{x})

    \vspace{-1.5ex}

    \rule{\textwidth}{0.5\fboxrule}

Call interface to the object.

This method applies the activation function over a vector of activation
potentials, and returns the results.
    \vspace{1ex}

      \textbf{Parameters}
      \begin{quote}
        \begin{Ventry}{x}

          \item[x]


A real number or a vector of real numbers representing the
activation potential of a neuron or a layer of neurons.
        \end{Ventry}

      \end{quote}

    \vspace{1ex}

      \textbf{Return Value}
      \begin{quote}

The activation function applied over the input vector.
      \end{quote}

    \vspace{1ex}

      Overrides: peach.nn.af.Activation.\_\_call\_\_

    \end{boxedminipage}

    \vspace{0.5ex}

    \begin{boxedminipage}{\textwidth}

    \raggedright \textbf{derivative}(\textit{self}, \textit{x})

    \vspace{-1.5ex}

    \rule{\textwidth}{0.5\fboxrule}

The function derivative.
    \vspace{1ex}

      \textbf{Parameters}
      \begin{quote}
        \begin{Ventry}{x}

          \item[x]


A real number or a vector of real numbers representing the
activation potential of a neuron or a layer of neurons.
        \end{Ventry}

      \end{quote}

    \vspace{1ex}

      \textbf{Return Value}
      \begin{quote}

The derivative of the activation function applied over the input
vector.
      \end{quote}

    \vspace{1ex}

      Overrides: peach.nn.af.Activation.derivative

    \end{boxedminipage}

    \label{object:__delattr__}
    \index{object.\_\_delattr\_\_ \textit{(function)}}

    \vspace{0.5ex}

    \begin{boxedminipage}{\textwidth}

    \raggedright \textbf{\_\_delattr\_\_}(\textit{...})

    \vspace{-1.5ex}

    \rule{\textwidth}{0.5\fboxrule}

x.{\_}{\_}delattr{\_}{\_}('name') {\textless}=={\textgreater} del x.name
    \vspace{1ex}

    \end{boxedminipage}

    \label{object:__getattribute__}
    \index{object.\_\_getattribute\_\_ \textit{(function)}}

    \vspace{0.5ex}

    \begin{boxedminipage}{\textwidth}

    \raggedright \textbf{\_\_getattribute\_\_}(\textit{...})

    \vspace{-1.5ex}

    \rule{\textwidth}{0.5\fboxrule}

x.{\_}{\_}getattribute{\_}{\_}('name') {\textless}=={\textgreater} x.name
    \vspace{1ex}

    \end{boxedminipage}

    \label{object:__hash__}
    \index{object.\_\_hash\_\_ \textit{(function)}}

    \vspace{0.5ex}

    \begin{boxedminipage}{\textwidth}

    \raggedright \textbf{\_\_hash\_\_}(\textit{x})

    \vspace{-1.5ex}

    \rule{\textwidth}{0.5\fboxrule}

hash(x)
    \vspace{1ex}

    \end{boxedminipage}

    \label{object:__new__}
    \index{object.\_\_new\_\_ \textit{(function)}}

    \vspace{0.5ex}

    \begin{boxedminipage}{\textwidth}

    \raggedright \textbf{\_\_new\_\_}(\textit{T}, \textit{S}, \textit{...})

      \textbf{Return Value}
      \begin{quote}
\begin{alltt}
a new object with type S, a subtype of T
\end{alltt}

      \end{quote}

    \vspace{1ex}

    \end{boxedminipage}

    \label{object:__reduce__}
    \index{object.\_\_reduce\_\_ \textit{(function)}}

    \vspace{0.5ex}

    \begin{boxedminipage}{\textwidth}

    \raggedright \textbf{\_\_reduce\_\_}(\textit{...})

    \vspace{-1.5ex}

    \rule{\textwidth}{0.5\fboxrule}

helper for pickle
    \vspace{1ex}

    \end{boxedminipage}

    \label{object:__reduce_ex__}
    \index{object.\_\_reduce\_ex\_\_ \textit{(function)}}

    \vspace{0.5ex}

    \begin{boxedminipage}{\textwidth}

    \raggedright \textbf{\_\_reduce\_ex\_\_}(\textit{...})

    \vspace{-1.5ex}

    \rule{\textwidth}{0.5\fboxrule}

helper for pickle
    \vspace{1ex}

    \end{boxedminipage}

    \label{object:__repr__}
    \index{object.\_\_repr\_\_ \textit{(function)}}

    \vspace{0.5ex}

    \begin{boxedminipage}{\textwidth}

    \raggedright \textbf{\_\_repr\_\_}(\textit{x})

    \vspace{-1.5ex}

    \rule{\textwidth}{0.5\fboxrule}

repr(x)
    \vspace{1ex}

    \end{boxedminipage}

    \label{object:__setattr__}
    \index{object.\_\_setattr\_\_ \textit{(function)}}

    \vspace{0.5ex}

    \begin{boxedminipage}{\textwidth}

    \raggedright \textbf{\_\_setattr\_\_}(\textit{...})

    \vspace{-1.5ex}

    \rule{\textwidth}{0.5\fboxrule}

x.{\_}{\_}setattr{\_}{\_}('name', value) {\textless}=={\textgreater} x.name = value
    \vspace{1ex}

    \end{boxedminipage}

    \label{object:__str__}
    \index{object.\_\_str\_\_ \textit{(function)}}

    \vspace{0.5ex}

    \begin{boxedminipage}{\textwidth}

    \raggedright \textbf{\_\_str\_\_}(\textit{x})

    \vspace{-1.5ex}

    \rule{\textwidth}{0.5\fboxrule}

str(x)
    \vspace{1ex}

    \end{boxedminipage}


%%%%%%%%%%%%%%%%%%%%%%%%%%%%%%%%%%%%%%%%%%%%%%%%%%%%%%%%%%%%%%%%%%%%%%%%%%%
%%                              Properties                               %%
%%%%%%%%%%%%%%%%%%%%%%%%%%%%%%%%%%%%%%%%%%%%%%%%%%%%%%%%%%%%%%%%%%%%%%%%%%%

  \subsubsection{Properties}

\begin{longtable}{|p{.30\textwidth}|p{.62\textwidth}|l}
\cline{1-2}
\cline{1-2} \centering \textbf{Name} & \centering \textbf{Description}& \\
\cline{1-2}
\endhead\cline{1-2}\multicolumn{3}{r}{\small\textit{continued on next page}}\\\endfoot\cline{1-2}
\endlastfoot\raggedright \_\-\_\-c\-l\-a\-s\-s\-\_\-\_\- & \raggedright \textbf{Value:} 
{\tt {\textless}attribute '\_\_class\_\_' of 'object' objects{\textgreater}}&\\
\cline{1-2}
\end{longtable}


%%%%%%%%%%%%%%%%%%%%%%%%%%%%%%%%%%%%%%%%%%%%%%%%%%%%%%%%%%%%%%%%%%%%%%%%%%%
%%                          Instance Variables                           %%
%%%%%%%%%%%%%%%%%%%%%%%%%%%%%%%%%%%%%%%%%%%%%%%%%%%%%%%%%%%%%%%%%%%%%%%%%%%

  \subsubsection{Instance Variables}

\begin{longtable}{|p{.30\textwidth}|p{.62\textwidth}|l}
\cline{1-2}
\cline{1-2} \centering \textbf{Name} & \centering \textbf{Description}& \\
\cline{1-2}
\endhead\cline{1-2}\multicolumn{3}{r}{\small\textit{continued on next page}}\\\endfoot\cline{1-2}
\endlastfoot\raggedright d\- & An alias to the derivative of the function.&\\
\cline{1-2}
\end{longtable}

    \index{peach \textit{(package)}!peach.nn \textit{(package)}!peach.nn.af \textit{(module)}!peach.nn.af.TanH \textit{(class)}|)}
    \index{peach \textit{(package)}!peach.nn \textit{(package)}!peach.nn.af \textit{(module)}|)}
