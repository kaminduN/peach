%
% API Documentation for Peach - Computational Intelligence for Python
% Module peach.fuzzy.defuzzy
%
% Generated by epydoc 3.0beta1
% [Mon Dec 21 08:51:35 2009]
%

%%%%%%%%%%%%%%%%%%%%%%%%%%%%%%%%%%%%%%%%%%%%%%%%%%%%%%%%%%%%%%%%%%%%%%%%%%%
%%                          Module Description                           %%
%%%%%%%%%%%%%%%%%%%%%%%%%%%%%%%%%%%%%%%%%%%%%%%%%%%%%%%%%%%%%%%%%%%%%%%%%%%

    \index{peach \textit{(package)}!peach.fuzzy \textit{(package)}!peach.fuzzy.defuzzy \textit{(module)}|(}
\section{Module peach.fuzzy.defuzzy}

    \label{peach:fuzzy:defuzzy}

This package implements defuzzification methods for use with fuzzy controllers.

Defuzzification methods take a set of numerical values, their corresponding
fuzzy membership values and calculate a defuzzified value for them. They're
implemented as functions, not as classes. So, to implement your own, use the
directions below.

These methods are implemented as functions with the signature \texttt{(mf, y)}, where
\texttt{mf} is the fuzzy set, and \texttt{y} is an array of values. That is, \texttt{mf} is a
fuzzy set containing the membership values of each one in the \texttt{y} array, in
the respective order. Both arrays should have the same dimensions, or else the
methods won't work.

See the example:
\begin{quote}{\ttfamily \raggedright \noindent
>{}>{}>~import~numpy~\\
>{}>{}>~from~peach~import~*~\\
>{}>{}>~y~=~numpy.linspace(0.,~5.,~100)~\\
>{}>{}>~m{\_}y~=~Triangle(1.,~2.,~3.)~\\
>{}>{}>~Centroid(m{\_}y(y),~y)~\\
2.0001030715316435
}\end{quote}

The methods defined here are the most commonly used.

%%%%%%%%%%%%%%%%%%%%%%%%%%%%%%%%%%%%%%%%%%%%%%%%%%%%%%%%%%%%%%%%%%%%%%%%%%%
%%                               Functions                               %%
%%%%%%%%%%%%%%%%%%%%%%%%%%%%%%%%%%%%%%%%%%%%%%%%%%%%%%%%%%%%%%%%%%%%%%%%%%%

  \subsection{Functions}

    \label{peach:fuzzy:defuzzy:Centroid}
    \index{peach \textit{(package)}!peach.fuzzy \textit{(package)}!peach.fuzzy.defuzzy \textit{(module)}!peach.fuzzy.defuzzy.Centroid \textit{(function)}}

    \vspace{0.5ex}

    \begin{boxedminipage}{\textwidth}

    \raggedright \textbf{Centroid}(\textit{mf}, \textit{y})

    \vspace{-1.5ex}

    \rule{\textwidth}{0.5\fboxrule}

Center of gravity method.

The center of gravity is calculate using the standard formula found in any
calculus book. The integrals are calculated using the trapezoid method.
    \vspace{1ex}

      \textbf{Parameters}
      \begin{quote}
        \begin{Ventry}{xx}

          \item[mf]


Fuzzy set containing the membership values of the elements in the
vector given in sequence
          \item[y]


Array of domain values of the defuzzified variable.
        \end{Ventry}

      \end{quote}

    \vspace{1ex}

      \textbf{Return Value}
      \begin{quote}

The center of gravity of the fuzzy set.
      \end{quote}

    \vspace{1ex}

    \end{boxedminipage}

    \label{peach:fuzzy:defuzzy:Bissector}
    \index{peach \textit{(package)}!peach.fuzzy \textit{(package)}!peach.fuzzy.defuzzy \textit{(module)}!peach.fuzzy.defuzzy.Bissector \textit{(function)}}

    \vspace{0.5ex}

    \begin{boxedminipage}{\textwidth}

    \raggedright \textbf{Bissector}(\textit{mf}, \textit{y})

    \vspace{-1.5ex}

    \rule{\textwidth}{0.5\fboxrule}

Bissection method

The bissection method finds a coordinate \texttt{y} in domain that divides the
fuzzy set in two subsets with the same area. Integrals are calculated using
the trapezoid method. This method only works if the values in \texttt{y} are
equally spaced, otherwise, the method will fail.
    \vspace{1ex}

      \textbf{Parameters}
      \begin{quote}
        \begin{Ventry}{xx}

          \item[mf]


Fuzzy set containing the membership values of the elements in the
vector given in sequence
          \item[y]


Array of domain values of the defuzzified variable.
        \end{Ventry}

      \end{quote}

    \vspace{1ex}

      \textbf{Return Value}
      \begin{quote}

Defuzzified value by the bissection method.
      \end{quote}

    \vspace{1ex}

    \end{boxedminipage}

    \label{peach:fuzzy:defuzzy:SmallestOfMaxima}
    \index{peach \textit{(package)}!peach.fuzzy \textit{(package)}!peach.fuzzy.defuzzy \textit{(module)}!peach.fuzzy.defuzzy.SmallestOfMaxima \textit{(function)}}

    \vspace{0.5ex}

    \begin{boxedminipage}{\textwidth}

    \raggedright \textbf{SmallestOfMaxima}(\textit{mf}, \textit{y})

    \vspace{-1.5ex}

    \rule{\textwidth}{0.5\fboxrule}

Smallest of maxima method.

This method finds all the points in the domain which have maximum membership
value in the fuzzy set, and returns the smallest of them.
    \vspace{1ex}

      \textbf{Parameters}
      \begin{quote}
        \begin{Ventry}{xx}

          \item[mf]


Fuzzy set containing the membership values of the elements in the
vector given in sequence
          \item[y]


Array of domain values of the defuzzified variable.
        \end{Ventry}

      \end{quote}

    \vspace{1ex}

      \textbf{Return Value}
      \begin{quote}

Defuzzified value by the smallest of maxima method.
      \end{quote}

    \vspace{1ex}

    \end{boxedminipage}

    \label{peach:fuzzy:defuzzy:LargestOfMaxima}
    \index{peach \textit{(package)}!peach.fuzzy \textit{(package)}!peach.fuzzy.defuzzy \textit{(module)}!peach.fuzzy.defuzzy.LargestOfMaxima \textit{(function)}}

    \vspace{0.5ex}

    \begin{boxedminipage}{\textwidth}

    \raggedright \textbf{LargestOfMaxima}(\textit{mf}, \textit{y})

    \vspace{-1.5ex}

    \rule{\textwidth}{0.5\fboxrule}

Largest of maxima method.

This method finds all the points in the domain which have maximum membership
value in the fuzzy set, and returns the largest of them.
    \vspace{1ex}

      \textbf{Parameters}
      \begin{quote}
        \begin{Ventry}{xx}

          \item[mf]


Fuzzy set containing the membership values of the elements in the
vector given in sequence
          \item[y]


Array of domain values of the defuzzified variable.
        \end{Ventry}

      \end{quote}

    \vspace{1ex}

      \textbf{Return Value}
      \begin{quote}

Defuzzified value by the largest of maxima method.
      \end{quote}

    \vspace{1ex}

    \end{boxedminipage}

    \label{peach:fuzzy:defuzzy:MeanOfMaxima}
    \index{peach \textit{(package)}!peach.fuzzy \textit{(package)}!peach.fuzzy.defuzzy \textit{(module)}!peach.fuzzy.defuzzy.MeanOfMaxima \textit{(function)}}

    \vspace{0.5ex}

    \begin{boxedminipage}{\textwidth}

    \raggedright \textbf{MeanOfMaxima}(\textit{mf}, \textit{y})

    \vspace{-1.5ex}

    \rule{\textwidth}{0.5\fboxrule}

Mean of maxima method.

This method finds the smallest and largest of maxima, and returns their
average.
    \vspace{1ex}

      \textbf{Parameters}
      \begin{quote}
        \begin{Ventry}{xx}

          \item[mf]


Fuzzy set containing the membership values of the elements in the
vector given in sequence
          \item[y]


Array of domain values of the defuzzified variable.
        \end{Ventry}

      \end{quote}

    \vspace{1ex}

      \textbf{Return Value}
      \begin{quote}

Defuzzified value by the  of maxima method.
      \end{quote}

    \vspace{1ex}

    \end{boxedminipage}


%%%%%%%%%%%%%%%%%%%%%%%%%%%%%%%%%%%%%%%%%%%%%%%%%%%%%%%%%%%%%%%%%%%%%%%%%%%
%%                               Variables                               %%
%%%%%%%%%%%%%%%%%%%%%%%%%%%%%%%%%%%%%%%%%%%%%%%%%%%%%%%%%%%%%%%%%%%%%%%%%%%

  \subsection{Variables}

\begin{longtable}{|p{.30\textwidth}|p{.62\textwidth}|l}
\cline{1-2}
\cline{1-2} \centering \textbf{Name} & \centering \textbf{Description}& \\
\cline{1-2}
\endhead\cline{1-2}\multicolumn{3}{r}{\small\textit{continued on next page}}\\\endfoot\cline{1-2}
\endlastfoot\raggedright \_\-\_\-d\-o\-c\-\_\-\_\- & \raggedright \textbf{Value:} 
{\tt \texttt{...}}&\\
\cline{1-2}
\end{longtable}

    \index{peach \textit{(package)}!peach.fuzzy \textit{(package)}!peach.fuzzy.defuzzy \textit{(module)}|)}
