%
% API Documentation for Peach - Computational Intelligence for Python
% Include File
%
% Generated by epydoc 3.0.1
% [Sun Jul 31 17:00:38 2011]
%
\documentclass{article}
\usepackage{alltt, parskip, fancyhdr, boxedminipage}
\usepackage{makeidx, multirow, longtable, tocbibind, amssymb}
\usepackage{fullpage}
\usepackage[usenames]{color}
\setlength{\headheight}{16pt}
\setlength{\headsep}{24pt}
\setlength{\topmargin}{-\headsep}
\setlength{\parindent}{0ex}
\setlength{\parskip}{2ex}
\setlength{\fboxrule}{2\fboxrule}
\newlength{\BCL} % base class length, for base trees.
\pagestyle{fancy}
\renewcommand{\sectionmark}[1]{\markboth{#1}{}}
\renewcommand{\subsectionmark}[1]{\markright{#1}}
\definecolor{py@keywordcolour}{rgb}{1,0.45882,0}
\definecolor{py@stringcolour}{rgb}{0,0.666666,0}
\definecolor{py@commentcolour}{rgb}{1,0,0}
\definecolor{py@ps1colour}{rgb}{0.60784,0,0}
\definecolor{py@ps2colour}{rgb}{0.60784,0,1}
\definecolor{py@inputcolour}{rgb}{0,0,0}
\definecolor{py@outputcolour}{rgb}{0,0,1}
\definecolor{py@exceptcolour}{rgb}{1,0,0}
\definecolor{py@defnamecolour}{rgb}{1,0.5,0.5}
\definecolor{py@builtincolour}{rgb}{0.58039,0,0.58039}
\definecolor{py@identifiercolour}{rgb}{0,0,0}
\definecolor{py@linenumcolour}{rgb}{0.4,0.4,0.4}
\definecolor{py@inputcolour}{rgb}{0,0,0}
% Prompt
\newcommand{\pysrcprompt}[1]{\textcolor{py@ps1colour}{\small\textbf{#1}}}
\newcommand{\pysrcmore}[1]{\textcolor{py@ps2colour}{\small\textbf{#1}}}
% Source code
\newcommand{\pysrckeyword}[1]{\textcolor{py@keywordcolour}{\small\textbf{#1}}}
\newcommand{\pysrcbuiltin}[1]{\textcolor{py@builtincolour}{\small\textbf{#1}}}
\newcommand{\pysrcstring}[1]{\textcolor{py@stringcolour}{\small\textbf{#1}}}
\newcommand{\pysrcdefname}[1]{\textcolor{py@defnamecolour}{\small\textbf{#1}}}
\newcommand{\pysrcother}[1]{\small\textbf{#1}}
% Comments
\newcommand{\pysrccomment}[1]{\textcolor{py@commentcolour}{\small\textbf{#1}}}
% Output
\newcommand{\pysrcoutput}[1]{\textcolor{py@outputcolour}{\small\textbf{#1}}}
% Exceptions
\newcommand{\pysrcexcept}[1]{\textcolor{py@exceptcolour}{\small\textbf{#1}}}
\newlength{\funcindent}
\newlength{\funcwidth}
\setlength{\funcindent}{1cm}
\setlength{\funcwidth}{\textwidth}
\addtolength{\funcwidth}{-2\funcindent}
\newlength{\varindent}
\newlength{\varnamewidth}
\newlength{\vardescrwidth}
\newlength{\varwidth}
\setlength{\varindent}{1cm}
\setlength{\varnamewidth}{.3\textwidth}
\setlength{\varwidth}{\textwidth}
\addtolength{\varwidth}{-4\tabcolsep}
\addtolength{\varwidth}{-3\arrayrulewidth}
\addtolength{\varwidth}{-2\varindent}
\setlength{\vardescrwidth}{\varwidth}
\addtolength{\vardescrwidth}{-\varnamewidth}
\newenvironment{Ventry}[1]%
 {\begin{list}{}{%
   \renewcommand{\makelabel}[1]{\texttt{##1:}\hfil}%
   \settowidth{\labelwidth}{\texttt{#1:}}%
   \setlength{\leftmargin}{\labelsep}%
   \addtolength{\leftmargin}{\labelwidth}}}%
 {\end{list}}
\usepackage[utf8]{inputenc}
\definecolor{UrlColor}{rgb}{0,0.08,0.45}
\usepackage[dvips, pagebackref, pdftitle={Peach - Computational Intelligence for Python}, pdfcreator={epydoc 3.0.1}, bookmarks=true, bookmarksopen=false, pdfpagemode=UseOutlines, colorlinks=true, linkcolor=black, anchorcolor=black, citecolor=black, filecolor=black, menucolor=black, pagecolor=black, urlcolor=UrlColor]{hyperref}
\makeindex

\begin{document}


%%%%%%%%%%%%%%%%%%%%%%%%%%%%%%%%%%%%%%%%%%%%%%%%%%%%%%%%%%%%%%%%%%%%%%%%%%%
%%                                Header                                 %%
%%%%%%%%%%%%%%%%%%%%%%%%%%%%%%%%%%%%%%%%%%%%%%%%%%%%%%%%%%%%%%%%%%%%%%%%%%%


%%%%%%%%%%%%%%%%%%%%%%%%%%%%%%%%%%%%%%%%%%%%%%%%%%%%%%%%%%%%%%%%%%%%%%%%%%%
%%                                 Title                                 %%
%%%%%%%%%%%%%%%%%%%%%%%%%%%%%%%%%%%%%%%%%%%%%%%%%%%%%%%%%%%%%%%%%%%%%%%%%%%

\title{Peach~-~Computational~Intelligence~for~Python}
\author{API Documentation}
\maketitle

%%%%%%%%%%%%%%%%%%%%%%%%%%%%%%%%%%%%%%%%%%%%%%%%%%%%%%%%%%%%%%%%%%%%%%%%%%%
%%                           Table of Contents                           %%
%%%%%%%%%%%%%%%%%%%%%%%%%%%%%%%%%%%%%%%%%%%%%%%%%%%%%%%%%%%%%%%%%%%%%%%%%%%

\addtolength{\parskip}{-2ex}
\tableofcontents
\addtolength{\parskip}{2ex}

%%%%%%%%%%%%%%%%%%%%%%%%%%%%%%%%%%%%%%%%%%%%%%%%%%%%%%%%%%%%%%%%%%%%%%%%%%%
%%                               Includes                                %%
%%%%%%%%%%%%%%%%%%%%%%%%%%%%%%%%%%%%%%%%%%%%%%%%%%%%%%%%%%%%%%%%%%%%%%%%%%%

%
% API Documentation for Peach - Computational Intelligence for Python
% Package peach
%
% Generated by epydoc 3.0.1
% [Thu Jul 28 16:37:45 2011]
%

%%%%%%%%%%%%%%%%%%%%%%%%%%%%%%%%%%%%%%%%%%%%%%%%%%%%%%%%%%%%%%%%%%%%%%%%%%%
%%                          Module Description                           %%
%%%%%%%%%%%%%%%%%%%%%%%%%%%%%%%%%%%%%%%%%%%%%%%%%%%%%%%%%%%%%%%%%%%%%%%%%%%

    \index{peach \textit{(package)}|(}
\section{Package peach}

    \label{peach}

\emph{Peach} is a pure-Python package with aims to implement techniques of machine
learning and computational intelligence. It contains packages for
%
\begin{itemize}

\item Neural Networks, including, but not limited to, multi-layer perceptrons and
self-organizing maps;

\item Fuzzy logic and fuzzy inference systems, including Mamdani-type and
Sugeno-type controllers;

\item Optimization packages, including multidimensional optimization;

\item Stochastic Optimizations, including genetic algorithms, simulated annealing,
particle swarm optimization;

\item A lot more.

\end{itemize}
\textbf{Author:} 
José Alexandre Nalon


\textbf{Version:} 0.1.0




%%%%%%%%%%%%%%%%%%%%%%%%%%%%%%%%%%%%%%%%%%%%%%%%%%%%%%%%%%%%%%%%%%%%%%%%%%%
%%                                Modules                                %%
%%%%%%%%%%%%%%%%%%%%%%%%%%%%%%%%%%%%%%%%%%%%%%%%%%%%%%%%%%%%%%%%%%%%%%%%%%%

\subsection{Modules}

\begin{itemize}
\setlength{\parskip}{0ex}
\item \textbf{fuzzy}: 
This package implements fuzzy logic. Consult:


  \textit{(Section \ref{peach:fuzzy}, p.~\pageref{peach:fuzzy})}

  \begin{itemize}
\setlength{\parskip}{0ex}
    \item \textbf{base}: 
This package implements basic definitions for fuzzy logic


  \textit{(Section \ref{peach:fuzzy:base}, p.~\pageref{peach:fuzzy:base})}

    \item \textbf{cmeans}: 
Fuzzy C-Means


  \textit{(Section \ref{peach:fuzzy:cmeans}, p.~\pageref{peach:fuzzy:cmeans})}

    \item \textbf{control}: 
This package implements fuzzy controllers, of fuzzy inference systems.


  \textit{(Section \ref{peach:fuzzy:control}, p.~\pageref{peach:fuzzy:control})}

    \item \textbf{defuzzy}: 
This package implements defuzzification methods for use with fuzzy controllers.


  \textit{(Section \ref{peach:fuzzy:defuzzy}, p.~\pageref{peach:fuzzy:defuzzy})}

    \item \textbf{mf}: 
Membership functions


  \textit{(Section \ref{peach:fuzzy:mf}, p.~\pageref{peach:fuzzy:mf})}

    \item \textbf{norms}: 
This package implements operations of fuzzy logic.


  \textit{(Section \ref{peach:fuzzy:norms}, p.~\pageref{peach:fuzzy:norms})}

  \end{itemize}
\item \textbf{ga}: 
This package implements genetic algorithms. Consult:


  \textit{(Section \ref{peach:ga}, p.~\pageref{peach:ga})}

  \begin{itemize}
\setlength{\parskip}{0ex}
    \item \textbf{base}: 
Basic Genetic Algorithm (GA)


  \textit{(Section \ref{peach:ga:base}, p.~\pageref{peach:ga:base})}

    \item \textbf{chromosome}: 
Basic definitions and classes for manipulating chromosomes


  \textit{(Section \ref{peach:ga:chromosome}, p.~\pageref{peach:ga:chromosome})}

    \item \textbf{crossover}: 
Basic definitions for crossover operations and base classes.


  \textit{(Section \ref{peach:ga:crossover}, p.~\pageref{peach:ga:crossover})}

    \item \textbf{fitness}: 
Basic definitions and base classes for definition of fitness functions for use
with genetic algorithms.


  \textit{(Section \ref{peach:ga:fitness}, p.~\pageref{peach:ga:fitness})}

    \item \textbf{mutation}: 
Basic definitions and classes for operating mutation on chromosomes.


  \textit{(Section \ref{peach:ga:mutation}, p.~\pageref{peach:ga:mutation})}

    \item \textbf{selection}: 
Basic classes and definitions for selection operator.


  \textit{(Section \ref{peach:ga:selection}, p.~\pageref{peach:ga:selection})}

  \end{itemize}
\item \textbf{nn}: 
This package implements support for neural networks. Consult:


  \textit{(Section \ref{peach:nn}, p.~\pageref{peach:nn})}

  \begin{itemize}
\setlength{\parskip}{0ex}
    \item \textbf{af}: 
Base activation functions and base class


  \textit{(Section \ref{peach:nn:af}, p.~\pageref{peach:nn:af})}

    \item \textbf{base}: 
Basic definitions for layers of neurons.


  \textit{(Section \ref{peach:nn:base}, p.~\pageref{peach:nn:base})}

    \item \textbf{kmeans}: 
K-Means clustering algorithm


  \textit{(Section \ref{peach:nn:kmeans}, p.~\pageref{peach:nn:kmeans})}

    \item \textbf{lrules}: 
Learning rules for neural networks and base classes for custom learning.


  \textit{(Section \ref{peach:nn:lrules}, p.~\pageref{peach:nn:lrules})}

    \item \textbf{mem}: 
Associative memories and Hopfield network model.


  \textit{(Section \ref{peach:nn:mem}, p.~\pageref{peach:nn:mem})}

    \item \textbf{nnet}: 
Basic topologies of neural networks.


  \textit{(Section \ref{peach:nn:nnet}, p.~\pageref{peach:nn:nnet})}

  \end{itemize}
\item \textbf{optm}: 
This package implements deterministic optimization methods. Consult:


  \textit{(Section \ref{peach:optm}, p.~\pageref{peach:optm})}

  \begin{itemize}
\setlength{\parskip}{0ex}
    \item \textbf{base}: 
Basic definitons and base class for optimizers


  \textit{(Section \ref{peach:optm:base}, p.~\pageref{peach:optm:base})}

    \item \textbf{linear}: 
This package implements basic one variable only optimizers.


  \textit{(Section \ref{peach:optm:linear}, p.~\pageref{peach:optm:linear})}

    \item \textbf{multivar}: 
This package implements basic multivariable optimizers, including gradient and
Newton searches.


  \textit{(Section \ref{peach:optm:multivar}, p.~\pageref{peach:optm:multivar})}

    \item \textbf{quasinewton}: 
This package implements basic quasi-Newton optimizers. Newton optimizer is very
efficient, except that inverse matrices need to be calculated at each
convergence step. These methods try to estimate the hessian inverse iteratively,
thus increasing performance.


  \textit{(Section \ref{peach:optm:quasinewton}, p.~\pageref{peach:optm:quasinewton})}

    \item \textbf{stochastic}
  \textit{(Section \ref{peach:optm:stochastic}, p.~\pageref{peach:optm:stochastic})}

  \end{itemize}
\item \textbf{pso}: 
Basic Particle Swarm Optimization (PSO)


  \textit{(Section \ref{peach:pso}, p.~\pageref{peach:pso})}

  \begin{itemize}
\setlength{\parskip}{0ex}
    \item \textbf{acc}: 
Functions to update the velocity (ie, accelerate) of the particles in a swarm.


  \textit{(Section \ref{peach:pso:acc}, p.~\pageref{peach:pso:acc})}

    \item \textbf{base}: 
This package implements the simple continuous version of the particle swarm
optimizer. In this implementation, it is possible to specify, besides the
objective function and the first estimates, the ranges of search, which will
influence the max velocity of the particles, and the population size. Other
parameters are available too, please refer to the rest of this documentation for
further details.


  \textit{(Section \ref{peach:pso:base}, p.~\pageref{peach:pso:base})}

  \end{itemize}
\item \textbf{sa}: 
This package implements optimization by simulated annealing. Consult:


  \textit{(Section \ref{peach:sa}, p.~\pageref{peach:sa})}

  \begin{itemize}
\setlength{\parskip}{0ex}
    \item \textbf{base}: 
This package implements two versions of simulated annealing optimization. One
works with numeric data, and the other with a codified bit string. This last
method can be used in discrete optimization problems.


  \textit{(Section \ref{peach:sa:base}, p.~\pageref{peach:sa:base})}

    \item \textbf{neighbor}: 
This module implements a general class to compute neighbors for continuous and
binary simulated annealing algorithms. The continuous neighbor functions return
an array with a neighbor of a given estimate; the binary neighbor functions
return a \texttt{bitarray} object.


  \textit{(Section \ref{peach:sa:neighbor}, p.~\pageref{peach:sa:neighbor})}

  \end{itemize}
\end{itemize}

    \index{peach \textit{(package)}|)}

%
% API Documentation for Peach - Computational Intelligence for Python
% Package peach.fuzzy
%
% Generated by epydoc 3.0.1
% [Sun Jul 31 17:00:39 2011]
%

%%%%%%%%%%%%%%%%%%%%%%%%%%%%%%%%%%%%%%%%%%%%%%%%%%%%%%%%%%%%%%%%%%%%%%%%%%%
%%                          Module Description                           %%
%%%%%%%%%%%%%%%%%%%%%%%%%%%%%%%%%%%%%%%%%%%%%%%%%%%%%%%%%%%%%%%%%%%%%%%%%%%

    \index{peach \textit{(package)}!peach.fuzzy \textit{(package)}|(}
\section{Package peach.fuzzy}

    \label{peach:fuzzy}

This package implements fuzzy logic. Consult:
%
\begin{quote}
%
\begin{description}
\item[{base}] \leavevmode 
Basic definitions, classes and operations in fuzzy logic;

\item[{mf}] \leavevmode 
Membership functions;

\item[{defuzzy}] \leavevmode 
Defuzzification methods;

\item[{control}] \leavevmode 
Fuzzy controllers (FIS - Fuzzy Inference Systems), for Mamdani- and
Sugeno-type controllers and others;

\item[{cmeans}] \leavevmode 
Fuzzy C-Means clustering algorithm;

\end{description}

\end{quote}

%%%%%%%%%%%%%%%%%%%%%%%%%%%%%%%%%%%%%%%%%%%%%%%%%%%%%%%%%%%%%%%%%%%%%%%%%%%
%%                                Modules                                %%
%%%%%%%%%%%%%%%%%%%%%%%%%%%%%%%%%%%%%%%%%%%%%%%%%%%%%%%%%%%%%%%%%%%%%%%%%%%

\subsection{Modules}

\begin{itemize}
\setlength{\parskip}{0ex}
\item \textbf{base}: 
This package implements basic definitions for fuzzy logic


  \textit{(Section \ref{peach:fuzzy:base}, p.~\pageref{peach:fuzzy:base})}

\item \textbf{cmeans}: 
Fuzzy C-Means


  \textit{(Section \ref{peach:fuzzy:cmeans}, p.~\pageref{peach:fuzzy:cmeans})}

\item \textbf{control}: 
This package implements fuzzy controllers, of fuzzy inference systems.


  \textit{(Section \ref{peach:fuzzy:control}, p.~\pageref{peach:fuzzy:control})}

\item \textbf{defuzzy}: 
This package implements defuzzification methods for use with fuzzy controllers.


  \textit{(Section \ref{peach:fuzzy:defuzzy}, p.~\pageref{peach:fuzzy:defuzzy})}

\item \textbf{mf}: 
Membership functions


  \textit{(Section \ref{peach:fuzzy:mf}, p.~\pageref{peach:fuzzy:mf})}

\item \textbf{norms}: 
This package implements operations of fuzzy logic.


  \textit{(Section \ref{peach:fuzzy:norms}, p.~\pageref{peach:fuzzy:norms})}

\end{itemize}

    \index{peach \textit{(package)}!peach.fuzzy \textit{(package)}|)}

%
% API Documentation for Peach - Computational Intelligence for Python
% Module peach.fuzzy.base
%
% Generated by epydoc 3.0.1
% [Thu Jul 28 16:37:45 2011]
%

%%%%%%%%%%%%%%%%%%%%%%%%%%%%%%%%%%%%%%%%%%%%%%%%%%%%%%%%%%%%%%%%%%%%%%%%%%%
%%                          Module Description                           %%
%%%%%%%%%%%%%%%%%%%%%%%%%%%%%%%%%%%%%%%%%%%%%%%%%%%%%%%%%%%%%%%%%%%%%%%%%%%

    \index{peach \textit{(package)}!peach.fuzzy \textit{(package)}!peach.fuzzy.base \textit{(module)}|(}
\section{Module peach.fuzzy.base}

    \label{peach:fuzzy:base}

This package implements basic definitions for fuzzy logic

%%%%%%%%%%%%%%%%%%%%%%%%%%%%%%%%%%%%%%%%%%%%%%%%%%%%%%%%%%%%%%%%%%%%%%%%%%%
%%                               Variables                               %%
%%%%%%%%%%%%%%%%%%%%%%%%%%%%%%%%%%%%%%%%%%%%%%%%%%%%%%%%%%%%%%%%%%%%%%%%%%%

  \subsection{Variables}

    \vspace{-1cm}
\hspace{\varindent}\begin{longtable}{|p{\varnamewidth}|p{\vardescrwidth}|l}
\cline{1-2}
\cline{1-2} \centering \textbf{Name} & \centering \textbf{Description}& \\
\cline{1-2}
\endhead\cline{1-2}\multicolumn{3}{r}{\small\textit{continued on next page}}\\\endfoot\cline{1-2}
\endlastfoot\raggedright \_\-\_\-d\-o\-c\-\_\-\_\- & \raggedright \textbf{Value:} 
{\tt \texttt{...}}&\\
\cline{1-2}
\raggedright \_\-\_\-p\-a\-c\-k\-a\-g\-e\-\_\-\_\- & \raggedright \textbf{Value:} 
{\tt \texttt{'}\texttt{peach.fuzzy}\texttt{'}}&\\
\cline{1-2}
\end{longtable}


%%%%%%%%%%%%%%%%%%%%%%%%%%%%%%%%%%%%%%%%%%%%%%%%%%%%%%%%%%%%%%%%%%%%%%%%%%%
%%                           Class Description                           %%
%%%%%%%%%%%%%%%%%%%%%%%%%%%%%%%%%%%%%%%%%%%%%%%%%%%%%%%%%%%%%%%%%%%%%%%%%%%

    \index{peach \textit{(package)}!peach.fuzzy \textit{(package)}!peach.fuzzy.base \textit{(module)}!peach.fuzzy.base.FuzzySet \textit{(class)}|(}
\subsection{Class FuzzySet}

    \label{peach:fuzzy:base:FuzzySet}
\begin{tabular}{cccccccc}
% Line for object, linespec=[False, False]
\multicolumn{2}{r}{\settowidth{\BCL}{object}\multirow{2}{\BCL}{object}}
&&
&&
  \\\cline{3-3}
  &&\multicolumn{1}{c|}{}
&&
&&
  \\
% Line for numpy.ndarray, linespec=[False]
\multicolumn{4}{r}{\settowidth{\BCL}{numpy.ndarray}\multirow{2}{\BCL}{numpy.ndarray}}
&&
  \\\cline{5-5}
  &&&&\multicolumn{1}{c|}{}
&&
  \\
&&&&\multicolumn{2}{l}{\textbf{peach.fuzzy.base.FuzzySet}}
\end{tabular}


Array containing fuzzy values for a set.

This class defines the behavior of a fuzzy set. It is an array of values in
the range from 0 to 1, and the basic operations of the logic -{}- and (using
the \texttt{\&} operator); or (using the \texttt{|} operator); not (using \texttt{\textasciitilde{}}
operator) -{}- can be defined according to a set of norms. The norms can be
redefined using the appropriated methods.

To create a FuzzySet, instantiate this class with a sequence as argument,
for example:
%
\begin{quote}{\ttfamily \raggedright \noindent
fuzzy\_set~=~FuzzySet({[}~0.,~0.25,~0.5,~0.75,~1.0~{]})
}
\end{quote}

%%%%%%%%%%%%%%%%%%%%%%%%%%%%%%%%%%%%%%%%%%%%%%%%%%%%%%%%%%%%%%%%%%%%%%%%%%%
%%                                Methods                                %%
%%%%%%%%%%%%%%%%%%%%%%%%%%%%%%%%%%%%%%%%%%%%%%%%%%%%%%%%%%%%%%%%%%%%%%%%%%%

  \subsubsection{Methods}

    \label{peach:fuzzy:norms:ZadehAnd}
    \index{peach \textit{(package)}!peach.fuzzy \textit{(package)}!peach.fuzzy.norms \textit{(module)}!peach.fuzzy.norms.ZadehAnd \textit{(function)}}

    \vspace{0.5ex}

\hspace{.8\funcindent}\begin{boxedminipage}{\funcwidth}

    \raggedright \textbf{\_\_AND\_\_}(\textit{x}, \textit{y})

    \vspace{-1.5ex}

    \rule{\textwidth}{0.5\fboxrule}
\setlength{\parskip}{2ex}

And operation as defined by Lofti Zadeh.

And operation is the minimum of the two values.
\setlength{\parskip}{1ex}
      \textbf{Return Value}
    \vspace{-1ex}

      \begin{quote}

The result of the and operation.
      \end{quote}

    \end{boxedminipage}

    \label{peach:fuzzy:norms:ZadehOr}
    \index{peach \textit{(package)}!peach.fuzzy \textit{(package)}!peach.fuzzy.norms \textit{(module)}!peach.fuzzy.norms.ZadehOr \textit{(function)}}

    \vspace{0.5ex}

\hspace{.8\funcindent}\begin{boxedminipage}{\funcwidth}

    \raggedright \textbf{\_\_OR\_\_}(\textit{x}, \textit{y})

    \vspace{-1.5ex}

    \rule{\textwidth}{0.5\fboxrule}
\setlength{\parskip}{2ex}

Or operation as defined by Lofti Zadeh.

Or operation is the maximum of the two values.
\setlength{\parskip}{1ex}
      \textbf{Return Value}
    \vspace{-1ex}

      \begin{quote}

The result of the or operation.
      \end{quote}

    \end{boxedminipage}

    \label{peach:fuzzy:norms:ZadehNot}
    \index{peach \textit{(package)}!peach.fuzzy \textit{(package)}!peach.fuzzy.norms \textit{(module)}!peach.fuzzy.norms.ZadehNot \textit{(function)}}

    \vspace{0.5ex}

\hspace{.8\funcindent}\begin{boxedminipage}{\funcwidth}

    \raggedright \textbf{\_\_NOT\_\_}(\textit{x})

    \vspace{-1.5ex}

    \rule{\textwidth}{0.5\fboxrule}
\setlength{\parskip}{2ex}

Not operation as defined by Lofti Zadeh.

Not operation is the complement to 1 of the given value, that is, \texttt{1 - x}.
\setlength{\parskip}{1ex}
      \textbf{Return Value}
    \vspace{-1ex}

      \begin{quote}

The result of the not operation.
      \end{quote}

    \end{boxedminipage}

    \vspace{0.5ex}

\hspace{.8\funcindent}\begin{boxedminipage}{\funcwidth}

    \raggedright \textbf{\_\_new\_\_}(\textit{cls}, \textit{data})

    \vspace{-1.5ex}

    \rule{\textwidth}{0.5\fboxrule}
\setlength{\parskip}{2ex}

Allocates space for the array.

A fuzzy set is derived from the basic NumPy array, so the appropriate
functions and methods are called to allocate the space. In theory, the
values for a fuzzy set should be in the range \texttt{0.0 <= x <= 1.0}, but
to increase efficiency, no verification is made.
\setlength{\parskip}{1ex}
      \textbf{Return Value}
    \vspace{-1ex}

      \begin{quote}

A new array object with the fuzzy set definitions.
      {\it (type=a new object with type S, a subtype of T)}

      \end{quote}

      Overrides: object.\_\_new\_\_

    \end{boxedminipage}

    \vspace{0.5ex}

\hspace{.8\funcindent}\begin{boxedminipage}{\funcwidth}

    \raggedright \textbf{\_\_init\_\_}(\textit{self}, \textit{data}={\tt \texttt{[}\texttt{]}})

    \vspace{-1.5ex}

    \rule{\textwidth}{0.5\fboxrule}
\setlength{\parskip}{2ex}

Initializes the object.

Operations are defaulted to Zadeh norms \texttt{(max, min, 1-x)}
\setlength{\parskip}{1ex}
      Overrides: object.\_\_init\_\_

    \end{boxedminipage}

    \vspace{0.5ex}

\hspace{.8\funcindent}\begin{boxedminipage}{\funcwidth}

    \raggedright \textbf{\_\_and\_\_}(\textit{self}, \textit{a})

    \vspace{-1.5ex}

    \rule{\textwidth}{0.5\fboxrule}
\setlength{\parskip}{2ex}

Fuzzy and (\texttt{\&}) operation.
\setlength{\parskip}{1ex}
      Overrides: numpy.ndarray.\_\_and\_\_

    \end{boxedminipage}

    \vspace{0.5ex}

\hspace{.8\funcindent}\begin{boxedminipage}{\funcwidth}

    \raggedright \textbf{\_\_or\_\_}(\textit{self}, \textit{a})

    \vspace{-1.5ex}

    \rule{\textwidth}{0.5\fboxrule}
\setlength{\parskip}{2ex}

Fuzzy or (\texttt{|}) operation.
\setlength{\parskip}{1ex}
      Overrides: numpy.ndarray.\_\_or\_\_

    \end{boxedminipage}

    \vspace{0.5ex}

\hspace{.8\funcindent}\begin{boxedminipage}{\funcwidth}

    \raggedright \textbf{\_\_invert\_\_}(\textit{self})

    \vspace{-1.5ex}

    \rule{\textwidth}{0.5\fboxrule}
\setlength{\parskip}{2ex}

Fuzzy not (\texttt{\textasciitilde{}}) operation.
\setlength{\parskip}{1ex}
      Overrides: numpy.ndarray.\_\_invert\_\_

    \end{boxedminipage}

    \label{peach:fuzzy:base:FuzzySet:set_norm}
    \index{peach \textit{(package)}!peach.fuzzy \textit{(package)}!peach.fuzzy.base \textit{(module)}!peach.fuzzy.base.FuzzySet \textit{(class)}!peach.fuzzy.base.FuzzySet.set\_norm \textit{(class method)}}

    \vspace{0.5ex}

\hspace{.8\funcindent}\begin{boxedminipage}{\funcwidth}

    \raggedright \textbf{set\_norm}(\textit{cls}, \textit{f})

    \vspace{-1.5ex}

    \rule{\textwidth}{0.5\fboxrule}
\setlength{\parskip}{2ex}

Selects a t-norm (and operation)

Use this method to change the behaviour of the and operation.
\setlength{\parskip}{1ex}
      \textbf{Parameters}
      \vspace{-1ex}

      \begin{quote}
        \begin{Ventry}{x}

          \item[f]


A function of two parameters which must return the \texttt{and} of the
values.
        \end{Ventry}

      \end{quote}

    \end{boxedminipage}

    \label{peach:fuzzy:base:FuzzySet:set_conorm}
    \index{peach \textit{(package)}!peach.fuzzy \textit{(package)}!peach.fuzzy.base \textit{(module)}!peach.fuzzy.base.FuzzySet \textit{(class)}!peach.fuzzy.base.FuzzySet.set\_conorm \textit{(class method)}}

    \vspace{0.5ex}

\hspace{.8\funcindent}\begin{boxedminipage}{\funcwidth}

    \raggedright \textbf{set\_conorm}(\textit{cls}, \textit{f})

    \vspace{-1.5ex}

    \rule{\textwidth}{0.5\fboxrule}
\setlength{\parskip}{2ex}

Selects a t-conorm (or operation)

Use this method to change the behaviour of the or operation.
\setlength{\parskip}{1ex}
      \textbf{Parameters}
      \vspace{-1ex}

      \begin{quote}
        \begin{Ventry}{x}

          \item[f]


A function of two parameters which must return the \texttt{or} of the
values.
        \end{Ventry}

      \end{quote}

    \end{boxedminipage}

    \label{peach:fuzzy:base:FuzzySet:set_negation}
    \index{peach \textit{(package)}!peach.fuzzy \textit{(package)}!peach.fuzzy.base \textit{(module)}!peach.fuzzy.base.FuzzySet \textit{(class)}!peach.fuzzy.base.FuzzySet.set\_negation \textit{(class method)}}

    \vspace{0.5ex}

\hspace{.8\funcindent}\begin{boxedminipage}{\funcwidth}

    \raggedright \textbf{set\_negation}(\textit{cls}, \textit{f})

    \vspace{-1.5ex}

    \rule{\textwidth}{0.5\fboxrule}
\setlength{\parskip}{2ex}

Selects a negation (not operation)

Use this method to change the behaviour of the not operation.
\setlength{\parskip}{1ex}
      \textbf{Parameters}
      \vspace{-1ex}

      \begin{quote}
        \begin{Ventry}{x}

          \item[f]


A function of one parameter which must return the \texttt{not} of the
value.
        \end{Ventry}

      \end{quote}

    \end{boxedminipage}


\large{\textbf{\textit{Inherited from numpy.ndarray}}}

\begin{quote}
\_\_abs\_\_(), \_\_add\_\_(), \_\_array\_\_(), \_\_array\_wrap\_\_(), \_\_contains\_\_(), \_\_copy\_\_(), \_\_deepcopy\_\_(), \_\_delitem\_\_(), \_\_delslice\_\_(), \_\_div\_\_(), \_\_divmod\_\_(), \_\_eq\_\_(), \_\_float\_\_(), \_\_floordiv\_\_(), \_\_ge\_\_(), \_\_getitem\_\_(), \_\_getslice\_\_(), \_\_gt\_\_(), \_\_hex\_\_(), \_\_iadd\_\_(), \_\_iand\_\_(), \_\_idiv\_\_(), \_\_ifloordiv\_\_(), \_\_ilshift\_\_(), \_\_imod\_\_(), \_\_imul\_\_(), \_\_index\_\_(), \_\_int\_\_(), \_\_ior\_\_(), \_\_ipow\_\_(), \_\_irshift\_\_(), \_\_isub\_\_(), \_\_iter\_\_(), \_\_itruediv\_\_(), \_\_ixor\_\_(), \_\_le\_\_(), \_\_len\_\_(), \_\_long\_\_(), \_\_lshift\_\_(), \_\_lt\_\_(), \_\_mod\_\_(), \_\_mul\_\_(), \_\_ne\_\_(), \_\_neg\_\_(), \_\_nonzero\_\_(), \_\_oct\_\_(), \_\_pos\_\_(), \_\_pow\_\_(), \_\_radd\_\_(), \_\_rand\_\_(), \_\_rdiv\_\_(), \_\_rdivmod\_\_(), \_\_reduce\_\_(), \_\_repr\_\_(), \_\_rfloordiv\_\_(), \_\_rlshift\_\_(), \_\_rmod\_\_(), \_\_rmul\_\_(), \_\_ror\_\_(), \_\_rpow\_\_(), \_\_rrshift\_\_(), \_\_rshift\_\_(), \_\_rsub\_\_(), \_\_rtruediv\_\_(), \_\_rxor\_\_(), \_\_setitem\_\_(), \_\_setslice\_\_(), \_\_setstate\_\_(), \_\_str\_\_(), \_\_sub\_\_(), \_\_truediv\_\_(), \_\_xor\_\_(), all(), any(), argmax(), argmin(), argsort(), astype(), byteswap(), choose(), clip(), compress(), conj(), conjugate(), copy(), cumprod(), cumsum(), diagonal(), dump(), dumps(), fill(), flatten(), getfield(), item(), itemset(), max(), mean(), min(), newbyteorder(), nonzero(), prod(), ptp(), put(), ravel(), repeat(), reshape(), resize(), round(), searchsorted(), setfield(), setflags(), sort(), squeeze(), std(), sum(), swapaxes(), take(), tofile(), tolist(), tostring(), trace(), transpose(), var(), view()
\end{quote}

\large{\textbf{\textit{Inherited from object}}}

\begin{quote}
\_\_delattr\_\_(), \_\_format\_\_(), \_\_getattribute\_\_(), \_\_hash\_\_(), \_\_reduce\_ex\_\_(), \_\_setattr\_\_(), \_\_sizeof\_\_(), \_\_subclasshook\_\_()
\end{quote}

%%%%%%%%%%%%%%%%%%%%%%%%%%%%%%%%%%%%%%%%%%%%%%%%%%%%%%%%%%%%%%%%%%%%%%%%%%%
%%                              Properties                               %%
%%%%%%%%%%%%%%%%%%%%%%%%%%%%%%%%%%%%%%%%%%%%%%%%%%%%%%%%%%%%%%%%%%%%%%%%%%%

  \subsubsection{Properties}

    \vspace{-1cm}
\hspace{\varindent}\begin{longtable}{|p{\varnamewidth}|p{\vardescrwidth}|l}
\cline{1-2}
\cline{1-2} \centering \textbf{Name} & \centering \textbf{Description}& \\
\cline{1-2}
\endhead\cline{1-2}\multicolumn{3}{r}{\small\textit{continued on next page}}\\\endfoot\cline{1-2}
\endlastfoot\multicolumn{2}{|l|}{\textit{Inherited from numpy.ndarray}}\\
\multicolumn{2}{|p{\varwidth}|}{\raggedright T, \_\_array\_finalize\_\_, \_\_array\_interface\_\_, \_\_array\_priority\_\_, \_\_array\_struct\_\_, base, ctypes, data, dtype, flags, flat, imag, itemsize, nbytes, ndim, real, shape, size, strides}\\
\cline{1-2}
\multicolumn{2}{|l|}{\textit{Inherited from object}}\\
\multicolumn{2}{|p{\varwidth}|}{\raggedright \_\_class\_\_}\\
\cline{1-2}
\end{longtable}

    \index{peach \textit{(package)}!peach.fuzzy \textit{(package)}!peach.fuzzy.base \textit{(module)}!peach.fuzzy.base.FuzzySet \textit{(class)}|)}
    \index{peach \textit{(package)}!peach.fuzzy \textit{(package)}!peach.fuzzy.base \textit{(module)}|)}

%
% API Documentation for Peach - Computational Intelligence for Python
% Module peach.fuzzy.cmeans
%
% Generated by epydoc 3.0.1
% [Mon Jan 24 15:39:50 2011]
%

%%%%%%%%%%%%%%%%%%%%%%%%%%%%%%%%%%%%%%%%%%%%%%%%%%%%%%%%%%%%%%%%%%%%%%%%%%%
%%                          Module Description                           %%
%%%%%%%%%%%%%%%%%%%%%%%%%%%%%%%%%%%%%%%%%%%%%%%%%%%%%%%%%%%%%%%%%%%%%%%%%%%

    \index{peach \textit{(package)}!peach.fuzzy \textit{(package)}!peach.fuzzy.cmeans \textit{(module)}|(}
\section{Module peach.fuzzy.cmeans}

    \label{peach:fuzzy:cmeans}

Fuzzy C-Means

Fuzzy C-Means is a clustering algorithm based no fuzzy logic.

This package implements the fuzzy c-means algorithm for clustering and
classification. This algorithm is very simple, yet very efficient. From a
training set and an initial condition which gives the membership values of each
example in the training set to the clusters, it converges very fastly to crisper
sets.

The initial conditions, ie, the starting membership, must follow some rules.
Please, refer to any bibliography about the subject to see why. Those rules are:
no example might have membership 1 in every class, and the sum of the membership
of every component must be equal to 1. This means that the initial condition is
a fuzzy partition of the universe.

%%%%%%%%%%%%%%%%%%%%%%%%%%%%%%%%%%%%%%%%%%%%%%%%%%%%%%%%%%%%%%%%%%%%%%%%%%%
%%                               Variables                               %%
%%%%%%%%%%%%%%%%%%%%%%%%%%%%%%%%%%%%%%%%%%%%%%%%%%%%%%%%%%%%%%%%%%%%%%%%%%%

  \subsection{Variables}

    \vspace{-1cm}
\hspace{\varindent}\begin{longtable}{|p{\varnamewidth}|p{\vardescrwidth}|l}
\cline{1-2}
\cline{1-2} \centering \textbf{Name} & \centering \textbf{Description}& \\
\cline{1-2}
\endhead\cline{1-2}\multicolumn{3}{r}{\small\textit{continued on next page}}\\\endfoot\cline{1-2}
\endlastfoot\raggedright \_\-\_\-d\-o\-c\-\_\-\_\- & \raggedright \textbf{Value:} 
{\tt \texttt{...}}&\\
\cline{1-2}
\raggedright \_\-\_\-p\-a\-c\-k\-a\-g\-e\-\_\-\_\- & \raggedright \textbf{Value:} 
{\tt \texttt{'}\texttt{peach.fuzzy}\texttt{'}}&\\
\cline{1-2}
\end{longtable}


%%%%%%%%%%%%%%%%%%%%%%%%%%%%%%%%%%%%%%%%%%%%%%%%%%%%%%%%%%%%%%%%%%%%%%%%%%%
%%                           Class Description                           %%
%%%%%%%%%%%%%%%%%%%%%%%%%%%%%%%%%%%%%%%%%%%%%%%%%%%%%%%%%%%%%%%%%%%%%%%%%%%

    \index{peach \textit{(package)}!peach.fuzzy \textit{(package)}!peach.fuzzy.cmeans \textit{(module)}!peach.fuzzy.cmeans.FuzzyCMeans \textit{(class)}|(}
\subsection{Class FuzzyCMeans}

    \label{peach:fuzzy:cmeans:FuzzyCMeans}
\begin{tabular}{cccccc}
% Line for object, linespec=[False]
\multicolumn{2}{r}{\settowidth{\BCL}{object}\multirow{2}{\BCL}{object}}
&&
  \\\cline{3-3}
  &&\multicolumn{1}{c|}{}
&&
  \\
&&\multicolumn{2}{l}{\textbf{peach.fuzzy.cmeans.FuzzyCMeans}}
\end{tabular}


Fuzzy C-Means convergence.

Use this class to instantiate a fuzzy c-means object. The object must be
given a training set and initial conditions. The training set is a list or
an array of N-dimensional vectors; the initial conditions are a list of the
initial membership values for every vector in the training set -{}- thus, the
length of both lists must be the same. The number of columns in the initial
conditions must be the same number of classes. That is, if you are, for
example, classifying in \texttt{C} classes, then the initial conditions must have
\texttt{C} columns.

There are restrictions in the initial conditions: first, no column can be
all zeros or all ones -{}- if that happened, then the class described by this
column is unnecessary; second, the sum of the memberships of every example
must be one -{}- that is, the sum of the membership in every column in each
line must be one. This means that the initial condition is a perfect
partition of \texttt{C} subsets.

Notice, however, that \emph{no checking} is done. If your algorithm seems to be
behaving strangely, try to check these conditions.

%%%%%%%%%%%%%%%%%%%%%%%%%%%%%%%%%%%%%%%%%%%%%%%%%%%%%%%%%%%%%%%%%%%%%%%%%%%
%%                                Methods                                %%
%%%%%%%%%%%%%%%%%%%%%%%%%%%%%%%%%%%%%%%%%%%%%%%%%%%%%%%%%%%%%%%%%%%%%%%%%%%

  \subsubsection{Methods}

    \vspace{0.5ex}

\hspace{.8\funcindent}\begin{boxedminipage}{\funcwidth}

    \raggedright \textbf{\_\_init\_\_}(\textit{self}, \textit{training\_set}, \textit{initial\_conditions}, \textit{m}={\tt 2.0})

    \vspace{-1.5ex}

    \rule{\textwidth}{0.5\fboxrule}
\setlength{\parskip}{2ex}

Initializes the algorithm.
\setlength{\parskip}{1ex}
      \textbf{Parameters}
      \vspace{-1ex}

      \begin{quote}
        \begin{Ventry}{xxxxxxxxxxxxxxxxxx}

          \item[training\_set]


A list or array of vectors containing the data to be classified.
Each of the vectors in this list \emph{must} have the same dimension, or
the algorithm won't behave correctly. Notice that each vector can be
given as a tuple -{}- internally, everything is converted to arrays.
          \item[initial\_conditions]


A list or array of vectors containing the initial membership values
associated to each example in the training set. Each column of this
array contains the membership assigned to the corresponding class
for that vector. Notice that each vector can be given as a tuple -{}-
internally, everything is converted to arrays.
          \item[m]


This is the aggregation value. The bigger it is, the smoother will
be the classification. Please, consult the bibliography about the
subject. \texttt{m} must be bigger than 1. Its default value is 2
        \end{Ventry}

      \end{quote}

      Overrides: object.\_\_init\_\_

    \end{boxedminipage}

    \label{peach:fuzzy:cmeans:FuzzyCMeans:centers}
    \index{peach \textit{(package)}!peach.fuzzy \textit{(package)}!peach.fuzzy.cmeans \textit{(module)}!peach.fuzzy.cmeans.FuzzyCMeans \textit{(class)}!peach.fuzzy.cmeans.FuzzyCMeans.centers \textit{(method)}}

    \vspace{0.5ex}

\hspace{.8\funcindent}\begin{boxedminipage}{\funcwidth}

    \raggedright \textbf{centers}(\textit{self})

    \vspace{-1.5ex}

    \rule{\textwidth}{0.5\fboxrule}
\setlength{\parskip}{2ex}

Given the present state of the algorithm, recalculates the centers, that
is, the position of the vectors representing each of the classes. Notice
that this method modifies the state of the algorithm if any change was
made to any parameter. This method receives no arguments and will seldom
be used externally. It can be useful if you want to step over the
algorithm. \emph{This method has a colateral effect!} If you use it, the
\texttt{c} property (see above) will be modified.
\setlength{\parskip}{1ex}
      \textbf{Return Value}
    \vspace{-1ex}

      \begin{quote}

A vector containing, in each line, the position of the centers of the
algorithm.
      \end{quote}

    \end{boxedminipage}

    \label{peach:fuzzy:cmeans:FuzzyCMeans:membership}
    \index{peach \textit{(package)}!peach.fuzzy \textit{(package)}!peach.fuzzy.cmeans \textit{(module)}!peach.fuzzy.cmeans.FuzzyCMeans \textit{(class)}!peach.fuzzy.cmeans.FuzzyCMeans.membership \textit{(method)}}

    \vspace{0.5ex}

\hspace{.8\funcindent}\begin{boxedminipage}{\funcwidth}

    \raggedright \textbf{membership}(\textit{self})

    \vspace{-1.5ex}

    \rule{\textwidth}{0.5\fboxrule}
\setlength{\parskip}{2ex}

Given the present state of the algorithm, recalculates the membership of
each example on each class. That is, it modifies the initial conditions
to represent an evolved state of the algorithm. Notice that this method
modifies the state of the algorithm if any change was made to any
parameter.
\setlength{\parskip}{1ex}
      \textbf{Return Value}
    \vspace{-1ex}

      \begin{quote}

A vector containing, in each line, the membership of the corresponding
example in each class.
      \end{quote}

    \end{boxedminipage}

    \label{peach:fuzzy:cmeans:FuzzyCMeans:step}
    \index{peach \textit{(package)}!peach.fuzzy \textit{(package)}!peach.fuzzy.cmeans \textit{(module)}!peach.fuzzy.cmeans.FuzzyCMeans \textit{(class)}!peach.fuzzy.cmeans.FuzzyCMeans.step \textit{(method)}}

    \vspace{0.5ex}

\hspace{.8\funcindent}\begin{boxedminipage}{\funcwidth}

    \raggedright \textbf{step}(\textit{self})

    \vspace{-1.5ex}

    \rule{\textwidth}{0.5\fboxrule}
\setlength{\parskip}{2ex}

This method runs one step of the algorithm. It might be useful to track
the changes in the parameters.
\setlength{\parskip}{1ex}
      \textbf{Return Value}
    \vspace{-1ex}

      \begin{quote}

The norm of the change in the membership values of the examples. It
can be used to track convergence and as an estimate of the error.
      \end{quote}

    \end{boxedminipage}

    \label{peach:fuzzy:cmeans:FuzzyCMeans:__call__}
    \index{peach \textit{(package)}!peach.fuzzy \textit{(package)}!peach.fuzzy.cmeans \textit{(module)}!peach.fuzzy.cmeans.FuzzyCMeans \textit{(class)}!peach.fuzzy.cmeans.FuzzyCMeans.\_\_call\_\_ \textit{(method)}}

    \vspace{0.5ex}

\hspace{.8\funcindent}\begin{boxedminipage}{\funcwidth}

    \raggedright \textbf{\_\_call\_\_}(\textit{self}, \textit{emax}={\tt 1e-10}, \textit{imax}={\tt 20})

    \vspace{-1.5ex}

    \rule{\textwidth}{0.5\fboxrule}
\setlength{\parskip}{2ex}

The \texttt{\_\_call\_\_} interface is used to run the algorithm until
convergence is found.
\setlength{\parskip}{1ex}
      \textbf{Parameters}
      \vspace{-1ex}

      \begin{quote}
        \begin{Ventry}{xxxx}

          \item[emax]


Specifies the maximum error admitted in the execution of the
algorithm. It defaults to 1.e-10. The error is tracked according to
the norm returned by the \texttt{step()} method.
          \item[imax]


Specifies the maximum number of iterations admitted in the execution
of the algorithm. It defaults to 20.
        \end{Ventry}

      \end{quote}

      \textbf{Return Value}
    \vspace{-1ex}

      \begin{quote}

An array containing, at each line, the vectors representing the
centers of the clustered regions.
      \end{quote}

    \end{boxedminipage}


\large{\textbf{\textit{Inherited from object}}}

\begin{quote}
\_\_delattr\_\_(), \_\_format\_\_(), \_\_getattribute\_\_(), \_\_hash\_\_(), \_\_new\_\_(), \_\_reduce\_\_(), \_\_reduce\_ex\_\_(), \_\_repr\_\_(), \_\_setattr\_\_(), \_\_sizeof\_\_(), \_\_str\_\_(), \_\_subclasshook\_\_()
\end{quote}

%%%%%%%%%%%%%%%%%%%%%%%%%%%%%%%%%%%%%%%%%%%%%%%%%%%%%%%%%%%%%%%%%%%%%%%%%%%
%%                              Properties                               %%
%%%%%%%%%%%%%%%%%%%%%%%%%%%%%%%%%%%%%%%%%%%%%%%%%%%%%%%%%%%%%%%%%%%%%%%%%%%

  \subsubsection{Properties}

    \vspace{-1cm}
\hspace{\varindent}\begin{longtable}{|p{\varnamewidth}|p{\vardescrwidth}|l}
\cline{1-2}
\cline{1-2} \centering \textbf{Name} & \centering \textbf{Description}& \\
\cline{1-2}
\endhead\cline{1-2}\multicolumn{3}{r}{\small\textit{continued on next page}}\\\endfoot\cline{1-2}
\endlastfoot\raggedright c\- & &\\
\cline{1-2}
\raggedright m\-u\- & &\\
\cline{1-2}
\raggedright x\- & &\\
\cline{1-2}
\multicolumn{2}{|l|}{\textit{Inherited from object}}\\
\multicolumn{2}{|p{\varwidth}|}{\raggedright \_\_class\_\_}\\
\cline{1-2}
\end{longtable}


%%%%%%%%%%%%%%%%%%%%%%%%%%%%%%%%%%%%%%%%%%%%%%%%%%%%%%%%%%%%%%%%%%%%%%%%%%%
%%                          Instance Variables                           %%
%%%%%%%%%%%%%%%%%%%%%%%%%%%%%%%%%%%%%%%%%%%%%%%%%%%%%%%%%%%%%%%%%%%%%%%%%%%

  \subsubsection{Instance Variables}

    \vspace{-1cm}
\hspace{\varindent}\begin{longtable}{|p{\varnamewidth}|p{\vardescrwidth}|l}
\cline{1-2}
\cline{1-2} \centering \textbf{Name} & \centering \textbf{Description}& \\
\cline{1-2}
\endhead\cline{1-2}\multicolumn{3}{r}{\small\textit{continued on next page}}\\\endfoot\cline{1-2}
\endlastfoot\raggedright m\- & The fuzzyness coefficient. Must be bigger than 1, the closest it is
to 1, the smoother the membership curves will be.&\\
\cline{1-2}
\end{longtable}

    \index{peach \textit{(package)}!peach.fuzzy \textit{(package)}!peach.fuzzy.cmeans \textit{(module)}!peach.fuzzy.cmeans.FuzzyCMeans \textit{(class)}|)}
    \index{peach \textit{(package)}!peach.fuzzy \textit{(package)}!peach.fuzzy.cmeans \textit{(module)}|)}

%
% API Documentation for Peach - Computational Intelligence for Python
% Module peach.fuzzy.control
%
% Generated by epydoc 3.0beta1
% [Mon Dec 21 08:51:35 2009]
%

%%%%%%%%%%%%%%%%%%%%%%%%%%%%%%%%%%%%%%%%%%%%%%%%%%%%%%%%%%%%%%%%%%%%%%%%%%%
%%                          Module Description                           %%
%%%%%%%%%%%%%%%%%%%%%%%%%%%%%%%%%%%%%%%%%%%%%%%%%%%%%%%%%%%%%%%%%%%%%%%%%%%

    \index{peach \textit{(package)}!peach.fuzzy \textit{(package)}!peach.fuzzy.control \textit{(module)}|(}
\section{Module peach.fuzzy.control}

    \label{peach:fuzzy:control}

This package implements fuzzy controllers, of fuzzy inference systems.

There are two types of controllers implemented in this package. The Mamdani
controller is the traditional approach, where input (or controlled) variables
are fuzzified, a set of decision rules determine the outcome in a fuzzified way,
and a defuzzification method is applied to obtain the numerical result.

The Sugeno controller operates in a similar way, but there is no defuzzification
step. Instead, the value of the output (or manipulated) variable is determined
by parametric models, and the final result is determined by a weighted average
based on the decision rules. This type of controller is also known as parametric
controller.

%%%%%%%%%%%%%%%%%%%%%%%%%%%%%%%%%%%%%%%%%%%%%%%%%%%%%%%%%%%%%%%%%%%%%%%%%%%
%%                               Variables                               %%
%%%%%%%%%%%%%%%%%%%%%%%%%%%%%%%%%%%%%%%%%%%%%%%%%%%%%%%%%%%%%%%%%%%%%%%%%%%

  \subsection{Variables}

\begin{longtable}{|p{.30\textwidth}|p{.62\textwidth}|l}
\cline{1-2}
\cline{1-2} \centering \textbf{Name} & \centering \textbf{Description}& \\
\cline{1-2}
\endhead\cline{1-2}\multicolumn{3}{r}{\small\textit{continued on next page}}\\\endfoot\cline{1-2}
\endlastfoot\raggedright \_\-\_\-d\-o\-c\-\_\-\_\- & \raggedright \textbf{Value:} 
{\tt \texttt{...}}&\\
\cline{1-2}
\raggedright M\-A\-M\-D\-A\-N\-I\-\_\-I\-N\-F\-E\-R\-E\-N\-C\-E\- & \raggedright \textbf{Value:} 
{\tt MamdaniImplication, MamdaniAglutination}&\\
\cline{1-2}
\raggedright P\-R\-O\-B\-\_\-I\-N\-F\-E\-R\-E\-N\-C\-E\- & \raggedright \textbf{Value:} 
{\tt ProbabilisticImplication, ProbabilisticAglutination}&\\
\cline{1-2}
\raggedright P\-R\-O\-B\-\_\-N\-O\-R\-M\-S\- & \raggedright \textbf{Value:} 
{\tt ProbabilisticAnd, ProbabilisticOr, ProbabilisticNot}&\\
\cline{1-2}
\raggedright Z\-A\-D\-E\-H\-\_\-N\-O\-R\-M\-S\- & \raggedright \textbf{Value:} 
{\tt ZadehAnd, ZadehOr, ZadehNot}&\\
\cline{1-2}
\raggedright c\-o\-s\- & \raggedright \textbf{Value:} 
{\tt {\textless}ufunc 'cos'{\textgreater}}&\\
\cline{1-2}
\raggedright e\-x\-p\- & \raggedright \textbf{Value:} 
{\tt {\textless}ufunc 'exp'{\textgreater}}&\\
\cline{1-2}
\raggedright p\-i\- & \raggedright \textbf{Value:} 
{\tt 3.14159265359}&\\
\cline{1-2}
\end{longtable}


%%%%%%%%%%%%%%%%%%%%%%%%%%%%%%%%%%%%%%%%%%%%%%%%%%%%%%%%%%%%%%%%%%%%%%%%%%%
%%                           Class Description                           %%
%%%%%%%%%%%%%%%%%%%%%%%%%%%%%%%%%%%%%%%%%%%%%%%%%%%%%%%%%%%%%%%%%%%%%%%%%%%

    \index{peach \textit{(package)}!peach.fuzzy \textit{(package)}!peach.fuzzy.control \textit{(module)}!peach.fuzzy.control.Controller \textit{(class)}|(}
\subsection{Class Controller}

    \label{peach:fuzzy:control:Controller}
\begin{tabular}{cccccc}
% Line for object, linespec=[False]
\multicolumn{2}{r}{\settowidth{\BCL}{object}\multirow{2}{\BCL}{object}}
&&
  \\\cline{3-3}
  &&\multicolumn{1}{c|}{}
&&
  \\
&&\multicolumn{2}{l}{\textbf{peach.fuzzy.control.Controller}}
\end{tabular}

\textbf{Known Subclasses:} peach.fuzzy.control.Mamdani


Basic Mamdani controller

This class implements a standard Mamdani controller. A controller based on
fuzzy logic has a somewhat complex behaviour, so it is not explained here.
There are numerous references that can be consulted.

It is essential to understand the format that decision rules must follow to
obtain correct behaviour of the controller. A rule is a tuple given by:
\begin{quote}{\ttfamily \raggedright \noindent
((mx0,~mx1,~...,~mxn),~my)
}\end{quote}

where \texttt{mx0} is a membership function of the first input variable, \texttt{mx1}
is a membership function of the second input variable and so on; and \texttt{my}
is a membership function or a fuzzy set of the output variable.

Notice that \texttt{mx}'s are \emph{functions} not fuzzy sets! They will be applied to
the values of the input variables given in the function call, so, if they
are anything different from a membership function, an exception will be
raised. Please, consult the examples to see how they must be used.

%%%%%%%%%%%%%%%%%%%%%%%%%%%%%%%%%%%%%%%%%%%%%%%%%%%%%%%%%%%%%%%%%%%%%%%%%%%
%%                                Methods                                %%
%%%%%%%%%%%%%%%%%%%%%%%%%%%%%%%%%%%%%%%%%%%%%%%%%%%%%%%%%%%%%%%%%%%%%%%%%%%

  \subsubsection{Methods}

    \vspace{0.5ex}

    \begin{boxedminipage}{\textwidth}

    \raggedright \textbf{\_\_init\_\_}(\textit{self}, \textit{yrange}, \textit{rules}=\texttt{\texttt{[}\texttt{]}}, \textit{defuzzy}=\texttt{{\textless}function Centroid at 0x885ecdc{\textgreater}}, \textit{norm}=\texttt{{\textless}function ZadehAnd at 0x885eaac{\textgreater}}, \textit{conorm}=\texttt{{\textless}function ZadehOr at 0x885eae4{\textgreater}}, \textit{negation}=\texttt{{\textless}function ZadehNot at 0x885eb1c{\textgreater}}, \textit{imply}=\texttt{{\textless}function MamdaniImplication at 0x885eb54{\textgreater}}, \textit{aglutinate}=\texttt{{\textless}function MamdaniAglutination at 0x885eb8c{\textgreater}})

    \vspace{-1.5ex}

    \rule{\textwidth}{0.5\fboxrule}

Creates and initialize the controller.
    \vspace{1ex}

      \textbf{Parameters}
      \begin{quote}
        \begin{Ventry}{xxxxxxxxxx}

          \item[yrange]


The range of the output variable. This must be given as a set of
points belonging to the interval where the output variable is
defined, not only the start and end points. It is strongly suggested
that the interval is divided in some (eg.: 100) points equally
spaced;
          \item[rules]


The set of decision rules, as defined above. If none is given, an
empty set of rules is assumed;
          \item[defuzzy]


The defuzzification method to be used. If none is given, the
Centroid method is used;
          \item[norm]


The norm (\texttt{and} operation) to be used. Defaults to Zadeh and.
          \item[conorm]


The conorm (\texttt{or} operation) to be used. Defaults to Zadeh or.
          \item[negation]


The negation (\texttt{not} operation) to be used. Defaults to Zadeh not.
          \item[imply]


The implication method to be used. Defaults to Mamdani implication.
          \item[aglutinate]


The aglutination method to be used. Defaults to Mamdani
aglutination.
        \end{Ventry}

      \end{quote}

    \vspace{1ex}

      Overrides: object.\_\_init\_\_

    \end{boxedminipage}

    \label{peach:fuzzy:control:Controller:__gety}
    \index{peach \textit{(package)}!peach.fuzzy \textit{(package)}!peach.fuzzy.control \textit{(module)}!peach.fuzzy.control.Controller \textit{(class)}!peach.fuzzy.control.Controller.\_\_gety \textit{(method)}}

    \vspace{0.5ex}

    \begin{boxedminipage}{\textwidth}

    \raggedright \textbf{\_\_gety}(\textit{self})

    \end{boxedminipage}

    \label{peach:fuzzy:control:Controller:__getrules}
    \index{peach \textit{(package)}!peach.fuzzy \textit{(package)}!peach.fuzzy.control \textit{(module)}!peach.fuzzy.control.Controller \textit{(class)}!peach.fuzzy.control.Controller.\_\_getrules \textit{(method)}}

    \vspace{0.5ex}

    \begin{boxedminipage}{\textwidth}

    \raggedright \textbf{\_\_getrules}(\textit{self})

    \end{boxedminipage}

    \label{peach:fuzzy:control:Controller:set_norm}
    \index{peach \textit{(package)}!peach.fuzzy \textit{(package)}!peach.fuzzy.control \textit{(module)}!peach.fuzzy.control.Controller \textit{(class)}!peach.fuzzy.control.Controller.set\_norm \textit{(method)}}

    \vspace{0.5ex}

    \begin{boxedminipage}{\textwidth}

    \raggedright \textbf{set\_norm}(\textit{self}, \textit{f})

    \vspace{-1.5ex}

    \rule{\textwidth}{0.5\fboxrule}

Sets the norm (\texttt{and}) to be used.

This method must be used to change the behavior of the \texttt{and} operation
of the controller.
    \vspace{1ex}

      \textbf{Parameters}
      \begin{quote}
        \begin{Ventry}{x}

          \item[f]


The function can be any function that takes two numerical values and
return one numerical value, that corresponds to the \texttt{and} result.
        \end{Ventry}

      \end{quote}

    \vspace{1ex}

    \end{boxedminipage}

    \label{peach:fuzzy:control:Controller:set_conorm}
    \index{peach \textit{(package)}!peach.fuzzy \textit{(package)}!peach.fuzzy.control \textit{(module)}!peach.fuzzy.control.Controller \textit{(class)}!peach.fuzzy.control.Controller.set\_conorm \textit{(method)}}

    \vspace{0.5ex}

    \begin{boxedminipage}{\textwidth}

    \raggedright \textbf{set\_conorm}(\textit{self}, \textit{f})

    \vspace{-1.5ex}

    \rule{\textwidth}{0.5\fboxrule}

Sets the conorm (\texttt{or}) to be used.

This method must be used to change the behavior of the \texttt{or} operation
of the controller.
    \vspace{1ex}

      \textbf{Parameters}
      \begin{quote}
        \begin{Ventry}{x}

          \item[f]


The function can be any function that takes two numerical values and
return one numerical value, that corresponds to the \texttt{or} result.
        \end{Ventry}

      \end{quote}

    \vspace{1ex}

    \end{boxedminipage}

    \label{peach:fuzzy:control:Controller:set_negation}
    \index{peach \textit{(package)}!peach.fuzzy \textit{(package)}!peach.fuzzy.control \textit{(module)}!peach.fuzzy.control.Controller \textit{(class)}!peach.fuzzy.control.Controller.set\_negation \textit{(method)}}

    \vspace{0.5ex}

    \begin{boxedminipage}{\textwidth}

    \raggedright \textbf{set\_negation}(\textit{self}, \textit{f})

    \vspace{-1.5ex}

    \rule{\textwidth}{0.5\fboxrule}

Sets the negation (\texttt{not}) to be used.

This method must be used to change the behavior of the \texttt{not} operation
of the controller.
    \vspace{1ex}

      \textbf{Parameters}
      \begin{quote}
        \begin{Ventry}{x}

          \item[f]


The function can be any function that takes one numerical value and
return one numerical value, that corresponds to the \texttt{not} result.
        \end{Ventry}

      \end{quote}

    \vspace{1ex}

    \end{boxedminipage}

    \label{peach:fuzzy:control:Controller:set_implication}
    \index{peach \textit{(package)}!peach.fuzzy \textit{(package)}!peach.fuzzy.control \textit{(module)}!peach.fuzzy.control.Controller \textit{(class)}!peach.fuzzy.control.Controller.set\_implication \textit{(method)}}

    \vspace{0.5ex}

    \begin{boxedminipage}{\textwidth}

    \raggedright \textbf{set\_implication}(\textit{self}, \textit{f})

    \vspace{-1.5ex}

    \rule{\textwidth}{0.5\fboxrule}

Sets the implication to be used.

This method must be used to change the behavior of the implication
operation of the controller.
    \vspace{1ex}

      \textbf{Parameters}
      \begin{quote}
        \begin{Ventry}{x}

          \item[f]


The function can be any function that takes two numerical values and
return one numerical value, that corresponds to the implication
result.
        \end{Ventry}

      \end{quote}

    \vspace{1ex}

    \end{boxedminipage}

    \label{peach:fuzzy:control:Controller:set_aglutination}
    \index{peach \textit{(package)}!peach.fuzzy \textit{(package)}!peach.fuzzy.control \textit{(module)}!peach.fuzzy.control.Controller \textit{(class)}!peach.fuzzy.control.Controller.set\_aglutination \textit{(method)}}

    \vspace{0.5ex}

    \begin{boxedminipage}{\textwidth}

    \raggedright \textbf{set\_aglutination}(\textit{self}, \textit{f})

    \vspace{-1.5ex}

    \rule{\textwidth}{0.5\fboxrule}

Sets the aglutination to be used.

This method must be used to change the behavior of the aglutination
operation of the controller.
    \vspace{1ex}

      \textbf{Parameters}
      \begin{quote}
        \begin{Ventry}{x}

          \item[f]


The function can be any function that takes two numerical values and
return one numerical value, that corresponds to the aglutination
result.
        \end{Ventry}

      \end{quote}

    \vspace{1ex}

    \end{boxedminipage}

    \label{peach:fuzzy:control:Controller:add_rule}
    \index{peach \textit{(package)}!peach.fuzzy \textit{(package)}!peach.fuzzy.control \textit{(module)}!peach.fuzzy.control.Controller \textit{(class)}!peach.fuzzy.control.Controller.add\_rule \textit{(method)}}

    \vspace{0.5ex}

    \begin{boxedminipage}{\textwidth}

    \raggedright \textbf{add\_rule}(\textit{self}, \textit{rule})

    \vspace{-1.5ex}

    \rule{\textwidth}{0.5\fboxrule}

Adds a decision rule to the knowledge base.

It is essential to understand the format that decision rules must follow
to obtain correct behaviour of the controller. A rule is a tuple must
have the following format:
\begin{quote}{\ttfamily \raggedright \noindent
((mx0,~mx1,~...,~mxn),~my)
}\end{quote}

where \texttt{mx0} is a membership function of the first input variable,
\texttt{mx1} is a membership function of the second input variable and so on;
and \texttt{my} is a membership function or a fuzzy set of the output
variable.

Notice that \texttt{mx}'s are \emph{functions} not fuzzy sets! They will be
applied to the values of the input variables given in the function call,
so, if they are anything different from a membership function, an
exception will be raised when the controller is used. Please, consult
the examples to see how they must be used.
    \vspace{1ex}

    \end{boxedminipage}

    \label{peach:fuzzy:control:Controller:add_table}
    \index{peach \textit{(package)}!peach.fuzzy \textit{(package)}!peach.fuzzy.control \textit{(module)}!peach.fuzzy.control.Controller \textit{(class)}!peach.fuzzy.control.Controller.add\_table \textit{(method)}}

    \vspace{0.5ex}

    \begin{boxedminipage}{\textwidth}

    \raggedright \textbf{add\_table}(\textit{self}, \textit{lx1}, \textit{lx2}, \textit{table})

    \vspace{-1.5ex}

    \rule{\textwidth}{0.5\fboxrule}

Adds a table of decision rules in a two variable controller.

Typically, fuzzy controllers are used to control two variables. In that
case, the set of decision rules are given in the form of a table, since
that is a more compact format and very easy to visualize. This is a
convenience function that allows to add decision rules in the form of a
table. Notice that the resulting knowledge base will be the same if this
function is used or the \texttt{add{\_}rule} method is used with every single
rule. The second method is in general easier to read in a script, so
consider well.
    \vspace{1ex}

      \textbf{Parameters}
      \begin{quote}
        \begin{Ventry}{xxxxx}

          \item[lx1]


The set of membership functions to the variable \texttt{x1}, or the
lines of the table
          \item[lx2]


The set of membership functions to the variable \texttt{x2}, or the
columns of the table
          \item[table]


The consequent of the rule where the condition is the line \texttt{and}
the column. These can be the membership functions or fuzzy sets.
        \end{Ventry}

      \end{quote}

    \vspace{1ex}

    \end{boxedminipage}

    \label{peach:fuzzy:control:Controller:eval}
    \index{peach \textit{(package)}!peach.fuzzy \textit{(package)}!peach.fuzzy.control \textit{(module)}!peach.fuzzy.control.Controller \textit{(class)}!peach.fuzzy.control.Controller.eval \textit{(method)}}

    \vspace{0.5ex}

    \begin{boxedminipage}{\textwidth}

    \raggedright \textbf{eval}(\textit{self}, \textit{r}, \textit{xs})

    \vspace{-1.5ex}

    \rule{\textwidth}{0.5\fboxrule}

Evaluates one decision rule in this controller

Takes a rule from the controller and evaluates it given the values of
the input variables.
    \vspace{1ex}

      \textbf{Parameters}
      \begin{quote}
        \begin{Ventry}{xx}

          \item[r]


The rule in the standard format, or an integer number. If \texttt{r} is
an integer, then the \texttt{r} th rule in the knowledge base will be
evaluated.
          \item[xs]


A tuple, a list or an array containing the values of the input
variables. The dimension must be coherent with the given rule.
        \end{Ventry}

      \end{quote}

    \vspace{1ex}

      \textbf{Return Value}
      \begin{quote}

This method evaluates each membership function in the rule for each
given value, and \texttt{and} 's the results to obtain the condition. If
the condition is zero, a tuple \texttt{(0.0, None) is returned. Otherwise,
the condition is `{}`imply} ed in the membership function of the output
variable. A tuple containing \texttt{(condition, imply)} (the membership
value associated to the condition and the result of the implication)
is returned.
      \end{quote}

    \vspace{1ex}

    \end{boxedminipage}

    \label{peach:fuzzy:control:Controller:eval_all}
    \index{peach \textit{(package)}!peach.fuzzy \textit{(package)}!peach.fuzzy.control \textit{(module)}!peach.fuzzy.control.Controller \textit{(class)}!peach.fuzzy.control.Controller.eval\_all \textit{(method)}}

    \vspace{0.5ex}

    \begin{boxedminipage}{\textwidth}

    \raggedright \textbf{eval\_all}(\textit{self}, *\textit{xs})

    \vspace{-1.5ex}

    \rule{\textwidth}{0.5\fboxrule}

Evaluates all the rules and aglutinates the results.

Given the values of the input variables, evaluate and apply every rule
in the knowledge base (with the \texttt{eval} method) and aglutinates the
results.
    \vspace{1ex}

      \textbf{Parameters}
      \begin{quote}
        \begin{Ventry}{xx}

          \item[xs]


A tuple, a list or an array with the values of the input variables.
        \end{Ventry}

      \end{quote}

    \vspace{1ex}

      \textbf{Return Value}
      \begin{quote}

A fuzzy set containing the result of the evaluation of every rule in
the knowledge base, with the results aglutinated.
      \end{quote}

    \vspace{1ex}

    \end{boxedminipage}

    \label{peach:fuzzy:control:Controller:__call__}
    \index{peach \textit{(package)}!peach.fuzzy \textit{(package)}!peach.fuzzy.control \textit{(module)}!peach.fuzzy.control.Controller \textit{(class)}!peach.fuzzy.control.Controller.\_\_call\_\_ \textit{(method)}}

    \vspace{0.5ex}

    \begin{boxedminipage}{\textwidth}

    \raggedright \textbf{\_\_call\_\_}(\textit{self}, *\textit{xs})

    \vspace{-1.5ex}

    \rule{\textwidth}{0.5\fboxrule}

Apply the controller to the set of input variables

Given the values of the input variables, evaluates every decision rule,
aglutinates the results and defuzzify it. Returns the response of the
controller.
    \vspace{1ex}

      \textbf{Parameters}
      \begin{quote}
        \begin{Ventry}{xx}

          \item[xs]


A tuple, a list or an array with the values of the input variables.
        \end{Ventry}

      \end{quote}

    \vspace{1ex}

      \textbf{Return Value}
      \begin{quote}

The response of the controller.
      \end{quote}

    \vspace{1ex}

    \end{boxedminipage}

    \label{object:__delattr__}
    \index{object.\_\_delattr\_\_ \textit{(function)}}

    \vspace{0.5ex}

    \begin{boxedminipage}{\textwidth}

    \raggedright \textbf{\_\_delattr\_\_}(\textit{...})

    \vspace{-1.5ex}

    \rule{\textwidth}{0.5\fboxrule}

x.{\_}{\_}delattr{\_}{\_}('name') {\textless}=={\textgreater} del x.name
    \vspace{1ex}

    \end{boxedminipage}

    \label{object:__getattribute__}
    \index{object.\_\_getattribute\_\_ \textit{(function)}}

    \vspace{0.5ex}

    \begin{boxedminipage}{\textwidth}

    \raggedright \textbf{\_\_getattribute\_\_}(\textit{...})

    \vspace{-1.5ex}

    \rule{\textwidth}{0.5\fboxrule}

x.{\_}{\_}getattribute{\_}{\_}('name') {\textless}=={\textgreater} x.name
    \vspace{1ex}

    \end{boxedminipage}

    \label{object:__hash__}
    \index{object.\_\_hash\_\_ \textit{(function)}}

    \vspace{0.5ex}

    \begin{boxedminipage}{\textwidth}

    \raggedright \textbf{\_\_hash\_\_}(\textit{x})

    \vspace{-1.5ex}

    \rule{\textwidth}{0.5\fboxrule}

hash(x)
    \vspace{1ex}

    \end{boxedminipage}

    \label{object:__new__}
    \index{object.\_\_new\_\_ \textit{(function)}}

    \vspace{0.5ex}

    \begin{boxedminipage}{\textwidth}

    \raggedright \textbf{\_\_new\_\_}(\textit{T}, \textit{S}, \textit{...})

      \textbf{Return Value}
      \begin{quote}
\begin{alltt}
a new object with type S, a subtype of T
\end{alltt}

      \end{quote}

    \vspace{1ex}

    \end{boxedminipage}

    \label{object:__reduce__}
    \index{object.\_\_reduce\_\_ \textit{(function)}}

    \vspace{0.5ex}

    \begin{boxedminipage}{\textwidth}

    \raggedright \textbf{\_\_reduce\_\_}(\textit{...})

    \vspace{-1.5ex}

    \rule{\textwidth}{0.5\fboxrule}

helper for pickle
    \vspace{1ex}

    \end{boxedminipage}

    \label{object:__reduce_ex__}
    \index{object.\_\_reduce\_ex\_\_ \textit{(function)}}

    \vspace{0.5ex}

    \begin{boxedminipage}{\textwidth}

    \raggedright \textbf{\_\_reduce\_ex\_\_}(\textit{...})

    \vspace{-1.5ex}

    \rule{\textwidth}{0.5\fboxrule}

helper for pickle
    \vspace{1ex}

    \end{boxedminipage}

    \label{object:__repr__}
    \index{object.\_\_repr\_\_ \textit{(function)}}

    \vspace{0.5ex}

    \begin{boxedminipage}{\textwidth}

    \raggedright \textbf{\_\_repr\_\_}(\textit{x})

    \vspace{-1.5ex}

    \rule{\textwidth}{0.5\fboxrule}

repr(x)
    \vspace{1ex}

    \end{boxedminipage}

    \label{object:__setattr__}
    \index{object.\_\_setattr\_\_ \textit{(function)}}

    \vspace{0.5ex}

    \begin{boxedminipage}{\textwidth}

    \raggedright \textbf{\_\_setattr\_\_}(\textit{...})

    \vspace{-1.5ex}

    \rule{\textwidth}{0.5\fboxrule}

x.{\_}{\_}setattr{\_}{\_}('name', value) {\textless}=={\textgreater} x.name = value
    \vspace{1ex}

    \end{boxedminipage}

    \label{object:__str__}
    \index{object.\_\_str\_\_ \textit{(function)}}

    \vspace{0.5ex}

    \begin{boxedminipage}{\textwidth}

    \raggedright \textbf{\_\_str\_\_}(\textit{x})

    \vspace{-1.5ex}

    \rule{\textwidth}{0.5\fboxrule}

str(x)
    \vspace{1ex}

    \end{boxedminipage}


%%%%%%%%%%%%%%%%%%%%%%%%%%%%%%%%%%%%%%%%%%%%%%%%%%%%%%%%%%%%%%%%%%%%%%%%%%%
%%                              Properties                               %%
%%%%%%%%%%%%%%%%%%%%%%%%%%%%%%%%%%%%%%%%%%%%%%%%%%%%%%%%%%%%%%%%%%%%%%%%%%%

  \subsubsection{Properties}

\begin{longtable}{|p{.30\textwidth}|p{.62\textwidth}|l}
\cline{1-2}
\cline{1-2} \centering \textbf{Name} & \centering \textbf{Description}& \\
\cline{1-2}
\endhead\cline{1-2}\multicolumn{3}{r}{\small\textit{continued on next page}}\\\endfoot\cline{1-2}
\endlastfoot\raggedright y\- & \raggedright Property that returns the output variable interval. Not writable

\textbf{Value:} 
{\tt {\textless}property object at 0x887be3c{\textgreater}}&\\
\cline{1-2}
\raggedright r\-u\-l\-e\-s\- & \raggedright Property that returns the list of decision rules. Not writable

\textbf{Value:} 
{\tt {\textless}property object at 0x887be64{\textgreater}}&\\
\cline{1-2}
\raggedright \_\-\_\-c\-l\-a\-s\-s\-\_\-\_\- & \raggedright \textbf{Value:} 
{\tt {\textless}attribute '\_\_class\_\_' of 'object' objects{\textgreater}}&\\
\cline{1-2}
\end{longtable}

    \index{peach \textit{(package)}!peach.fuzzy \textit{(package)}!peach.fuzzy.control \textit{(module)}!peach.fuzzy.control.Controller \textit{(class)}|)}

%%%%%%%%%%%%%%%%%%%%%%%%%%%%%%%%%%%%%%%%%%%%%%%%%%%%%%%%%%%%%%%%%%%%%%%%%%%
%%                           Class Description                           %%
%%%%%%%%%%%%%%%%%%%%%%%%%%%%%%%%%%%%%%%%%%%%%%%%%%%%%%%%%%%%%%%%%%%%%%%%%%%

    \index{peach \textit{(package)}!peach.fuzzy \textit{(package)}!peach.fuzzy.control \textit{(module)}!peach.fuzzy.control.Mamdani \textit{(class)}|(}
\subsection{Class Mamdani}

    \label{peach:fuzzy:control:Mamdani}
\begin{tabular}{cccccccc}
% Line for object, linespec=[False, False]
\multicolumn{2}{r}{\settowidth{\BCL}{object}\multirow{2}{\BCL}{object}}
&&
&&
  \\\cline{3-3}
  &&\multicolumn{1}{c|}{}
&&
&&
  \\
% Line for peach.fuzzy.control.Controller, linespec=[False]
\multicolumn{4}{r}{\settowidth{\BCL}{peach.fuzzy.control.Controller}\multirow{2}{\BCL}{peach.fuzzy.control.Controller}}
&&
  \\\cline{5-5}
  &&&&\multicolumn{1}{c|}{}
&&
  \\
&&&&\multicolumn{2}{l}{\textbf{peach.fuzzy.control.Mamdani}}
\end{tabular}


\texttt{Mandani} is an alias to \texttt{Controller}

%%%%%%%%%%%%%%%%%%%%%%%%%%%%%%%%%%%%%%%%%%%%%%%%%%%%%%%%%%%%%%%%%%%%%%%%%%%
%%                                Methods                                %%
%%%%%%%%%%%%%%%%%%%%%%%%%%%%%%%%%%%%%%%%%%%%%%%%%%%%%%%%%%%%%%%%%%%%%%%%%%%

  \subsubsection{Methods}

    \label{peach:fuzzy:control:Controller:__call__}
    \index{peach \textit{(package)}!peach.fuzzy \textit{(package)}!peach.fuzzy.control \textit{(module)}!peach.fuzzy.control.Controller \textit{(class)}!peach.fuzzy.control.Controller.\_\_call\_\_ \textit{(method)}}

    \vspace{0.5ex}

    \begin{boxedminipage}{\textwidth}

    \raggedright \textbf{\_\_call\_\_}(\textit{self}, *\textit{xs})

    \vspace{-1.5ex}

    \rule{\textwidth}{0.5\fboxrule}

Apply the controller to the set of input variables

Given the values of the input variables, evaluates every decision rule,
aglutinates the results and defuzzify it. Returns the response of the
controller.
    \vspace{1ex}

      \textbf{Parameters}
      \begin{quote}
        \begin{Ventry}{xx}

          \item[xs]


A tuple, a list or an array with the values of the input variables.
        \end{Ventry}

      \end{quote}

    \vspace{1ex}

      \textbf{Return Value}
      \begin{quote}

The response of the controller.
      \end{quote}

    \vspace{1ex}

    \end{boxedminipage}

    \label{object:__delattr__}
    \index{object.\_\_delattr\_\_ \textit{(function)}}

    \vspace{0.5ex}

    \begin{boxedminipage}{\textwidth}

    \raggedright \textbf{\_\_delattr\_\_}(\textit{...})

    \vspace{-1.5ex}

    \rule{\textwidth}{0.5\fboxrule}

x.{\_}{\_}delattr{\_}{\_}('name') {\textless}=={\textgreater} del x.name
    \vspace{1ex}

    \end{boxedminipage}

    \label{object:__getattribute__}
    \index{object.\_\_getattribute\_\_ \textit{(function)}}

    \vspace{0.5ex}

    \begin{boxedminipage}{\textwidth}

    \raggedright \textbf{\_\_getattribute\_\_}(\textit{...})

    \vspace{-1.5ex}

    \rule{\textwidth}{0.5\fboxrule}

x.{\_}{\_}getattribute{\_}{\_}('name') {\textless}=={\textgreater} x.name
    \vspace{1ex}

    \end{boxedminipage}

    \label{object:__hash__}
    \index{object.\_\_hash\_\_ \textit{(function)}}

    \vspace{0.5ex}

    \begin{boxedminipage}{\textwidth}

    \raggedright \textbf{\_\_hash\_\_}(\textit{x})

    \vspace{-1.5ex}

    \rule{\textwidth}{0.5\fboxrule}

hash(x)
    \vspace{1ex}

    \end{boxedminipage}

    \vspace{0.5ex}

    \begin{boxedminipage}{\textwidth}

    \raggedright \textbf{\_\_init\_\_}(\textit{self}, \textit{yrange}, \textit{rules}=\texttt{\texttt{[}\texttt{]}}, \textit{defuzzy}=\texttt{{\textless}function Centroid at 0x885ecdc{\textgreater}}, \textit{norm}=\texttt{{\textless}function ZadehAnd at 0x885eaac{\textgreater}}, \textit{conorm}=\texttt{{\textless}function ZadehOr at 0x885eae4{\textgreater}}, \textit{negation}=\texttt{{\textless}function ZadehNot at 0x885eb1c{\textgreater}}, \textit{imply}=\texttt{{\textless}function MamdaniImplication at 0x885eb54{\textgreater}}, \textit{aglutinate}=\texttt{{\textless}function MamdaniAglutination at 0x885eb8c{\textgreater}})

    \vspace{-1.5ex}

    \rule{\textwidth}{0.5\fboxrule}

Creates and initialize the controller.
    \vspace{1ex}

      \textbf{Parameters}
      \begin{quote}
        \begin{Ventry}{xxxxxxxxxx}

          \item[yrange]


The range of the output variable. This must be given as a set of
points belonging to the interval where the output variable is
defined, not only the start and end points. It is strongly suggested
that the interval is divided in some (eg.: 100) points equally
spaced;
          \item[rules]


The set of decision rules, as defined above. If none is given, an
empty set of rules is assumed;
          \item[defuzzy]


The defuzzification method to be used. If none is given, the
Centroid method is used;
          \item[norm]


The norm (\texttt{and} operation) to be used. Defaults to Zadeh and.
          \item[conorm]


The conorm (\texttt{or} operation) to be used. Defaults to Zadeh or.
          \item[negation]


The negation (\texttt{not} operation) to be used. Defaults to Zadeh not.
          \item[imply]


The implication method to be used. Defaults to Mamdani implication.
          \item[aglutinate]


The aglutination method to be used. Defaults to Mamdani
aglutination.
        \end{Ventry}

      \end{quote}

    \vspace{1ex}

      Overrides: object.\_\_init\_\_

    \end{boxedminipage}

    \label{object:__new__}
    \index{object.\_\_new\_\_ \textit{(function)}}

    \vspace{0.5ex}

    \begin{boxedminipage}{\textwidth}

    \raggedright \textbf{\_\_new\_\_}(\textit{T}, \textit{S}, \textit{...})

      \textbf{Return Value}
      \begin{quote}
\begin{alltt}
a new object with type S, a subtype of T
\end{alltt}

      \end{quote}

    \vspace{1ex}

    \end{boxedminipage}

    \label{object:__reduce__}
    \index{object.\_\_reduce\_\_ \textit{(function)}}

    \vspace{0.5ex}

    \begin{boxedminipage}{\textwidth}

    \raggedright \textbf{\_\_reduce\_\_}(\textit{...})

    \vspace{-1.5ex}

    \rule{\textwidth}{0.5\fboxrule}

helper for pickle
    \vspace{1ex}

    \end{boxedminipage}

    \label{object:__reduce_ex__}
    \index{object.\_\_reduce\_ex\_\_ \textit{(function)}}

    \vspace{0.5ex}

    \begin{boxedminipage}{\textwidth}

    \raggedright \textbf{\_\_reduce\_ex\_\_}(\textit{...})

    \vspace{-1.5ex}

    \rule{\textwidth}{0.5\fboxrule}

helper for pickle
    \vspace{1ex}

    \end{boxedminipage}

    \label{object:__repr__}
    \index{object.\_\_repr\_\_ \textit{(function)}}

    \vspace{0.5ex}

    \begin{boxedminipage}{\textwidth}

    \raggedright \textbf{\_\_repr\_\_}(\textit{x})

    \vspace{-1.5ex}

    \rule{\textwidth}{0.5\fboxrule}

repr(x)
    \vspace{1ex}

    \end{boxedminipage}

    \label{object:__setattr__}
    \index{object.\_\_setattr\_\_ \textit{(function)}}

    \vspace{0.5ex}

    \begin{boxedminipage}{\textwidth}

    \raggedright \textbf{\_\_setattr\_\_}(\textit{...})

    \vspace{-1.5ex}

    \rule{\textwidth}{0.5\fboxrule}

x.{\_}{\_}setattr{\_}{\_}('name', value) {\textless}=={\textgreater} x.name = value
    \vspace{1ex}

    \end{boxedminipage}

    \label{object:__str__}
    \index{object.\_\_str\_\_ \textit{(function)}}

    \vspace{0.5ex}

    \begin{boxedminipage}{\textwidth}

    \raggedright \textbf{\_\_str\_\_}(\textit{x})

    \vspace{-1.5ex}

    \rule{\textwidth}{0.5\fboxrule}

str(x)
    \vspace{1ex}

    \end{boxedminipage}

    \label{peach:fuzzy:control:Controller:add_rule}
    \index{peach \textit{(package)}!peach.fuzzy \textit{(package)}!peach.fuzzy.control \textit{(module)}!peach.fuzzy.control.Controller \textit{(class)}!peach.fuzzy.control.Controller.add\_rule \textit{(method)}}

    \vspace{0.5ex}

    \begin{boxedminipage}{\textwidth}

    \raggedright \textbf{add\_rule}(\textit{self}, \textit{rule})

    \vspace{-1.5ex}

    \rule{\textwidth}{0.5\fboxrule}

Adds a decision rule to the knowledge base.

It is essential to understand the format that decision rules must follow
to obtain correct behaviour of the controller. A rule is a tuple must
have the following format:
\begin{quote}{\ttfamily \raggedright \noindent
((mx0,~mx1,~...,~mxn),~my)
}\end{quote}

where \texttt{mx0} is a membership function of the first input variable,
\texttt{mx1} is a membership function of the second input variable and so on;
and \texttt{my} is a membership function or a fuzzy set of the output
variable.

Notice that \texttt{mx}'s are \emph{functions} not fuzzy sets! They will be
applied to the values of the input variables given in the function call,
so, if they are anything different from a membership function, an
exception will be raised when the controller is used. Please, consult
the examples to see how they must be used.
    \vspace{1ex}

    \end{boxedminipage}

    \label{peach:fuzzy:control:Controller:add_table}
    \index{peach \textit{(package)}!peach.fuzzy \textit{(package)}!peach.fuzzy.control \textit{(module)}!peach.fuzzy.control.Controller \textit{(class)}!peach.fuzzy.control.Controller.add\_table \textit{(method)}}

    \vspace{0.5ex}

    \begin{boxedminipage}{\textwidth}

    \raggedright \textbf{add\_table}(\textit{self}, \textit{lx1}, \textit{lx2}, \textit{table})

    \vspace{-1.5ex}

    \rule{\textwidth}{0.5\fboxrule}

Adds a table of decision rules in a two variable controller.

Typically, fuzzy controllers are used to control two variables. In that
case, the set of decision rules are given in the form of a table, since
that is a more compact format and very easy to visualize. This is a
convenience function that allows to add decision rules in the form of a
table. Notice that the resulting knowledge base will be the same if this
function is used or the \texttt{add{\_}rule} method is used with every single
rule. The second method is in general easier to read in a script, so
consider well.
    \vspace{1ex}

      \textbf{Parameters}
      \begin{quote}
        \begin{Ventry}{xxxxx}

          \item[lx1]


The set of membership functions to the variable \texttt{x1}, or the
lines of the table
          \item[lx2]


The set of membership functions to the variable \texttt{x2}, or the
columns of the table
          \item[table]


The consequent of the rule where the condition is the line \texttt{and}
the column. These can be the membership functions or fuzzy sets.
        \end{Ventry}

      \end{quote}

    \vspace{1ex}

    \end{boxedminipage}

    \label{peach:fuzzy:control:Controller:eval}
    \index{peach \textit{(package)}!peach.fuzzy \textit{(package)}!peach.fuzzy.control \textit{(module)}!peach.fuzzy.control.Controller \textit{(class)}!peach.fuzzy.control.Controller.eval \textit{(method)}}

    \vspace{0.5ex}

    \begin{boxedminipage}{\textwidth}

    \raggedright \textbf{eval}(\textit{self}, \textit{r}, \textit{xs})

    \vspace{-1.5ex}

    \rule{\textwidth}{0.5\fboxrule}

Evaluates one decision rule in this controller

Takes a rule from the controller and evaluates it given the values of
the input variables.
    \vspace{1ex}

      \textbf{Parameters}
      \begin{quote}
        \begin{Ventry}{xx}

          \item[r]


The rule in the standard format, or an integer number. If \texttt{r} is
an integer, then the \texttt{r} th rule in the knowledge base will be
evaluated.
          \item[xs]


A tuple, a list or an array containing the values of the input
variables. The dimension must be coherent with the given rule.
        \end{Ventry}

      \end{quote}

    \vspace{1ex}

      \textbf{Return Value}
      \begin{quote}

This method evaluates each membership function in the rule for each
given value, and \texttt{and} 's the results to obtain the condition. If
the condition is zero, a tuple \texttt{(0.0, None) is returned. Otherwise,
the condition is `{}`imply} ed in the membership function of the output
variable. A tuple containing \texttt{(condition, imply)} (the membership
value associated to the condition and the result of the implication)
is returned.
      \end{quote}

    \vspace{1ex}

    \end{boxedminipage}

    \label{peach:fuzzy:control:Controller:eval_all}
    \index{peach \textit{(package)}!peach.fuzzy \textit{(package)}!peach.fuzzy.control \textit{(module)}!peach.fuzzy.control.Controller \textit{(class)}!peach.fuzzy.control.Controller.eval\_all \textit{(method)}}

    \vspace{0.5ex}

    \begin{boxedminipage}{\textwidth}

    \raggedright \textbf{eval\_all}(\textit{self}, *\textit{xs})

    \vspace{-1.5ex}

    \rule{\textwidth}{0.5\fboxrule}

Evaluates all the rules and aglutinates the results.

Given the values of the input variables, evaluate and apply every rule
in the knowledge base (with the \texttt{eval} method) and aglutinates the
results.
    \vspace{1ex}

      \textbf{Parameters}
      \begin{quote}
        \begin{Ventry}{xx}

          \item[xs]


A tuple, a list or an array with the values of the input variables.
        \end{Ventry}

      \end{quote}

    \vspace{1ex}

      \textbf{Return Value}
      \begin{quote}

A fuzzy set containing the result of the evaluation of every rule in
the knowledge base, with the results aglutinated.
      \end{quote}

    \vspace{1ex}

    \end{boxedminipage}

    \label{peach:fuzzy:control:Controller:set_aglutination}
    \index{peach \textit{(package)}!peach.fuzzy \textit{(package)}!peach.fuzzy.control \textit{(module)}!peach.fuzzy.control.Controller \textit{(class)}!peach.fuzzy.control.Controller.set\_aglutination \textit{(method)}}

    \vspace{0.5ex}

    \begin{boxedminipage}{\textwidth}

    \raggedright \textbf{set\_aglutination}(\textit{self}, \textit{f})

    \vspace{-1.5ex}

    \rule{\textwidth}{0.5\fboxrule}

Sets the aglutination to be used.

This method must be used to change the behavior of the aglutination
operation of the controller.
    \vspace{1ex}

      \textbf{Parameters}
      \begin{quote}
        \begin{Ventry}{x}

          \item[f]


The function can be any function that takes two numerical values and
return one numerical value, that corresponds to the aglutination
result.
        \end{Ventry}

      \end{quote}

    \vspace{1ex}

    \end{boxedminipage}

    \label{peach:fuzzy:control:Controller:set_conorm}
    \index{peach \textit{(package)}!peach.fuzzy \textit{(package)}!peach.fuzzy.control \textit{(module)}!peach.fuzzy.control.Controller \textit{(class)}!peach.fuzzy.control.Controller.set\_conorm \textit{(method)}}

    \vspace{0.5ex}

    \begin{boxedminipage}{\textwidth}

    \raggedright \textbf{set\_conorm}(\textit{self}, \textit{f})

    \vspace{-1.5ex}

    \rule{\textwidth}{0.5\fboxrule}

Sets the conorm (\texttt{or}) to be used.

This method must be used to change the behavior of the \texttt{or} operation
of the controller.
    \vspace{1ex}

      \textbf{Parameters}
      \begin{quote}
        \begin{Ventry}{x}

          \item[f]


The function can be any function that takes two numerical values and
return one numerical value, that corresponds to the \texttt{or} result.
        \end{Ventry}

      \end{quote}

    \vspace{1ex}

    \end{boxedminipage}

    \label{peach:fuzzy:control:Controller:set_implication}
    \index{peach \textit{(package)}!peach.fuzzy \textit{(package)}!peach.fuzzy.control \textit{(module)}!peach.fuzzy.control.Controller \textit{(class)}!peach.fuzzy.control.Controller.set\_implication \textit{(method)}}

    \vspace{0.5ex}

    \begin{boxedminipage}{\textwidth}

    \raggedright \textbf{set\_implication}(\textit{self}, \textit{f})

    \vspace{-1.5ex}

    \rule{\textwidth}{0.5\fboxrule}

Sets the implication to be used.

This method must be used to change the behavior of the implication
operation of the controller.
    \vspace{1ex}

      \textbf{Parameters}
      \begin{quote}
        \begin{Ventry}{x}

          \item[f]


The function can be any function that takes two numerical values and
return one numerical value, that corresponds to the implication
result.
        \end{Ventry}

      \end{quote}

    \vspace{1ex}

    \end{boxedminipage}

    \label{peach:fuzzy:control:Controller:set_negation}
    \index{peach \textit{(package)}!peach.fuzzy \textit{(package)}!peach.fuzzy.control \textit{(module)}!peach.fuzzy.control.Controller \textit{(class)}!peach.fuzzy.control.Controller.set\_negation \textit{(method)}}

    \vspace{0.5ex}

    \begin{boxedminipage}{\textwidth}

    \raggedright \textbf{set\_negation}(\textit{self}, \textit{f})

    \vspace{-1.5ex}

    \rule{\textwidth}{0.5\fboxrule}

Sets the negation (\texttt{not}) to be used.

This method must be used to change the behavior of the \texttt{not} operation
of the controller.
    \vspace{1ex}

      \textbf{Parameters}
      \begin{quote}
        \begin{Ventry}{x}

          \item[f]


The function can be any function that takes one numerical value and
return one numerical value, that corresponds to the \texttt{not} result.
        \end{Ventry}

      \end{quote}

    \vspace{1ex}

    \end{boxedminipage}

    \label{peach:fuzzy:control:Controller:set_norm}
    \index{peach \textit{(package)}!peach.fuzzy \textit{(package)}!peach.fuzzy.control \textit{(module)}!peach.fuzzy.control.Controller \textit{(class)}!peach.fuzzy.control.Controller.set\_norm \textit{(method)}}

    \vspace{0.5ex}

    \begin{boxedminipage}{\textwidth}

    \raggedright \textbf{set\_norm}(\textit{self}, \textit{f})

    \vspace{-1.5ex}

    \rule{\textwidth}{0.5\fboxrule}

Sets the norm (\texttt{and}) to be used.

This method must be used to change the behavior of the \texttt{and} operation
of the controller.
    \vspace{1ex}

      \textbf{Parameters}
      \begin{quote}
        \begin{Ventry}{x}

          \item[f]


The function can be any function that takes two numerical values and
return one numerical value, that corresponds to the \texttt{and} result.
        \end{Ventry}

      \end{quote}

    \vspace{1ex}

    \end{boxedminipage}


%%%%%%%%%%%%%%%%%%%%%%%%%%%%%%%%%%%%%%%%%%%%%%%%%%%%%%%%%%%%%%%%%%%%%%%%%%%
%%                              Properties                               %%
%%%%%%%%%%%%%%%%%%%%%%%%%%%%%%%%%%%%%%%%%%%%%%%%%%%%%%%%%%%%%%%%%%%%%%%%%%%

  \subsubsection{Properties}

\begin{longtable}{|p{.30\textwidth}|p{.62\textwidth}|l}
\cline{1-2}
\cline{1-2} \centering \textbf{Name} & \centering \textbf{Description}& \\
\cline{1-2}
\endhead\cline{1-2}\multicolumn{3}{r}{\small\textit{continued on next page}}\\\endfoot\cline{1-2}
\endlastfoot\raggedright \_\-\_\-c\-l\-a\-s\-s\-\_\-\_\- & \raggedright \textbf{Value:} 
{\tt {\textless}attribute '\_\_class\_\_' of 'object' objects{\textgreater}}&\\
\cline{1-2}
\raggedright r\-u\-l\-e\-s\- & \raggedright Property that returns the list of decision rules. Not writable

\textbf{Value:} 
{\tt {\textless}property object at 0x887be64{\textgreater}}&\\
\cline{1-2}
\raggedright y\- & \raggedright Property that returns the output variable interval. Not writable

\textbf{Value:} 
{\tt {\textless}property object at 0x887be3c{\textgreater}}&\\
\cline{1-2}
\end{longtable}

    \index{peach \textit{(package)}!peach.fuzzy \textit{(package)}!peach.fuzzy.control \textit{(module)}!peach.fuzzy.control.Mamdani \textit{(class)}|)}

%%%%%%%%%%%%%%%%%%%%%%%%%%%%%%%%%%%%%%%%%%%%%%%%%%%%%%%%%%%%%%%%%%%%%%%%%%%
%%                           Class Description                           %%
%%%%%%%%%%%%%%%%%%%%%%%%%%%%%%%%%%%%%%%%%%%%%%%%%%%%%%%%%%%%%%%%%%%%%%%%%%%

    \index{peach \textit{(package)}!peach.fuzzy \textit{(package)}!peach.fuzzy.control \textit{(module)}!peach.fuzzy.control.Parametric \textit{(class)}|(}
\subsection{Class Parametric}

    \label{peach:fuzzy:control:Parametric}
\begin{tabular}{cccccc}
% Line for object, linespec=[False]
\multicolumn{2}{r}{\settowidth{\BCL}{object}\multirow{2}{\BCL}{object}}
&&
  \\\cline{3-3}
  &&\multicolumn{1}{c|}{}
&&
  \\
&&\multicolumn{2}{l}{\textbf{peach.fuzzy.control.Parametric}}
\end{tabular}

\textbf{Known Subclasses:} peach.fuzzy.control.Sugeno


Basic Parametric controller

This class implements a standard parametric (or Takagi-Sugeno) controller. A
controller based on fuzzy logic has a somewhat complex behaviour, so it is
not explained here. There are numerous references that can be consulted.

It is essential to understand the format that decision rules must follow to
obtain correct behaviour of the controller. A rule is a tuple given by:
\begin{quote}{\ttfamily \raggedright \noindent
((mx0,~mx1,~...,~mxn),~(a0,~a1,~...,~an))
}\end{quote}

where \texttt{mx0} is a membership function of the first input variable, \texttt{mx1}
is a membership function of the second input variable and so on; and \texttt{a0}
is the linear parameter, \texttt{a1} is the parameter associated with the first
input variable, \texttt{a2} is the parameter associated with the second input
variable and so on. The response to the rule is calculated by:
\begin{quote}{\ttfamily \raggedright \noindent
y~=~a0~+~a1*x1~+~a2*x2~+~...~+~an*xn
}\end{quote}

Notice that \texttt{mx}'s are \emph{functions} not fuzzy sets! They will be applied to
the values of the input variables given in the function call, so, if they
are anything different from a membership function, an exception will be
raised. Please, consult the examples to see how they must be used.

%%%%%%%%%%%%%%%%%%%%%%%%%%%%%%%%%%%%%%%%%%%%%%%%%%%%%%%%%%%%%%%%%%%%%%%%%%%
%%                                Methods                                %%
%%%%%%%%%%%%%%%%%%%%%%%%%%%%%%%%%%%%%%%%%%%%%%%%%%%%%%%%%%%%%%%%%%%%%%%%%%%

  \subsubsection{Methods}

    \vspace{0.5ex}

    \begin{boxedminipage}{\textwidth}

    \raggedright \textbf{\_\_init\_\_}(\textit{self}, \textit{rules}=\texttt{\texttt{[}\texttt{]}}, \textit{norm}=\texttt{{\textless}function ProbabilisticAnd at 0x885ebc4{\textgreater}}, \textit{conorm}=\texttt{{\textless}function ProbabilisticOr at 0x885ebfc{\textgreater}}, \textit{negation}=\texttt{{\textless}function ProbabilisticNot at 0x885ec34{\textgreater}})

    \vspace{-1.5ex}

    \rule{\textwidth}{0.5\fboxrule}

Creates and initializes the controller.
    \vspace{1ex}

      \textbf{Parameters}
      \begin{quote}
        \begin{Ventry}{xxxxxxxx}

          \item[rules]


List containing the decision rules for the controller. If not given,
an empty set of decision rules is used.
          \item[norm]


The norm (\texttt{and} operation) to be used. Defaults to Probabilistic
and.
          \item[conorm]


The conorm (\texttt{or} operation) to be used. Defaults to Probabilistic
or.
          \item[negation]


The negation (\texttt{not} operation) to be used. Defaults to
Probabilistic not.
        \end{Ventry}

      \end{quote}

    \vspace{1ex}

      Overrides: object.\_\_init\_\_

    \end{boxedminipage}

    \label{peach:fuzzy:control:Parametric:__getrules}
    \index{peach \textit{(package)}!peach.fuzzy \textit{(package)}!peach.fuzzy.control \textit{(module)}!peach.fuzzy.control.Parametric \textit{(class)}!peach.fuzzy.control.Parametric.\_\_getrules \textit{(method)}}

    \vspace{0.5ex}

    \begin{boxedminipage}{\textwidth}

    \raggedright \textbf{\_\_getrules}(\textit{self})

    \end{boxedminipage}

    \label{peach:fuzzy:control:Parametric:add_rule}
    \index{peach \textit{(package)}!peach.fuzzy \textit{(package)}!peach.fuzzy.control \textit{(module)}!peach.fuzzy.control.Parametric \textit{(class)}!peach.fuzzy.control.Parametric.add\_rule \textit{(method)}}

    \vspace{0.5ex}

    \begin{boxedminipage}{\textwidth}

    \raggedright \textbf{add\_rule}(\textit{self}, \textit{rule})

    \vspace{-1.5ex}

    \rule{\textwidth}{0.5\fboxrule}

Adds a decision rule to the knowledge base.

It is essential to understand the format that decision rules must follow
to obtain correct behaviour of the controller. A rule is a tuple given
by:
\begin{quote}{\ttfamily \raggedright \noindent
((mx0,~mx1,~...,~mxn),~(a0,~a1,~...,~an))
}\end{quote}

where \texttt{mx0} is a membership function of the first input variable,
\texttt{mx1} is a membership function of the second input variable and so on;
and \texttt{a0} is the linear parameter, \texttt{a1} is the parameter associated
with the first input variable, \texttt{a2} is the parameter associated with
the second input variable and so on.

Notice that \texttt{mx}'s are \emph{functions} not fuzzy sets! They will be
applied to the values of the input variables given in the function call,
so, if they are anything different from a membership function, an
exception will be raised. Please, consult the examples to see how they
must be used.
    \vspace{1ex}

    \end{boxedminipage}

    \label{peach:fuzzy:control:Parametric:eval}
    \index{peach \textit{(package)}!peach.fuzzy \textit{(package)}!peach.fuzzy.control \textit{(module)}!peach.fuzzy.control.Parametric \textit{(class)}!peach.fuzzy.control.Parametric.eval \textit{(method)}}

    \vspace{0.5ex}

    \begin{boxedminipage}{\textwidth}

    \raggedright \textbf{eval}(\textit{self}, \textit{r}, \textit{xs})

    \vspace{-1.5ex}

    \rule{\textwidth}{0.5\fboxrule}

Evaluates one decision rule in this controller

Takes a rule from the controller and evaluates it given the values of
the input variables. The format of the rule is as given, and the
response to the rule is calculated by:
\begin{quote}{\ttfamily \raggedright \noindent
y~=~a0~+~a1*x1~+~a2*x2~+~...~+~an*xn
}\end{quote}
    \vspace{1ex}

      \textbf{Parameters}
      \begin{quote}
        \begin{Ventry}{xx}

          \item[r]


The rule in the standard format, or an integer number. If \texttt{r} is
an integer, then the \texttt{r} th rule in the knowledge base will be
evaluated.
          \item[xs]


A tuple, a list or an array containing the values of the input
variables. The dimension must be coherent with the given rule.
        \end{Ventry}

      \end{quote}

    \vspace{1ex}

      \textbf{Return Value}
      \begin{quote}

This method evaluates each membership function in the rule for each
given value, and \texttt{and} 's the results to obtain the condition. If
the condition is zero, a tuple \texttt{(0.0, 0.0) is returned. Otherwise,
the result as given above is calculate, and a tuple containing
`{}`(condition, result)} (the membership value associated to the
condition and the result of the calculation) is returned.
      \end{quote}

    \vspace{1ex}

    \end{boxedminipage}

    \label{peach:fuzzy:control:Parametric:__call__}
    \index{peach \textit{(package)}!peach.fuzzy \textit{(package)}!peach.fuzzy.control \textit{(module)}!peach.fuzzy.control.Parametric \textit{(class)}!peach.fuzzy.control.Parametric.\_\_call\_\_ \textit{(method)}}

    \vspace{0.5ex}

    \begin{boxedminipage}{\textwidth}

    \raggedright \textbf{\_\_call\_\_}(\textit{self}, *\textit{xs})

    \vspace{-1.5ex}

    \rule{\textwidth}{0.5\fboxrule}

Apply the controller to the set of input variables

Given the values of the input variables, evaluates every decision rule,
and calculates the weighted average of the results. Returns the response
of the controller.
    \vspace{1ex}

      \textbf{Parameters}
      \begin{quote}
        \begin{Ventry}{xx}

          \item[xs]


A tuple, a list or an array with the values of the input variables.
        \end{Ventry}

      \end{quote}

    \vspace{1ex}

      \textbf{Return Value}
      \begin{quote}

The response of the controller.
      \end{quote}

    \vspace{1ex}

    \end{boxedminipage}

    \label{object:__delattr__}
    \index{object.\_\_delattr\_\_ \textit{(function)}}

    \vspace{0.5ex}

    \begin{boxedminipage}{\textwidth}

    \raggedright \textbf{\_\_delattr\_\_}(\textit{...})

    \vspace{-1.5ex}

    \rule{\textwidth}{0.5\fboxrule}

x.{\_}{\_}delattr{\_}{\_}('name') {\textless}=={\textgreater} del x.name
    \vspace{1ex}

    \end{boxedminipage}

    \label{object:__getattribute__}
    \index{object.\_\_getattribute\_\_ \textit{(function)}}

    \vspace{0.5ex}

    \begin{boxedminipage}{\textwidth}

    \raggedright \textbf{\_\_getattribute\_\_}(\textit{...})

    \vspace{-1.5ex}

    \rule{\textwidth}{0.5\fboxrule}

x.{\_}{\_}getattribute{\_}{\_}('name') {\textless}=={\textgreater} x.name
    \vspace{1ex}

    \end{boxedminipage}

    \label{object:__hash__}
    \index{object.\_\_hash\_\_ \textit{(function)}}

    \vspace{0.5ex}

    \begin{boxedminipage}{\textwidth}

    \raggedright \textbf{\_\_hash\_\_}(\textit{x})

    \vspace{-1.5ex}

    \rule{\textwidth}{0.5\fboxrule}

hash(x)
    \vspace{1ex}

    \end{boxedminipage}

    \label{object:__new__}
    \index{object.\_\_new\_\_ \textit{(function)}}

    \vspace{0.5ex}

    \begin{boxedminipage}{\textwidth}

    \raggedright \textbf{\_\_new\_\_}(\textit{T}, \textit{S}, \textit{...})

      \textbf{Return Value}
      \begin{quote}
\begin{alltt}
a new object with type S, a subtype of T
\end{alltt}

      \end{quote}

    \vspace{1ex}

    \end{boxedminipage}

    \label{object:__reduce__}
    \index{object.\_\_reduce\_\_ \textit{(function)}}

    \vspace{0.5ex}

    \begin{boxedminipage}{\textwidth}

    \raggedright \textbf{\_\_reduce\_\_}(\textit{...})

    \vspace{-1.5ex}

    \rule{\textwidth}{0.5\fboxrule}

helper for pickle
    \vspace{1ex}

    \end{boxedminipage}

    \label{object:__reduce_ex__}
    \index{object.\_\_reduce\_ex\_\_ \textit{(function)}}

    \vspace{0.5ex}

    \begin{boxedminipage}{\textwidth}

    \raggedright \textbf{\_\_reduce\_ex\_\_}(\textit{...})

    \vspace{-1.5ex}

    \rule{\textwidth}{0.5\fboxrule}

helper for pickle
    \vspace{1ex}

    \end{boxedminipage}

    \label{object:__repr__}
    \index{object.\_\_repr\_\_ \textit{(function)}}

    \vspace{0.5ex}

    \begin{boxedminipage}{\textwidth}

    \raggedright \textbf{\_\_repr\_\_}(\textit{x})

    \vspace{-1.5ex}

    \rule{\textwidth}{0.5\fboxrule}

repr(x)
    \vspace{1ex}

    \end{boxedminipage}

    \label{object:__setattr__}
    \index{object.\_\_setattr\_\_ \textit{(function)}}

    \vspace{0.5ex}

    \begin{boxedminipage}{\textwidth}

    \raggedright \textbf{\_\_setattr\_\_}(\textit{...})

    \vspace{-1.5ex}

    \rule{\textwidth}{0.5\fboxrule}

x.{\_}{\_}setattr{\_}{\_}('name', value) {\textless}=={\textgreater} x.name = value
    \vspace{1ex}

    \end{boxedminipage}

    \label{object:__str__}
    \index{object.\_\_str\_\_ \textit{(function)}}

    \vspace{0.5ex}

    \begin{boxedminipage}{\textwidth}

    \raggedright \textbf{\_\_str\_\_}(\textit{x})

    \vspace{-1.5ex}

    \rule{\textwidth}{0.5\fboxrule}

str(x)
    \vspace{1ex}

    \end{boxedminipage}


%%%%%%%%%%%%%%%%%%%%%%%%%%%%%%%%%%%%%%%%%%%%%%%%%%%%%%%%%%%%%%%%%%%%%%%%%%%
%%                              Properties                               %%
%%%%%%%%%%%%%%%%%%%%%%%%%%%%%%%%%%%%%%%%%%%%%%%%%%%%%%%%%%%%%%%%%%%%%%%%%%%

  \subsubsection{Properties}

\begin{longtable}{|p{.30\textwidth}|p{.62\textwidth}|l}
\cline{1-2}
\cline{1-2} \centering \textbf{Name} & \centering \textbf{Description}& \\
\cline{1-2}
\endhead\cline{1-2}\multicolumn{3}{r}{\small\textit{continued on next page}}\\\endfoot\cline{1-2}
\endlastfoot\raggedright r\-u\-l\-e\-s\- & \raggedright Property that returns the list of decision rules. Not writable

\textbf{Value:} 
{\tt {\textless}property object at 0x887bedc{\textgreater}}&\\
\cline{1-2}
\raggedright \_\-\_\-c\-l\-a\-s\-s\-\_\-\_\- & \raggedright \textbf{Value:} 
{\tt {\textless}attribute '\_\_class\_\_' of 'object' objects{\textgreater}}&\\
\cline{1-2}
\end{longtable}

    \index{peach \textit{(package)}!peach.fuzzy \textit{(package)}!peach.fuzzy.control \textit{(module)}!peach.fuzzy.control.Parametric \textit{(class)}|)}

%%%%%%%%%%%%%%%%%%%%%%%%%%%%%%%%%%%%%%%%%%%%%%%%%%%%%%%%%%%%%%%%%%%%%%%%%%%
%%                           Class Description                           %%
%%%%%%%%%%%%%%%%%%%%%%%%%%%%%%%%%%%%%%%%%%%%%%%%%%%%%%%%%%%%%%%%%%%%%%%%%%%

    \index{peach \textit{(package)}!peach.fuzzy \textit{(package)}!peach.fuzzy.control \textit{(module)}!peach.fuzzy.control.Sugeno \textit{(class)}|(}
\subsection{Class Sugeno}

    \label{peach:fuzzy:control:Sugeno}
\begin{tabular}{cccccccc}
% Line for object, linespec=[False, False]
\multicolumn{2}{r}{\settowidth{\BCL}{object}\multirow{2}{\BCL}{object}}
&&
&&
  \\\cline{3-3}
  &&\multicolumn{1}{c|}{}
&&
&&
  \\
% Line for peach.fuzzy.control.Parametric, linespec=[False]
\multicolumn{4}{r}{\settowidth{\BCL}{peach.fuzzy.control.Parametric}\multirow{2}{\BCL}{peach.fuzzy.control.Parametric}}
&&
  \\\cline{5-5}
  &&&&\multicolumn{1}{c|}{}
&&
  \\
&&&&\multicolumn{2}{l}{\textbf{peach.fuzzy.control.Sugeno}}
\end{tabular}


\texttt{Sugeno} is an alias to \texttt{Parametric}

%%%%%%%%%%%%%%%%%%%%%%%%%%%%%%%%%%%%%%%%%%%%%%%%%%%%%%%%%%%%%%%%%%%%%%%%%%%
%%                                Methods                                %%
%%%%%%%%%%%%%%%%%%%%%%%%%%%%%%%%%%%%%%%%%%%%%%%%%%%%%%%%%%%%%%%%%%%%%%%%%%%

  \subsubsection{Methods}

    \label{peach:fuzzy:control:Parametric:__call__}
    \index{peach \textit{(package)}!peach.fuzzy \textit{(package)}!peach.fuzzy.control \textit{(module)}!peach.fuzzy.control.Parametric \textit{(class)}!peach.fuzzy.control.Parametric.\_\_call\_\_ \textit{(method)}}

    \vspace{0.5ex}

    \begin{boxedminipage}{\textwidth}

    \raggedright \textbf{\_\_call\_\_}(\textit{self}, *\textit{xs})

    \vspace{-1.5ex}

    \rule{\textwidth}{0.5\fboxrule}

Apply the controller to the set of input variables

Given the values of the input variables, evaluates every decision rule,
and calculates the weighted average of the results. Returns the response
of the controller.
    \vspace{1ex}

      \textbf{Parameters}
      \begin{quote}
        \begin{Ventry}{xx}

          \item[xs]


A tuple, a list or an array with the values of the input variables.
        \end{Ventry}

      \end{quote}

    \vspace{1ex}

      \textbf{Return Value}
      \begin{quote}

The response of the controller.
      \end{quote}

    \vspace{1ex}

    \end{boxedminipage}

    \label{object:__delattr__}
    \index{object.\_\_delattr\_\_ \textit{(function)}}

    \vspace{0.5ex}

    \begin{boxedminipage}{\textwidth}

    \raggedright \textbf{\_\_delattr\_\_}(\textit{...})

    \vspace{-1.5ex}

    \rule{\textwidth}{0.5\fboxrule}

x.{\_}{\_}delattr{\_}{\_}('name') {\textless}=={\textgreater} del x.name
    \vspace{1ex}

    \end{boxedminipage}

    \label{object:__getattribute__}
    \index{object.\_\_getattribute\_\_ \textit{(function)}}

    \vspace{0.5ex}

    \begin{boxedminipage}{\textwidth}

    \raggedright \textbf{\_\_getattribute\_\_}(\textit{...})

    \vspace{-1.5ex}

    \rule{\textwidth}{0.5\fboxrule}

x.{\_}{\_}getattribute{\_}{\_}('name') {\textless}=={\textgreater} x.name
    \vspace{1ex}

    \end{boxedminipage}

    \label{object:__hash__}
    \index{object.\_\_hash\_\_ \textit{(function)}}

    \vspace{0.5ex}

    \begin{boxedminipage}{\textwidth}

    \raggedright \textbf{\_\_hash\_\_}(\textit{x})

    \vspace{-1.5ex}

    \rule{\textwidth}{0.5\fboxrule}

hash(x)
    \vspace{1ex}

    \end{boxedminipage}

    \vspace{0.5ex}

    \begin{boxedminipage}{\textwidth}

    \raggedright \textbf{\_\_init\_\_}(\textit{self}, \textit{rules}=\texttt{\texttt{[}\texttt{]}}, \textit{norm}=\texttt{{\textless}function ProbabilisticAnd at 0x885ebc4{\textgreater}}, \textit{conorm}=\texttt{{\textless}function ProbabilisticOr at 0x885ebfc{\textgreater}}, \textit{negation}=\texttt{{\textless}function ProbabilisticNot at 0x885ec34{\textgreater}})

    \vspace{-1.5ex}

    \rule{\textwidth}{0.5\fboxrule}

Creates and initializes the controller.
    \vspace{1ex}

      \textbf{Parameters}
      \begin{quote}
        \begin{Ventry}{xxxxxxxx}

          \item[rules]


List containing the decision rules for the controller. If not given,
an empty set of decision rules is used.
          \item[norm]


The norm (\texttt{and} operation) to be used. Defaults to Probabilistic
and.
          \item[conorm]


The conorm (\texttt{or} operation) to be used. Defaults to Probabilistic
or.
          \item[negation]


The negation (\texttt{not} operation) to be used. Defaults to
Probabilistic not.
        \end{Ventry}

      \end{quote}

    \vspace{1ex}

      Overrides: object.\_\_init\_\_

    \end{boxedminipage}

    \label{object:__new__}
    \index{object.\_\_new\_\_ \textit{(function)}}

    \vspace{0.5ex}

    \begin{boxedminipage}{\textwidth}

    \raggedright \textbf{\_\_new\_\_}(\textit{T}, \textit{S}, \textit{...})

      \textbf{Return Value}
      \begin{quote}
\begin{alltt}
a new object with type S, a subtype of T
\end{alltt}

      \end{quote}

    \vspace{1ex}

    \end{boxedminipage}

    \label{object:__reduce__}
    \index{object.\_\_reduce\_\_ \textit{(function)}}

    \vspace{0.5ex}

    \begin{boxedminipage}{\textwidth}

    \raggedright \textbf{\_\_reduce\_\_}(\textit{...})

    \vspace{-1.5ex}

    \rule{\textwidth}{0.5\fboxrule}

helper for pickle
    \vspace{1ex}

    \end{boxedminipage}

    \label{object:__reduce_ex__}
    \index{object.\_\_reduce\_ex\_\_ \textit{(function)}}

    \vspace{0.5ex}

    \begin{boxedminipage}{\textwidth}

    \raggedright \textbf{\_\_reduce\_ex\_\_}(\textit{...})

    \vspace{-1.5ex}

    \rule{\textwidth}{0.5\fboxrule}

helper for pickle
    \vspace{1ex}

    \end{boxedminipage}

    \label{object:__repr__}
    \index{object.\_\_repr\_\_ \textit{(function)}}

    \vspace{0.5ex}

    \begin{boxedminipage}{\textwidth}

    \raggedright \textbf{\_\_repr\_\_}(\textit{x})

    \vspace{-1.5ex}

    \rule{\textwidth}{0.5\fboxrule}

repr(x)
    \vspace{1ex}

    \end{boxedminipage}

    \label{object:__setattr__}
    \index{object.\_\_setattr\_\_ \textit{(function)}}

    \vspace{0.5ex}

    \begin{boxedminipage}{\textwidth}

    \raggedright \textbf{\_\_setattr\_\_}(\textit{...})

    \vspace{-1.5ex}

    \rule{\textwidth}{0.5\fboxrule}

x.{\_}{\_}setattr{\_}{\_}('name', value) {\textless}=={\textgreater} x.name = value
    \vspace{1ex}

    \end{boxedminipage}

    \label{object:__str__}
    \index{object.\_\_str\_\_ \textit{(function)}}

    \vspace{0.5ex}

    \begin{boxedminipage}{\textwidth}

    \raggedright \textbf{\_\_str\_\_}(\textit{x})

    \vspace{-1.5ex}

    \rule{\textwidth}{0.5\fboxrule}

str(x)
    \vspace{1ex}

    \end{boxedminipage}

    \label{peach:fuzzy:control:Parametric:add_rule}
    \index{peach \textit{(package)}!peach.fuzzy \textit{(package)}!peach.fuzzy.control \textit{(module)}!peach.fuzzy.control.Parametric \textit{(class)}!peach.fuzzy.control.Parametric.add\_rule \textit{(method)}}

    \vspace{0.5ex}

    \begin{boxedminipage}{\textwidth}

    \raggedright \textbf{add\_rule}(\textit{self}, \textit{rule})

    \vspace{-1.5ex}

    \rule{\textwidth}{0.5\fboxrule}

Adds a decision rule to the knowledge base.

It is essential to understand the format that decision rules must follow
to obtain correct behaviour of the controller. A rule is a tuple given
by:
\begin{quote}{\ttfamily \raggedright \noindent
((mx0,~mx1,~...,~mxn),~(a0,~a1,~...,~an))
}\end{quote}

where \texttt{mx0} is a membership function of the first input variable,
\texttt{mx1} is a membership function of the second input variable and so on;
and \texttt{a0} is the linear parameter, \texttt{a1} is the parameter associated
with the first input variable, \texttt{a2} is the parameter associated with
the second input variable and so on.

Notice that \texttt{mx}'s are \emph{functions} not fuzzy sets! They will be
applied to the values of the input variables given in the function call,
so, if they are anything different from a membership function, an
exception will be raised. Please, consult the examples to see how they
must be used.
    \vspace{1ex}

    \end{boxedminipage}

    \label{peach:fuzzy:control:Parametric:eval}
    \index{peach \textit{(package)}!peach.fuzzy \textit{(package)}!peach.fuzzy.control \textit{(module)}!peach.fuzzy.control.Parametric \textit{(class)}!peach.fuzzy.control.Parametric.eval \textit{(method)}}

    \vspace{0.5ex}

    \begin{boxedminipage}{\textwidth}

    \raggedright \textbf{eval}(\textit{self}, \textit{r}, \textit{xs})

    \vspace{-1.5ex}

    \rule{\textwidth}{0.5\fboxrule}

Evaluates one decision rule in this controller

Takes a rule from the controller and evaluates it given the values of
the input variables. The format of the rule is as given, and the
response to the rule is calculated by:
\begin{quote}{\ttfamily \raggedright \noindent
y~=~a0~+~a1*x1~+~a2*x2~+~...~+~an*xn
}\end{quote}
    \vspace{1ex}

      \textbf{Parameters}
      \begin{quote}
        \begin{Ventry}{xx}

          \item[r]


The rule in the standard format, or an integer number. If \texttt{r} is
an integer, then the \texttt{r} th rule in the knowledge base will be
evaluated.
          \item[xs]


A tuple, a list or an array containing the values of the input
variables. The dimension must be coherent with the given rule.
        \end{Ventry}

      \end{quote}

    \vspace{1ex}

      \textbf{Return Value}
      \begin{quote}

This method evaluates each membership function in the rule for each
given value, and \texttt{and} 's the results to obtain the condition. If
the condition is zero, a tuple \texttt{(0.0, 0.0) is returned. Otherwise,
the result as given above is calculate, and a tuple containing
`{}`(condition, result)} (the membership value associated to the
condition and the result of the calculation) is returned.
      \end{quote}

    \vspace{1ex}

    \end{boxedminipage}


%%%%%%%%%%%%%%%%%%%%%%%%%%%%%%%%%%%%%%%%%%%%%%%%%%%%%%%%%%%%%%%%%%%%%%%%%%%
%%                              Properties                               %%
%%%%%%%%%%%%%%%%%%%%%%%%%%%%%%%%%%%%%%%%%%%%%%%%%%%%%%%%%%%%%%%%%%%%%%%%%%%

  \subsubsection{Properties}

\begin{longtable}{|p{.30\textwidth}|p{.62\textwidth}|l}
\cline{1-2}
\cline{1-2} \centering \textbf{Name} & \centering \textbf{Description}& \\
\cline{1-2}
\endhead\cline{1-2}\multicolumn{3}{r}{\small\textit{continued on next page}}\\\endfoot\cline{1-2}
\endlastfoot\raggedright \_\-\_\-c\-l\-a\-s\-s\-\_\-\_\- & \raggedright \textbf{Value:} 
{\tt {\textless}attribute '\_\_class\_\_' of 'object' objects{\textgreater}}&\\
\cline{1-2}
\raggedright r\-u\-l\-e\-s\- & \raggedright Property that returns the list of decision rules. Not writable

\textbf{Value:} 
{\tt {\textless}property object at 0x887bedc{\textgreater}}&\\
\cline{1-2}
\end{longtable}

    \index{peach \textit{(package)}!peach.fuzzy \textit{(package)}!peach.fuzzy.control \textit{(module)}!peach.fuzzy.control.Sugeno \textit{(class)}|)}
    \index{peach \textit{(package)}!peach.fuzzy \textit{(package)}!peach.fuzzy.control \textit{(module)}|)}

%
% API Documentation for Peach - Computational Intelligence for Python
% Module peach.fuzzy.defuzzy
%
% Generated by epydoc 3.0beta1
% [Mon Dec 21 08:51:35 2009]
%

%%%%%%%%%%%%%%%%%%%%%%%%%%%%%%%%%%%%%%%%%%%%%%%%%%%%%%%%%%%%%%%%%%%%%%%%%%%
%%                          Module Description                           %%
%%%%%%%%%%%%%%%%%%%%%%%%%%%%%%%%%%%%%%%%%%%%%%%%%%%%%%%%%%%%%%%%%%%%%%%%%%%

    \index{peach \textit{(package)}!peach.fuzzy \textit{(package)}!peach.fuzzy.defuzzy \textit{(module)}|(}
\section{Module peach.fuzzy.defuzzy}

    \label{peach:fuzzy:defuzzy}

This package implements defuzzification methods for use with fuzzy controllers.

Defuzzification methods take a set of numerical values, their corresponding
fuzzy membership values and calculate a defuzzified value for them. They're
implemented as functions, not as classes. So, to implement your own, use the
directions below.

These methods are implemented as functions with the signature \texttt{(mf, y)}, where
\texttt{mf} is the fuzzy set, and \texttt{y} is an array of values. That is, \texttt{mf} is a
fuzzy set containing the membership values of each one in the \texttt{y} array, in
the respective order. Both arrays should have the same dimensions, or else the
methods won't work.

See the example:
\begin{quote}{\ttfamily \raggedright \noindent
>{}>{}>~import~numpy~\\
>{}>{}>~from~peach~import~*~\\
>{}>{}>~y~=~numpy.linspace(0.,~5.,~100)~\\
>{}>{}>~m{\_}y~=~Triangle(1.,~2.,~3.)~\\
>{}>{}>~Centroid(m{\_}y(y),~y)~\\
2.0001030715316435
}\end{quote}

The methods defined here are the most commonly used.

%%%%%%%%%%%%%%%%%%%%%%%%%%%%%%%%%%%%%%%%%%%%%%%%%%%%%%%%%%%%%%%%%%%%%%%%%%%
%%                               Functions                               %%
%%%%%%%%%%%%%%%%%%%%%%%%%%%%%%%%%%%%%%%%%%%%%%%%%%%%%%%%%%%%%%%%%%%%%%%%%%%

  \subsection{Functions}

    \label{peach:fuzzy:defuzzy:Centroid}
    \index{peach \textit{(package)}!peach.fuzzy \textit{(package)}!peach.fuzzy.defuzzy \textit{(module)}!peach.fuzzy.defuzzy.Centroid \textit{(function)}}

    \vspace{0.5ex}

    \begin{boxedminipage}{\textwidth}

    \raggedright \textbf{Centroid}(\textit{mf}, \textit{y})

    \vspace{-1.5ex}

    \rule{\textwidth}{0.5\fboxrule}

Center of gravity method.

The center of gravity is calculate using the standard formula found in any
calculus book. The integrals are calculated using the trapezoid method.
    \vspace{1ex}

      \textbf{Parameters}
      \begin{quote}
        \begin{Ventry}{xx}

          \item[mf]


Fuzzy set containing the membership values of the elements in the
vector given in sequence
          \item[y]


Array of domain values of the defuzzified variable.
        \end{Ventry}

      \end{quote}

    \vspace{1ex}

      \textbf{Return Value}
      \begin{quote}

The center of gravity of the fuzzy set.
      \end{quote}

    \vspace{1ex}

    \end{boxedminipage}

    \label{peach:fuzzy:defuzzy:Bissector}
    \index{peach \textit{(package)}!peach.fuzzy \textit{(package)}!peach.fuzzy.defuzzy \textit{(module)}!peach.fuzzy.defuzzy.Bissector \textit{(function)}}

    \vspace{0.5ex}

    \begin{boxedminipage}{\textwidth}

    \raggedright \textbf{Bissector}(\textit{mf}, \textit{y})

    \vspace{-1.5ex}

    \rule{\textwidth}{0.5\fboxrule}

Bissection method

The bissection method finds a coordinate \texttt{y} in domain that divides the
fuzzy set in two subsets with the same area. Integrals are calculated using
the trapezoid method. This method only works if the values in \texttt{y} are
equally spaced, otherwise, the method will fail.
    \vspace{1ex}

      \textbf{Parameters}
      \begin{quote}
        \begin{Ventry}{xx}

          \item[mf]


Fuzzy set containing the membership values of the elements in the
vector given in sequence
          \item[y]


Array of domain values of the defuzzified variable.
        \end{Ventry}

      \end{quote}

    \vspace{1ex}

      \textbf{Return Value}
      \begin{quote}

Defuzzified value by the bissection method.
      \end{quote}

    \vspace{1ex}

    \end{boxedminipage}

    \label{peach:fuzzy:defuzzy:SmallestOfMaxima}
    \index{peach \textit{(package)}!peach.fuzzy \textit{(package)}!peach.fuzzy.defuzzy \textit{(module)}!peach.fuzzy.defuzzy.SmallestOfMaxima \textit{(function)}}

    \vspace{0.5ex}

    \begin{boxedminipage}{\textwidth}

    \raggedright \textbf{SmallestOfMaxima}(\textit{mf}, \textit{y})

    \vspace{-1.5ex}

    \rule{\textwidth}{0.5\fboxrule}

Smallest of maxima method.

This method finds all the points in the domain which have maximum membership
value in the fuzzy set, and returns the smallest of them.
    \vspace{1ex}

      \textbf{Parameters}
      \begin{quote}
        \begin{Ventry}{xx}

          \item[mf]


Fuzzy set containing the membership values of the elements in the
vector given in sequence
          \item[y]


Array of domain values of the defuzzified variable.
        \end{Ventry}

      \end{quote}

    \vspace{1ex}

      \textbf{Return Value}
      \begin{quote}

Defuzzified value by the smallest of maxima method.
      \end{quote}

    \vspace{1ex}

    \end{boxedminipage}

    \label{peach:fuzzy:defuzzy:LargestOfMaxima}
    \index{peach \textit{(package)}!peach.fuzzy \textit{(package)}!peach.fuzzy.defuzzy \textit{(module)}!peach.fuzzy.defuzzy.LargestOfMaxima \textit{(function)}}

    \vspace{0.5ex}

    \begin{boxedminipage}{\textwidth}

    \raggedright \textbf{LargestOfMaxima}(\textit{mf}, \textit{y})

    \vspace{-1.5ex}

    \rule{\textwidth}{0.5\fboxrule}

Largest of maxima method.

This method finds all the points in the domain which have maximum membership
value in the fuzzy set, and returns the largest of them.
    \vspace{1ex}

      \textbf{Parameters}
      \begin{quote}
        \begin{Ventry}{xx}

          \item[mf]


Fuzzy set containing the membership values of the elements in the
vector given in sequence
          \item[y]


Array of domain values of the defuzzified variable.
        \end{Ventry}

      \end{quote}

    \vspace{1ex}

      \textbf{Return Value}
      \begin{quote}

Defuzzified value by the largest of maxima method.
      \end{quote}

    \vspace{1ex}

    \end{boxedminipage}

    \label{peach:fuzzy:defuzzy:MeanOfMaxima}
    \index{peach \textit{(package)}!peach.fuzzy \textit{(package)}!peach.fuzzy.defuzzy \textit{(module)}!peach.fuzzy.defuzzy.MeanOfMaxima \textit{(function)}}

    \vspace{0.5ex}

    \begin{boxedminipage}{\textwidth}

    \raggedright \textbf{MeanOfMaxima}(\textit{mf}, \textit{y})

    \vspace{-1.5ex}

    \rule{\textwidth}{0.5\fboxrule}

Mean of maxima method.

This method finds the smallest and largest of maxima, and returns their
average.
    \vspace{1ex}

      \textbf{Parameters}
      \begin{quote}
        \begin{Ventry}{xx}

          \item[mf]


Fuzzy set containing the membership values of the elements in the
vector given in sequence
          \item[y]


Array of domain values of the defuzzified variable.
        \end{Ventry}

      \end{quote}

    \vspace{1ex}

      \textbf{Return Value}
      \begin{quote}

Defuzzified value by the  of maxima method.
      \end{quote}

    \vspace{1ex}

    \end{boxedminipage}


%%%%%%%%%%%%%%%%%%%%%%%%%%%%%%%%%%%%%%%%%%%%%%%%%%%%%%%%%%%%%%%%%%%%%%%%%%%
%%                               Variables                               %%
%%%%%%%%%%%%%%%%%%%%%%%%%%%%%%%%%%%%%%%%%%%%%%%%%%%%%%%%%%%%%%%%%%%%%%%%%%%

  \subsection{Variables}

\begin{longtable}{|p{.30\textwidth}|p{.62\textwidth}|l}
\cline{1-2}
\cline{1-2} \centering \textbf{Name} & \centering \textbf{Description}& \\
\cline{1-2}
\endhead\cline{1-2}\multicolumn{3}{r}{\small\textit{continued on next page}}\\\endfoot\cline{1-2}
\endlastfoot\raggedright \_\-\_\-d\-o\-c\-\_\-\_\- & \raggedright \textbf{Value:} 
{\tt \texttt{...}}&\\
\cline{1-2}
\end{longtable}

    \index{peach \textit{(package)}!peach.fuzzy \textit{(package)}!peach.fuzzy.defuzzy \textit{(module)}|)}

%
% API Documentation for Peach - Computational Intelligence for Python
% Module peach.fuzzy.mf
%
% Generated by epydoc 3.0beta1
% [Mon Dec 21 08:51:36 2009]
%

%%%%%%%%%%%%%%%%%%%%%%%%%%%%%%%%%%%%%%%%%%%%%%%%%%%%%%%%%%%%%%%%%%%%%%%%%%%
%%                          Module Description                           %%
%%%%%%%%%%%%%%%%%%%%%%%%%%%%%%%%%%%%%%%%%%%%%%%%%%%%%%%%%%%%%%%%%%%%%%%%%%%

    \index{peach \textit{(package)}!peach.fuzzy \textit{(package)}!peach.fuzzy.mf \textit{(module)}|(}
\section{Module peach.fuzzy.mf}

    \label{peach:fuzzy:mf}

Membership functions

Membership functions are actually subclasses of a main class called Membership,
see below. Instantiate a class to generate a function, optional arguments can be
specified to configure the function as needed. For example, to create a triangle
function starting at 0, with peak in 3, and ending in 4, use:
\begin{quote}{\ttfamily \raggedright \noindent
mu~=~Triangle(0,~3,~4)
}\end{quote}

Please notice that the return value is a \emph{function}. To use it, apply it as a
normal function. For example, the function above, applied to the value 1.5
should return 0.5:
\begin{quote}{\ttfamily \raggedright \noindent
>{}>{}>~print~mu(1.5)~\\
0.5
}\end{quote}

%%%%%%%%%%%%%%%%%%%%%%%%%%%%%%%%%%%%%%%%%%%%%%%%%%%%%%%%%%%%%%%%%%%%%%%%%%%
%%                               Functions                               %%
%%%%%%%%%%%%%%%%%%%%%%%%%%%%%%%%%%%%%%%%%%%%%%%%%%%%%%%%%%%%%%%%%%%%%%%%%%%

  \subsection{Functions}

    \label{peach:fuzzy:mf:Saw}
    \index{peach \textit{(package)}!peach.fuzzy \textit{(package)}!peach.fuzzy.mf \textit{(module)}!peach.fuzzy.mf.Saw \textit{(function)}}

    \vspace{0.5ex}

    \begin{boxedminipage}{\textwidth}

    \raggedright \textbf{Saw}(\textit{interval}, \textit{n})

    \vspace{-1.5ex}

    \rule{\textwidth}{0.5\fboxrule}

Splits an \texttt{interval} into \texttt{n} triangle functions.

Given an interval in any domain, this function will create \texttt{n} triangle
functions of the same size equally spaced in the interval. It is very
useful to create membership functions for controllers. The command below
will create 3 triangle functions equally spaced in the interval (0, 4):
\begin{quote}{\ttfamily \raggedright \noindent
mf1,~mf2,~mf3~=~Saw((0,~4),~3)
}\end{quote}

This is the same as the following commands:
\begin{quote}{\ttfamily \raggedright \noindent
mf1~=~Triangle(0,~1,~2)~\\
mf2~=~Triangle(1,~2,~3)~\\
mf3~=~Triangle(2,~3,~4)
}\end{quote}
    \vspace{1ex}

      \textbf{Parameters}
      \begin{quote}
        \begin{Ventry}{xxxxxxxx}

          \item[interval]


A tuple containing the start and the end of the interval, in the format
\texttt{(start, end)};
          \item[n]


The number of functions in which the interval must be split.
        \end{Ventry}

      \end{quote}

    \vspace{1ex}

      \textbf{Return Value}
      \begin{quote}

A list of triangle membership functions, in order.
      \end{quote}

    \vspace{1ex}

    \end{boxedminipage}

    \label{peach:fuzzy:mf:FlatSaw}
    \index{peach \textit{(package)}!peach.fuzzy \textit{(package)}!peach.fuzzy.mf \textit{(module)}!peach.fuzzy.mf.FlatSaw \textit{(function)}}

    \vspace{0.5ex}

    \begin{boxedminipage}{\textwidth}

    \raggedright \textbf{FlatSaw}(\textit{interval}, \textit{n})

    \vspace{-1.5ex}

    \rule{\textwidth}{0.5\fboxrule}

Splits an \texttt{interval} into a decreasing ramp, \texttt{n-2} triangle functions
and an increasing ramp.

Given an interval in any domain, this function will create a decreasing ramp
in the start of the interval, \texttt{n-2} triangle functions of the same size
equally spaced in the interval, and a increasing ramp in the end of the
interval. It is very useful to create membership functions for controllers.
The command below will create a decreasing ramp, a triangle function and an
increasing ramp equally spaced in the interval (0, 2):
\begin{quote}{\ttfamily \raggedright \noindent
mf1,~mf2,~mf3~=~FlatSaw((0,~2),~3)
}\end{quote}

This is the same as the following commands:
\begin{quote}{\ttfamily \raggedright \noindent
mf1~=~DecreasingRamp(0,~1)~\\
mf2~=~Triangle(0,~1,~2)~\\
mf3~=~Increasingramp(1,~2)
}\end{quote}
    \vspace{1ex}

      \textbf{Parameters}
      \begin{quote}
        \begin{Ventry}{xxxxxxxx}

          \item[interval]


A tuple containing the start and the end of the interval, in the format
\texttt{(start, end)};
          \item[n]


The number of functions in which the interval must be split.
        \end{Ventry}

      \end{quote}

    \vspace{1ex}

      \textbf{Return Value}
      \begin{quote}

A list of corresponding functions, in order.
      \end{quote}

    \vspace{1ex}

    \end{boxedminipage}


%%%%%%%%%%%%%%%%%%%%%%%%%%%%%%%%%%%%%%%%%%%%%%%%%%%%%%%%%%%%%%%%%%%%%%%%%%%
%%                               Variables                               %%
%%%%%%%%%%%%%%%%%%%%%%%%%%%%%%%%%%%%%%%%%%%%%%%%%%%%%%%%%%%%%%%%%%%%%%%%%%%

  \subsection{Variables}

\begin{longtable}{|p{.30\textwidth}|p{.62\textwidth}|l}
\cline{1-2}
\cline{1-2} \centering \textbf{Name} & \centering \textbf{Description}& \\
\cline{1-2}
\endhead\cline{1-2}\multicolumn{3}{r}{\small\textit{continued on next page}}\\\endfoot\cline{1-2}
\endlastfoot\raggedright \_\-\_\-d\-o\-c\-\_\-\_\- & \raggedright \textbf{Value:} 
{\tt \texttt{...}}&\\
\cline{1-2}
\end{longtable}


%%%%%%%%%%%%%%%%%%%%%%%%%%%%%%%%%%%%%%%%%%%%%%%%%%%%%%%%%%%%%%%%%%%%%%%%%%%
%%                           Class Description                           %%
%%%%%%%%%%%%%%%%%%%%%%%%%%%%%%%%%%%%%%%%%%%%%%%%%%%%%%%%%%%%%%%%%%%%%%%%%%%

    \index{peach \textit{(package)}!peach.fuzzy \textit{(package)}!peach.fuzzy.mf \textit{(module)}!peach.fuzzy.mf.Membership \textit{(class)}|(}
\subsection{Class Membership}

    \label{peach:fuzzy:mf:Membership}
\begin{tabular}{cccccc}
% Line for object, linespec=[False]
\multicolumn{2}{r}{\settowidth{\BCL}{object}\multirow{2}{\BCL}{object}}
&&
  \\\cline{3-3}
  &&\multicolumn{1}{c|}{}
&&
  \\
&&\multicolumn{2}{l}{\textbf{peach.fuzzy.mf.Membership}}
\end{tabular}

\textbf{Known Subclasses:}
peach.fuzzy.mf.Bell,
    peach.fuzzy.mf.DecreasingRamp,
    peach.fuzzy.mf.DecreasingSigmoid,
    peach.fuzzy.mf.Gaussian,
    peach.fuzzy.mf.IncreasingRamp,
    peach.fuzzy.mf.IncreasingSigmoid,
    peach.fuzzy.mf.RaisedCosine,
    peach.fuzzy.mf.Trapezoid,
    peach.fuzzy.mf.Triangle


Base class of all membership functions.

This class is used as base of the implemented membership functions, and can
also be used to transform a regular function in a membership function that
can be used with the fuzzy logic package.

To create a membership function from a regular function \texttt{f}, use:
\begin{quote}{\ttfamily \raggedright \noindent
mf~=~Membership(f)
}\end{quote}

A function this converted can be used with vectors and matrices and always
return a FuzzySet object. Notice that the value range is not verified so
that it fits in the range {[} 0, 1 {]}. It is responsibility of the programmer
to warrant that.

To subclass Membership, just use it as a base class. It is suggested that
the \texttt{{\_}{\_}init{\_}{\_}} method of the derived class allows configuration, and the
\texttt{{\_}{\_}call{\_}{\_}} method is used to apply the function over its arguments.

%%%%%%%%%%%%%%%%%%%%%%%%%%%%%%%%%%%%%%%%%%%%%%%%%%%%%%%%%%%%%%%%%%%%%%%%%%%
%%                                Methods                                %%
%%%%%%%%%%%%%%%%%%%%%%%%%%%%%%%%%%%%%%%%%%%%%%%%%%%%%%%%%%%%%%%%%%%%%%%%%%%

  \subsubsection{Methods}

    \vspace{0.5ex}

    \begin{boxedminipage}{\textwidth}

    \raggedright \textbf{\_\_init\_\_}(\textit{self}, \textit{f})

    \vspace{-1.5ex}

    \rule{\textwidth}{0.5\fboxrule}

Builds a membership function from a regular function
    \vspace{1ex}

      \textbf{Parameters}
      \begin{quote}
        \begin{Ventry}{x}

          \item[f]


Function to be transformed into a membership function. It must be
given, and it must be a \texttt{FunctionType} object, otherwise, a
\texttt{ValueError} is raised.
        \end{Ventry}

      \end{quote}

    \vspace{1ex}

      Overrides: object.\_\_init\_\_

    \end{boxedminipage}

    \label{peach:fuzzy:mf:Membership:__call__}
    \index{peach \textit{(package)}!peach.fuzzy \textit{(package)}!peach.fuzzy.mf \textit{(module)}!peach.fuzzy.mf.Membership \textit{(class)}!peach.fuzzy.mf.Membership.\_\_call\_\_ \textit{(method)}}

    \vspace{0.5ex}

    \begin{boxedminipage}{\textwidth}

    \raggedright \textbf{\_\_call\_\_}(\textit{self}, \textit{x})

    \vspace{-1.5ex}

    \rule{\textwidth}{0.5\fboxrule}

Maps the function on a vector
    \vspace{1ex}

      \textbf{Parameters}
      \begin{quote}
        \begin{Ventry}{x}

          \item[x]


A value, vector or matrix over which the function is evaluated.
        \end{Ventry}

      \end{quote}

    \vspace{1ex}

      \textbf{Return Value}
      \begin{quote}

A \texttt{FuzzySet} object containing the evaluation of the function over
each of the components of the input.
      \end{quote}

    \vspace{1ex}

    \end{boxedminipage}

    \label{object:__delattr__}
    \index{object.\_\_delattr\_\_ \textit{(function)}}

    \vspace{0.5ex}

    \begin{boxedminipage}{\textwidth}

    \raggedright \textbf{\_\_delattr\_\_}(\textit{...})

    \vspace{-1.5ex}

    \rule{\textwidth}{0.5\fboxrule}

x.{\_}{\_}delattr{\_}{\_}('name') {\textless}=={\textgreater} del x.name
    \vspace{1ex}

    \end{boxedminipage}

    \label{object:__getattribute__}
    \index{object.\_\_getattribute\_\_ \textit{(function)}}

    \vspace{0.5ex}

    \begin{boxedminipage}{\textwidth}

    \raggedright \textbf{\_\_getattribute\_\_}(\textit{...})

    \vspace{-1.5ex}

    \rule{\textwidth}{0.5\fboxrule}

x.{\_}{\_}getattribute{\_}{\_}('name') {\textless}=={\textgreater} x.name
    \vspace{1ex}

    \end{boxedminipage}

    \label{object:__hash__}
    \index{object.\_\_hash\_\_ \textit{(function)}}

    \vspace{0.5ex}

    \begin{boxedminipage}{\textwidth}

    \raggedright \textbf{\_\_hash\_\_}(\textit{x})

    \vspace{-1.5ex}

    \rule{\textwidth}{0.5\fboxrule}

hash(x)
    \vspace{1ex}

    \end{boxedminipage}

    \label{object:__new__}
    \index{object.\_\_new\_\_ \textit{(function)}}

    \vspace{0.5ex}

    \begin{boxedminipage}{\textwidth}

    \raggedright \textbf{\_\_new\_\_}(\textit{T}, \textit{S}, \textit{...})

      \textbf{Return Value}
      \begin{quote}
\begin{alltt}
a new object with type S, a subtype of T
\end{alltt}

      \end{quote}

    \vspace{1ex}

    \end{boxedminipage}

    \label{object:__reduce__}
    \index{object.\_\_reduce\_\_ \textit{(function)}}

    \vspace{0.5ex}

    \begin{boxedminipage}{\textwidth}

    \raggedright \textbf{\_\_reduce\_\_}(\textit{...})

    \vspace{-1.5ex}

    \rule{\textwidth}{0.5\fboxrule}

helper for pickle
    \vspace{1ex}

    \end{boxedminipage}

    \label{object:__reduce_ex__}
    \index{object.\_\_reduce\_ex\_\_ \textit{(function)}}

    \vspace{0.5ex}

    \begin{boxedminipage}{\textwidth}

    \raggedright \textbf{\_\_reduce\_ex\_\_}(\textit{...})

    \vspace{-1.5ex}

    \rule{\textwidth}{0.5\fboxrule}

helper for pickle
    \vspace{1ex}

    \end{boxedminipage}

    \label{object:__repr__}
    \index{object.\_\_repr\_\_ \textit{(function)}}

    \vspace{0.5ex}

    \begin{boxedminipage}{\textwidth}

    \raggedright \textbf{\_\_repr\_\_}(\textit{x})

    \vspace{-1.5ex}

    \rule{\textwidth}{0.5\fboxrule}

repr(x)
    \vspace{1ex}

    \end{boxedminipage}

    \label{object:__setattr__}
    \index{object.\_\_setattr\_\_ \textit{(function)}}

    \vspace{0.5ex}

    \begin{boxedminipage}{\textwidth}

    \raggedright \textbf{\_\_setattr\_\_}(\textit{...})

    \vspace{-1.5ex}

    \rule{\textwidth}{0.5\fboxrule}

x.{\_}{\_}setattr{\_}{\_}('name', value) {\textless}=={\textgreater} x.name = value
    \vspace{1ex}

    \end{boxedminipage}

    \label{object:__str__}
    \index{object.\_\_str\_\_ \textit{(function)}}

    \vspace{0.5ex}

    \begin{boxedminipage}{\textwidth}

    \raggedright \textbf{\_\_str\_\_}(\textit{x})

    \vspace{-1.5ex}

    \rule{\textwidth}{0.5\fboxrule}

str(x)
    \vspace{1ex}

    \end{boxedminipage}


%%%%%%%%%%%%%%%%%%%%%%%%%%%%%%%%%%%%%%%%%%%%%%%%%%%%%%%%%%%%%%%%%%%%%%%%%%%
%%                              Properties                               %%
%%%%%%%%%%%%%%%%%%%%%%%%%%%%%%%%%%%%%%%%%%%%%%%%%%%%%%%%%%%%%%%%%%%%%%%%%%%

  \subsubsection{Properties}

\begin{longtable}{|p{.30\textwidth}|p{.62\textwidth}|l}
\cline{1-2}
\cline{1-2} \centering \textbf{Name} & \centering \textbf{Description}& \\
\cline{1-2}
\endhead\cline{1-2}\multicolumn{3}{r}{\small\textit{continued on next page}}\\\endfoot\cline{1-2}
\endlastfoot\raggedright \_\-\_\-c\-l\-a\-s\-s\-\_\-\_\- & \raggedright \textbf{Value:} 
{\tt {\textless}attribute '\_\_class\_\_' of 'object' objects{\textgreater}}&\\
\cline{1-2}
\end{longtable}

    \index{peach \textit{(package)}!peach.fuzzy \textit{(package)}!peach.fuzzy.mf \textit{(module)}!peach.fuzzy.mf.Membership \textit{(class)}|)}

%%%%%%%%%%%%%%%%%%%%%%%%%%%%%%%%%%%%%%%%%%%%%%%%%%%%%%%%%%%%%%%%%%%%%%%%%%%
%%                           Class Description                           %%
%%%%%%%%%%%%%%%%%%%%%%%%%%%%%%%%%%%%%%%%%%%%%%%%%%%%%%%%%%%%%%%%%%%%%%%%%%%

    \index{peach \textit{(package)}!peach.fuzzy \textit{(package)}!peach.fuzzy.mf \textit{(module)}!peach.fuzzy.mf.IncreasingRamp \textit{(class)}|(}
\subsection{Class IncreasingRamp}

    \label{peach:fuzzy:mf:IncreasingRamp}
\begin{tabular}{cccccccc}
% Line for object, linespec=[False, False]
\multicolumn{2}{r}{\settowidth{\BCL}{object}\multirow{2}{\BCL}{object}}
&&
&&
  \\\cline{3-3}
  &&\multicolumn{1}{c|}{}
&&
&&
  \\
% Line for peach.fuzzy.mf.Membership, linespec=[False]
\multicolumn{4}{r}{\settowidth{\BCL}{peach.fuzzy.mf.Membership}\multirow{2}{\BCL}{peach.fuzzy.mf.Membership}}
&&
  \\\cline{5-5}
  &&&&\multicolumn{1}{c|}{}
&&
  \\
&&&&\multicolumn{2}{l}{\textbf{peach.fuzzy.mf.IncreasingRamp}}
\end{tabular}


Increasing ramp.

Given two points, \texttt{x0} and \texttt{x1}, with \texttt{x0 < x1}, creates a function
which returns:
\begin{quote}

0, if \texttt{x <= x0};

\texttt{(x - x0) / (x1 - x0)}, if \texttt{x0 < x <= x1};

1, if \texttt{x > x1}.
\end{quote}

%%%%%%%%%%%%%%%%%%%%%%%%%%%%%%%%%%%%%%%%%%%%%%%%%%%%%%%%%%%%%%%%%%%%%%%%%%%
%%                                Methods                                %%
%%%%%%%%%%%%%%%%%%%%%%%%%%%%%%%%%%%%%%%%%%%%%%%%%%%%%%%%%%%%%%%%%%%%%%%%%%%

  \subsubsection{Methods}

    \vspace{0.5ex}

    \begin{boxedminipage}{\textwidth}

    \raggedright \textbf{\_\_init\_\_}(\textit{self}, \textit{x0}, \textit{x1})

    \vspace{-1.5ex}

    \rule{\textwidth}{0.5\fboxrule}

Initializes the function.
    \vspace{1ex}

      \textbf{Parameters}
      \begin{quote}
        \begin{Ventry}{xx}

          \item[x0]


Start of the ramp;
          \item[x1]


End of the ramp.
        \end{Ventry}

      \end{quote}

    \vspace{1ex}

      Overrides: peach.fuzzy.mf.Membership.\_\_init\_\_

    \end{boxedminipage}

    \vspace{0.5ex}

    \begin{boxedminipage}{\textwidth}

    \raggedright \textbf{\_\_call\_\_}(\textit{self}, \textit{x})


Maps the function on a vector
    \vspace{1ex}

      \textbf{Return Value}
      \begin{quote}

A \texttt{FuzzySet} object containing the evaluation of the function over
each of the components of the input.
      \end{quote}

    \vspace{1ex}

      Overrides: peach.fuzzy.mf.Membership.\_\_call\_\_ 	extit{(inherited documentation)}

    \end{boxedminipage}

    \label{object:__delattr__}
    \index{object.\_\_delattr\_\_ \textit{(function)}}

    \vspace{0.5ex}

    \begin{boxedminipage}{\textwidth}

    \raggedright \textbf{\_\_delattr\_\_}(\textit{...})

    \vspace{-1.5ex}

    \rule{\textwidth}{0.5\fboxrule}

x.{\_}{\_}delattr{\_}{\_}('name') {\textless}=={\textgreater} del x.name
    \vspace{1ex}

    \end{boxedminipage}

    \label{object:__getattribute__}
    \index{object.\_\_getattribute\_\_ \textit{(function)}}

    \vspace{0.5ex}

    \begin{boxedminipage}{\textwidth}

    \raggedright \textbf{\_\_getattribute\_\_}(\textit{...})

    \vspace{-1.5ex}

    \rule{\textwidth}{0.5\fboxrule}

x.{\_}{\_}getattribute{\_}{\_}('name') {\textless}=={\textgreater} x.name
    \vspace{1ex}

    \end{boxedminipage}

    \label{object:__hash__}
    \index{object.\_\_hash\_\_ \textit{(function)}}

    \vspace{0.5ex}

    \begin{boxedminipage}{\textwidth}

    \raggedright \textbf{\_\_hash\_\_}(\textit{x})

    \vspace{-1.5ex}

    \rule{\textwidth}{0.5\fboxrule}

hash(x)
    \vspace{1ex}

    \end{boxedminipage}

    \label{object:__new__}
    \index{object.\_\_new\_\_ \textit{(function)}}

    \vspace{0.5ex}

    \begin{boxedminipage}{\textwidth}

    \raggedright \textbf{\_\_new\_\_}(\textit{T}, \textit{S}, \textit{...})

      \textbf{Return Value}
      \begin{quote}
\begin{alltt}
a new object with type S, a subtype of T
\end{alltt}

      \end{quote}

    \vspace{1ex}

    \end{boxedminipage}

    \label{object:__reduce__}
    \index{object.\_\_reduce\_\_ \textit{(function)}}

    \vspace{0.5ex}

    \begin{boxedminipage}{\textwidth}

    \raggedright \textbf{\_\_reduce\_\_}(\textit{...})

    \vspace{-1.5ex}

    \rule{\textwidth}{0.5\fboxrule}

helper for pickle
    \vspace{1ex}

    \end{boxedminipage}

    \label{object:__reduce_ex__}
    \index{object.\_\_reduce\_ex\_\_ \textit{(function)}}

    \vspace{0.5ex}

    \begin{boxedminipage}{\textwidth}

    \raggedright \textbf{\_\_reduce\_ex\_\_}(\textit{...})

    \vspace{-1.5ex}

    \rule{\textwidth}{0.5\fboxrule}

helper for pickle
    \vspace{1ex}

    \end{boxedminipage}

    \label{object:__repr__}
    \index{object.\_\_repr\_\_ \textit{(function)}}

    \vspace{0.5ex}

    \begin{boxedminipage}{\textwidth}

    \raggedright \textbf{\_\_repr\_\_}(\textit{x})

    \vspace{-1.5ex}

    \rule{\textwidth}{0.5\fboxrule}

repr(x)
    \vspace{1ex}

    \end{boxedminipage}

    \label{object:__setattr__}
    \index{object.\_\_setattr\_\_ \textit{(function)}}

    \vspace{0.5ex}

    \begin{boxedminipage}{\textwidth}

    \raggedright \textbf{\_\_setattr\_\_}(\textit{...})

    \vspace{-1.5ex}

    \rule{\textwidth}{0.5\fboxrule}

x.{\_}{\_}setattr{\_}{\_}('name', value) {\textless}=={\textgreater} x.name = value
    \vspace{1ex}

    \end{boxedminipage}

    \label{object:__str__}
    \index{object.\_\_str\_\_ \textit{(function)}}

    \vspace{0.5ex}

    \begin{boxedminipage}{\textwidth}

    \raggedright \textbf{\_\_str\_\_}(\textit{x})

    \vspace{-1.5ex}

    \rule{\textwidth}{0.5\fboxrule}

str(x)
    \vspace{1ex}

    \end{boxedminipage}


%%%%%%%%%%%%%%%%%%%%%%%%%%%%%%%%%%%%%%%%%%%%%%%%%%%%%%%%%%%%%%%%%%%%%%%%%%%
%%                              Properties                               %%
%%%%%%%%%%%%%%%%%%%%%%%%%%%%%%%%%%%%%%%%%%%%%%%%%%%%%%%%%%%%%%%%%%%%%%%%%%%

  \subsubsection{Properties}

\begin{longtable}{|p{.30\textwidth}|p{.62\textwidth}|l}
\cline{1-2}
\cline{1-2} \centering \textbf{Name} & \centering \textbf{Description}& \\
\cline{1-2}
\endhead\cline{1-2}\multicolumn{3}{r}{\small\textit{continued on next page}}\\\endfoot\cline{1-2}
\endlastfoot\raggedright \_\-\_\-c\-l\-a\-s\-s\-\_\-\_\- & \raggedright \textbf{Value:} 
{\tt {\textless}attribute '\_\_class\_\_' of 'object' objects{\textgreater}}&\\
\cline{1-2}
\end{longtable}

    \index{peach \textit{(package)}!peach.fuzzy \textit{(package)}!peach.fuzzy.mf \textit{(module)}!peach.fuzzy.mf.IncreasingRamp \textit{(class)}|)}

%%%%%%%%%%%%%%%%%%%%%%%%%%%%%%%%%%%%%%%%%%%%%%%%%%%%%%%%%%%%%%%%%%%%%%%%%%%
%%                           Class Description                           %%
%%%%%%%%%%%%%%%%%%%%%%%%%%%%%%%%%%%%%%%%%%%%%%%%%%%%%%%%%%%%%%%%%%%%%%%%%%%

    \index{peach \textit{(package)}!peach.fuzzy \textit{(package)}!peach.fuzzy.mf \textit{(module)}!peach.fuzzy.mf.DecreasingRamp \textit{(class)}|(}
\subsection{Class DecreasingRamp}

    \label{peach:fuzzy:mf:DecreasingRamp}
\begin{tabular}{cccccccc}
% Line for object, linespec=[False, False]
\multicolumn{2}{r}{\settowidth{\BCL}{object}\multirow{2}{\BCL}{object}}
&&
&&
  \\\cline{3-3}
  &&\multicolumn{1}{c|}{}
&&
&&
  \\
% Line for peach.fuzzy.mf.Membership, linespec=[False]
\multicolumn{4}{r}{\settowidth{\BCL}{peach.fuzzy.mf.Membership}\multirow{2}{\BCL}{peach.fuzzy.mf.Membership}}
&&
  \\\cline{5-5}
  &&&&\multicolumn{1}{c|}{}
&&
  \\
&&&&\multicolumn{2}{l}{\textbf{peach.fuzzy.mf.DecreasingRamp}}
\end{tabular}


Decreasing ramp.

Given two points, \texttt{x0} and \texttt{x1}, with \texttt{x0 < x1}, creates a function
which returns:
\begin{quote}

1, if \texttt{x <= x0};

\texttt{(x1 - x) / (x1 - x0)}, if \texttt{x0 < x <= x1};

0, if \texttt{x > x1}.
\end{quote}

%%%%%%%%%%%%%%%%%%%%%%%%%%%%%%%%%%%%%%%%%%%%%%%%%%%%%%%%%%%%%%%%%%%%%%%%%%%
%%                                Methods                                %%
%%%%%%%%%%%%%%%%%%%%%%%%%%%%%%%%%%%%%%%%%%%%%%%%%%%%%%%%%%%%%%%%%%%%%%%%%%%

  \subsubsection{Methods}

    \vspace{0.5ex}

    \begin{boxedminipage}{\textwidth}

    \raggedright \textbf{\_\_init\_\_}(\textit{self}, \textit{x0}, \textit{x1})

    \vspace{-1.5ex}

    \rule{\textwidth}{0.5\fboxrule}

Initializes the function.
    \vspace{1ex}

      \textbf{Parameters}
      \begin{quote}
        \begin{Ventry}{xx}

          \item[x0]


Start of the ramp;
          \item[x1]


End of the ramp.
        \end{Ventry}

      \end{quote}

    \vspace{1ex}

      Overrides: peach.fuzzy.mf.Membership.\_\_init\_\_

    \end{boxedminipage}

    \vspace{0.5ex}

    \begin{boxedminipage}{\textwidth}

    \raggedright \textbf{\_\_call\_\_}(\textit{self}, \textit{x})


Maps the function on a vector
    \vspace{1ex}

      \textbf{Return Value}
      \begin{quote}

A \texttt{FuzzySet} object containing the evaluation of the function over
each of the components of the input.
      \end{quote}

    \vspace{1ex}

      Overrides: peach.fuzzy.mf.Membership.\_\_call\_\_ 	extit{(inherited documentation)}

    \end{boxedminipage}

    \label{object:__delattr__}
    \index{object.\_\_delattr\_\_ \textit{(function)}}

    \vspace{0.5ex}

    \begin{boxedminipage}{\textwidth}

    \raggedright \textbf{\_\_delattr\_\_}(\textit{...})

    \vspace{-1.5ex}

    \rule{\textwidth}{0.5\fboxrule}

x.{\_}{\_}delattr{\_}{\_}('name') {\textless}=={\textgreater} del x.name
    \vspace{1ex}

    \end{boxedminipage}

    \label{object:__getattribute__}
    \index{object.\_\_getattribute\_\_ \textit{(function)}}

    \vspace{0.5ex}

    \begin{boxedminipage}{\textwidth}

    \raggedright \textbf{\_\_getattribute\_\_}(\textit{...})

    \vspace{-1.5ex}

    \rule{\textwidth}{0.5\fboxrule}

x.{\_}{\_}getattribute{\_}{\_}('name') {\textless}=={\textgreater} x.name
    \vspace{1ex}

    \end{boxedminipage}

    \label{object:__hash__}
    \index{object.\_\_hash\_\_ \textit{(function)}}

    \vspace{0.5ex}

    \begin{boxedminipage}{\textwidth}

    \raggedright \textbf{\_\_hash\_\_}(\textit{x})

    \vspace{-1.5ex}

    \rule{\textwidth}{0.5\fboxrule}

hash(x)
    \vspace{1ex}

    \end{boxedminipage}

    \label{object:__new__}
    \index{object.\_\_new\_\_ \textit{(function)}}

    \vspace{0.5ex}

    \begin{boxedminipage}{\textwidth}

    \raggedright \textbf{\_\_new\_\_}(\textit{T}, \textit{S}, \textit{...})

      \textbf{Return Value}
      \begin{quote}
\begin{alltt}
a new object with type S, a subtype of T
\end{alltt}

      \end{quote}

    \vspace{1ex}

    \end{boxedminipage}

    \label{object:__reduce__}
    \index{object.\_\_reduce\_\_ \textit{(function)}}

    \vspace{0.5ex}

    \begin{boxedminipage}{\textwidth}

    \raggedright \textbf{\_\_reduce\_\_}(\textit{...})

    \vspace{-1.5ex}

    \rule{\textwidth}{0.5\fboxrule}

helper for pickle
    \vspace{1ex}

    \end{boxedminipage}

    \label{object:__reduce_ex__}
    \index{object.\_\_reduce\_ex\_\_ \textit{(function)}}

    \vspace{0.5ex}

    \begin{boxedminipage}{\textwidth}

    \raggedright \textbf{\_\_reduce\_ex\_\_}(\textit{...})

    \vspace{-1.5ex}

    \rule{\textwidth}{0.5\fboxrule}

helper for pickle
    \vspace{1ex}

    \end{boxedminipage}

    \label{object:__repr__}
    \index{object.\_\_repr\_\_ \textit{(function)}}

    \vspace{0.5ex}

    \begin{boxedminipage}{\textwidth}

    \raggedright \textbf{\_\_repr\_\_}(\textit{x})

    \vspace{-1.5ex}

    \rule{\textwidth}{0.5\fboxrule}

repr(x)
    \vspace{1ex}

    \end{boxedminipage}

    \label{object:__setattr__}
    \index{object.\_\_setattr\_\_ \textit{(function)}}

    \vspace{0.5ex}

    \begin{boxedminipage}{\textwidth}

    \raggedright \textbf{\_\_setattr\_\_}(\textit{...})

    \vspace{-1.5ex}

    \rule{\textwidth}{0.5\fboxrule}

x.{\_}{\_}setattr{\_}{\_}('name', value) {\textless}=={\textgreater} x.name = value
    \vspace{1ex}

    \end{boxedminipage}

    \label{object:__str__}
    \index{object.\_\_str\_\_ \textit{(function)}}

    \vspace{0.5ex}

    \begin{boxedminipage}{\textwidth}

    \raggedright \textbf{\_\_str\_\_}(\textit{x})

    \vspace{-1.5ex}

    \rule{\textwidth}{0.5\fboxrule}

str(x)
    \vspace{1ex}

    \end{boxedminipage}


%%%%%%%%%%%%%%%%%%%%%%%%%%%%%%%%%%%%%%%%%%%%%%%%%%%%%%%%%%%%%%%%%%%%%%%%%%%
%%                              Properties                               %%
%%%%%%%%%%%%%%%%%%%%%%%%%%%%%%%%%%%%%%%%%%%%%%%%%%%%%%%%%%%%%%%%%%%%%%%%%%%

  \subsubsection{Properties}

\begin{longtable}{|p{.30\textwidth}|p{.62\textwidth}|l}
\cline{1-2}
\cline{1-2} \centering \textbf{Name} & \centering \textbf{Description}& \\
\cline{1-2}
\endhead\cline{1-2}\multicolumn{3}{r}{\small\textit{continued on next page}}\\\endfoot\cline{1-2}
\endlastfoot\raggedright \_\-\_\-c\-l\-a\-s\-s\-\_\-\_\- & \raggedright \textbf{Value:} 
{\tt {\textless}attribute '\_\_class\_\_' of 'object' objects{\textgreater}}&\\
\cline{1-2}
\end{longtable}

    \index{peach \textit{(package)}!peach.fuzzy \textit{(package)}!peach.fuzzy.mf \textit{(module)}!peach.fuzzy.mf.DecreasingRamp \textit{(class)}|)}

%%%%%%%%%%%%%%%%%%%%%%%%%%%%%%%%%%%%%%%%%%%%%%%%%%%%%%%%%%%%%%%%%%%%%%%%%%%
%%                           Class Description                           %%
%%%%%%%%%%%%%%%%%%%%%%%%%%%%%%%%%%%%%%%%%%%%%%%%%%%%%%%%%%%%%%%%%%%%%%%%%%%

    \index{peach \textit{(package)}!peach.fuzzy \textit{(package)}!peach.fuzzy.mf \textit{(module)}!peach.fuzzy.mf.Triangle \textit{(class)}|(}
\subsection{Class Triangle}

    \label{peach:fuzzy:mf:Triangle}
\begin{tabular}{cccccccc}
% Line for object, linespec=[False, False]
\multicolumn{2}{r}{\settowidth{\BCL}{object}\multirow{2}{\BCL}{object}}
&&
&&
  \\\cline{3-3}
  &&\multicolumn{1}{c|}{}
&&
&&
  \\
% Line for peach.fuzzy.mf.Membership, linespec=[False]
\multicolumn{4}{r}{\settowidth{\BCL}{peach.fuzzy.mf.Membership}\multirow{2}{\BCL}{peach.fuzzy.mf.Membership}}
&&
  \\\cline{5-5}
  &&&&\multicolumn{1}{c|}{}
&&
  \\
&&&&\multicolumn{2}{l}{\textbf{peach.fuzzy.mf.Triangle}}
\end{tabular}


Triangle function.

Given three points, \texttt{x0}, \texttt{x1} and \texttt{x2}, with \texttt{x0 < x1 < x2},
creates a function which returns:
\begin{quote}

0, if \texttt{x <= x0} or \texttt{x > x2};

\texttt{(x - x0) / (x1 - x0)}, if \texttt{x0 < x <= x1};

\texttt{(x2 - x) / (x2 - x1)}, if \texttt{x1 < x <= x2}.
\end{quote}

%%%%%%%%%%%%%%%%%%%%%%%%%%%%%%%%%%%%%%%%%%%%%%%%%%%%%%%%%%%%%%%%%%%%%%%%%%%
%%                                Methods                                %%
%%%%%%%%%%%%%%%%%%%%%%%%%%%%%%%%%%%%%%%%%%%%%%%%%%%%%%%%%%%%%%%%%%%%%%%%%%%

  \subsubsection{Methods}

    \vspace{0.5ex}

    \begin{boxedminipage}{\textwidth}

    \raggedright \textbf{\_\_init\_\_}(\textit{self}, \textit{x0}, \textit{x1}, \textit{x2})

    \vspace{-1.5ex}

    \rule{\textwidth}{0.5\fboxrule}

Initializes the function.
    \vspace{1ex}

      \textbf{Parameters}
      \begin{quote}
        \begin{Ventry}{xx}

          \item[x0]


Start of the triangle;
          \item[x1]


Peak of the triangle;
          \item[x2]


End of triangle.
        \end{Ventry}

      \end{quote}

    \vspace{1ex}

      Overrides: peach.fuzzy.mf.Membership.\_\_init\_\_

    \end{boxedminipage}

    \vspace{0.5ex}

    \begin{boxedminipage}{\textwidth}

    \raggedright \textbf{\_\_call\_\_}(\textit{self}, \textit{x})


Maps the function on a vector
    \vspace{1ex}

      \textbf{Return Value}
      \begin{quote}

A \texttt{FuzzySet} object containing the evaluation of the function over
each of the components of the input.
      \end{quote}

    \vspace{1ex}

      Overrides: peach.fuzzy.mf.Membership.\_\_call\_\_ 	extit{(inherited documentation)}

    \end{boxedminipage}

    \label{object:__delattr__}
    \index{object.\_\_delattr\_\_ \textit{(function)}}

    \vspace{0.5ex}

    \begin{boxedminipage}{\textwidth}

    \raggedright \textbf{\_\_delattr\_\_}(\textit{...})

    \vspace{-1.5ex}

    \rule{\textwidth}{0.5\fboxrule}

x.{\_}{\_}delattr{\_}{\_}('name') {\textless}=={\textgreater} del x.name
    \vspace{1ex}

    \end{boxedminipage}

    \label{object:__getattribute__}
    \index{object.\_\_getattribute\_\_ \textit{(function)}}

    \vspace{0.5ex}

    \begin{boxedminipage}{\textwidth}

    \raggedright \textbf{\_\_getattribute\_\_}(\textit{...})

    \vspace{-1.5ex}

    \rule{\textwidth}{0.5\fboxrule}

x.{\_}{\_}getattribute{\_}{\_}('name') {\textless}=={\textgreater} x.name
    \vspace{1ex}

    \end{boxedminipage}

    \label{object:__hash__}
    \index{object.\_\_hash\_\_ \textit{(function)}}

    \vspace{0.5ex}

    \begin{boxedminipage}{\textwidth}

    \raggedright \textbf{\_\_hash\_\_}(\textit{x})

    \vspace{-1.5ex}

    \rule{\textwidth}{0.5\fboxrule}

hash(x)
    \vspace{1ex}

    \end{boxedminipage}

    \label{object:__new__}
    \index{object.\_\_new\_\_ \textit{(function)}}

    \vspace{0.5ex}

    \begin{boxedminipage}{\textwidth}

    \raggedright \textbf{\_\_new\_\_}(\textit{T}, \textit{S}, \textit{...})

      \textbf{Return Value}
      \begin{quote}
\begin{alltt}
a new object with type S, a subtype of T
\end{alltt}

      \end{quote}

    \vspace{1ex}

    \end{boxedminipage}

    \label{object:__reduce__}
    \index{object.\_\_reduce\_\_ \textit{(function)}}

    \vspace{0.5ex}

    \begin{boxedminipage}{\textwidth}

    \raggedright \textbf{\_\_reduce\_\_}(\textit{...})

    \vspace{-1.5ex}

    \rule{\textwidth}{0.5\fboxrule}

helper for pickle
    \vspace{1ex}

    \end{boxedminipage}

    \label{object:__reduce_ex__}
    \index{object.\_\_reduce\_ex\_\_ \textit{(function)}}

    \vspace{0.5ex}

    \begin{boxedminipage}{\textwidth}

    \raggedright \textbf{\_\_reduce\_ex\_\_}(\textit{...})

    \vspace{-1.5ex}

    \rule{\textwidth}{0.5\fboxrule}

helper for pickle
    \vspace{1ex}

    \end{boxedminipage}

    \label{object:__repr__}
    \index{object.\_\_repr\_\_ \textit{(function)}}

    \vspace{0.5ex}

    \begin{boxedminipage}{\textwidth}

    \raggedright \textbf{\_\_repr\_\_}(\textit{x})

    \vspace{-1.5ex}

    \rule{\textwidth}{0.5\fboxrule}

repr(x)
    \vspace{1ex}

    \end{boxedminipage}

    \label{object:__setattr__}
    \index{object.\_\_setattr\_\_ \textit{(function)}}

    \vspace{0.5ex}

    \begin{boxedminipage}{\textwidth}

    \raggedright \textbf{\_\_setattr\_\_}(\textit{...})

    \vspace{-1.5ex}

    \rule{\textwidth}{0.5\fboxrule}

x.{\_}{\_}setattr{\_}{\_}('name', value) {\textless}=={\textgreater} x.name = value
    \vspace{1ex}

    \end{boxedminipage}

    \label{object:__str__}
    \index{object.\_\_str\_\_ \textit{(function)}}

    \vspace{0.5ex}

    \begin{boxedminipage}{\textwidth}

    \raggedright \textbf{\_\_str\_\_}(\textit{x})

    \vspace{-1.5ex}

    \rule{\textwidth}{0.5\fboxrule}

str(x)
    \vspace{1ex}

    \end{boxedminipage}


%%%%%%%%%%%%%%%%%%%%%%%%%%%%%%%%%%%%%%%%%%%%%%%%%%%%%%%%%%%%%%%%%%%%%%%%%%%
%%                              Properties                               %%
%%%%%%%%%%%%%%%%%%%%%%%%%%%%%%%%%%%%%%%%%%%%%%%%%%%%%%%%%%%%%%%%%%%%%%%%%%%

  \subsubsection{Properties}

\begin{longtable}{|p{.30\textwidth}|p{.62\textwidth}|l}
\cline{1-2}
\cline{1-2} \centering \textbf{Name} & \centering \textbf{Description}& \\
\cline{1-2}
\endhead\cline{1-2}\multicolumn{3}{r}{\small\textit{continued on next page}}\\\endfoot\cline{1-2}
\endlastfoot\raggedright \_\-\_\-c\-l\-a\-s\-s\-\_\-\_\- & \raggedright \textbf{Value:} 
{\tt {\textless}attribute '\_\_class\_\_' of 'object' objects{\textgreater}}&\\
\cline{1-2}
\end{longtable}

    \index{peach \textit{(package)}!peach.fuzzy \textit{(package)}!peach.fuzzy.mf \textit{(module)}!peach.fuzzy.mf.Triangle \textit{(class)}|)}

%%%%%%%%%%%%%%%%%%%%%%%%%%%%%%%%%%%%%%%%%%%%%%%%%%%%%%%%%%%%%%%%%%%%%%%%%%%
%%                           Class Description                           %%
%%%%%%%%%%%%%%%%%%%%%%%%%%%%%%%%%%%%%%%%%%%%%%%%%%%%%%%%%%%%%%%%%%%%%%%%%%%

    \index{peach \textit{(package)}!peach.fuzzy \textit{(package)}!peach.fuzzy.mf \textit{(module)}!peach.fuzzy.mf.Trapezoid \textit{(class)}|(}
\subsection{Class Trapezoid}

    \label{peach:fuzzy:mf:Trapezoid}
\begin{tabular}{cccccccc}
% Line for object, linespec=[False, False]
\multicolumn{2}{r}{\settowidth{\BCL}{object}\multirow{2}{\BCL}{object}}
&&
&&
  \\\cline{3-3}
  &&\multicolumn{1}{c|}{}
&&
&&
  \\
% Line for peach.fuzzy.mf.Membership, linespec=[False]
\multicolumn{4}{r}{\settowidth{\BCL}{peach.fuzzy.mf.Membership}\multirow{2}{\BCL}{peach.fuzzy.mf.Membership}}
&&
  \\\cline{5-5}
  &&&&\multicolumn{1}{c|}{}
&&
  \\
&&&&\multicolumn{2}{l}{\textbf{peach.fuzzy.mf.Trapezoid}}
\end{tabular}


Trapezoid function.

Given four points, \texttt{x0}, \texttt{x1}, \texttt{x2} and \texttt{x3}, with
\texttt{x0 < x1 < x2 < x3}, creates a function which returns:
\begin{quote}

0, if \texttt{x <= x0} or \texttt{x > x3};

\texttt{(x - x0)/(x1 - x0)}, if \texttt{x0 <= x < x1};

1, if \texttt{x1 <= x < x2};

\texttt{(x3 - x)/(x3 - x2)}, if \texttt{x2 <= x < x3}.
\end{quote}

%%%%%%%%%%%%%%%%%%%%%%%%%%%%%%%%%%%%%%%%%%%%%%%%%%%%%%%%%%%%%%%%%%%%%%%%%%%
%%                                Methods                                %%
%%%%%%%%%%%%%%%%%%%%%%%%%%%%%%%%%%%%%%%%%%%%%%%%%%%%%%%%%%%%%%%%%%%%%%%%%%%

  \subsubsection{Methods}

    \vspace{0.5ex}

    \begin{boxedminipage}{\textwidth}

    \raggedright \textbf{\_\_init\_\_}(\textit{self}, \textit{x0}, \textit{x1}, \textit{x2}, \textit{x3})

    \vspace{-1.5ex}

    \rule{\textwidth}{0.5\fboxrule}

Initializes the function.
    \vspace{1ex}

      \textbf{Parameters}
      \begin{quote}
        \begin{Ventry}{xx}

          \item[x0]


Start of the trapezoid;
          \item[x1]


First peak of the trapezoid;
          \item[x2]


Last peak of the trapezoid;
          \item[x3]


End of trapezoid.
        \end{Ventry}

      \end{quote}

    \vspace{1ex}

      Overrides: peach.fuzzy.mf.Membership.\_\_init\_\_

    \end{boxedminipage}

    \vspace{0.5ex}

    \begin{boxedminipage}{\textwidth}

    \raggedright \textbf{\_\_call\_\_}(\textit{self}, \textit{x})


Maps the function on a vector
    \vspace{1ex}

      \textbf{Return Value}
      \begin{quote}

A \texttt{FuzzySet} object containing the evaluation of the function over
each of the components of the input.
      \end{quote}

    \vspace{1ex}

      Overrides: peach.fuzzy.mf.Membership.\_\_call\_\_ 	extit{(inherited documentation)}

    \end{boxedminipage}

    \label{object:__delattr__}
    \index{object.\_\_delattr\_\_ \textit{(function)}}

    \vspace{0.5ex}

    \begin{boxedminipage}{\textwidth}

    \raggedright \textbf{\_\_delattr\_\_}(\textit{...})

    \vspace{-1.5ex}

    \rule{\textwidth}{0.5\fboxrule}

x.{\_}{\_}delattr{\_}{\_}('name') {\textless}=={\textgreater} del x.name
    \vspace{1ex}

    \end{boxedminipage}

    \label{object:__getattribute__}
    \index{object.\_\_getattribute\_\_ \textit{(function)}}

    \vspace{0.5ex}

    \begin{boxedminipage}{\textwidth}

    \raggedright \textbf{\_\_getattribute\_\_}(\textit{...})

    \vspace{-1.5ex}

    \rule{\textwidth}{0.5\fboxrule}

x.{\_}{\_}getattribute{\_}{\_}('name') {\textless}=={\textgreater} x.name
    \vspace{1ex}

    \end{boxedminipage}

    \label{object:__hash__}
    \index{object.\_\_hash\_\_ \textit{(function)}}

    \vspace{0.5ex}

    \begin{boxedminipage}{\textwidth}

    \raggedright \textbf{\_\_hash\_\_}(\textit{x})

    \vspace{-1.5ex}

    \rule{\textwidth}{0.5\fboxrule}

hash(x)
    \vspace{1ex}

    \end{boxedminipage}

    \label{object:__new__}
    \index{object.\_\_new\_\_ \textit{(function)}}

    \vspace{0.5ex}

    \begin{boxedminipage}{\textwidth}

    \raggedright \textbf{\_\_new\_\_}(\textit{T}, \textit{S}, \textit{...})

      \textbf{Return Value}
      \begin{quote}
\begin{alltt}
a new object with type S, a subtype of T
\end{alltt}

      \end{quote}

    \vspace{1ex}

    \end{boxedminipage}

    \label{object:__reduce__}
    \index{object.\_\_reduce\_\_ \textit{(function)}}

    \vspace{0.5ex}

    \begin{boxedminipage}{\textwidth}

    \raggedright \textbf{\_\_reduce\_\_}(\textit{...})

    \vspace{-1.5ex}

    \rule{\textwidth}{0.5\fboxrule}

helper for pickle
    \vspace{1ex}

    \end{boxedminipage}

    \label{object:__reduce_ex__}
    \index{object.\_\_reduce\_ex\_\_ \textit{(function)}}

    \vspace{0.5ex}

    \begin{boxedminipage}{\textwidth}

    \raggedright \textbf{\_\_reduce\_ex\_\_}(\textit{...})

    \vspace{-1.5ex}

    \rule{\textwidth}{0.5\fboxrule}

helper for pickle
    \vspace{1ex}

    \end{boxedminipage}

    \label{object:__repr__}
    \index{object.\_\_repr\_\_ \textit{(function)}}

    \vspace{0.5ex}

    \begin{boxedminipage}{\textwidth}

    \raggedright \textbf{\_\_repr\_\_}(\textit{x})

    \vspace{-1.5ex}

    \rule{\textwidth}{0.5\fboxrule}

repr(x)
    \vspace{1ex}

    \end{boxedminipage}

    \label{object:__setattr__}
    \index{object.\_\_setattr\_\_ \textit{(function)}}

    \vspace{0.5ex}

    \begin{boxedminipage}{\textwidth}

    \raggedright \textbf{\_\_setattr\_\_}(\textit{...})

    \vspace{-1.5ex}

    \rule{\textwidth}{0.5\fboxrule}

x.{\_}{\_}setattr{\_}{\_}('name', value) {\textless}=={\textgreater} x.name = value
    \vspace{1ex}

    \end{boxedminipage}

    \label{object:__str__}
    \index{object.\_\_str\_\_ \textit{(function)}}

    \vspace{0.5ex}

    \begin{boxedminipage}{\textwidth}

    \raggedright \textbf{\_\_str\_\_}(\textit{x})

    \vspace{-1.5ex}

    \rule{\textwidth}{0.5\fboxrule}

str(x)
    \vspace{1ex}

    \end{boxedminipage}


%%%%%%%%%%%%%%%%%%%%%%%%%%%%%%%%%%%%%%%%%%%%%%%%%%%%%%%%%%%%%%%%%%%%%%%%%%%
%%                              Properties                               %%
%%%%%%%%%%%%%%%%%%%%%%%%%%%%%%%%%%%%%%%%%%%%%%%%%%%%%%%%%%%%%%%%%%%%%%%%%%%

  \subsubsection{Properties}

\begin{longtable}{|p{.30\textwidth}|p{.62\textwidth}|l}
\cline{1-2}
\cline{1-2} \centering \textbf{Name} & \centering \textbf{Description}& \\
\cline{1-2}
\endhead\cline{1-2}\multicolumn{3}{r}{\small\textit{continued on next page}}\\\endfoot\cline{1-2}
\endlastfoot\raggedright \_\-\_\-c\-l\-a\-s\-s\-\_\-\_\- & \raggedright \textbf{Value:} 
{\tt {\textless}attribute '\_\_class\_\_' of 'object' objects{\textgreater}}&\\
\cline{1-2}
\end{longtable}

    \index{peach \textit{(package)}!peach.fuzzy \textit{(package)}!peach.fuzzy.mf \textit{(module)}!peach.fuzzy.mf.Trapezoid \textit{(class)}|)}

%%%%%%%%%%%%%%%%%%%%%%%%%%%%%%%%%%%%%%%%%%%%%%%%%%%%%%%%%%%%%%%%%%%%%%%%%%%
%%                           Class Description                           %%
%%%%%%%%%%%%%%%%%%%%%%%%%%%%%%%%%%%%%%%%%%%%%%%%%%%%%%%%%%%%%%%%%%%%%%%%%%%

    \index{peach \textit{(package)}!peach.fuzzy \textit{(package)}!peach.fuzzy.mf \textit{(module)}!peach.fuzzy.mf.Gaussian \textit{(class)}|(}
\subsection{Class Gaussian}

    \label{peach:fuzzy:mf:Gaussian}
\begin{tabular}{cccccccc}
% Line for object, linespec=[False, False]
\multicolumn{2}{r}{\settowidth{\BCL}{object}\multirow{2}{\BCL}{object}}
&&
&&
  \\\cline{3-3}
  &&\multicolumn{1}{c|}{}
&&
&&
  \\
% Line for peach.fuzzy.mf.Membership, linespec=[False]
\multicolumn{4}{r}{\settowidth{\BCL}{peach.fuzzy.mf.Membership}\multirow{2}{\BCL}{peach.fuzzy.mf.Membership}}
&&
  \\\cline{5-5}
  &&&&\multicolumn{1}{c|}{}
&&
  \\
&&&&\multicolumn{2}{l}{\textbf{peach.fuzzy.mf.Gaussian}}
\end{tabular}


Gaussian function.

Given the center and the width, creates a function which returns a gaussian
fit to these parameters, that is:
\begin{quote}

\texttt{exp(-a*(x - x0)**2)}
\end{quote}

%%%%%%%%%%%%%%%%%%%%%%%%%%%%%%%%%%%%%%%%%%%%%%%%%%%%%%%%%%%%%%%%%%%%%%%%%%%
%%                                Methods                                %%
%%%%%%%%%%%%%%%%%%%%%%%%%%%%%%%%%%%%%%%%%%%%%%%%%%%%%%%%%%%%%%%%%%%%%%%%%%%

  \subsubsection{Methods}

    \vspace{0.5ex}

    \begin{boxedminipage}{\textwidth}

    \raggedright \textbf{\_\_init\_\_}(\textit{self}, \textit{x0}=\texttt{0.0}, \textit{a}=\texttt{1.0})

    \vspace{-1.5ex}

    \rule{\textwidth}{0.5\fboxrule}

Initializes the function.
    \vspace{1ex}

      \textbf{Parameters}
      \begin{quote}
        \begin{Ventry}{xx}

          \item[x0]


Center of the gaussian. Default value \texttt{0.0};
          \item[a]


Width of the gaussian. Default value \texttt{1.0}.
        \end{Ventry}

      \end{quote}

    \vspace{1ex}

      Overrides: peach.fuzzy.mf.Membership.\_\_init\_\_

    \end{boxedminipage}

    \vspace{0.5ex}

    \begin{boxedminipage}{\textwidth}

    \raggedright \textbf{\_\_call\_\_}(\textit{self}, \textit{x})


Maps the function on a vector
    \vspace{1ex}

      \textbf{Return Value}
      \begin{quote}

A \texttt{FuzzySet} object containing the evaluation of the function over
each of the components of the input.
      \end{quote}

    \vspace{1ex}

      Overrides: peach.fuzzy.mf.Membership.\_\_call\_\_ 	extit{(inherited documentation)}

    \end{boxedminipage}

    \label{object:__delattr__}
    \index{object.\_\_delattr\_\_ \textit{(function)}}

    \vspace{0.5ex}

    \begin{boxedminipage}{\textwidth}

    \raggedright \textbf{\_\_delattr\_\_}(\textit{...})

    \vspace{-1.5ex}

    \rule{\textwidth}{0.5\fboxrule}

x.{\_}{\_}delattr{\_}{\_}('name') {\textless}=={\textgreater} del x.name
    \vspace{1ex}

    \end{boxedminipage}

    \label{object:__getattribute__}
    \index{object.\_\_getattribute\_\_ \textit{(function)}}

    \vspace{0.5ex}

    \begin{boxedminipage}{\textwidth}

    \raggedright \textbf{\_\_getattribute\_\_}(\textit{...})

    \vspace{-1.5ex}

    \rule{\textwidth}{0.5\fboxrule}

x.{\_}{\_}getattribute{\_}{\_}('name') {\textless}=={\textgreater} x.name
    \vspace{1ex}

    \end{boxedminipage}

    \label{object:__hash__}
    \index{object.\_\_hash\_\_ \textit{(function)}}

    \vspace{0.5ex}

    \begin{boxedminipage}{\textwidth}

    \raggedright \textbf{\_\_hash\_\_}(\textit{x})

    \vspace{-1.5ex}

    \rule{\textwidth}{0.5\fboxrule}

hash(x)
    \vspace{1ex}

    \end{boxedminipage}

    \label{object:__new__}
    \index{object.\_\_new\_\_ \textit{(function)}}

    \vspace{0.5ex}

    \begin{boxedminipage}{\textwidth}

    \raggedright \textbf{\_\_new\_\_}(\textit{T}, \textit{S}, \textit{...})

      \textbf{Return Value}
      \begin{quote}
\begin{alltt}
a new object with type S, a subtype of T
\end{alltt}

      \end{quote}

    \vspace{1ex}

    \end{boxedminipage}

    \label{object:__reduce__}
    \index{object.\_\_reduce\_\_ \textit{(function)}}

    \vspace{0.5ex}

    \begin{boxedminipage}{\textwidth}

    \raggedright \textbf{\_\_reduce\_\_}(\textit{...})

    \vspace{-1.5ex}

    \rule{\textwidth}{0.5\fboxrule}

helper for pickle
    \vspace{1ex}

    \end{boxedminipage}

    \label{object:__reduce_ex__}
    \index{object.\_\_reduce\_ex\_\_ \textit{(function)}}

    \vspace{0.5ex}

    \begin{boxedminipage}{\textwidth}

    \raggedright \textbf{\_\_reduce\_ex\_\_}(\textit{...})

    \vspace{-1.5ex}

    \rule{\textwidth}{0.5\fboxrule}

helper for pickle
    \vspace{1ex}

    \end{boxedminipage}

    \label{object:__repr__}
    \index{object.\_\_repr\_\_ \textit{(function)}}

    \vspace{0.5ex}

    \begin{boxedminipage}{\textwidth}

    \raggedright \textbf{\_\_repr\_\_}(\textit{x})

    \vspace{-1.5ex}

    \rule{\textwidth}{0.5\fboxrule}

repr(x)
    \vspace{1ex}

    \end{boxedminipage}

    \label{object:__setattr__}
    \index{object.\_\_setattr\_\_ \textit{(function)}}

    \vspace{0.5ex}

    \begin{boxedminipage}{\textwidth}

    \raggedright \textbf{\_\_setattr\_\_}(\textit{...})

    \vspace{-1.5ex}

    \rule{\textwidth}{0.5\fboxrule}

x.{\_}{\_}setattr{\_}{\_}('name', value) {\textless}=={\textgreater} x.name = value
    \vspace{1ex}

    \end{boxedminipage}

    \label{object:__str__}
    \index{object.\_\_str\_\_ \textit{(function)}}

    \vspace{0.5ex}

    \begin{boxedminipage}{\textwidth}

    \raggedright \textbf{\_\_str\_\_}(\textit{x})

    \vspace{-1.5ex}

    \rule{\textwidth}{0.5\fboxrule}

str(x)
    \vspace{1ex}

    \end{boxedminipage}


%%%%%%%%%%%%%%%%%%%%%%%%%%%%%%%%%%%%%%%%%%%%%%%%%%%%%%%%%%%%%%%%%%%%%%%%%%%
%%                              Properties                               %%
%%%%%%%%%%%%%%%%%%%%%%%%%%%%%%%%%%%%%%%%%%%%%%%%%%%%%%%%%%%%%%%%%%%%%%%%%%%

  \subsubsection{Properties}

\begin{longtable}{|p{.30\textwidth}|p{.62\textwidth}|l}
\cline{1-2}
\cline{1-2} \centering \textbf{Name} & \centering \textbf{Description}& \\
\cline{1-2}
\endhead\cline{1-2}\multicolumn{3}{r}{\small\textit{continued on next page}}\\\endfoot\cline{1-2}
\endlastfoot\raggedright \_\-\_\-c\-l\-a\-s\-s\-\_\-\_\- & \raggedright \textbf{Value:} 
{\tt {\textless}attribute '\_\_class\_\_' of 'object' objects{\textgreater}}&\\
\cline{1-2}
\end{longtable}

    \index{peach \textit{(package)}!peach.fuzzy \textit{(package)}!peach.fuzzy.mf \textit{(module)}!peach.fuzzy.mf.Gaussian \textit{(class)}|)}

%%%%%%%%%%%%%%%%%%%%%%%%%%%%%%%%%%%%%%%%%%%%%%%%%%%%%%%%%%%%%%%%%%%%%%%%%%%
%%                           Class Description                           %%
%%%%%%%%%%%%%%%%%%%%%%%%%%%%%%%%%%%%%%%%%%%%%%%%%%%%%%%%%%%%%%%%%%%%%%%%%%%

    \index{peach \textit{(package)}!peach.fuzzy \textit{(package)}!peach.fuzzy.mf \textit{(module)}!peach.fuzzy.mf.IncreasingSigmoid \textit{(class)}|(}
\subsection{Class IncreasingSigmoid}

    \label{peach:fuzzy:mf:IncreasingSigmoid}
\begin{tabular}{cccccccc}
% Line for object, linespec=[False, False]
\multicolumn{2}{r}{\settowidth{\BCL}{object}\multirow{2}{\BCL}{object}}
&&
&&
  \\\cline{3-3}
  &&\multicolumn{1}{c|}{}
&&
&&
  \\
% Line for peach.fuzzy.mf.Membership, linespec=[False]
\multicolumn{4}{r}{\settowidth{\BCL}{peach.fuzzy.mf.Membership}\multirow{2}{\BCL}{peach.fuzzy.mf.Membership}}
&&
  \\\cline{5-5}
  &&&&\multicolumn{1}{c|}{}
&&
  \\
&&&&\multicolumn{2}{l}{\textbf{peach.fuzzy.mf.IncreasingSigmoid}}
\end{tabular}


Increasing Sigmoid function.

Given the center and the slope, creates an increasing sigmoidal function.
It goes to \texttt{0} as \texttt{x} approaches to -infinity, and goes to \texttt{1} as
\texttt{x} approaches infinity, that is:
\begin{quote}

\texttt{1 / (1 + exp(-a*(x - x0))}
\end{quote}

%%%%%%%%%%%%%%%%%%%%%%%%%%%%%%%%%%%%%%%%%%%%%%%%%%%%%%%%%%%%%%%%%%%%%%%%%%%
%%                                Methods                                %%
%%%%%%%%%%%%%%%%%%%%%%%%%%%%%%%%%%%%%%%%%%%%%%%%%%%%%%%%%%%%%%%%%%%%%%%%%%%

  \subsubsection{Methods}

    \vspace{0.5ex}

    \begin{boxedminipage}{\textwidth}

    \raggedright \textbf{\_\_init\_\_}(\textit{self}, \textit{x0}=\texttt{0.0}, \textit{a}=\texttt{1.0})

    \vspace{-1.5ex}

    \rule{\textwidth}{0.5\fboxrule}

Initializes the function.
    \vspace{1ex}

      \textbf{Parameters}
      \begin{quote}
        \begin{Ventry}{xx}

          \item[x0]


Center of the sigmoid. Default value \texttt{0.0}. The function evaluates
to \texttt{0.5} if \texttt{x = x0};
          \item[a]


Slope of the sigmoid. Default value \texttt{1.0}.
        \end{Ventry}

      \end{quote}

    \vspace{1ex}

      Overrides: peach.fuzzy.mf.Membership.\_\_init\_\_

    \end{boxedminipage}

    \vspace{0.5ex}

    \begin{boxedminipage}{\textwidth}

    \raggedright \textbf{\_\_call\_\_}(\textit{self}, \textit{x})


Maps the function on a vector
    \vspace{1ex}

      \textbf{Return Value}
      \begin{quote}

A \texttt{FuzzySet} object containing the evaluation of the function over
each of the components of the input.
      \end{quote}

    \vspace{1ex}

      Overrides: peach.fuzzy.mf.Membership.\_\_call\_\_ 	extit{(inherited documentation)}

    \end{boxedminipage}

    \label{object:__delattr__}
    \index{object.\_\_delattr\_\_ \textit{(function)}}

    \vspace{0.5ex}

    \begin{boxedminipage}{\textwidth}

    \raggedright \textbf{\_\_delattr\_\_}(\textit{...})

    \vspace{-1.5ex}

    \rule{\textwidth}{0.5\fboxrule}

x.{\_}{\_}delattr{\_}{\_}('name') {\textless}=={\textgreater} del x.name
    \vspace{1ex}

    \end{boxedminipage}

    \label{object:__getattribute__}
    \index{object.\_\_getattribute\_\_ \textit{(function)}}

    \vspace{0.5ex}

    \begin{boxedminipage}{\textwidth}

    \raggedright \textbf{\_\_getattribute\_\_}(\textit{...})

    \vspace{-1.5ex}

    \rule{\textwidth}{0.5\fboxrule}

x.{\_}{\_}getattribute{\_}{\_}('name') {\textless}=={\textgreater} x.name
    \vspace{1ex}

    \end{boxedminipage}

    \label{object:__hash__}
    \index{object.\_\_hash\_\_ \textit{(function)}}

    \vspace{0.5ex}

    \begin{boxedminipage}{\textwidth}

    \raggedright \textbf{\_\_hash\_\_}(\textit{x})

    \vspace{-1.5ex}

    \rule{\textwidth}{0.5\fboxrule}

hash(x)
    \vspace{1ex}

    \end{boxedminipage}

    \label{object:__new__}
    \index{object.\_\_new\_\_ \textit{(function)}}

    \vspace{0.5ex}

    \begin{boxedminipage}{\textwidth}

    \raggedright \textbf{\_\_new\_\_}(\textit{T}, \textit{S}, \textit{...})

      \textbf{Return Value}
      \begin{quote}
\begin{alltt}
a new object with type S, a subtype of T
\end{alltt}

      \end{quote}

    \vspace{1ex}

    \end{boxedminipage}

    \label{object:__reduce__}
    \index{object.\_\_reduce\_\_ \textit{(function)}}

    \vspace{0.5ex}

    \begin{boxedminipage}{\textwidth}

    \raggedright \textbf{\_\_reduce\_\_}(\textit{...})

    \vspace{-1.5ex}

    \rule{\textwidth}{0.5\fboxrule}

helper for pickle
    \vspace{1ex}

    \end{boxedminipage}

    \label{object:__reduce_ex__}
    \index{object.\_\_reduce\_ex\_\_ \textit{(function)}}

    \vspace{0.5ex}

    \begin{boxedminipage}{\textwidth}

    \raggedright \textbf{\_\_reduce\_ex\_\_}(\textit{...})

    \vspace{-1.5ex}

    \rule{\textwidth}{0.5\fboxrule}

helper for pickle
    \vspace{1ex}

    \end{boxedminipage}

    \label{object:__repr__}
    \index{object.\_\_repr\_\_ \textit{(function)}}

    \vspace{0.5ex}

    \begin{boxedminipage}{\textwidth}

    \raggedright \textbf{\_\_repr\_\_}(\textit{x})

    \vspace{-1.5ex}

    \rule{\textwidth}{0.5\fboxrule}

repr(x)
    \vspace{1ex}

    \end{boxedminipage}

    \label{object:__setattr__}
    \index{object.\_\_setattr\_\_ \textit{(function)}}

    \vspace{0.5ex}

    \begin{boxedminipage}{\textwidth}

    \raggedright \textbf{\_\_setattr\_\_}(\textit{...})

    \vspace{-1.5ex}

    \rule{\textwidth}{0.5\fboxrule}

x.{\_}{\_}setattr{\_}{\_}('name', value) {\textless}=={\textgreater} x.name = value
    \vspace{1ex}

    \end{boxedminipage}

    \label{object:__str__}
    \index{object.\_\_str\_\_ \textit{(function)}}

    \vspace{0.5ex}

    \begin{boxedminipage}{\textwidth}

    \raggedright \textbf{\_\_str\_\_}(\textit{x})

    \vspace{-1.5ex}

    \rule{\textwidth}{0.5\fboxrule}

str(x)
    \vspace{1ex}

    \end{boxedminipage}


%%%%%%%%%%%%%%%%%%%%%%%%%%%%%%%%%%%%%%%%%%%%%%%%%%%%%%%%%%%%%%%%%%%%%%%%%%%
%%                              Properties                               %%
%%%%%%%%%%%%%%%%%%%%%%%%%%%%%%%%%%%%%%%%%%%%%%%%%%%%%%%%%%%%%%%%%%%%%%%%%%%

  \subsubsection{Properties}

\begin{longtable}{|p{.30\textwidth}|p{.62\textwidth}|l}
\cline{1-2}
\cline{1-2} \centering \textbf{Name} & \centering \textbf{Description}& \\
\cline{1-2}
\endhead\cline{1-2}\multicolumn{3}{r}{\small\textit{continued on next page}}\\\endfoot\cline{1-2}
\endlastfoot\raggedright \_\-\_\-c\-l\-a\-s\-s\-\_\-\_\- & \raggedright \textbf{Value:} 
{\tt {\textless}attribute '\_\_class\_\_' of 'object' objects{\textgreater}}&\\
\cline{1-2}
\end{longtable}

    \index{peach \textit{(package)}!peach.fuzzy \textit{(package)}!peach.fuzzy.mf \textit{(module)}!peach.fuzzy.mf.IncreasingSigmoid \textit{(class)}|)}

%%%%%%%%%%%%%%%%%%%%%%%%%%%%%%%%%%%%%%%%%%%%%%%%%%%%%%%%%%%%%%%%%%%%%%%%%%%
%%                           Class Description                           %%
%%%%%%%%%%%%%%%%%%%%%%%%%%%%%%%%%%%%%%%%%%%%%%%%%%%%%%%%%%%%%%%%%%%%%%%%%%%

    \index{peach \textit{(package)}!peach.fuzzy \textit{(package)}!peach.fuzzy.mf \textit{(module)}!peach.fuzzy.mf.DecreasingSigmoid \textit{(class)}|(}
\subsection{Class DecreasingSigmoid}

    \label{peach:fuzzy:mf:DecreasingSigmoid}
\begin{tabular}{cccccccc}
% Line for object, linespec=[False, False]
\multicolumn{2}{r}{\settowidth{\BCL}{object}\multirow{2}{\BCL}{object}}
&&
&&
  \\\cline{3-3}
  &&\multicolumn{1}{c|}{}
&&
&&
  \\
% Line for peach.fuzzy.mf.Membership, linespec=[False]
\multicolumn{4}{r}{\settowidth{\BCL}{peach.fuzzy.mf.Membership}\multirow{2}{\BCL}{peach.fuzzy.mf.Membership}}
&&
  \\\cline{5-5}
  &&&&\multicolumn{1}{c|}{}
&&
  \\
&&&&\multicolumn{2}{l}{\textbf{peach.fuzzy.mf.DecreasingSigmoid}}
\end{tabular}


Decreasing Sigmoid function.

Given the center and the slope, creates an decreasing sigmoidal function.
It goes to \texttt{1} as \texttt{x} approaches to -infinity, and goes to \texttt{0} as
\texttt{x} approaches infinity, that is:
\begin{quote}

\texttt{1 / (1 + exp(a*(x - x0))}
\end{quote}

%%%%%%%%%%%%%%%%%%%%%%%%%%%%%%%%%%%%%%%%%%%%%%%%%%%%%%%%%%%%%%%%%%%%%%%%%%%
%%                                Methods                                %%
%%%%%%%%%%%%%%%%%%%%%%%%%%%%%%%%%%%%%%%%%%%%%%%%%%%%%%%%%%%%%%%%%%%%%%%%%%%

  \subsubsection{Methods}

    \vspace{0.5ex}

    \begin{boxedminipage}{\textwidth}

    \raggedright \textbf{\_\_init\_\_}(\textit{self}, \textit{x0}=\texttt{0.0}, \textit{a}=\texttt{1.0})

    \vspace{-1.5ex}

    \rule{\textwidth}{0.5\fboxrule}

Initializes the function.
    \vspace{1ex}

      \textbf{Parameters}
      \begin{quote}
        \begin{Ventry}{xx}

          \item[x0]


Center of the sigmoid. Default value \texttt{0.0}. The function evaluates
to \texttt{0.5} if \texttt{x = x0};
          \item[a]


Slope of the sigmoid. Default value \texttt{1.0}.
        \end{Ventry}

      \end{quote}

    \vspace{1ex}

      Overrides: peach.fuzzy.mf.Membership.\_\_init\_\_

    \end{boxedminipage}

    \vspace{0.5ex}

    \begin{boxedminipage}{\textwidth}

    \raggedright \textbf{\_\_call\_\_}(\textit{self}, \textit{x})


Maps the function on a vector
    \vspace{1ex}

      \textbf{Return Value}
      \begin{quote}

A \texttt{FuzzySet} object containing the evaluation of the function over
each of the components of the input.
      \end{quote}

    \vspace{1ex}

      Overrides: peach.fuzzy.mf.Membership.\_\_call\_\_ 	extit{(inherited documentation)}

    \end{boxedminipage}

    \label{object:__delattr__}
    \index{object.\_\_delattr\_\_ \textit{(function)}}

    \vspace{0.5ex}

    \begin{boxedminipage}{\textwidth}

    \raggedright \textbf{\_\_delattr\_\_}(\textit{...})

    \vspace{-1.5ex}

    \rule{\textwidth}{0.5\fboxrule}

x.{\_}{\_}delattr{\_}{\_}('name') {\textless}=={\textgreater} del x.name
    \vspace{1ex}

    \end{boxedminipage}

    \label{object:__getattribute__}
    \index{object.\_\_getattribute\_\_ \textit{(function)}}

    \vspace{0.5ex}

    \begin{boxedminipage}{\textwidth}

    \raggedright \textbf{\_\_getattribute\_\_}(\textit{...})

    \vspace{-1.5ex}

    \rule{\textwidth}{0.5\fboxrule}

x.{\_}{\_}getattribute{\_}{\_}('name') {\textless}=={\textgreater} x.name
    \vspace{1ex}

    \end{boxedminipage}

    \label{object:__hash__}
    \index{object.\_\_hash\_\_ \textit{(function)}}

    \vspace{0.5ex}

    \begin{boxedminipage}{\textwidth}

    \raggedright \textbf{\_\_hash\_\_}(\textit{x})

    \vspace{-1.5ex}

    \rule{\textwidth}{0.5\fboxrule}

hash(x)
    \vspace{1ex}

    \end{boxedminipage}

    \label{object:__new__}
    \index{object.\_\_new\_\_ \textit{(function)}}

    \vspace{0.5ex}

    \begin{boxedminipage}{\textwidth}

    \raggedright \textbf{\_\_new\_\_}(\textit{T}, \textit{S}, \textit{...})

      \textbf{Return Value}
      \begin{quote}
\begin{alltt}
a new object with type S, a subtype of T
\end{alltt}

      \end{quote}

    \vspace{1ex}

    \end{boxedminipage}

    \label{object:__reduce__}
    \index{object.\_\_reduce\_\_ \textit{(function)}}

    \vspace{0.5ex}

    \begin{boxedminipage}{\textwidth}

    \raggedright \textbf{\_\_reduce\_\_}(\textit{...})

    \vspace{-1.5ex}

    \rule{\textwidth}{0.5\fboxrule}

helper for pickle
    \vspace{1ex}

    \end{boxedminipage}

    \label{object:__reduce_ex__}
    \index{object.\_\_reduce\_ex\_\_ \textit{(function)}}

    \vspace{0.5ex}

    \begin{boxedminipage}{\textwidth}

    \raggedright \textbf{\_\_reduce\_ex\_\_}(\textit{...})

    \vspace{-1.5ex}

    \rule{\textwidth}{0.5\fboxrule}

helper for pickle
    \vspace{1ex}

    \end{boxedminipage}

    \label{object:__repr__}
    \index{object.\_\_repr\_\_ \textit{(function)}}

    \vspace{0.5ex}

    \begin{boxedminipage}{\textwidth}

    \raggedright \textbf{\_\_repr\_\_}(\textit{x})

    \vspace{-1.5ex}

    \rule{\textwidth}{0.5\fboxrule}

repr(x)
    \vspace{1ex}

    \end{boxedminipage}

    \label{object:__setattr__}
    \index{object.\_\_setattr\_\_ \textit{(function)}}

    \vspace{0.5ex}

    \begin{boxedminipage}{\textwidth}

    \raggedright \textbf{\_\_setattr\_\_}(\textit{...})

    \vspace{-1.5ex}

    \rule{\textwidth}{0.5\fboxrule}

x.{\_}{\_}setattr{\_}{\_}('name', value) {\textless}=={\textgreater} x.name = value
    \vspace{1ex}

    \end{boxedminipage}

    \label{object:__str__}
    \index{object.\_\_str\_\_ \textit{(function)}}

    \vspace{0.5ex}

    \begin{boxedminipage}{\textwidth}

    \raggedright \textbf{\_\_str\_\_}(\textit{x})

    \vspace{-1.5ex}

    \rule{\textwidth}{0.5\fboxrule}

str(x)
    \vspace{1ex}

    \end{boxedminipage}


%%%%%%%%%%%%%%%%%%%%%%%%%%%%%%%%%%%%%%%%%%%%%%%%%%%%%%%%%%%%%%%%%%%%%%%%%%%
%%                              Properties                               %%
%%%%%%%%%%%%%%%%%%%%%%%%%%%%%%%%%%%%%%%%%%%%%%%%%%%%%%%%%%%%%%%%%%%%%%%%%%%

  \subsubsection{Properties}

\begin{longtable}{|p{.30\textwidth}|p{.62\textwidth}|l}
\cline{1-2}
\cline{1-2} \centering \textbf{Name} & \centering \textbf{Description}& \\
\cline{1-2}
\endhead\cline{1-2}\multicolumn{3}{r}{\small\textit{continued on next page}}\\\endfoot\cline{1-2}
\endlastfoot\raggedright \_\-\_\-c\-l\-a\-s\-s\-\_\-\_\- & \raggedright \textbf{Value:} 
{\tt {\textless}attribute '\_\_class\_\_' of 'object' objects{\textgreater}}&\\
\cline{1-2}
\end{longtable}

    \index{peach \textit{(package)}!peach.fuzzy \textit{(package)}!peach.fuzzy.mf \textit{(module)}!peach.fuzzy.mf.DecreasingSigmoid \textit{(class)}|)}

%%%%%%%%%%%%%%%%%%%%%%%%%%%%%%%%%%%%%%%%%%%%%%%%%%%%%%%%%%%%%%%%%%%%%%%%%%%
%%                           Class Description                           %%
%%%%%%%%%%%%%%%%%%%%%%%%%%%%%%%%%%%%%%%%%%%%%%%%%%%%%%%%%%%%%%%%%%%%%%%%%%%

    \index{peach \textit{(package)}!peach.fuzzy \textit{(package)}!peach.fuzzy.mf \textit{(module)}!peach.fuzzy.mf.RaisedCosine \textit{(class)}|(}
\subsection{Class RaisedCosine}

    \label{peach:fuzzy:mf:RaisedCosine}
\begin{tabular}{cccccccc}
% Line for object, linespec=[False, False]
\multicolumn{2}{r}{\settowidth{\BCL}{object}\multirow{2}{\BCL}{object}}
&&
&&
  \\\cline{3-3}
  &&\multicolumn{1}{c|}{}
&&
&&
  \\
% Line for peach.fuzzy.mf.Membership, linespec=[False]
\multicolumn{4}{r}{\settowidth{\BCL}{peach.fuzzy.mf.Membership}\multirow{2}{\BCL}{peach.fuzzy.mf.Membership}}
&&
  \\\cline{5-5}
  &&&&\multicolumn{1}{c|}{}
&&
  \\
&&&&\multicolumn{2}{l}{\textbf{peach.fuzzy.mf.RaisedCosine}}
\end{tabular}


Raised Cosine function.

Given the center and the frequency, creates a function that is a period of
a raised cosine, that is:
\begin{quote}

0, if \texttt{x <= xm - pi/w} or \texttt{x > xm + pi/w};

\texttt{0.5 + 0.5 * cos(w*(x - xm))}, if \texttt{xm - pi/w <= x < xm + pi/w};
\end{quote}

%%%%%%%%%%%%%%%%%%%%%%%%%%%%%%%%%%%%%%%%%%%%%%%%%%%%%%%%%%%%%%%%%%%%%%%%%%%
%%                                Methods                                %%
%%%%%%%%%%%%%%%%%%%%%%%%%%%%%%%%%%%%%%%%%%%%%%%%%%%%%%%%%%%%%%%%%%%%%%%%%%%

  \subsubsection{Methods}

    \vspace{0.5ex}

    \begin{boxedminipage}{\textwidth}

    \raggedright \textbf{\_\_init\_\_}(\textit{self}, \textit{xm}=\texttt{0.0}, \textit{w}=\texttt{1.0})

    \vspace{-1.5ex}

    \rule{\textwidth}{0.5\fboxrule}

Initializes the function.
    \vspace{1ex}

      \textbf{Parameters}
      \begin{quote}
        \begin{Ventry}{xx}

          \item[xm]


Center of the cosine. Default value \texttt{0.0}. The function evaluates
to \texttt{1} if \texttt{x = xm};
          \item[w]


Frequency of the cosine. Default value \texttt{1.0}.
        \end{Ventry}

      \end{quote}

    \vspace{1ex}

      Overrides: peach.fuzzy.mf.Membership.\_\_init\_\_

    \end{boxedminipage}

    \vspace{0.5ex}

    \begin{boxedminipage}{\textwidth}

    \raggedright \textbf{\_\_call\_\_}(\textit{self}, \textit{x})


Maps the function on a vector
    \vspace{1ex}

      \textbf{Return Value}
      \begin{quote}

A \texttt{FuzzySet} object containing the evaluation of the function over
each of the components of the input.
      \end{quote}

    \vspace{1ex}

      Overrides: peach.fuzzy.mf.Membership.\_\_call\_\_ 	extit{(inherited documentation)}

    \end{boxedminipage}

    \label{object:__delattr__}
    \index{object.\_\_delattr\_\_ \textit{(function)}}

    \vspace{0.5ex}

    \begin{boxedminipage}{\textwidth}

    \raggedright \textbf{\_\_delattr\_\_}(\textit{...})

    \vspace{-1.5ex}

    \rule{\textwidth}{0.5\fboxrule}

x.{\_}{\_}delattr{\_}{\_}('name') {\textless}=={\textgreater} del x.name
    \vspace{1ex}

    \end{boxedminipage}

    \label{object:__getattribute__}
    \index{object.\_\_getattribute\_\_ \textit{(function)}}

    \vspace{0.5ex}

    \begin{boxedminipage}{\textwidth}

    \raggedright \textbf{\_\_getattribute\_\_}(\textit{...})

    \vspace{-1.5ex}

    \rule{\textwidth}{0.5\fboxrule}

x.{\_}{\_}getattribute{\_}{\_}('name') {\textless}=={\textgreater} x.name
    \vspace{1ex}

    \end{boxedminipage}

    \label{object:__hash__}
    \index{object.\_\_hash\_\_ \textit{(function)}}

    \vspace{0.5ex}

    \begin{boxedminipage}{\textwidth}

    \raggedright \textbf{\_\_hash\_\_}(\textit{x})

    \vspace{-1.5ex}

    \rule{\textwidth}{0.5\fboxrule}

hash(x)
    \vspace{1ex}

    \end{boxedminipage}

    \label{object:__new__}
    \index{object.\_\_new\_\_ \textit{(function)}}

    \vspace{0.5ex}

    \begin{boxedminipage}{\textwidth}

    \raggedright \textbf{\_\_new\_\_}(\textit{T}, \textit{S}, \textit{...})

      \textbf{Return Value}
      \begin{quote}
\begin{alltt}
a new object with type S, a subtype of T
\end{alltt}

      \end{quote}

    \vspace{1ex}

    \end{boxedminipage}

    \label{object:__reduce__}
    \index{object.\_\_reduce\_\_ \textit{(function)}}

    \vspace{0.5ex}

    \begin{boxedminipage}{\textwidth}

    \raggedright \textbf{\_\_reduce\_\_}(\textit{...})

    \vspace{-1.5ex}

    \rule{\textwidth}{0.5\fboxrule}

helper for pickle
    \vspace{1ex}

    \end{boxedminipage}

    \label{object:__reduce_ex__}
    \index{object.\_\_reduce\_ex\_\_ \textit{(function)}}

    \vspace{0.5ex}

    \begin{boxedminipage}{\textwidth}

    \raggedright \textbf{\_\_reduce\_ex\_\_}(\textit{...})

    \vspace{-1.5ex}

    \rule{\textwidth}{0.5\fboxrule}

helper for pickle
    \vspace{1ex}

    \end{boxedminipage}

    \label{object:__repr__}
    \index{object.\_\_repr\_\_ \textit{(function)}}

    \vspace{0.5ex}

    \begin{boxedminipage}{\textwidth}

    \raggedright \textbf{\_\_repr\_\_}(\textit{x})

    \vspace{-1.5ex}

    \rule{\textwidth}{0.5\fboxrule}

repr(x)
    \vspace{1ex}

    \end{boxedminipage}

    \label{object:__setattr__}
    \index{object.\_\_setattr\_\_ \textit{(function)}}

    \vspace{0.5ex}

    \begin{boxedminipage}{\textwidth}

    \raggedright \textbf{\_\_setattr\_\_}(\textit{...})

    \vspace{-1.5ex}

    \rule{\textwidth}{0.5\fboxrule}

x.{\_}{\_}setattr{\_}{\_}('name', value) {\textless}=={\textgreater} x.name = value
    \vspace{1ex}

    \end{boxedminipage}

    \label{object:__str__}
    \index{object.\_\_str\_\_ \textit{(function)}}

    \vspace{0.5ex}

    \begin{boxedminipage}{\textwidth}

    \raggedright \textbf{\_\_str\_\_}(\textit{x})

    \vspace{-1.5ex}

    \rule{\textwidth}{0.5\fboxrule}

str(x)
    \vspace{1ex}

    \end{boxedminipage}


%%%%%%%%%%%%%%%%%%%%%%%%%%%%%%%%%%%%%%%%%%%%%%%%%%%%%%%%%%%%%%%%%%%%%%%%%%%
%%                              Properties                               %%
%%%%%%%%%%%%%%%%%%%%%%%%%%%%%%%%%%%%%%%%%%%%%%%%%%%%%%%%%%%%%%%%%%%%%%%%%%%

  \subsubsection{Properties}

\begin{longtable}{|p{.30\textwidth}|p{.62\textwidth}|l}
\cline{1-2}
\cline{1-2} \centering \textbf{Name} & \centering \textbf{Description}& \\
\cline{1-2}
\endhead\cline{1-2}\multicolumn{3}{r}{\small\textit{continued on next page}}\\\endfoot\cline{1-2}
\endlastfoot\raggedright \_\-\_\-c\-l\-a\-s\-s\-\_\-\_\- & \raggedright \textbf{Value:} 
{\tt {\textless}attribute '\_\_class\_\_' of 'object' objects{\textgreater}}&\\
\cline{1-2}
\end{longtable}

    \index{peach \textit{(package)}!peach.fuzzy \textit{(package)}!peach.fuzzy.mf \textit{(module)}!peach.fuzzy.mf.RaisedCosine \textit{(class)}|)}

%%%%%%%%%%%%%%%%%%%%%%%%%%%%%%%%%%%%%%%%%%%%%%%%%%%%%%%%%%%%%%%%%%%%%%%%%%%
%%                           Class Description                           %%
%%%%%%%%%%%%%%%%%%%%%%%%%%%%%%%%%%%%%%%%%%%%%%%%%%%%%%%%%%%%%%%%%%%%%%%%%%%

    \index{peach \textit{(package)}!peach.fuzzy \textit{(package)}!peach.fuzzy.mf \textit{(module)}!peach.fuzzy.mf.Bell \textit{(class)}|(}
\subsection{Class Bell}

    \label{peach:fuzzy:mf:Bell}
\begin{tabular}{cccccccc}
% Line for object, linespec=[False, False]
\multicolumn{2}{r}{\settowidth{\BCL}{object}\multirow{2}{\BCL}{object}}
&&
&&
  \\\cline{3-3}
  &&\multicolumn{1}{c|}{}
&&
&&
  \\
% Line for peach.fuzzy.mf.Membership, linespec=[False]
\multicolumn{4}{r}{\settowidth{\BCL}{peach.fuzzy.mf.Membership}\multirow{2}{\BCL}{peach.fuzzy.mf.Membership}}
&&
  \\\cline{5-5}
  &&&&\multicolumn{1}{c|}{}
&&
  \\
&&&&\multicolumn{2}{l}{\textbf{peach.fuzzy.mf.Bell}}
\end{tabular}


Generalized Bell function.

A generalized bell is a symmetric function with its peak in its center and
fast decreasing to \texttt{0} outside a given interval, that is:
\begin{quote}

\texttt{1 / (1 + ((x - x0)/a)**(2*b))}
\end{quote}

%%%%%%%%%%%%%%%%%%%%%%%%%%%%%%%%%%%%%%%%%%%%%%%%%%%%%%%%%%%%%%%%%%%%%%%%%%%
%%                                Methods                                %%
%%%%%%%%%%%%%%%%%%%%%%%%%%%%%%%%%%%%%%%%%%%%%%%%%%%%%%%%%%%%%%%%%%%%%%%%%%%

  \subsubsection{Methods}

    \vspace{0.5ex}

    \begin{boxedminipage}{\textwidth}

    \raggedright \textbf{\_\_init\_\_}(\textit{self}, \textit{x0}=\texttt{0.0}, \textit{a}=\texttt{1.0}, \textit{b}=\texttt{1.0})

    \vspace{-1.5ex}

    \rule{\textwidth}{0.5\fboxrule}

Initializes the function.
    \vspace{1ex}

      \textbf{Parameters}
      \begin{quote}
        \begin{Ventry}{xx}

          \item[x0]


Center of the bell. Default value \texttt{0.0}. The function evaluates to
\texttt{1} if \texttt{x = xm};
          \item[a]


Size of the interval. Default value \texttt{1.0}. A generalized bell
evaluates to \texttt{0.5} if \texttt{x = -a} or \texttt{x = a};
          \item[b]


Measure of \emph{flatness} of the bell. The bigger the value of \texttt{b},
the flatter is the resulting function. Default value \texttt{1.0}.
        \end{Ventry}

      \end{quote}

    \vspace{1ex}

      Overrides: peach.fuzzy.mf.Membership.\_\_init\_\_

    \end{boxedminipage}

    \vspace{0.5ex}

    \begin{boxedminipage}{\textwidth}

    \raggedright \textbf{\_\_call\_\_}(\textit{self}, \textit{x})


Maps the function on a vector
    \vspace{1ex}

      \textbf{Return Value}
      \begin{quote}

A \texttt{FuzzySet} object containing the evaluation of the function over
each of the components of the input.
      \end{quote}

    \vspace{1ex}

      Overrides: peach.fuzzy.mf.Membership.\_\_call\_\_ 	extit{(inherited documentation)}

    \end{boxedminipage}

    \label{object:__delattr__}
    \index{object.\_\_delattr\_\_ \textit{(function)}}

    \vspace{0.5ex}

    \begin{boxedminipage}{\textwidth}

    \raggedright \textbf{\_\_delattr\_\_}(\textit{...})

    \vspace{-1.5ex}

    \rule{\textwidth}{0.5\fboxrule}

x.{\_}{\_}delattr{\_}{\_}('name') {\textless}=={\textgreater} del x.name
    \vspace{1ex}

    \end{boxedminipage}

    \label{object:__getattribute__}
    \index{object.\_\_getattribute\_\_ \textit{(function)}}

    \vspace{0.5ex}

    \begin{boxedminipage}{\textwidth}

    \raggedright \textbf{\_\_getattribute\_\_}(\textit{...})

    \vspace{-1.5ex}

    \rule{\textwidth}{0.5\fboxrule}

x.{\_}{\_}getattribute{\_}{\_}('name') {\textless}=={\textgreater} x.name
    \vspace{1ex}

    \end{boxedminipage}

    \label{object:__hash__}
    \index{object.\_\_hash\_\_ \textit{(function)}}

    \vspace{0.5ex}

    \begin{boxedminipage}{\textwidth}

    \raggedright \textbf{\_\_hash\_\_}(\textit{x})

    \vspace{-1.5ex}

    \rule{\textwidth}{0.5\fboxrule}

hash(x)
    \vspace{1ex}

    \end{boxedminipage}

    \label{object:__new__}
    \index{object.\_\_new\_\_ \textit{(function)}}

    \vspace{0.5ex}

    \begin{boxedminipage}{\textwidth}

    \raggedright \textbf{\_\_new\_\_}(\textit{T}, \textit{S}, \textit{...})

      \textbf{Return Value}
      \begin{quote}
\begin{alltt}
a new object with type S, a subtype of T
\end{alltt}

      \end{quote}

    \vspace{1ex}

    \end{boxedminipage}

    \label{object:__reduce__}
    \index{object.\_\_reduce\_\_ \textit{(function)}}

    \vspace{0.5ex}

    \begin{boxedminipage}{\textwidth}

    \raggedright \textbf{\_\_reduce\_\_}(\textit{...})

    \vspace{-1.5ex}

    \rule{\textwidth}{0.5\fboxrule}

helper for pickle
    \vspace{1ex}

    \end{boxedminipage}

    \label{object:__reduce_ex__}
    \index{object.\_\_reduce\_ex\_\_ \textit{(function)}}

    \vspace{0.5ex}

    \begin{boxedminipage}{\textwidth}

    \raggedright \textbf{\_\_reduce\_ex\_\_}(\textit{...})

    \vspace{-1.5ex}

    \rule{\textwidth}{0.5\fboxrule}

helper for pickle
    \vspace{1ex}

    \end{boxedminipage}

    \label{object:__repr__}
    \index{object.\_\_repr\_\_ \textit{(function)}}

    \vspace{0.5ex}

    \begin{boxedminipage}{\textwidth}

    \raggedright \textbf{\_\_repr\_\_}(\textit{x})

    \vspace{-1.5ex}

    \rule{\textwidth}{0.5\fboxrule}

repr(x)
    \vspace{1ex}

    \end{boxedminipage}

    \label{object:__setattr__}
    \index{object.\_\_setattr\_\_ \textit{(function)}}

    \vspace{0.5ex}

    \begin{boxedminipage}{\textwidth}

    \raggedright \textbf{\_\_setattr\_\_}(\textit{...})

    \vspace{-1.5ex}

    \rule{\textwidth}{0.5\fboxrule}

x.{\_}{\_}setattr{\_}{\_}('name', value) {\textless}=={\textgreater} x.name = value
    \vspace{1ex}

    \end{boxedminipage}

    \label{object:__str__}
    \index{object.\_\_str\_\_ \textit{(function)}}

    \vspace{0.5ex}

    \begin{boxedminipage}{\textwidth}

    \raggedright \textbf{\_\_str\_\_}(\textit{x})

    \vspace{-1.5ex}

    \rule{\textwidth}{0.5\fboxrule}

str(x)
    \vspace{1ex}

    \end{boxedminipage}


%%%%%%%%%%%%%%%%%%%%%%%%%%%%%%%%%%%%%%%%%%%%%%%%%%%%%%%%%%%%%%%%%%%%%%%%%%%
%%                              Properties                               %%
%%%%%%%%%%%%%%%%%%%%%%%%%%%%%%%%%%%%%%%%%%%%%%%%%%%%%%%%%%%%%%%%%%%%%%%%%%%

  \subsubsection{Properties}

\begin{longtable}{|p{.30\textwidth}|p{.62\textwidth}|l}
\cline{1-2}
\cline{1-2} \centering \textbf{Name} & \centering \textbf{Description}& \\
\cline{1-2}
\endhead\cline{1-2}\multicolumn{3}{r}{\small\textit{continued on next page}}\\\endfoot\cline{1-2}
\endlastfoot\raggedright \_\-\_\-c\-l\-a\-s\-s\-\_\-\_\- & \raggedright \textbf{Value:} 
{\tt {\textless}attribute '\_\_class\_\_' of 'object' objects{\textgreater}}&\\
\cline{1-2}
\end{longtable}

    \index{peach \textit{(package)}!peach.fuzzy \textit{(package)}!peach.fuzzy.mf \textit{(module)}!peach.fuzzy.mf.Bell \textit{(class)}|)}
    \index{peach \textit{(package)}!peach.fuzzy \textit{(package)}!peach.fuzzy.mf \textit{(module)}|)}

%
% API Documentation for Peach - Computational Intelligence for Python
% Module peach.fuzzy.norms
%
% Generated by epydoc 3.0beta1
% [Mon Dec 21 08:51:36 2009]
%

%%%%%%%%%%%%%%%%%%%%%%%%%%%%%%%%%%%%%%%%%%%%%%%%%%%%%%%%%%%%%%%%%%%%%%%%%%%
%%                          Module Description                           %%
%%%%%%%%%%%%%%%%%%%%%%%%%%%%%%%%%%%%%%%%%%%%%%%%%%%%%%%%%%%%%%%%%%%%%%%%%%%

    \index{peach \textit{(package)}!peach.fuzzy \textit{(package)}!peach.fuzzy.norms \textit{(module)}|(}
\section{Module peach.fuzzy.norms}

    \label{peach:fuzzy:norms}

This package implements operations of fuzzy logic.

Basic operations are \texttt{and ({\&})}, \texttt{or (|)} and \texttt{not ({\textasciitilde})}. Those are
implemented as functions of, respectively, two, two and one values. The \texttt{and}
is the t-norm of the fuzzy logic, and it is a function that takes two values and
returns the result of the \texttt{and} operation. The \texttt{or} is a function that takes
two values and returns the result of the \texttt{or} operation. the \texttt{not} is a
function that takes one value and returns the result of the \texttt{not} operation.
To implement your own operations there is no need to subclass -{}- just create the
functions and use them where appropriate.

Also, implication and aglutination functions are defined here. Implication is
the result of the generalized modus ponens used in fuzzy inference systems.
Aglutination is the generalization from two different conclusions used in fuzzy
inference systems. Both are implemented as functions that take two values and
return the result of the operation. As above, to implement your own operations,
there is no need to subclass -{}- just create the functions and use them where
appropriate.

The functions here are provided as convenience.

%%%%%%%%%%%%%%%%%%%%%%%%%%%%%%%%%%%%%%%%%%%%%%%%%%%%%%%%%%%%%%%%%%%%%%%%%%%
%%                               Functions                               %%
%%%%%%%%%%%%%%%%%%%%%%%%%%%%%%%%%%%%%%%%%%%%%%%%%%%%%%%%%%%%%%%%%%%%%%%%%%%

  \subsection{Functions}

    \label{peach:fuzzy:norms:ZadehAnd}
    \index{peach \textit{(package)}!peach.fuzzy \textit{(package)}!peach.fuzzy.norms \textit{(module)}!peach.fuzzy.norms.ZadehAnd \textit{(function)}}

    \vspace{0.5ex}

    \begin{boxedminipage}{\textwidth}

    \raggedright \textbf{ZadehAnd}(\textit{x}, \textit{y})

    \vspace{-1.5ex}

    \rule{\textwidth}{0.5\fboxrule}

And operation as defined by Lofti Zadeh.

And operation is the minimum of the two values.
    \vspace{1ex}

      \textbf{Return Value}
      \begin{quote}

The result of the and operation.
      \end{quote}

    \vspace{1ex}

    \end{boxedminipage}

    \label{peach:fuzzy:norms:ZadehOr}
    \index{peach \textit{(package)}!peach.fuzzy \textit{(package)}!peach.fuzzy.norms \textit{(module)}!peach.fuzzy.norms.ZadehOr \textit{(function)}}

    \vspace{0.5ex}

    \begin{boxedminipage}{\textwidth}

    \raggedright \textbf{ZadehOr}(\textit{x}, \textit{y})

    \vspace{-1.5ex}

    \rule{\textwidth}{0.5\fboxrule}

Or operation as defined by Lofti Zadeh.

Or operation is the maximum of the two values.
    \vspace{1ex}

      \textbf{Return Value}
      \begin{quote}

The result of the or operation.
      \end{quote}

    \vspace{1ex}

    \end{boxedminipage}

    \label{peach:fuzzy:norms:ZadehNot}
    \index{peach \textit{(package)}!peach.fuzzy \textit{(package)}!peach.fuzzy.norms \textit{(module)}!peach.fuzzy.norms.ZadehNot \textit{(function)}}

    \vspace{0.5ex}

    \begin{boxedminipage}{\textwidth}

    \raggedright \textbf{ZadehNot}(\textit{x})

    \vspace{-1.5ex}

    \rule{\textwidth}{0.5\fboxrule}

Not operation as defined by Lofti Zadeh.

Not operation is the complement to 1 of the given value, that is, \texttt{1 - x}.
    \vspace{1ex}

      \textbf{Return Value}
      \begin{quote}

The result of the not operation.
      \end{quote}

    \vspace{1ex}

    \end{boxedminipage}

    \label{peach:fuzzy:norms:MamdaniImplication}
    \index{peach \textit{(package)}!peach.fuzzy \textit{(package)}!peach.fuzzy.norms \textit{(module)}!peach.fuzzy.norms.MamdaniImplication \textit{(function)}}

    \vspace{0.5ex}

    \begin{boxedminipage}{\textwidth}

    \raggedright \textbf{MamdaniImplication}(\textit{x}, \textit{y})

    \vspace{-1.5ex}

    \rule{\textwidth}{0.5\fboxrule}

Implication operation as defined by Mamdani.

Implication is the minimum of the two values.
    \vspace{1ex}

      \textbf{Return Value}
      \begin{quote}

The result of the implication.
      \end{quote}

    \vspace{1ex}

    \end{boxedminipage}

    \label{peach:fuzzy:norms:MamdaniAglutination}
    \index{peach \textit{(package)}!peach.fuzzy \textit{(package)}!peach.fuzzy.norms \textit{(module)}!peach.fuzzy.norms.MamdaniAglutination \textit{(function)}}

    \vspace{0.5ex}

    \begin{boxedminipage}{\textwidth}

    \raggedright \textbf{MamdaniAglutination}(\textit{x}, \textit{y})

    \vspace{-1.5ex}

    \rule{\textwidth}{0.5\fboxrule}

Aglutination as defined by Mamdani.

Aglutination is the maximum of the two values.
    \vspace{1ex}

      \textbf{Return Value}
      \begin{quote}

The result of the aglutination.
      \end{quote}

    \vspace{1ex}

    \end{boxedminipage}

    \label{peach:fuzzy:norms:ProbabilisticAnd}
    \index{peach \textit{(package)}!peach.fuzzy \textit{(package)}!peach.fuzzy.norms \textit{(module)}!peach.fuzzy.norms.ProbabilisticAnd \textit{(function)}}

    \vspace{0.5ex}

    \begin{boxedminipage}{\textwidth}

    \raggedright \textbf{ProbabilisticAnd}(\textit{x}, \textit{y})

    \vspace{-1.5ex}

    \rule{\textwidth}{0.5\fboxrule}

And operation as a probabilistic operation.

And operation is the product of the two values.
    \vspace{1ex}

      \textbf{Return Value}
      \begin{quote}

The result of the and operation.
      \end{quote}

    \vspace{1ex}

    \end{boxedminipage}

    \label{peach:fuzzy:norms:ProbabilisticOr}
    \index{peach \textit{(package)}!peach.fuzzy \textit{(package)}!peach.fuzzy.norms \textit{(module)}!peach.fuzzy.norms.ProbabilisticOr \textit{(function)}}

    \vspace{0.5ex}

    \begin{boxedminipage}{\textwidth}

    \raggedright \textbf{ProbabilisticOr}(\textit{x}, \textit{y})

    \vspace{-1.5ex}

    \rule{\textwidth}{0.5\fboxrule}

Or operation as a probabilistic operation.

Or operation is given as the probability of the intersection of two events,
that is, x + y - xy.
    \vspace{1ex}

      \textbf{Return Value}
      \begin{quote}

The result of the or operation.
      \end{quote}

    \vspace{1ex}

    \end{boxedminipage}

    \label{peach:fuzzy:norms:ProbabilisticNot}
    \index{peach \textit{(package)}!peach.fuzzy \textit{(package)}!peach.fuzzy.norms \textit{(module)}!peach.fuzzy.norms.ProbabilisticNot \textit{(function)}}

    \vspace{0.5ex}

    \begin{boxedminipage}{\textwidth}

    \raggedright \textbf{ProbabilisticNot}(\textit{x})

    \vspace{-1.5ex}

    \rule{\textwidth}{0.5\fboxrule}

Not operation as a probabilistic operation.

Not operation is the complement to 1 of the given value, that is, \texttt{1 - x}.
    \vspace{1ex}

      \textbf{Return Value}
      \begin{quote}

The result of the not operation.
      \end{quote}

    \vspace{1ex}

    \end{boxedminipage}

    \label{peach:fuzzy:norms:ProbabilisticImplication}
    \index{peach \textit{(package)}!peach.fuzzy \textit{(package)}!peach.fuzzy.norms \textit{(module)}!peach.fuzzy.norms.ProbabilisticImplication \textit{(function)}}

    \vspace{0.5ex}

    \begin{boxedminipage}{\textwidth}

    \raggedright \textbf{ProbabilisticImplication}(\textit{x}, \textit{y})

    \vspace{-1.5ex}

    \rule{\textwidth}{0.5\fboxrule}

Implication as a probabilistic operation.

Implication is the product of the two values.
    \vspace{1ex}

      \textbf{Return Value}
      \begin{quote}

The result of the and implication.
      \end{quote}

    \vspace{1ex}

    \end{boxedminipage}

    \label{peach:fuzzy:norms:ProbabilisticAglutination}
    \index{peach \textit{(package)}!peach.fuzzy \textit{(package)}!peach.fuzzy.norms \textit{(module)}!peach.fuzzy.norms.ProbabilisticAglutination \textit{(function)}}

    \vspace{0.5ex}

    \begin{boxedminipage}{\textwidth}

    \raggedright \textbf{ProbabilisticAglutination}(\textit{x}, \textit{y})

    \vspace{-1.5ex}

    \rule{\textwidth}{0.5\fboxrule}

Implication as a probabilistic operation.

Implication is given as the probability of the intersection of two events,
that is, x + y - xy.
    \vspace{1ex}

      \textbf{Return Value}
      \begin{quote}

The result of the and implication.
      \end{quote}

    \vspace{1ex}

    \end{boxedminipage}


%%%%%%%%%%%%%%%%%%%%%%%%%%%%%%%%%%%%%%%%%%%%%%%%%%%%%%%%%%%%%%%%%%%%%%%%%%%
%%                               Variables                               %%
%%%%%%%%%%%%%%%%%%%%%%%%%%%%%%%%%%%%%%%%%%%%%%%%%%%%%%%%%%%%%%%%%%%%%%%%%%%

  \subsection{Variables}

\begin{longtable}{|p{.30\textwidth}|p{.62\textwidth}|l}
\cline{1-2}
\cline{1-2} \centering \textbf{Name} & \centering \textbf{Description}& \\
\cline{1-2}
\endhead\cline{1-2}\multicolumn{3}{r}{\small\textit{continued on next page}}\\\endfoot\cline{1-2}
\endlastfoot\raggedright \_\-\_\-d\-o\-c\-\_\-\_\- & \raggedright \textbf{Value:} 
{\tt \texttt{...}}&\\
\cline{1-2}
\raggedright Z\-A\-D\-E\-H\-\_\-N\-O\-R\-M\-S\- & \raggedright Tuple containing, in order, Zadeh and, or and not operations

\textbf{Value:} 
{\tt ZadehAnd, ZadehOr, ZadehNot}&\\
\cline{1-2}
\raggedright M\-A\-M\-D\-A\-N\-I\-\_\-I\-N\-F\-E\-R\-E\-N\-C\-E\- & \raggedright Tuple containing, in order, Mamdani implication and algutination

\textbf{Value:} 
{\tt MamdaniImplication, MamdaniAglutination}&\\
\cline{1-2}
\raggedright P\-R\-O\-B\-\_\-N\-O\-R\-M\-S\- & \raggedright Tuple containing, in order, probabilistic and, or and not operations

\textbf{Value:} 
{\tt ProbabilisticAnd, ProbabilisticOr, ProbabilisticNot}&\\
\cline{1-2}
\raggedright P\-R\-O\-B\-\_\-I\-N\-F\-E\-R\-E\-N\-C\-E\- & \raggedright Tuple containing, in order, probabilistic implication and algutination

\textbf{Value:} 
{\tt ProbabilisticImplication, ProbabilisticAglutination}&\\
\cline{1-2}
\end{longtable}

    \index{peach \textit{(package)}!peach.fuzzy \textit{(package)}!peach.fuzzy.norms \textit{(module)}|)}

%
% API Documentation for Peach - Computational Intelligence for Python
% Package peach.ga
%
% Generated by epydoc 3.0.1
% [Mon Jan 24 15:39:50 2011]
%

%%%%%%%%%%%%%%%%%%%%%%%%%%%%%%%%%%%%%%%%%%%%%%%%%%%%%%%%%%%%%%%%%%%%%%%%%%%
%%                          Module Description                           %%
%%%%%%%%%%%%%%%%%%%%%%%%%%%%%%%%%%%%%%%%%%%%%%%%%%%%%%%%%%%%%%%%%%%%%%%%%%%

    \index{peach \textit{(package)}!peach.ga \textit{(package)}|(}
\section{Package peach.ga}

    \label{peach:ga}

This package implements genetic algorithms. Consult:
%
\begin{quote}
%
\begin{description}
\item[{base}] \leavevmode 
Implementation of the basic genetic algorithm;

\item[{chromosome}] \leavevmode 
Basic definitions to work with chromosomes. Defined as arrays of bits;

\item[{crossover}] \leavevmode 
Defines crossover operators and base classes;

\item[{fitness}] \leavevmode 
Defines fitness functions and base classes;

\item[{mutation}] \leavevmode 
Defines mutation operators and base classes;

\item[{selection}] \leavevmode 
Defines selection operators and base classes;

\end{description}

\end{quote}

%%%%%%%%%%%%%%%%%%%%%%%%%%%%%%%%%%%%%%%%%%%%%%%%%%%%%%%%%%%%%%%%%%%%%%%%%%%
%%                                Modules                                %%
%%%%%%%%%%%%%%%%%%%%%%%%%%%%%%%%%%%%%%%%%%%%%%%%%%%%%%%%%%%%%%%%%%%%%%%%%%%

\subsection{Modules}

\begin{itemize}
\setlength{\parskip}{0ex}
\item \textbf{base}: 
Basic Genetic Algorithm (GA)


  \textit{(Section \ref{peach:ga:base}, p.~\pageref{peach:ga:base})}

\item \textbf{chromosome}: 
Basic definitions and classes for manipulating chromosomes


  \textit{(Section \ref{peach:ga:chromosome}, p.~\pageref{peach:ga:chromosome})}

\item \textbf{crossover}: 
Basic definitions for crossover operations and base classes.


  \textit{(Section \ref{peach:ga:crossover}, p.~\pageref{peach:ga:crossover})}

\item \textbf{fitness}: 
Basic definitions and base classes for definition of fitness functions for use
with genetic algorithms.


  \textit{(Section \ref{peach:ga:fitness}, p.~\pageref{peach:ga:fitness})}

\item \textbf{mutation}: 
Basic definitions and classes for operating mutation on chromosomes.


  \textit{(Section \ref{peach:ga:mutation}, p.~\pageref{peach:ga:mutation})}

\item \textbf{selection}: 
Basic classes and definitions for selection operator.


  \textit{(Section \ref{peach:ga:selection}, p.~\pageref{peach:ga:selection})}

\end{itemize}

    \index{peach \textit{(package)}!peach.ga \textit{(package)}|)}

%
% API Documentation for Peach - Computational Intelligence for Python
% Module peach.ga.base
%
% Generated by epydoc 3.0.1
% [Fri Feb  4 17:21:19 2011]
%

%%%%%%%%%%%%%%%%%%%%%%%%%%%%%%%%%%%%%%%%%%%%%%%%%%%%%%%%%%%%%%%%%%%%%%%%%%%
%%                          Module Description                           %%
%%%%%%%%%%%%%%%%%%%%%%%%%%%%%%%%%%%%%%%%%%%%%%%%%%%%%%%%%%%%%%%%%%%%%%%%%%%

    \index{peach \textit{(package)}!peach.ga \textit{(package)}!peach.ga.base \textit{(module)}|(}
\section{Module peach.ga.base}

    \label{peach:ga:base}

Basic Genetic Algorithm (GA)

This sub-package implements a traditional genetic algorithm as described in
books and papers. It consists of selecting, breeding and mutating a population
of chromosomes (arrays of bits) and reinserting the fittest individual from the
previous generation if the GA is elitist. Please, consult a good reference on
the subject, for the subject is way too complicated to be explained here.

Within the algorithm implemented here, it is possible to specify and configure
the selection, crossover and mutation methods using the classes in the
respective sub-modules and custom methods can be implemented (check
\texttt{selection}, \texttt{crossover} and \texttt{mutation} modules).

A GA object is actually a list of chromosomes. Please, refer to the
documentation of the class below for more information.

%%%%%%%%%%%%%%%%%%%%%%%%%%%%%%%%%%%%%%%%%%%%%%%%%%%%%%%%%%%%%%%%%%%%%%%%%%%
%%                               Variables                               %%
%%%%%%%%%%%%%%%%%%%%%%%%%%%%%%%%%%%%%%%%%%%%%%%%%%%%%%%%%%%%%%%%%%%%%%%%%%%

  \subsection{Variables}

    \vspace{-1cm}
\hspace{\varindent}\begin{longtable}{|p{\varnamewidth}|p{\vardescrwidth}|l}
\cline{1-2}
\cline{1-2} \centering \textbf{Name} & \centering \textbf{Description}& \\
\cline{1-2}
\endhead\cline{1-2}\multicolumn{3}{r}{\small\textit{continued on next page}}\\\endfoot\cline{1-2}
\endlastfoot\raggedright \_\-\_\-d\-o\-c\-\_\-\_\- & \raggedright \textbf{Value:} 
{\tt \texttt{...}}&\\
\cline{1-2}
\raggedright \_\-\_\-p\-a\-c\-k\-a\-g\-e\-\_\-\_\- & \raggedright \textbf{Value:} 
{\tt \texttt{'}\texttt{peach.ga}\texttt{'}}&\\
\cline{1-2}
\raggedright a\-d\-d\- & \raggedright \textbf{Value:} 
{\tt {\textless}ufunc 'add'{\textgreater}}&\\
\cline{1-2}
\end{longtable}


%%%%%%%%%%%%%%%%%%%%%%%%%%%%%%%%%%%%%%%%%%%%%%%%%%%%%%%%%%%%%%%%%%%%%%%%%%%
%%                           Class Description                           %%
%%%%%%%%%%%%%%%%%%%%%%%%%%%%%%%%%%%%%%%%%%%%%%%%%%%%%%%%%%%%%%%%%%%%%%%%%%%

    \index{peach \textit{(package)}!peach.ga \textit{(package)}!peach.ga.base \textit{(module)}!peach.ga.base.GeneticAlgorithm \textit{(class)}|(}
\subsection{Class GeneticAlgorithm}

    \label{peach:ga:base:GeneticAlgorithm}
\begin{tabular}{cccccccc}
% Line for object, linespec=[False, False]
\multicolumn{2}{r}{\settowidth{\BCL}{object}\multirow{2}{\BCL}{object}}
&&
&&
  \\\cline{3-3}
  &&\multicolumn{1}{c|}{}
&&
&&
  \\
% Line for list, linespec=[False]
\multicolumn{4}{r}{\settowidth{\BCL}{list}\multirow{2}{\BCL}{list}}
&&
  \\\cline{5-5}
  &&&&\multicolumn{1}{c|}{}
&&
  \\
&&&&\multicolumn{2}{l}{\textbf{peach.ga.base.GeneticAlgorithm}}
\end{tabular}

\textbf{Known Subclasses:} peach.ga.base.GA


A standard Genetic Algorithm

This class implements the methods to generate, initialize and evolve a
population of chromosomes according to a given fitness function. A standard
GA implements, in this order:
%
\begin{quote}
%
\begin{itemize}

\item A selection method, to choose, from this generation, which individuals
will be present in the next generation;

\item A crossover method, to exchange information between selected individuals
to add diversity to the population;

\item A mutation method, to change information in a selected individual, also
to add diversity to the population;

\item The reinsertion of the fittest individual, if the population is elitist
(which is almost always the case).

\end{itemize}

\end{quote}

A population is actually a list of chromosomes, and individuals can be
read and set as in a normal list. Use the \texttt{{[} {]}} operators to access
individual chromosomes but please be aware that modifying the information on
the list before the end of convergence can cause unpredictable results. The
population and the algorithm have also other properties, check below to see
more information on them.

%%%%%%%%%%%%%%%%%%%%%%%%%%%%%%%%%%%%%%%%%%%%%%%%%%%%%%%%%%%%%%%%%%%%%%%%%%%
%%                                Methods                                %%
%%%%%%%%%%%%%%%%%%%%%%%%%%%%%%%%%%%%%%%%%%%%%%%%%%%%%%%%%%%%%%%%%%%%%%%%%%%

  \subsubsection{Methods}

    \vspace{0.5ex}

\hspace{.8\funcindent}\begin{boxedminipage}{\funcwidth}

    \raggedright \textbf{\_\_init\_\_}(\textit{self}, \textit{f}, \textit{x0}, \textit{ranges}={\tt \texttt{[}\texttt{]}}, \textit{fmt}={\tt \texttt{'}\texttt{f}\texttt{'}}, \textit{fitness}={\tt {\textless}class 'peach.ga.fitness.Fitness'{\textgreater}}, \textit{selection}={\tt {\textless}class 'peach.ga.selection.RouletteWheel'{\textgreater}}, \textit{crossover}={\tt {\textless}class 'peach.ga.crossover.TwoPoint'{\textgreater}}, \textit{mutation}={\tt {\textless}class 'peach.ga.mutation.BitToBit'{\textgreater}}, \textit{elitist}={\tt True})

    \vspace{-1.5ex}

    \rule{\textwidth}{0.5\fboxrule}
\setlength{\parskip}{2ex}

Initializes the population and the algorithm.

On the initialization of the population, a lot of parameters can be set.
Those will deeply affect the results. The parameters are:
\setlength{\parskip}{1ex}
      \textbf{Parameters}
      \vspace{-1ex}

      \begin{quote}
        \begin{Ventry}{xxxxxxxxx}

          \item[f]


A multivariable function to be evaluated. The nature of the
parameters in the objective function will depend of the way you want
the genetic algorithm to process. It can be a standard function that
receives a one-dimensional array of values and computes the value of
the function. In this case, the values will be passed as a tuple,
instead of an array. This is so that integer, floats and other types
of values can be passed and processed. In this case, the values will
depend of the format string (see below)

If you don't supply a format, your objective function will receive a
\texttt{Chromosome} instance, and it is the responsability of the
function to decode the array of bits in any way. Notice that, while
it is more flexible, it is certainly more difficult to deal with.
Your function should process the bits and compute the return value
which, in any case, should be a scalar.

Please, note that genetic algorithms maximize functions, so project
your objective function accordingly. If you want to minimize a
function, return its negated value.
          \item[x0]


A population of first estimates. This is a list, array or tuple of
one-dimension arrays, each one corresponding to an estimate of the
position of the minimum. The population size of the algorithm will
be the same as the number of estimates in this list. Each component
of the vectors in this list are one of the variables in the function
to be optimized.
          \item[ranges]


Since messing with the bits can change substantially the values
obtained can diverge a lot from the maximum point. To avoid this,
you can specify a range for each of the variables. \texttt{range}
defaults to \texttt{{[} {]}}, this means that no range checkin will be done.
If given, then every variable will be checked. There are two ways to
specify the ranges.

It might be a tuple of two values, \texttt{(x0, x1)}, where \texttt{x0} is the
start of the interval, and \texttt{x1} its end. Obviously, \texttt{x0} should
be smaller than \texttt{x1}. If \texttt{range} is given in this way, then this
range will be used for every variable.

It can be specified as a list of tuples with the same format as
given above. In that case, the list must have one range for every
variable specified in the format and the ranges must appear in the
same order as there. That is, every variable must have a range
associated to it.
          \item[fmt]


A \texttt{struct}-format string. The \texttt{struct} module is a standard
Python module that packs and unpacks informations in bits. These
are used to inform the algorithm what types of data are to be used.
For example, if you are maximizing a function of three real
variables, the format should be something like \texttt{"fff"}. Any type
supported by the \texttt{struct} module can be used. The GA will decode
the bit array according to this format and send it as is to your
fitness function -{}- your function \emph{must} know what to do with them.

Alternatively, the format can be an integer. In that case, the GA
will not try to decode the bit sequence. Instead, the bits are
passed without modification to the objective function, which must
deal with them. Notice that, if this is used this way, the
\texttt{ranges} property (see below) makes no sense, so it is set to
\texttt{None}. Also, no sanity checks will be performed.

It defaults to \texttt{``f''}, that is, a single floating point variable.
          \item[fitness]


A fitness method to be applied over the objective function. This
parameter must be a \texttt{Fitness} instance or subclass. It will be
applied over the objective function to compute the fitness of every
individual in the population. Please, see the documentation on the
\texttt{Fitness} class.
          \item[selection]


This specifies the selection method. You can use one given in the
\texttt{selection} sub-module, or you can implement your own. In any
case, the \texttt{selection} parameter must be an instance of
\texttt{Selection} or of a subclass. Please, see the documentation on the
\texttt{selection} module for more information. Defaults to
\texttt{RouletteWheel}. If made \texttt{None}, then selection will not be
present in the GA.
          \item[crossover]


This specifies the crossover method. You can use one given in the
\texttt{crossover} sub-module, or you can implement your own. In any
case, the \texttt{crossover} parameter must be an instance of
\texttt{Crossover} or of a subclass. Please, see the documentation on the
\texttt{crossover} module for more information. Defaults to
\texttt{TwoPoint}. If made \texttt{None}, then crossover will not be
present in the GA.
          \item[mutation]


This specifies the mutation method. You can use one given in the
\texttt{mutation} sub-module, or you can implement your own. In any
case, the \texttt{mutation} parameter must be an instance of \texttt{Mutation}
or of a subclass. Please, see the documentation on the \texttt{mutation}
module for more information. Defaults to \texttt{BitToBit}.  If made
\texttt{None}, then mutation will not be present in the GA.
          \item[elitist]


Defines if the population is elitist or not. An elitist population
will never discard the fittest individual when a new generation is
computed. Defaults to \texttt{True}.
        \end{Ventry}

      \end{quote}

      \textbf{Return Value}
    \vspace{-1ex}

      \begin{quote}

new empty list
      \end{quote}

      Overrides: object.\_\_init\_\_

    \end{boxedminipage}

    \label{peach:ga:base:GeneticAlgorithm:sanity}
    \index{peach \textit{(package)}!peach.ga \textit{(package)}!peach.ga.base \textit{(module)}!peach.ga.base.GeneticAlgorithm \textit{(class)}!peach.ga.base.GeneticAlgorithm.sanity \textit{(method)}}

    \vspace{0.5ex}

\hspace{.8\funcindent}\begin{boxedminipage}{\funcwidth}

    \raggedright \textbf{sanity}(\textit{self})

    \vspace{-1.5ex}

    \rule{\textwidth}{0.5\fboxrule}
\setlength{\parskip}{2ex}

Sanitizes the chromosomes in the population.

Since not every individual generated by the crossover and mutation
operations might be a valid result, this method verifies if they are
inside the allowed ranges (or if it is a number at all). Each invalid
individual is discarded and a new one is generated.

This method has no parameters and returns no values.
\setlength{\parskip}{1ex}
    \end{boxedminipage}

    \label{peach:ga:base:GeneticAlgorithm:restart}
    \index{peach \textit{(package)}!peach.ga \textit{(package)}!peach.ga.base \textit{(module)}!peach.ga.base.GeneticAlgorithm \textit{(class)}!peach.ga.base.GeneticAlgorithm.restart \textit{(method)}}

    \vspace{0.5ex}

\hspace{.8\funcindent}\begin{boxedminipage}{\funcwidth}

    \raggedright \textbf{restart}(\textit{self}, \textit{x0})

    \vspace{-1.5ex}

    \rule{\textwidth}{0.5\fboxrule}
\setlength{\parskip}{2ex}

Resets the optimizer, allowing the use of a new set of estimates. This
can be used to avoid stagnation.
\setlength{\parskip}{1ex}
      \textbf{Parameters}
      \vspace{-1ex}

      \begin{quote}
        \begin{Ventry}{xx}

          \item[x0]


A new set of estimates. It doesn't need to have the same size of the
original population, but it must be a list of estimates in the same
format as in the object instantiation. Please, see the documentation
on the instantiation of the class.
        \end{Ventry}

      \end{quote}

    \end{boxedminipage}

    \label{peach:ga:base:GeneticAlgorithm:step}
    \index{peach \textit{(package)}!peach.ga \textit{(package)}!peach.ga.base \textit{(module)}!peach.ga.base.GeneticAlgorithm \textit{(class)}!peach.ga.base.GeneticAlgorithm.step \textit{(method)}}

    \vspace{0.5ex}

\hspace{.8\funcindent}\begin{boxedminipage}{\funcwidth}

    \raggedright \textbf{step}(\textit{self})

    \vspace{-1.5ex}

    \rule{\textwidth}{0.5\fboxrule}
\setlength{\parskip}{2ex}

Computes a new generation of the population, a step of the adaptation.

This method goes through all the steps of the GA, as described above. If
the selection, crossover and mutation operators are defined, they are
applied over the population. If the population is elitist, then the
fittest individual of the past generation is reinserted.

This method has no parameters and returns no values. The GA itself can
be consulted (using \texttt{{[} {]}}) to find the fittest individual which is the
result of the process.
\setlength{\parskip}{1ex}
    \end{boxedminipage}

    \label{peach:ga:base:GeneticAlgorithm:__call__}
    \index{peach \textit{(package)}!peach.ga \textit{(package)}!peach.ga.base \textit{(module)}!peach.ga.base.GeneticAlgorithm \textit{(class)}!peach.ga.base.GeneticAlgorithm.\_\_call\_\_ \textit{(method)}}

    \vspace{0.5ex}

\hspace{.8\funcindent}\begin{boxedminipage}{\funcwidth}

    \raggedright \textbf{\_\_call\_\_}(\textit{self})

    \vspace{-1.5ex}

    \rule{\textwidth}{0.5\fboxrule}
\setlength{\parskip}{2ex}

Transparently executes the search until the minimum is found. The stop
criteria are the maximum error or the maximum number of iterations,
whichever is reached first. Note that this is a \texttt{\_\_call\_\_} method, so
the object is called as a function. This method returns a tuple
\texttt{(x, e)}, with the best estimate of the minimum and the error.
\setlength{\parskip}{1ex}
      \textbf{Return Value}
    \vspace{-1ex}

      \begin{quote}

This method returns a tuple \texttt{(x, e)}, where \texttt{x} is the best
estimate of the minimum, and \texttt{e} is the estimated error.
      \end{quote}

    \end{boxedminipage}


\large{\textbf{\textit{Inherited from list}}}

\begin{quote}
\_\_add\_\_(), \_\_contains\_\_(), \_\_delitem\_\_(), \_\_delslice\_\_(), \_\_eq\_\_(), \_\_ge\_\_(), \_\_getattribute\_\_(), \_\_getitem\_\_(), \_\_getslice\_\_(), \_\_gt\_\_(), \_\_iadd\_\_(), \_\_imul\_\_(), \_\_iter\_\_(), \_\_le\_\_(), \_\_len\_\_(), \_\_lt\_\_(), \_\_mul\_\_(), \_\_ne\_\_(), \_\_new\_\_(), \_\_repr\_\_(), \_\_reversed\_\_(), \_\_rmul\_\_(), \_\_setitem\_\_(), \_\_setslice\_\_(), \_\_sizeof\_\_(), append(), count(), extend(), index(), insert(), pop(), remove(), reverse(), sort()
\end{quote}

\large{\textbf{\textit{Inherited from object}}}

\begin{quote}
\_\_delattr\_\_(), \_\_format\_\_(), \_\_reduce\_\_(), \_\_reduce\_ex\_\_(), \_\_setattr\_\_(), \_\_str\_\_(), \_\_subclasshook\_\_()
\end{quote}

%%%%%%%%%%%%%%%%%%%%%%%%%%%%%%%%%%%%%%%%%%%%%%%%%%%%%%%%%%%%%%%%%%%%%%%%%%%
%%                              Properties                               %%
%%%%%%%%%%%%%%%%%%%%%%%%%%%%%%%%%%%%%%%%%%%%%%%%%%%%%%%%%%%%%%%%%%%%%%%%%%%

  \subsubsection{Properties}

    \vspace{-1cm}
\hspace{\varindent}\begin{longtable}{|p{\varnamewidth}|p{\vardescrwidth}|l}
\cline{1-2}
\cline{1-2} \centering \textbf{Name} & \centering \textbf{Description}& \\
\cline{1-2}
\endhead\cline{1-2}\multicolumn{3}{r}{\small\textit{continued on next page}}\\\endfoot\cline{1-2}
\endlastfoot\raggedright c\-h\-r\-o\-m\-o\-s\-o\-m\-e\-\_\-s\-i\-z\-e\- & &\\
\cline{1-2}
\raggedright f\-x\- & &\\
\cline{1-2}
\raggedright b\-e\-s\-t\- & &\\
\cline{1-2}
\raggedright f\-b\-e\-s\-t\- & &\\
\cline{1-2}
\raggedright f\-i\-t\-n\-e\-s\-s\- & &\\
\cline{1-2}
\multicolumn{2}{|l|}{\textit{Inherited from object}}\\
\multicolumn{2}{|p{\varwidth}|}{\raggedright \_\_class\_\_}\\
\cline{1-2}
\end{longtable}


%%%%%%%%%%%%%%%%%%%%%%%%%%%%%%%%%%%%%%%%%%%%%%%%%%%%%%%%%%%%%%%%%%%%%%%%%%%
%%                            Class Variables                            %%
%%%%%%%%%%%%%%%%%%%%%%%%%%%%%%%%%%%%%%%%%%%%%%%%%%%%%%%%%%%%%%%%%%%%%%%%%%%

  \subsubsection{Class Variables}

    \vspace{-1cm}
\hspace{\varindent}\begin{longtable}{|p{\varnamewidth}|p{\vardescrwidth}|l}
\cline{1-2}
\cline{1-2} \centering \textbf{Name} & \centering \textbf{Description}& \\
\cline{1-2}
\endhead\cline{1-2}\multicolumn{3}{r}{\small\textit{continued on next page}}\\\endfoot\cline{1-2}
\endlastfoot\multicolumn{2}{|l|}{\textit{Inherited from list}}\\
\multicolumn{2}{|p{\varwidth}|}{\raggedright \_\_hash\_\_}\\
\cline{1-2}
\end{longtable}


%%%%%%%%%%%%%%%%%%%%%%%%%%%%%%%%%%%%%%%%%%%%%%%%%%%%%%%%%%%%%%%%%%%%%%%%%%%
%%                          Instance Variables                           %%
%%%%%%%%%%%%%%%%%%%%%%%%%%%%%%%%%%%%%%%%%%%%%%%%%%%%%%%%%%%%%%%%%%%%%%%%%%%

  \subsubsection{Instance Variables}

    \vspace{-1cm}
\hspace{\varindent}\begin{longtable}{|p{\varnamewidth}|p{\vardescrwidth}|l}
\cline{1-2}
\cline{1-2} \centering \textbf{Name} & \centering \textbf{Description}& \\
\cline{1-2}
\endhead\cline{1-2}\multicolumn{3}{r}{\small\textit{continued on next page}}\\\endfoot\cline{1-2}
\endlastfoot\raggedright e\-l\-i\-t\-i\-s\-t\- & If \texttt{True}, then the population is elitist.&\\
\cline{1-2}
\raggedright r\-a\-n\-g\-e\-s\- & Holds the ranges for every variable. Although it is a
writable property, care should be taken in changing parameters
before ending the convergence.&\\
\cline{1-2}
\end{longtable}

    \index{peach \textit{(package)}!peach.ga \textit{(package)}!peach.ga.base \textit{(module)}!peach.ga.base.GeneticAlgorithm \textit{(class)}|)}

%%%%%%%%%%%%%%%%%%%%%%%%%%%%%%%%%%%%%%%%%%%%%%%%%%%%%%%%%%%%%%%%%%%%%%%%%%%
%%                           Class Description                           %%
%%%%%%%%%%%%%%%%%%%%%%%%%%%%%%%%%%%%%%%%%%%%%%%%%%%%%%%%%%%%%%%%%%%%%%%%%%%

    \index{peach \textit{(package)}!peach.ga \textit{(package)}!peach.ga.base \textit{(module)}!peach.ga.base.GA \textit{(class)}|(}
\subsection{Class GA}

    \label{peach:ga:base:GA}
\begin{tabular}{cccccccccc}
% Line for object, linespec=[False, False, False]
\multicolumn{2}{r}{\settowidth{\BCL}{object}\multirow{2}{\BCL}{object}}
&&
&&
&&
  \\\cline{3-3}
  &&\multicolumn{1}{c|}{}
&&
&&
&&
  \\
% Line for list, linespec=[False, False]
\multicolumn{4}{r}{\settowidth{\BCL}{list}\multirow{2}{\BCL}{list}}
&&
&&
  \\\cline{5-5}
  &&&&\multicolumn{1}{c|}{}
&&
&&
  \\
% Line for peach.ga.base.GeneticAlgorithm, linespec=[False]
\multicolumn{6}{r}{\settowidth{\BCL}{peach.ga.base.GeneticAlgorithm}\multirow{2}{\BCL}{peach.ga.base.GeneticAlgorithm}}
&&
  \\\cline{7-7}
  &&&&&&\multicolumn{1}{c|}{}
&&
  \\
&&&&&&\multicolumn{2}{l}{\textbf{peach.ga.base.GA}}
\end{tabular}


GA is an alias to \texttt{GeneticAlgorithm}

%%%%%%%%%%%%%%%%%%%%%%%%%%%%%%%%%%%%%%%%%%%%%%%%%%%%%%%%%%%%%%%%%%%%%%%%%%%
%%                                Methods                                %%
%%%%%%%%%%%%%%%%%%%%%%%%%%%%%%%%%%%%%%%%%%%%%%%%%%%%%%%%%%%%%%%%%%%%%%%%%%%

  \subsubsection{Methods}


\large{\textbf{\textit{Inherited from peach.ga.base.GeneticAlgorithm\textit{(Section \ref{peach:ga:base:GeneticAlgorithm})}}}}

\begin{quote}
\_\_call\_\_(), \_\_init\_\_(), restart(), sanity(), step()
\end{quote}

\large{\textbf{\textit{Inherited from list}}}

\begin{quote}
\_\_add\_\_(), \_\_contains\_\_(), \_\_delitem\_\_(), \_\_delslice\_\_(), \_\_eq\_\_(), \_\_ge\_\_(), \_\_getattribute\_\_(), \_\_getitem\_\_(), \_\_getslice\_\_(), \_\_gt\_\_(), \_\_iadd\_\_(), \_\_imul\_\_(), \_\_iter\_\_(), \_\_le\_\_(), \_\_len\_\_(), \_\_lt\_\_(), \_\_mul\_\_(), \_\_ne\_\_(), \_\_new\_\_(), \_\_repr\_\_(), \_\_reversed\_\_(), \_\_rmul\_\_(), \_\_setitem\_\_(), \_\_setslice\_\_(), \_\_sizeof\_\_(), append(), count(), extend(), index(), insert(), pop(), remove(), reverse(), sort()
\end{quote}

\large{\textbf{\textit{Inherited from object}}}

\begin{quote}
\_\_delattr\_\_(), \_\_format\_\_(), \_\_reduce\_\_(), \_\_reduce\_ex\_\_(), \_\_setattr\_\_(), \_\_str\_\_(), \_\_subclasshook\_\_()
\end{quote}

%%%%%%%%%%%%%%%%%%%%%%%%%%%%%%%%%%%%%%%%%%%%%%%%%%%%%%%%%%%%%%%%%%%%%%%%%%%
%%                              Properties                               %%
%%%%%%%%%%%%%%%%%%%%%%%%%%%%%%%%%%%%%%%%%%%%%%%%%%%%%%%%%%%%%%%%%%%%%%%%%%%

  \subsubsection{Properties}

    \vspace{-1cm}
\hspace{\varindent}\begin{longtable}{|p{\varnamewidth}|p{\vardescrwidth}|l}
\cline{1-2}
\cline{1-2} \centering \textbf{Name} & \centering \textbf{Description}& \\
\cline{1-2}
\endhead\cline{1-2}\multicolumn{3}{r}{\small\textit{continued on next page}}\\\endfoot\cline{1-2}
\endlastfoot\multicolumn{2}{|l|}{\textit{Inherited from peach.ga.base.GeneticAlgorithm \textit{(Section \ref{peach:ga:base:GeneticAlgorithm})}}}\\
\multicolumn{2}{|p{\varwidth}|}{\raggedright best, chromosome\_size, fbest, fitness, fx}\\
\cline{1-2}
\multicolumn{2}{|l|}{\textit{Inherited from object}}\\
\multicolumn{2}{|p{\varwidth}|}{\raggedright \_\_class\_\_}\\
\cline{1-2}
\end{longtable}


%%%%%%%%%%%%%%%%%%%%%%%%%%%%%%%%%%%%%%%%%%%%%%%%%%%%%%%%%%%%%%%%%%%%%%%%%%%
%%                            Class Variables                            %%
%%%%%%%%%%%%%%%%%%%%%%%%%%%%%%%%%%%%%%%%%%%%%%%%%%%%%%%%%%%%%%%%%%%%%%%%%%%

  \subsubsection{Class Variables}

    \vspace{-1cm}
\hspace{\varindent}\begin{longtable}{|p{\varnamewidth}|p{\vardescrwidth}|l}
\cline{1-2}
\cline{1-2} \centering \textbf{Name} & \centering \textbf{Description}& \\
\cline{1-2}
\endhead\cline{1-2}\multicolumn{3}{r}{\small\textit{continued on next page}}\\\endfoot\cline{1-2}
\endlastfoot\multicolumn{2}{|l|}{\textit{Inherited from list}}\\
\multicolumn{2}{|p{\varwidth}|}{\raggedright \_\_hash\_\_}\\
\cline{1-2}
\end{longtable}


%%%%%%%%%%%%%%%%%%%%%%%%%%%%%%%%%%%%%%%%%%%%%%%%%%%%%%%%%%%%%%%%%%%%%%%%%%%
%%                          Instance Variables                           %%
%%%%%%%%%%%%%%%%%%%%%%%%%%%%%%%%%%%%%%%%%%%%%%%%%%%%%%%%%%%%%%%%%%%%%%%%%%%

  \subsubsection{Instance Variables}

    \vspace{-1cm}
\hspace{\varindent}\begin{longtable}{|p{\varnamewidth}|p{\vardescrwidth}|l}
\cline{1-2}
\cline{1-2} \centering \textbf{Name} & \centering \textbf{Description}& \\
\cline{1-2}
\endhead\cline{1-2}\multicolumn{3}{r}{\small\textit{continued on next page}}\\\endfoot\cline{1-2}
\endlastfoot\multicolumn{2}{|l|}{\textit{Inherited from peach.ga.base.GeneticAlgorithm \textit{(Section \ref{peach:ga:base:GeneticAlgorithm})}}}\\
\multicolumn{2}{|p{\varwidth}|}{\raggedright elitist, ranges}\\
\cline{1-2}
\end{longtable}

    \index{peach \textit{(package)}!peach.ga \textit{(package)}!peach.ga.base \textit{(module)}!peach.ga.base.GA \textit{(class)}|)}
    \index{peach \textit{(package)}!peach.ga \textit{(package)}!peach.ga.base \textit{(module)}|)}

%
% API Documentation for Peach - Computational Intelligence for Python
% Module peach.ga.chromosome
%
% Generated by epydoc 3.0.1
% [Sun Jul 31 17:00:40 2011]
%

%%%%%%%%%%%%%%%%%%%%%%%%%%%%%%%%%%%%%%%%%%%%%%%%%%%%%%%%%%%%%%%%%%%%%%%%%%%
%%                          Module Description                           %%
%%%%%%%%%%%%%%%%%%%%%%%%%%%%%%%%%%%%%%%%%%%%%%%%%%%%%%%%%%%%%%%%%%%%%%%%%%%

    \index{peach \textit{(package)}!peach.ga \textit{(package)}!peach.ga.chromosome \textit{(module)}|(}
\section{Module peach.ga.chromosome}

    \label{peach:ga:chromosome}

Basic definitions and classes for manipulating chromosomes

This sub-package is a vital part of the genetic algorithms framework within the
module. This uses the \texttt{bitarray} module to implement a chromosome as an array
of bits. It is, thus, necessary that this module is installed in your Python
system. Please, check within the Python website how to install the \texttt{bitarray}
module.

The class defined in this module is derived from \texttt{bitarray} and can also be
derived if needed. In general, users or programmers won't need to instance this
class directly -{}- it is manipulated by the genetic algorithm itself. Check the
class definition for more information.

%%%%%%%%%%%%%%%%%%%%%%%%%%%%%%%%%%%%%%%%%%%%%%%%%%%%%%%%%%%%%%%%%%%%%%%%%%%
%%                               Variables                               %%
%%%%%%%%%%%%%%%%%%%%%%%%%%%%%%%%%%%%%%%%%%%%%%%%%%%%%%%%%%%%%%%%%%%%%%%%%%%

  \subsection{Variables}

    \vspace{-1cm}
\hspace{\varindent}\begin{longtable}{|p{\varnamewidth}|p{\vardescrwidth}|l}
\cline{1-2}
\cline{1-2} \centering \textbf{Name} & \centering \textbf{Description}& \\
\cline{1-2}
\endhead\cline{1-2}\multicolumn{3}{r}{\small\textit{continued on next page}}\\\endfoot\cline{1-2}
\endlastfoot\raggedright \_\-\_\-d\-o\-c\-\_\-\_\- & \raggedright \textbf{Value:} 
{\tt \texttt{...}}&\\
\cline{1-2}
\raggedright \_\-\_\-p\-a\-c\-k\-a\-g\-e\-\_\-\_\- & \raggedright \textbf{Value:} 
{\tt \texttt{'}\texttt{peach.ga}\texttt{'}}&\\
\cline{1-2}
\end{longtable}


%%%%%%%%%%%%%%%%%%%%%%%%%%%%%%%%%%%%%%%%%%%%%%%%%%%%%%%%%%%%%%%%%%%%%%%%%%%
%%                           Class Description                           %%
%%%%%%%%%%%%%%%%%%%%%%%%%%%%%%%%%%%%%%%%%%%%%%%%%%%%%%%%%%%%%%%%%%%%%%%%%%%

    \index{peach \textit{(package)}!peach.ga \textit{(package)}!peach.ga.chromosome \textit{(module)}!peach.ga.chromosome.Chromosome \textit{(class)}|(}
\subsection{Class Chromosome}

    \label{peach:ga:chromosome:Chromosome}
\begin{tabular}{cccccccccc}
% Line for object, linespec=[False, False, False]
\multicolumn{2}{r}{\settowidth{\BCL}{object}\multirow{2}{\BCL}{object}}
&&
&&
&&
  \\\cline{3-3}
  &&\multicolumn{1}{c|}{}
&&
&&
&&
  \\
% Line for bitarray.\_bitarray, linespec=[False, False]
\multicolumn{4}{r}{\settowidth{\BCL}{bitarray.\_bitarray}\multirow{2}{\BCL}{bitarray.\_bitarray}}
&&
&&
  \\\cline{5-5}
  &&&&\multicolumn{1}{c|}{}
&&
&&
  \\
% Line for bitarray.bitarray, linespec=[False]
\multicolumn{6}{r}{\settowidth{\BCL}{bitarray.bitarray}\multirow{2}{\BCL}{bitarray.bitarray}}
&&
  \\\cline{7-7}
  &&&&&&\multicolumn{1}{c|}{}
&&
  \\
&&&&&&\multicolumn{2}{l}{\textbf{peach.ga.chromosome.Chromosome}}
\end{tabular}


Implements a chromosome as a bit array.

Data is structured according to the \texttt{struct} module that exists in the
Python standard library. Internally, data used in optimization with a
genetic algorithm are represented as arrays of bits, so the \texttt{bitarray}
module must be installed. Please consult the Python package index for more
information on how to install \texttt{bitarray}. In general, the user don't need
to worry about how the data is manipulated internally, but a specification
of the format as in the \texttt{struct} module is needed.

If the internal format of the data is specified as an \texttt{struct} format, the
genetic algorithm will take care of encoding and decoding data from and to
the optimizer. However, it is possible to specify, instead of a format, the
length of the chromosome. In that case, the fitness function must deal with
the encoding and decoding of the information. It is strongly suggested that
you use \texttt{struct} format strings, as they are much easier. This second
option is provided as a convenience.

The \texttt{Chromosome} class is derived from the \texttt{bitarray} class. So, every
property and method of this class should be accessible.

%%%%%%%%%%%%%%%%%%%%%%%%%%%%%%%%%%%%%%%%%%%%%%%%%%%%%%%%%%%%%%%%%%%%%%%%%%%
%%                                Methods                                %%
%%%%%%%%%%%%%%%%%%%%%%%%%%%%%%%%%%%%%%%%%%%%%%%%%%%%%%%%%%%%%%%%%%%%%%%%%%%

  \subsubsection{Methods}

    \vspace{0.5ex}

\hspace{.8\funcindent}\begin{boxedminipage}{\funcwidth}

    \raggedright \textbf{\_\_new\_\_}(\textit{cls}, \textit{fmt}={\tt \texttt{'}\texttt{}\texttt{'}}, \textit{endian}={\tt \texttt{'}\texttt{little}\texttt{'}})

    \vspace{-1.5ex}

    \rule{\textwidth}{0.5\fboxrule}
\setlength{\parskip}{2ex}

Allocates new memory space for the chromosome

This function overrides the \texttt{bitarray.\_\_new\_\_} function to deal with
the length of the chromosome. It should never be directly used, as it is
automatically called by the Python interpreter in the moment of object
creation.
\setlength{\parskip}{1ex}
      \textbf{Return Value}
    \vspace{-1ex}

      \begin{quote}

A new \texttt{Chromosome} object.
      {\it (type=a new object with type S, a subtype of T)}

      \end{quote}

      Overrides: object.\_\_new\_\_

    \end{boxedminipage}

    \vspace{0.5ex}

\hspace{.8\funcindent}\begin{boxedminipage}{\funcwidth}

    \raggedright \textbf{\_\_init\_\_}(\textit{self}, \textit{fmt}={\tt \texttt{'}\texttt{}\texttt{'}})

    \vspace{-1.5ex}

    \rule{\textwidth}{0.5\fboxrule}
\setlength{\parskip}{2ex}

Initializes the chromosome.

This method is automatically called by the Python interpreter and
initializes the data in the chromosome. No data should be provided to be
encoded in the chromosome, as it is usually better start with random
estimates. This method, in particular, does not clear the memory used in
the time of creation of the \texttt{bitarray} from which a \texttt{Chromosome}
derives -{}- so the random noise in the memory is used as initial value.
\setlength{\parskip}{1ex}
      \textbf{Parameters}
      \vspace{-1ex}

      \begin{quote}
        \begin{Ventry}{xxx}

          \item[fmt]


This parameter can be passed in two different ways. If \texttt{fmt} is a
string, then it is assumed to be a \texttt{struct}-format string. Its
size is calculated and a \texttt{bitarray} of the corresponding size is
created. Please, consult the \texttt{struct} documentation, since what is
explained there is exactly what is used here. For example, if you
are going to use the optimizer to deal with three-dimensional
vectors of continuous variables, the format would be something
like:
%
\begin{quote}{\ttfamily \raggedright \noindent
fmt~=~'fff'
}
\end{quote}

If \texttt{fmt}, however, is an integer, then a \texttt{bitarray} of the given
length is created. Note that, in this case, no format is given to
the chromosome, and it is responsability of the programmer and the
fitness function to provide for it.

Default value is an empty string.
        \end{Ventry}

      \end{quote}

      Overrides: object.\_\_init\_\_

    \end{boxedminipage}

    \vspace{0.5ex}

\hspace{.8\funcindent}\begin{boxedminipage}{\funcwidth}

    \raggedright \textbf{decode}(\textit{self})

    \vspace{-1.5ex}

    \rule{\textwidth}{0.5\fboxrule}
\setlength{\parskip}{2ex}

This method decodes the information given in the chromosome.

Data in the chromosome is encoded as a \texttt{struct}-formated string in a
\texttt{bitarray} object. This method decodes the information and returns the
encoded values. If a format string is not given, then it is assumed that
this chromosome is just an array of bits, which is returned.
\setlength{\parskip}{1ex}
      \textbf{Return Value}
    \vspace{-1ex}

      \begin{quote}

A tuple containing the decoded values, in the order specified by the
format string.
      \end{quote}

      Overrides: bitarray.bitarray.decode

    \end{boxedminipage}

    \vspace{0.5ex}

\hspace{.8\funcindent}\begin{boxedminipage}{\funcwidth}

    \raggedright \textbf{encode}(\textit{self}, \textit{values})

    \vspace{-1.5ex}

    \rule{\textwidth}{0.5\fboxrule}
\setlength{\parskip}{2ex}

This method encodes the information into the chromosome.

Data in the chromosome is encoded as a \texttt{struct}-formated string in a
\texttt{bitarray} object. This method encodes the given information in the
bitarray. If a format string is not given, this method raises a
\texttt{TypeError} exception.
\setlength{\parskip}{1ex}
      \textbf{Parameters}
      \vspace{-1ex}

      \begin{quote}
        \begin{Ventry}{xxxxxx}

          \item[values]


A tuple containing the values to be encoded in an order consistent
with the given \texttt{struct}-format.
        \end{Ventry}

      \end{quote}

      Overrides: bitarray.bitarray.encode

    \end{boxedminipage}


\large{\textbf{\textit{Inherited from bitarray.bitarray}}}

\begin{quote}
\_\_contains\_\_(), search()
\end{quote}

\large{\textbf{\textit{Inherited from bitarray.\_bitarray}}}

\begin{quote}
\_\_add\_\_(), \_\_and\_\_(), \_\_copy\_\_(), \_\_deepcopy\_\_(), \_\_delitem\_\_(), \_\_eq\_\_(), \_\_ge\_\_(), \_\_getattribute\_\_(), \_\_getitem\_\_(), \_\_gt\_\_(), \_\_iadd\_\_(), \_\_iand\_\_(), \_\_imul\_\_(), \_\_invert\_\_(), \_\_ior\_\_(), \_\_iter\_\_(), \_\_ixor\_\_(), \_\_le\_\_(), \_\_len\_\_(), \_\_lt\_\_(), \_\_mul\_\_(), \_\_ne\_\_(), \_\_or\_\_(), \_\_reduce\_\_(), \_\_repr\_\_(), \_\_rmul\_\_(), \_\_setitem\_\_(), \_\_xor\_\_(), all(), any(), append(), buffer\_info(), bytereverse(), copy(), count(), endian(), extend(), fill(), fromfile(), fromstring(), index(), insert(), invert(), length(), pack(), pop(), remove(), reverse(), setall(), sort(), to01(), tofile(), tolist(), tostring(), unpack()
\end{quote}

\large{\textbf{\textit{Inherited from object}}}

\begin{quote}
\_\_delattr\_\_(), \_\_format\_\_(), \_\_hash\_\_(), \_\_reduce\_ex\_\_(), \_\_setattr\_\_(), \_\_sizeof\_\_(), \_\_str\_\_(), \_\_subclasshook\_\_()
\end{quote}

%%%%%%%%%%%%%%%%%%%%%%%%%%%%%%%%%%%%%%%%%%%%%%%%%%%%%%%%%%%%%%%%%%%%%%%%%%%
%%                              Properties                               %%
%%%%%%%%%%%%%%%%%%%%%%%%%%%%%%%%%%%%%%%%%%%%%%%%%%%%%%%%%%%%%%%%%%%%%%%%%%%

  \subsubsection{Properties}

    \vspace{-1cm}
\hspace{\varindent}\begin{longtable}{|p{\varnamewidth}|p{\vardescrwidth}|l}
\cline{1-2}
\cline{1-2} \centering \textbf{Name} & \centering \textbf{Description}& \\
\cline{1-2}
\endhead\cline{1-2}\multicolumn{3}{r}{\small\textit{continued on next page}}\\\endfoot\cline{1-2}
\endlastfoot\raggedright s\-i\-z\-e\- & &\\
\cline{1-2}
\multicolumn{2}{|l|}{\textit{Inherited from object}}\\
\multicolumn{2}{|p{\varwidth}|}{\raggedright \_\_class\_\_}\\
\cline{1-2}
\end{longtable}


%%%%%%%%%%%%%%%%%%%%%%%%%%%%%%%%%%%%%%%%%%%%%%%%%%%%%%%%%%%%%%%%%%%%%%%%%%%
%%                          Instance Variables                           %%
%%%%%%%%%%%%%%%%%%%%%%%%%%%%%%%%%%%%%%%%%%%%%%%%%%%%%%%%%%%%%%%%%%%%%%%%%%%

  \subsubsection{Instance Variables}

    \vspace{-1cm}
\hspace{\varindent}\begin{longtable}{|p{\varnamewidth}|p{\vardescrwidth}|l}
\cline{1-2}
\cline{1-2} \centering \textbf{Name} & \centering \textbf{Description}& \\
\cline{1-2}
\endhead\cline{1-2}\multicolumn{3}{r}{\small\textit{continued on next page}}\\\endfoot\cline{1-2}
\endlastfoot\raggedright f\-o\-r\-m\-a\-t\- & Property that contains the chromosome \texttt{struct} format.&\\
\cline{1-2}
\end{longtable}

    \index{peach \textit{(package)}!peach.ga \textit{(package)}!peach.ga.chromosome \textit{(module)}!peach.ga.chromosome.Chromosome \textit{(class)}|)}
    \index{peach \textit{(package)}!peach.ga \textit{(package)}!peach.ga.chromosome \textit{(module)}|)}

%
% API Documentation for Peach - Computational Intelligence for Python
% Module peach.ga.crossover
%
% Generated by epydoc 3.0beta1
% [Mon Dec 21 08:51:36 2009]
%

%%%%%%%%%%%%%%%%%%%%%%%%%%%%%%%%%%%%%%%%%%%%%%%%%%%%%%%%%%%%%%%%%%%%%%%%%%%
%%                          Module Description                           %%
%%%%%%%%%%%%%%%%%%%%%%%%%%%%%%%%%%%%%%%%%%%%%%%%%%%%%%%%%%%%%%%%%%%%%%%%%%%

    \index{peach \textit{(package)}!peach.ga \textit{(package)}!peach.ga.crossover \textit{(module)}|(}
\section{Module peach.ga.crossover}

    \label{peach:ga:crossover}

Basic definitions for crossover operations and base classes.

Crossover is a very basic and important operation in genetic algorithms. It is
by means of crossover among the chromosomes that population gains diversity,
thus exploring more completelly the solution space and giving better answers.
This sub-module provides definitions of the most common crossover operations,
and provides a class that can be subclassed to construct different types of
crossover for experimentation.

%%%%%%%%%%%%%%%%%%%%%%%%%%%%%%%%%%%%%%%%%%%%%%%%%%%%%%%%%%%%%%%%%%%%%%%%%%%
%%                               Variables                               %%
%%%%%%%%%%%%%%%%%%%%%%%%%%%%%%%%%%%%%%%%%%%%%%%%%%%%%%%%%%%%%%%%%%%%%%%%%%%

  \subsection{Variables}

\begin{longtable}{|p{.30\textwidth}|p{.62\textwidth}|l}
\cline{1-2}
\cline{1-2} \centering \textbf{Name} & \centering \textbf{Description}& \\
\cline{1-2}
\endhead\cline{1-2}\multicolumn{3}{r}{\small\textit{continued on next page}}\\\endfoot\cline{1-2}
\endlastfoot\raggedright \_\-\_\-d\-o\-c\-\_\-\_\- & \raggedright \textbf{Value:} 
{\tt \texttt{...}}&\\
\cline{1-2}
\end{longtable}


%%%%%%%%%%%%%%%%%%%%%%%%%%%%%%%%%%%%%%%%%%%%%%%%%%%%%%%%%%%%%%%%%%%%%%%%%%%
%%                           Class Description                           %%
%%%%%%%%%%%%%%%%%%%%%%%%%%%%%%%%%%%%%%%%%%%%%%%%%%%%%%%%%%%%%%%%%%%%%%%%%%%

    \index{peach \textit{(package)}!peach.ga \textit{(package)}!peach.ga.crossover \textit{(module)}!peach.ga.crossover.Crossover \textit{(class)}|(}
\subsection{Class Crossover}

    \label{peach:ga:crossover:Crossover}
\begin{tabular}{cccccc}
% Line for object, linespec=[False]
\multicolumn{2}{r}{\settowidth{\BCL}{object}\multirow{2}{\BCL}{object}}
&&
  \\\cline{3-3}
  &&\multicolumn{1}{c|}{}
&&
  \\
&&\multicolumn{2}{l}{\textbf{peach.ga.crossover.Crossover}}
\end{tabular}

\textbf{Known Subclasses:}
peach.ga.crossover.OnePoint,
    peach.ga.crossover.TwoPoint,
    peach.ga.crossover.Uniform


Base class for crossover operators.

This class should be subclassed if you want to create your own crossover
operator. The base class doesn't do much, it is only a prototype. As is done
with all the base classes within this library, use the \texttt{{\_}{\_}init{\_}{\_}} method
to configure your crossover behaviour -{}- if needed -{}- and the \texttt{{\_}{\_}call{\_}{\_}}
method to operate over a population.

A class derived from this one should implement at least 2 methods, defined
below:
\begin{quote}
\begin{description}
%[visit_definition_list_item]
\item[{{\_}{\_}init{\_}{\_}(self, {\color{red}\bfseries{}*}cnf, {\color{red}\bfseries{}**}kw)}] %[visit_definition]

Initializes the object. There is no mandatory arguments, but any
parameters can be used here to configure the operator. For example, a
class can define a crossover rate -{}- this should be defined here:
\begin{quote}{\ttfamily \raggedright \noindent
{\_}{\_}init{\_}{\_}(self,~rate=0.75)
}\end{quote}

A default value should always be offered, if possible.

%[depart_definition]
%[depart_definition_list_item]
%[visit_definition_list_item]
\item[{{\_}{\_}call{\_}{\_}(self, population)}] %[visit_definition]

The \texttt{{\_}{\_}call{\_}{\_}} implementation should receive a population and operate
over it. Please, consult the \texttt{ga} module to see more information on
populations. It should return the processed population. No recomendation
on the internals of the method is made. That being said, in general the
crossover operators pairs chromosomes and swap bits among them (but
there is nothing to say that you can't do it differently).

%[depart_definition]
%[depart_definition_list_item]
\end{description}
\end{quote}

Please, note that the GA implementations relies on this behaviour: it will
pass a population to your \texttt{{\_}{\_}call{\_}{\_}} method and expects to received the
result back.

%%%%%%%%%%%%%%%%%%%%%%%%%%%%%%%%%%%%%%%%%%%%%%%%%%%%%%%%%%%%%%%%%%%%%%%%%%%
%%                                Methods                                %%
%%%%%%%%%%%%%%%%%%%%%%%%%%%%%%%%%%%%%%%%%%%%%%%%%%%%%%%%%%%%%%%%%%%%%%%%%%%

  \subsubsection{Methods}

    \label{object:__delattr__}
    \index{object.\_\_delattr\_\_ \textit{(function)}}

    \vspace{0.5ex}

    \begin{boxedminipage}{\textwidth}

    \raggedright \textbf{\_\_delattr\_\_}(\textit{...})

    \vspace{-1.5ex}

    \rule{\textwidth}{0.5\fboxrule}

x.{\_}{\_}delattr{\_}{\_}('name') {\textless}=={\textgreater} del x.name
    \vspace{1ex}

    \end{boxedminipage}

    \label{object:__getattribute__}
    \index{object.\_\_getattribute\_\_ \textit{(function)}}

    \vspace{0.5ex}

    \begin{boxedminipage}{\textwidth}

    \raggedright \textbf{\_\_getattribute\_\_}(\textit{...})

    \vspace{-1.5ex}

    \rule{\textwidth}{0.5\fboxrule}

x.{\_}{\_}getattribute{\_}{\_}('name') {\textless}=={\textgreater} x.name
    \vspace{1ex}

    \end{boxedminipage}

    \label{object:__hash__}
    \index{object.\_\_hash\_\_ \textit{(function)}}

    \vspace{0.5ex}

    \begin{boxedminipage}{\textwidth}

    \raggedright \textbf{\_\_hash\_\_}(\textit{x})

    \vspace{-1.5ex}

    \rule{\textwidth}{0.5\fboxrule}

hash(x)
    \vspace{1ex}

    \end{boxedminipage}

    \label{object:__init__}
    \index{object.\_\_init\_\_ \textit{(function)}}

    \vspace{0.5ex}

    \begin{boxedminipage}{\textwidth}

    \raggedright \textbf{\_\_init\_\_}(\textit{...})

    \vspace{-1.5ex}

    \rule{\textwidth}{0.5\fboxrule}

x.{\_}{\_}init{\_}{\_}(...) initializes x; see x.{\_}{\_}class{\_}{\_}.{\_}{\_}doc{\_}{\_} for signature
    \vspace{1ex}

    \end{boxedminipage}

    \label{object:__new__}
    \index{object.\_\_new\_\_ \textit{(function)}}

    \vspace{0.5ex}

    \begin{boxedminipage}{\textwidth}

    \raggedright \textbf{\_\_new\_\_}(\textit{T}, \textit{S}, \textit{...})

      \textbf{Return Value}
      \begin{quote}
\begin{alltt}
a new object with type S, a subtype of T
\end{alltt}

      \end{quote}

    \vspace{1ex}

    \end{boxedminipage}

    \label{object:__reduce__}
    \index{object.\_\_reduce\_\_ \textit{(function)}}

    \vspace{0.5ex}

    \begin{boxedminipage}{\textwidth}

    \raggedright \textbf{\_\_reduce\_\_}(\textit{...})

    \vspace{-1.5ex}

    \rule{\textwidth}{0.5\fboxrule}

helper for pickle
    \vspace{1ex}

    \end{boxedminipage}

    \label{object:__reduce_ex__}
    \index{object.\_\_reduce\_ex\_\_ \textit{(function)}}

    \vspace{0.5ex}

    \begin{boxedminipage}{\textwidth}

    \raggedright \textbf{\_\_reduce\_ex\_\_}(\textit{...})

    \vspace{-1.5ex}

    \rule{\textwidth}{0.5\fboxrule}

helper for pickle
    \vspace{1ex}

    \end{boxedminipage}

    \label{object:__repr__}
    \index{object.\_\_repr\_\_ \textit{(function)}}

    \vspace{0.5ex}

    \begin{boxedminipage}{\textwidth}

    \raggedright \textbf{\_\_repr\_\_}(\textit{x})

    \vspace{-1.5ex}

    \rule{\textwidth}{0.5\fboxrule}

repr(x)
    \vspace{1ex}

    \end{boxedminipage}

    \label{object:__setattr__}
    \index{object.\_\_setattr\_\_ \textit{(function)}}

    \vspace{0.5ex}

    \begin{boxedminipage}{\textwidth}

    \raggedright \textbf{\_\_setattr\_\_}(\textit{...})

    \vspace{-1.5ex}

    \rule{\textwidth}{0.5\fboxrule}

x.{\_}{\_}setattr{\_}{\_}('name', value) {\textless}=={\textgreater} x.name = value
    \vspace{1ex}

    \end{boxedminipage}

    \label{object:__str__}
    \index{object.\_\_str\_\_ \textit{(function)}}

    \vspace{0.5ex}

    \begin{boxedminipage}{\textwidth}

    \raggedright \textbf{\_\_str\_\_}(\textit{x})

    \vspace{-1.5ex}

    \rule{\textwidth}{0.5\fboxrule}

str(x)
    \vspace{1ex}

    \end{boxedminipage}


%%%%%%%%%%%%%%%%%%%%%%%%%%%%%%%%%%%%%%%%%%%%%%%%%%%%%%%%%%%%%%%%%%%%%%%%%%%
%%                              Properties                               %%
%%%%%%%%%%%%%%%%%%%%%%%%%%%%%%%%%%%%%%%%%%%%%%%%%%%%%%%%%%%%%%%%%%%%%%%%%%%

  \subsubsection{Properties}

\begin{longtable}{|p{.30\textwidth}|p{.62\textwidth}|l}
\cline{1-2}
\cline{1-2} \centering \textbf{Name} & \centering \textbf{Description}& \\
\cline{1-2}
\endhead\cline{1-2}\multicolumn{3}{r}{\small\textit{continued on next page}}\\\endfoot\cline{1-2}
\endlastfoot\raggedright \_\-\_\-c\-l\-a\-s\-s\-\_\-\_\- & \raggedright \textbf{Value:} 
{\tt {\textless}attribute '\_\_class\_\_' of 'object' objects{\textgreater}}&\\
\cline{1-2}
\end{longtable}

    \index{peach \textit{(package)}!peach.ga \textit{(package)}!peach.ga.crossover \textit{(module)}!peach.ga.crossover.Crossover \textit{(class)}|)}

%%%%%%%%%%%%%%%%%%%%%%%%%%%%%%%%%%%%%%%%%%%%%%%%%%%%%%%%%%%%%%%%%%%%%%%%%%%
%%                           Class Description                           %%
%%%%%%%%%%%%%%%%%%%%%%%%%%%%%%%%%%%%%%%%%%%%%%%%%%%%%%%%%%%%%%%%%%%%%%%%%%%

    \index{peach \textit{(package)}!peach.ga \textit{(package)}!peach.ga.crossover \textit{(module)}!peach.ga.crossover.OnePoint \textit{(class)}|(}
\subsection{Class OnePoint}

    \label{peach:ga:crossover:OnePoint}
\begin{tabular}{cccccccc}
% Line for object, linespec=[False, False]
\multicolumn{2}{r}{\settowidth{\BCL}{object}\multirow{2}{\BCL}{object}}
&&
&&
  \\\cline{3-3}
  &&\multicolumn{1}{c|}{}
&&
&&
  \\
% Line for peach.ga.crossover.Crossover, linespec=[False]
\multicolumn{4}{r}{\settowidth{\BCL}{peach.ga.crossover.Crossover}\multirow{2}{\BCL}{peach.ga.crossover.Crossover}}
&&
  \\\cline{5-5}
  &&&&\multicolumn{1}{c|}{}
&&
  \\
&&&&\multicolumn{2}{l}{\textbf{peach.ga.crossover.OnePoint}}
\end{tabular}


A one-point crossover operator.

A one-point crossover randomly selects a single point in two chromosomes and
swaps the bits among them from that point until the end of the bit stream.
The crossover rate is the probability that two paired chromosomes will
exchange bits.

%%%%%%%%%%%%%%%%%%%%%%%%%%%%%%%%%%%%%%%%%%%%%%%%%%%%%%%%%%%%%%%%%%%%%%%%%%%
%%                                Methods                                %%
%%%%%%%%%%%%%%%%%%%%%%%%%%%%%%%%%%%%%%%%%%%%%%%%%%%%%%%%%%%%%%%%%%%%%%%%%%%

  \subsubsection{Methods}

    \vspace{0.5ex}

    \begin{boxedminipage}{\textwidth}

    \raggedright \textbf{\_\_init\_\_}(\textit{self}, \textit{rate}=\texttt{0.75})

    \vspace{-1.5ex}

    \rule{\textwidth}{0.5\fboxrule}

Initialize the crossover operator.
    \vspace{1ex}

      \textbf{Parameters}
      \begin{quote}
        \begin{Ventry}{xxxx}

          \item[rate]


Probability that two paired chromosomes will exchange bits.
        \end{Ventry}

      \end{quote}

    \vspace{1ex}

      Overrides: object.\_\_init\_\_

    \end{boxedminipage}

    \label{peach:ga:crossover:OnePoint:__call__}
    \index{peach \textit{(package)}!peach.ga \textit{(package)}!peach.ga.crossover \textit{(module)}!peach.ga.crossover.OnePoint \textit{(class)}!peach.ga.crossover.OnePoint.\_\_call\_\_ \textit{(method)}}

    \vspace{0.5ex}

    \begin{boxedminipage}{\textwidth}

    \raggedright \textbf{\_\_call\_\_}(\textit{self}, \textit{population})

    \vspace{-1.5ex}

    \rule{\textwidth}{0.5\fboxrule}

Proceeds the crossover over a population.

In one-point crossover, chromosomes from a population are randomly
paired. If a uniform random number is below the \texttt{rate} given in the
instantiation of the operator, then a random point is selected and bits
from that point until the end of the chromosomes are exchanged.
    \vspace{1ex}

      \textbf{Parameters}
      \begin{quote}
        \begin{Ventry}{xxxxxxxxxx}

          \item[population]


A list of \texttt{Chromosomes} containing the present population of the
algorithm. It is processed and the results of the exchange are
returned to the caller.
        \end{Ventry}

      \end{quote}

    \vspace{1ex}

      \textbf{Return Value}
      \begin{quote}

The processed population, a list of \texttt{Chromosomes}.
      \end{quote}

    \vspace{1ex}

    \end{boxedminipage}

    \label{object:__delattr__}
    \index{object.\_\_delattr\_\_ \textit{(function)}}

    \vspace{0.5ex}

    \begin{boxedminipage}{\textwidth}

    \raggedright \textbf{\_\_delattr\_\_}(\textit{...})

    \vspace{-1.5ex}

    \rule{\textwidth}{0.5\fboxrule}

x.{\_}{\_}delattr{\_}{\_}('name') {\textless}=={\textgreater} del x.name
    \vspace{1ex}

    \end{boxedminipage}

    \label{object:__getattribute__}
    \index{object.\_\_getattribute\_\_ \textit{(function)}}

    \vspace{0.5ex}

    \begin{boxedminipage}{\textwidth}

    \raggedright \textbf{\_\_getattribute\_\_}(\textit{...})

    \vspace{-1.5ex}

    \rule{\textwidth}{0.5\fboxrule}

x.{\_}{\_}getattribute{\_}{\_}('name') {\textless}=={\textgreater} x.name
    \vspace{1ex}

    \end{boxedminipage}

    \label{object:__hash__}
    \index{object.\_\_hash\_\_ \textit{(function)}}

    \vspace{0.5ex}

    \begin{boxedminipage}{\textwidth}

    \raggedright \textbf{\_\_hash\_\_}(\textit{x})

    \vspace{-1.5ex}

    \rule{\textwidth}{0.5\fboxrule}

hash(x)
    \vspace{1ex}

    \end{boxedminipage}

    \label{object:__new__}
    \index{object.\_\_new\_\_ \textit{(function)}}

    \vspace{0.5ex}

    \begin{boxedminipage}{\textwidth}

    \raggedright \textbf{\_\_new\_\_}(\textit{T}, \textit{S}, \textit{...})

      \textbf{Return Value}
      \begin{quote}
\begin{alltt}
a new object with type S, a subtype of T
\end{alltt}

      \end{quote}

    \vspace{1ex}

    \end{boxedminipage}

    \label{object:__reduce__}
    \index{object.\_\_reduce\_\_ \textit{(function)}}

    \vspace{0.5ex}

    \begin{boxedminipage}{\textwidth}

    \raggedright \textbf{\_\_reduce\_\_}(\textit{...})

    \vspace{-1.5ex}

    \rule{\textwidth}{0.5\fboxrule}

helper for pickle
    \vspace{1ex}

    \end{boxedminipage}

    \label{object:__reduce_ex__}
    \index{object.\_\_reduce\_ex\_\_ \textit{(function)}}

    \vspace{0.5ex}

    \begin{boxedminipage}{\textwidth}

    \raggedright \textbf{\_\_reduce\_ex\_\_}(\textit{...})

    \vspace{-1.5ex}

    \rule{\textwidth}{0.5\fboxrule}

helper for pickle
    \vspace{1ex}

    \end{boxedminipage}

    \label{object:__repr__}
    \index{object.\_\_repr\_\_ \textit{(function)}}

    \vspace{0.5ex}

    \begin{boxedminipage}{\textwidth}

    \raggedright \textbf{\_\_repr\_\_}(\textit{x})

    \vspace{-1.5ex}

    \rule{\textwidth}{0.5\fboxrule}

repr(x)
    \vspace{1ex}

    \end{boxedminipage}

    \label{object:__setattr__}
    \index{object.\_\_setattr\_\_ \textit{(function)}}

    \vspace{0.5ex}

    \begin{boxedminipage}{\textwidth}

    \raggedright \textbf{\_\_setattr\_\_}(\textit{...})

    \vspace{-1.5ex}

    \rule{\textwidth}{0.5\fboxrule}

x.{\_}{\_}setattr{\_}{\_}('name', value) {\textless}=={\textgreater} x.name = value
    \vspace{1ex}

    \end{boxedminipage}

    \label{object:__str__}
    \index{object.\_\_str\_\_ \textit{(function)}}

    \vspace{0.5ex}

    \begin{boxedminipage}{\textwidth}

    \raggedright \textbf{\_\_str\_\_}(\textit{x})

    \vspace{-1.5ex}

    \rule{\textwidth}{0.5\fboxrule}

str(x)
    \vspace{1ex}

    \end{boxedminipage}


%%%%%%%%%%%%%%%%%%%%%%%%%%%%%%%%%%%%%%%%%%%%%%%%%%%%%%%%%%%%%%%%%%%%%%%%%%%
%%                              Properties                               %%
%%%%%%%%%%%%%%%%%%%%%%%%%%%%%%%%%%%%%%%%%%%%%%%%%%%%%%%%%%%%%%%%%%%%%%%%%%%

  \subsubsection{Properties}

\begin{longtable}{|p{.30\textwidth}|p{.62\textwidth}|l}
\cline{1-2}
\cline{1-2} \centering \textbf{Name} & \centering \textbf{Description}& \\
\cline{1-2}
\endhead\cline{1-2}\multicolumn{3}{r}{\small\textit{continued on next page}}\\\endfoot\cline{1-2}
\endlastfoot\raggedright \_\-\_\-c\-l\-a\-s\-s\-\_\-\_\- & \raggedright \textbf{Value:} 
{\tt {\textless}attribute '\_\_class\_\_' of 'object' objects{\textgreater}}&\\
\cline{1-2}
\end{longtable}


%%%%%%%%%%%%%%%%%%%%%%%%%%%%%%%%%%%%%%%%%%%%%%%%%%%%%%%%%%%%%%%%%%%%%%%%%%%
%%                          Instance Variables                           %%
%%%%%%%%%%%%%%%%%%%%%%%%%%%%%%%%%%%%%%%%%%%%%%%%%%%%%%%%%%%%%%%%%%%%%%%%%%%

  \subsubsection{Instance Variables}

\begin{longtable}{|p{.30\textwidth}|p{.62\textwidth}|l}
\cline{1-2}
\cline{1-2} \centering \textbf{Name} & \centering \textbf{Description}& \\
\cline{1-2}
\endhead\cline{1-2}\multicolumn{3}{r}{\small\textit{continued on next page}}\\\endfoot\cline{1-2}
\endlastfoot\raggedright r\-a\-t\-e\- & Property that contains the crossover rate.&\\
\cline{1-2}
\end{longtable}

    \index{peach \textit{(package)}!peach.ga \textit{(package)}!peach.ga.crossover \textit{(module)}!peach.ga.crossover.OnePoint \textit{(class)}|)}

%%%%%%%%%%%%%%%%%%%%%%%%%%%%%%%%%%%%%%%%%%%%%%%%%%%%%%%%%%%%%%%%%%%%%%%%%%%
%%                           Class Description                           %%
%%%%%%%%%%%%%%%%%%%%%%%%%%%%%%%%%%%%%%%%%%%%%%%%%%%%%%%%%%%%%%%%%%%%%%%%%%%

    \index{peach \textit{(package)}!peach.ga \textit{(package)}!peach.ga.crossover \textit{(module)}!peach.ga.crossover.TwoPoint \textit{(class)}|(}
\subsection{Class TwoPoint}

    \label{peach:ga:crossover:TwoPoint}
\begin{tabular}{cccccccc}
% Line for object, linespec=[False, False]
\multicolumn{2}{r}{\settowidth{\BCL}{object}\multirow{2}{\BCL}{object}}
&&
&&
  \\\cline{3-3}
  &&\multicolumn{1}{c|}{}
&&
&&
  \\
% Line for peach.ga.crossover.Crossover, linespec=[False]
\multicolumn{4}{r}{\settowidth{\BCL}{peach.ga.crossover.Crossover}\multirow{2}{\BCL}{peach.ga.crossover.Crossover}}
&&
  \\\cline{5-5}
  &&&&\multicolumn{1}{c|}{}
&&
  \\
&&&&\multicolumn{2}{l}{\textbf{peach.ga.crossover.TwoPoint}}
\end{tabular}


A two-point crossover operator.

A two-point crossover randomly selects two points in two chromosomes and
swaps the bits among them between these points. The crossover rate is the
probability that two paired chromosomes will exchange bits.

%%%%%%%%%%%%%%%%%%%%%%%%%%%%%%%%%%%%%%%%%%%%%%%%%%%%%%%%%%%%%%%%%%%%%%%%%%%
%%                                Methods                                %%
%%%%%%%%%%%%%%%%%%%%%%%%%%%%%%%%%%%%%%%%%%%%%%%%%%%%%%%%%%%%%%%%%%%%%%%%%%%

  \subsubsection{Methods}

    \vspace{0.5ex}

    \begin{boxedminipage}{\textwidth}

    \raggedright \textbf{\_\_init\_\_}(\textit{self}, \textit{rate}=\texttt{0.75})

    \vspace{-1.5ex}

    \rule{\textwidth}{0.5\fboxrule}

Initialize the crossover operator.
    \vspace{1ex}

      \textbf{Parameters}
      \begin{quote}
        \begin{Ventry}{xxxx}

          \item[rate]


Probability that two paired chromosomes will exchange bits.
        \end{Ventry}

      \end{quote}

    \vspace{1ex}

      Overrides: object.\_\_init\_\_

    \end{boxedminipage}

    \label{peach:ga:crossover:TwoPoint:__call__}
    \index{peach \textit{(package)}!peach.ga \textit{(package)}!peach.ga.crossover \textit{(module)}!peach.ga.crossover.TwoPoint \textit{(class)}!peach.ga.crossover.TwoPoint.\_\_call\_\_ \textit{(method)}}

    \vspace{0.5ex}

    \begin{boxedminipage}{\textwidth}

    \raggedright \textbf{\_\_call\_\_}(\textit{self}, \textit{population})

    \vspace{-1.5ex}

    \rule{\textwidth}{0.5\fboxrule}

Proceeds the crossover over a population.

In two-point crossover, chromosomes from a population are randomly
paired. If a uniform random number is below the \texttt{rate} given in the
instantiation of the operator, then random points are selected and bits
between those points are exchanged.
    \vspace{1ex}

      \textbf{Parameters}
      \begin{quote}
        \begin{Ventry}{xxxxxxxxxx}

          \item[population]


A list of \texttt{Chromosomes} containing the present population of the
algorithm. It is processed and the results of the exchange are
returned to the caller.
        \end{Ventry}

      \end{quote}

    \vspace{1ex}

      \textbf{Return Value}
      \begin{quote}

The processed population, a list of \texttt{Chromosomes}.
      \end{quote}

    \vspace{1ex}

    \end{boxedminipage}

    \label{object:__delattr__}
    \index{object.\_\_delattr\_\_ \textit{(function)}}

    \vspace{0.5ex}

    \begin{boxedminipage}{\textwidth}

    \raggedright \textbf{\_\_delattr\_\_}(\textit{...})

    \vspace{-1.5ex}

    \rule{\textwidth}{0.5\fboxrule}

x.{\_}{\_}delattr{\_}{\_}('name') {\textless}=={\textgreater} del x.name
    \vspace{1ex}

    \end{boxedminipage}

    \label{object:__getattribute__}
    \index{object.\_\_getattribute\_\_ \textit{(function)}}

    \vspace{0.5ex}

    \begin{boxedminipage}{\textwidth}

    \raggedright \textbf{\_\_getattribute\_\_}(\textit{...})

    \vspace{-1.5ex}

    \rule{\textwidth}{0.5\fboxrule}

x.{\_}{\_}getattribute{\_}{\_}('name') {\textless}=={\textgreater} x.name
    \vspace{1ex}

    \end{boxedminipage}

    \label{object:__hash__}
    \index{object.\_\_hash\_\_ \textit{(function)}}

    \vspace{0.5ex}

    \begin{boxedminipage}{\textwidth}

    \raggedright \textbf{\_\_hash\_\_}(\textit{x})

    \vspace{-1.5ex}

    \rule{\textwidth}{0.5\fboxrule}

hash(x)
    \vspace{1ex}

    \end{boxedminipage}

    \label{object:__new__}
    \index{object.\_\_new\_\_ \textit{(function)}}

    \vspace{0.5ex}

    \begin{boxedminipage}{\textwidth}

    \raggedright \textbf{\_\_new\_\_}(\textit{T}, \textit{S}, \textit{...})

      \textbf{Return Value}
      \begin{quote}
\begin{alltt}
a new object with type S, a subtype of T
\end{alltt}

      \end{quote}

    \vspace{1ex}

    \end{boxedminipage}

    \label{object:__reduce__}
    \index{object.\_\_reduce\_\_ \textit{(function)}}

    \vspace{0.5ex}

    \begin{boxedminipage}{\textwidth}

    \raggedright \textbf{\_\_reduce\_\_}(\textit{...})

    \vspace{-1.5ex}

    \rule{\textwidth}{0.5\fboxrule}

helper for pickle
    \vspace{1ex}

    \end{boxedminipage}

    \label{object:__reduce_ex__}
    \index{object.\_\_reduce\_ex\_\_ \textit{(function)}}

    \vspace{0.5ex}

    \begin{boxedminipage}{\textwidth}

    \raggedright \textbf{\_\_reduce\_ex\_\_}(\textit{...})

    \vspace{-1.5ex}

    \rule{\textwidth}{0.5\fboxrule}

helper for pickle
    \vspace{1ex}

    \end{boxedminipage}

    \label{object:__repr__}
    \index{object.\_\_repr\_\_ \textit{(function)}}

    \vspace{0.5ex}

    \begin{boxedminipage}{\textwidth}

    \raggedright \textbf{\_\_repr\_\_}(\textit{x})

    \vspace{-1.5ex}

    \rule{\textwidth}{0.5\fboxrule}

repr(x)
    \vspace{1ex}

    \end{boxedminipage}

    \label{object:__setattr__}
    \index{object.\_\_setattr\_\_ \textit{(function)}}

    \vspace{0.5ex}

    \begin{boxedminipage}{\textwidth}

    \raggedright \textbf{\_\_setattr\_\_}(\textit{...})

    \vspace{-1.5ex}

    \rule{\textwidth}{0.5\fboxrule}

x.{\_}{\_}setattr{\_}{\_}('name', value) {\textless}=={\textgreater} x.name = value
    \vspace{1ex}

    \end{boxedminipage}

    \label{object:__str__}
    \index{object.\_\_str\_\_ \textit{(function)}}

    \vspace{0.5ex}

    \begin{boxedminipage}{\textwidth}

    \raggedright \textbf{\_\_str\_\_}(\textit{x})

    \vspace{-1.5ex}

    \rule{\textwidth}{0.5\fboxrule}

str(x)
    \vspace{1ex}

    \end{boxedminipage}


%%%%%%%%%%%%%%%%%%%%%%%%%%%%%%%%%%%%%%%%%%%%%%%%%%%%%%%%%%%%%%%%%%%%%%%%%%%
%%                              Properties                               %%
%%%%%%%%%%%%%%%%%%%%%%%%%%%%%%%%%%%%%%%%%%%%%%%%%%%%%%%%%%%%%%%%%%%%%%%%%%%

  \subsubsection{Properties}

\begin{longtable}{|p{.30\textwidth}|p{.62\textwidth}|l}
\cline{1-2}
\cline{1-2} \centering \textbf{Name} & \centering \textbf{Description}& \\
\cline{1-2}
\endhead\cline{1-2}\multicolumn{3}{r}{\small\textit{continued on next page}}\\\endfoot\cline{1-2}
\endlastfoot\raggedright \_\-\_\-c\-l\-a\-s\-s\-\_\-\_\- & \raggedright \textbf{Value:} 
{\tt {\textless}attribute '\_\_class\_\_' of 'object' objects{\textgreater}}&\\
\cline{1-2}
\end{longtable}


%%%%%%%%%%%%%%%%%%%%%%%%%%%%%%%%%%%%%%%%%%%%%%%%%%%%%%%%%%%%%%%%%%%%%%%%%%%
%%                          Instance Variables                           %%
%%%%%%%%%%%%%%%%%%%%%%%%%%%%%%%%%%%%%%%%%%%%%%%%%%%%%%%%%%%%%%%%%%%%%%%%%%%

  \subsubsection{Instance Variables}

\begin{longtable}{|p{.30\textwidth}|p{.62\textwidth}|l}
\cline{1-2}
\cline{1-2} \centering \textbf{Name} & \centering \textbf{Description}& \\
\cline{1-2}
\endhead\cline{1-2}\multicolumn{3}{r}{\small\textit{continued on next page}}\\\endfoot\cline{1-2}
\endlastfoot\raggedright r\-a\-t\-e\- & Property that contains the crossover rate.&\\
\cline{1-2}
\end{longtable}

    \index{peach \textit{(package)}!peach.ga \textit{(package)}!peach.ga.crossover \textit{(module)}!peach.ga.crossover.TwoPoint \textit{(class)}|)}

%%%%%%%%%%%%%%%%%%%%%%%%%%%%%%%%%%%%%%%%%%%%%%%%%%%%%%%%%%%%%%%%%%%%%%%%%%%
%%                           Class Description                           %%
%%%%%%%%%%%%%%%%%%%%%%%%%%%%%%%%%%%%%%%%%%%%%%%%%%%%%%%%%%%%%%%%%%%%%%%%%%%

    \index{peach \textit{(package)}!peach.ga \textit{(package)}!peach.ga.crossover \textit{(module)}!peach.ga.crossover.Uniform \textit{(class)}|(}
\subsection{Class Uniform}

    \label{peach:ga:crossover:Uniform}
\begin{tabular}{cccccccc}
% Line for object, linespec=[False, False]
\multicolumn{2}{r}{\settowidth{\BCL}{object}\multirow{2}{\BCL}{object}}
&&
&&
  \\\cline{3-3}
  &&\multicolumn{1}{c|}{}
&&
&&
  \\
% Line for peach.ga.crossover.Crossover, linespec=[False]
\multicolumn{4}{r}{\settowidth{\BCL}{peach.ga.crossover.Crossover}\multirow{2}{\BCL}{peach.ga.crossover.Crossover}}
&&
  \\\cline{5-5}
  &&&&\multicolumn{1}{c|}{}
&&
  \\
&&&&\multicolumn{2}{l}{\textbf{peach.ga.crossover.Uniform}}
\end{tabular}


A uniform crossover operator.

A uniform crossover scans two chromosomes in a bit-to-bit fashion. According
to a given crossover rate, the corresponding bits are exchanged. The
crossover rate is the probability that two bits will be exchanged.

%%%%%%%%%%%%%%%%%%%%%%%%%%%%%%%%%%%%%%%%%%%%%%%%%%%%%%%%%%%%%%%%%%%%%%%%%%%
%%                                Methods                                %%
%%%%%%%%%%%%%%%%%%%%%%%%%%%%%%%%%%%%%%%%%%%%%%%%%%%%%%%%%%%%%%%%%%%%%%%%%%%

  \subsubsection{Methods}

    \vspace{0.5ex}

    \begin{boxedminipage}{\textwidth}

    \raggedright \textbf{\_\_init\_\_}(\textit{self}, \textit{rate}=\texttt{0.75})

    \vspace{-1.5ex}

    \rule{\textwidth}{0.5\fboxrule}

Initialize the crossover operator.
    \vspace{1ex}

      \textbf{Parameters}
      \begin{quote}
        \begin{Ventry}{xxxx}

          \item[rate]


Probability that bits from two paired chromosomes will be exchanged.
        \end{Ventry}

      \end{quote}

    \vspace{1ex}

      Overrides: object.\_\_init\_\_

    \end{boxedminipage}

    \label{peach:ga:crossover:Uniform:__call__}
    \index{peach \textit{(package)}!peach.ga \textit{(package)}!peach.ga.crossover \textit{(module)}!peach.ga.crossover.Uniform \textit{(class)}!peach.ga.crossover.Uniform.\_\_call\_\_ \textit{(method)}}

    \vspace{0.5ex}

    \begin{boxedminipage}{\textwidth}

    \raggedright \textbf{\_\_call\_\_}(\textit{self}, \textit{population})

    \vspace{-1.5ex}

    \rule{\textwidth}{0.5\fboxrule}

Proceeds the crossover over a population.

In uniform crossover, chromosomes from a population are randomly paired,
and scaned in a bit-to-bit fashion. If a uniform random number is below
the \texttt{rate} given in the instantiation of the operator, then the bits
under scan will be exchanged in the chromosomes.
    \vspace{1ex}

      \textbf{Parameters}
      \begin{quote}
        \begin{Ventry}{xxxxxxxxxx}

          \item[population]


A list of \texttt{Chromosomes} containing the present population of the
algorithm. It is processed and the results of the exchange are
returned to the caller.
        \end{Ventry}

      \end{quote}

    \vspace{1ex}

      \textbf{Return Value}
      \begin{quote}

The processed population, a list of \texttt{Chromosomes}.
      \end{quote}

    \vspace{1ex}

    \end{boxedminipage}

    \label{object:__delattr__}
    \index{object.\_\_delattr\_\_ \textit{(function)}}

    \vspace{0.5ex}

    \begin{boxedminipage}{\textwidth}

    \raggedright \textbf{\_\_delattr\_\_}(\textit{...})

    \vspace{-1.5ex}

    \rule{\textwidth}{0.5\fboxrule}

x.{\_}{\_}delattr{\_}{\_}('name') {\textless}=={\textgreater} del x.name
    \vspace{1ex}

    \end{boxedminipage}

    \label{object:__getattribute__}
    \index{object.\_\_getattribute\_\_ \textit{(function)}}

    \vspace{0.5ex}

    \begin{boxedminipage}{\textwidth}

    \raggedright \textbf{\_\_getattribute\_\_}(\textit{...})

    \vspace{-1.5ex}

    \rule{\textwidth}{0.5\fboxrule}

x.{\_}{\_}getattribute{\_}{\_}('name') {\textless}=={\textgreater} x.name
    \vspace{1ex}

    \end{boxedminipage}

    \label{object:__hash__}
    \index{object.\_\_hash\_\_ \textit{(function)}}

    \vspace{0.5ex}

    \begin{boxedminipage}{\textwidth}

    \raggedright \textbf{\_\_hash\_\_}(\textit{x})

    \vspace{-1.5ex}

    \rule{\textwidth}{0.5\fboxrule}

hash(x)
    \vspace{1ex}

    \end{boxedminipage}

    \label{object:__new__}
    \index{object.\_\_new\_\_ \textit{(function)}}

    \vspace{0.5ex}

    \begin{boxedminipage}{\textwidth}

    \raggedright \textbf{\_\_new\_\_}(\textit{T}, \textit{S}, \textit{...})

      \textbf{Return Value}
      \begin{quote}
\begin{alltt}
a new object with type S, a subtype of T
\end{alltt}

      \end{quote}

    \vspace{1ex}

    \end{boxedminipage}

    \label{object:__reduce__}
    \index{object.\_\_reduce\_\_ \textit{(function)}}

    \vspace{0.5ex}

    \begin{boxedminipage}{\textwidth}

    \raggedright \textbf{\_\_reduce\_\_}(\textit{...})

    \vspace{-1.5ex}

    \rule{\textwidth}{0.5\fboxrule}

helper for pickle
    \vspace{1ex}

    \end{boxedminipage}

    \label{object:__reduce_ex__}
    \index{object.\_\_reduce\_ex\_\_ \textit{(function)}}

    \vspace{0.5ex}

    \begin{boxedminipage}{\textwidth}

    \raggedright \textbf{\_\_reduce\_ex\_\_}(\textit{...})

    \vspace{-1.5ex}

    \rule{\textwidth}{0.5\fboxrule}

helper for pickle
    \vspace{1ex}

    \end{boxedminipage}

    \label{object:__repr__}
    \index{object.\_\_repr\_\_ \textit{(function)}}

    \vspace{0.5ex}

    \begin{boxedminipage}{\textwidth}

    \raggedright \textbf{\_\_repr\_\_}(\textit{x})

    \vspace{-1.5ex}

    \rule{\textwidth}{0.5\fboxrule}

repr(x)
    \vspace{1ex}

    \end{boxedminipage}

    \label{object:__setattr__}
    \index{object.\_\_setattr\_\_ \textit{(function)}}

    \vspace{0.5ex}

    \begin{boxedminipage}{\textwidth}

    \raggedright \textbf{\_\_setattr\_\_}(\textit{...})

    \vspace{-1.5ex}

    \rule{\textwidth}{0.5\fboxrule}

x.{\_}{\_}setattr{\_}{\_}('name', value) {\textless}=={\textgreater} x.name = value
    \vspace{1ex}

    \end{boxedminipage}

    \label{object:__str__}
    \index{object.\_\_str\_\_ \textit{(function)}}

    \vspace{0.5ex}

    \begin{boxedminipage}{\textwidth}

    \raggedright \textbf{\_\_str\_\_}(\textit{x})

    \vspace{-1.5ex}

    \rule{\textwidth}{0.5\fboxrule}

str(x)
    \vspace{1ex}

    \end{boxedminipage}


%%%%%%%%%%%%%%%%%%%%%%%%%%%%%%%%%%%%%%%%%%%%%%%%%%%%%%%%%%%%%%%%%%%%%%%%%%%
%%                              Properties                               %%
%%%%%%%%%%%%%%%%%%%%%%%%%%%%%%%%%%%%%%%%%%%%%%%%%%%%%%%%%%%%%%%%%%%%%%%%%%%

  \subsubsection{Properties}

\begin{longtable}{|p{.30\textwidth}|p{.62\textwidth}|l}
\cline{1-2}
\cline{1-2} \centering \textbf{Name} & \centering \textbf{Description}& \\
\cline{1-2}
\endhead\cline{1-2}\multicolumn{3}{r}{\small\textit{continued on next page}}\\\endfoot\cline{1-2}
\endlastfoot\raggedright \_\-\_\-c\-l\-a\-s\-s\-\_\-\_\- & \raggedright \textbf{Value:} 
{\tt {\textless}attribute '\_\_class\_\_' of 'object' objects{\textgreater}}&\\
\cline{1-2}
\end{longtable}


%%%%%%%%%%%%%%%%%%%%%%%%%%%%%%%%%%%%%%%%%%%%%%%%%%%%%%%%%%%%%%%%%%%%%%%%%%%
%%                          Instance Variables                           %%
%%%%%%%%%%%%%%%%%%%%%%%%%%%%%%%%%%%%%%%%%%%%%%%%%%%%%%%%%%%%%%%%%%%%%%%%%%%

  \subsubsection{Instance Variables}

\begin{longtable}{|p{.30\textwidth}|p{.62\textwidth}|l}
\cline{1-2}
\cline{1-2} \centering \textbf{Name} & \centering \textbf{Description}& \\
\cline{1-2}
\endhead\cline{1-2}\multicolumn{3}{r}{\small\textit{continued on next page}}\\\endfoot\cline{1-2}
\endlastfoot\raggedright r\-a\-t\-e\- & Property that contains the crossover rate.&\\
\cline{1-2}
\end{longtable}

    \index{peach \textit{(package)}!peach.ga \textit{(package)}!peach.ga.crossover \textit{(module)}!peach.ga.crossover.Uniform \textit{(class)}|)}
    \index{peach \textit{(package)}!peach.ga \textit{(package)}!peach.ga.crossover \textit{(module)}|)}

%
% API Documentation for Peach - Computational Intelligence for Python
% Module peach.ga.fitness
%
% Generated by epydoc 3.0.1
% [Sun Jul 31 17:00:40 2011]
%

%%%%%%%%%%%%%%%%%%%%%%%%%%%%%%%%%%%%%%%%%%%%%%%%%%%%%%%%%%%%%%%%%%%%%%%%%%%
%%                          Module Description                           %%
%%%%%%%%%%%%%%%%%%%%%%%%%%%%%%%%%%%%%%%%%%%%%%%%%%%%%%%%%%%%%%%%%%%%%%%%%%%

    \index{peach \textit{(package)}!peach.ga \textit{(package)}!peach.ga.fitness \textit{(module)}|(}
\section{Module peach.ga.fitness}

    \label{peach:ga:fitness}

Basic definitions and base classes for definition of fitness functions for use
with genetic algorithms.

Fitness is a function that rates higher the chromosomes that perform better
according to the objective function. For example, if the minimum of a function
needs to be found, then the fitness function should rate better the chromosomes
that correspond to lower values of the objective function. This module gives
support to use common Python functions as fitness functions in genetic
algorithms.

The classes defined in this sub-module take a function and use some algorithm to
rank a population. There are some different ranking functions, some are provided
in this module. There is also a class that can be subclassed to generate other
fitness methods. See the documentation of the corresponding class for more
information.

%%%%%%%%%%%%%%%%%%%%%%%%%%%%%%%%%%%%%%%%%%%%%%%%%%%%%%%%%%%%%%%%%%%%%%%%%%%
%%                               Variables                               %%
%%%%%%%%%%%%%%%%%%%%%%%%%%%%%%%%%%%%%%%%%%%%%%%%%%%%%%%%%%%%%%%%%%%%%%%%%%%

  \subsection{Variables}

    \vspace{-1cm}
\hspace{\varindent}\begin{longtable}{|p{\varnamewidth}|p{\vardescrwidth}|l}
\cline{1-2}
\cline{1-2} \centering \textbf{Name} & \centering \textbf{Description}& \\
\cline{1-2}
\endhead\cline{1-2}\multicolumn{3}{r}{\small\textit{continued on next page}}\\\endfoot\cline{1-2}
\endlastfoot\raggedright \_\-\_\-d\-o\-c\-\_\-\_\- & \raggedright \textbf{Value:} 
{\tt \texttt{...}}&\\
\cline{1-2}
\raggedright \_\-\_\-p\-a\-c\-k\-a\-g\-e\-\_\-\_\- & \raggedright \textbf{Value:} 
{\tt \texttt{'}\texttt{peach.ga}\texttt{'}}&\\
\cline{1-2}
\end{longtable}


%%%%%%%%%%%%%%%%%%%%%%%%%%%%%%%%%%%%%%%%%%%%%%%%%%%%%%%%%%%%%%%%%%%%%%%%%%%
%%                           Class Description                           %%
%%%%%%%%%%%%%%%%%%%%%%%%%%%%%%%%%%%%%%%%%%%%%%%%%%%%%%%%%%%%%%%%%%%%%%%%%%%

    \index{peach \textit{(package)}!peach.ga \textit{(package)}!peach.ga.fitness \textit{(module)}!peach.ga.fitness.Fitness \textit{(class)}|(}
\subsection{Class Fitness}

    \label{peach:ga:fitness:Fitness}
\begin{tabular}{cccccc}
% Line for object, linespec=[False]
\multicolumn{2}{r}{\settowidth{\BCL}{object}\multirow{2}{\BCL}{object}}
&&
  \\\cline{3-3}
  &&\multicolumn{1}{c|}{}
&&
  \\
&&\multicolumn{2}{l}{\textbf{peach.ga.fitness.Fitness}}
\end{tabular}

\textbf{Known Subclasses:} peach.ga.fitness.Ranking


Base class for fitness function classifiers.

This class is used as the base of all fitness functions. However, even if
it is intended to be used as a base class, it also provides some
functionality, described below.

A subclass of this class should implement at least 2 methods:
%
\begin{quote}
%
\begin{description}
\item[{\_\_init\_\_(self, %
\raisebox{1em}{\hypertarget{id2}{}}\hyperlink{id1}{\textbf{\color{red}*}}args, %
\raisebox{1em}{\hypertarget{id4}{}}\hyperlink{id3}{\textbf{\color{red}**}}kwargs)}] \leavevmode 
Initialization method. The initialization procedure doesn't need to take
any parameters, but if any configuration must be done, it should be
passed as an argument to the \texttt{\_\_init\_\_} function. The genetic
algorithm, however, does not expect parameters in the instantiation, so
you should provide sensible defaults.

\item[{\_\_call\_\_(self, fx)}] \leavevmode 
This method is called to calculate population fitness. There is no
recomendation about the internals of the method, but its signature is
expected as defined above. This method receives the values of the
objective function applied over a population -{}- please, consult the
\texttt{ga} module for more information on populations -{}- and should return a
vector or list with the fitness value for each chromosome in the same
order that they appear in the population.

\end{description}

This class implements the standard normalization fitness, as described in
every book and article about GAs. The rank given to a chromosome is
proportional to its objective function value.

\end{quote}

%%%%%%%%%%%%%%%%%%%%%%%%%%%%%%%%%%%%%%%%%%%%%%%%%%%%%%%%%%%%%%%%%%%%%%%%%%%
%%                                Methods                                %%
%%%%%%%%%%%%%%%%%%%%%%%%%%%%%%%%%%%%%%%%%%%%%%%%%%%%%%%%%%%%%%%%%%%%%%%%%%%

  \subsubsection{Methods}

    \vspace{0.5ex}

\hspace{.8\funcindent}\begin{boxedminipage}{\funcwidth}

    \raggedright \textbf{\_\_init\_\_}(\textit{self})

    \vspace{-1.5ex}

    \rule{\textwidth}{0.5\fboxrule}
\setlength{\parskip}{2ex}

Initializes the operator.
\setlength{\parskip}{1ex}
      Overrides: object.\_\_init\_\_

    \end{boxedminipage}

    \label{peach:ga:fitness:Fitness:__call__}
    \index{peach \textit{(package)}!peach.ga \textit{(package)}!peach.ga.fitness \textit{(module)}!peach.ga.fitness.Fitness \textit{(class)}!peach.ga.fitness.Fitness.\_\_call\_\_ \textit{(method)}}

    \vspace{0.5ex}

\hspace{.8\funcindent}\begin{boxedminipage}{\funcwidth}

    \raggedright \textbf{\_\_call\_\_}(\textit{self}, \textit{fx})

    \vspace{-1.5ex}

    \rule{\textwidth}{0.5\fboxrule}
\setlength{\parskip}{2ex}

Calculates the fitness for all individuals in the population.
\setlength{\parskip}{1ex}
      \textbf{Parameters}
      \vspace{-1ex}

      \begin{quote}
        \begin{Ventry}{xx}

          \item[fx]


The values of the objective function for every individual on the
population to be processed. Please, consult the \texttt{ga} module for
more information on populations. This method calculates the fitness
according to the traditional normalization technique.
        \end{Ventry}

      \end{quote}

      \textbf{Return Value}
    \vspace{-1ex}

      \begin{quote}

A vector containing the fitness value for every individual in the
population, in the same order that they appear there.
      \end{quote}

    \end{boxedminipage}


\large{\textbf{\textit{Inherited from object}}}

\begin{quote}
\_\_delattr\_\_(), \_\_format\_\_(), \_\_getattribute\_\_(), \_\_hash\_\_(), \_\_new\_\_(), \_\_reduce\_\_(), \_\_reduce\_ex\_\_(), \_\_repr\_\_(), \_\_setattr\_\_(), \_\_sizeof\_\_(), \_\_str\_\_(), \_\_subclasshook\_\_()
\end{quote}

%%%%%%%%%%%%%%%%%%%%%%%%%%%%%%%%%%%%%%%%%%%%%%%%%%%%%%%%%%%%%%%%%%%%%%%%%%%
%%                              Properties                               %%
%%%%%%%%%%%%%%%%%%%%%%%%%%%%%%%%%%%%%%%%%%%%%%%%%%%%%%%%%%%%%%%%%%%%%%%%%%%

  \subsubsection{Properties}

    \vspace{-1cm}
\hspace{\varindent}\begin{longtable}{|p{\varnamewidth}|p{\vardescrwidth}|l}
\cline{1-2}
\cline{1-2} \centering \textbf{Name} & \centering \textbf{Description}& \\
\cline{1-2}
\endhead\cline{1-2}\multicolumn{3}{r}{\small\textit{continued on next page}}\\\endfoot\cline{1-2}
\endlastfoot\multicolumn{2}{|l|}{\textit{Inherited from object}}\\
\multicolumn{2}{|p{\varwidth}|}{\raggedright \_\_class\_\_}\\
\cline{1-2}
\end{longtable}

    \index{peach \textit{(package)}!peach.ga \textit{(package)}!peach.ga.fitness \textit{(module)}!peach.ga.fitness.Fitness \textit{(class)}|)}

%%%%%%%%%%%%%%%%%%%%%%%%%%%%%%%%%%%%%%%%%%%%%%%%%%%%%%%%%%%%%%%%%%%%%%%%%%%
%%                           Class Description                           %%
%%%%%%%%%%%%%%%%%%%%%%%%%%%%%%%%%%%%%%%%%%%%%%%%%%%%%%%%%%%%%%%%%%%%%%%%%%%

    \index{peach \textit{(package)}!peach.ga \textit{(package)}!peach.ga.fitness \textit{(module)}!peach.ga.fitness.Ranking \textit{(class)}|(}
\subsection{Class Ranking}

    \label{peach:ga:fitness:Ranking}
\begin{tabular}{cccccccc}
% Line for object, linespec=[False, False]
\multicolumn{2}{r}{\settowidth{\BCL}{object}\multirow{2}{\BCL}{object}}
&&
&&
  \\\cline{3-3}
  &&\multicolumn{1}{c|}{}
&&
&&
  \\
% Line for peach.ga.fitness.Fitness, linespec=[False]
\multicolumn{4}{r}{\settowidth{\BCL}{peach.ga.fitness.Fitness}\multirow{2}{\BCL}{peach.ga.fitness.Fitness}}
&&
  \\\cline{5-5}
  &&&&\multicolumn{1}{c|}{}
&&
  \\
&&&&\multicolumn{2}{l}{\textbf{peach.ga.fitness.Ranking}}
\end{tabular}


Ranking fitness for a population

Ranking gives fitness values equally spaced between 0 and 1. The fittest
individual receives fitness equals to 1, the second best equals to 1 - 1/N,
the third best 1 - 2/N, and so on, where N is the size of the population.
It is important to note that the worst fit individual receives a fitness
value of 1/N, not 0. That allows that no individuals are excluded from the
selection operator.

%%%%%%%%%%%%%%%%%%%%%%%%%%%%%%%%%%%%%%%%%%%%%%%%%%%%%%%%%%%%%%%%%%%%%%%%%%%
%%                                Methods                                %%
%%%%%%%%%%%%%%%%%%%%%%%%%%%%%%%%%%%%%%%%%%%%%%%%%%%%%%%%%%%%%%%%%%%%%%%%%%%

  \subsubsection{Methods}

    \vspace{0.5ex}

\hspace{.8\funcindent}\begin{boxedminipage}{\funcwidth}

    \raggedright \textbf{\_\_init\_\_}(\textit{self})

    \vspace{-1.5ex}

    \rule{\textwidth}{0.5\fboxrule}
\setlength{\parskip}{2ex}

Initializes the operator.
\setlength{\parskip}{1ex}
      Overrides: object.\_\_init\_\_

    \end{boxedminipage}

    \vspace{0.5ex}

\hspace{.8\funcindent}\begin{boxedminipage}{\funcwidth}

    \raggedright \textbf{\_\_call\_\_}(\textit{self}, \textit{fx})

    \vspace{-1.5ex}

    \rule{\textwidth}{0.5\fboxrule}
\setlength{\parskip}{2ex}

Calculates the fitness for all individuals in the population.
\setlength{\parskip}{1ex}
      \textbf{Parameters}
      \vspace{-1ex}

      \begin{quote}
        \begin{Ventry}{xx}

          \item[fx]


The values of the objective function for every individual on the
population to be processed. Please, consult the \texttt{ga} module for
more information on populations. This method calculates the fitness
according to the equally spaced ranking technique.
        \end{Ventry}

      \end{quote}

      \textbf{Return Value}
    \vspace{-1ex}

      \begin{quote}

A vector containing the fitness value for every individual in the
population, in the same order that they appear there.
      \end{quote}

      Overrides: peach.ga.fitness.Fitness.\_\_call\_\_

    \end{boxedminipage}


\large{\textbf{\textit{Inherited from object}}}

\begin{quote}
\_\_delattr\_\_(), \_\_format\_\_(), \_\_getattribute\_\_(), \_\_hash\_\_(), \_\_new\_\_(), \_\_reduce\_\_(), \_\_reduce\_ex\_\_(), \_\_repr\_\_(), \_\_setattr\_\_(), \_\_sizeof\_\_(), \_\_str\_\_(), \_\_subclasshook\_\_()
\end{quote}

%%%%%%%%%%%%%%%%%%%%%%%%%%%%%%%%%%%%%%%%%%%%%%%%%%%%%%%%%%%%%%%%%%%%%%%%%%%
%%                              Properties                               %%
%%%%%%%%%%%%%%%%%%%%%%%%%%%%%%%%%%%%%%%%%%%%%%%%%%%%%%%%%%%%%%%%%%%%%%%%%%%

  \subsubsection{Properties}

    \vspace{-1cm}
\hspace{\varindent}\begin{longtable}{|p{\varnamewidth}|p{\vardescrwidth}|l}
\cline{1-2}
\cline{1-2} \centering \textbf{Name} & \centering \textbf{Description}& \\
\cline{1-2}
\endhead\cline{1-2}\multicolumn{3}{r}{\small\textit{continued on next page}}\\\endfoot\cline{1-2}
\endlastfoot\multicolumn{2}{|l|}{\textit{Inherited from object}}\\
\multicolumn{2}{|p{\varwidth}|}{\raggedright \_\_class\_\_}\\
\cline{1-2}
\end{longtable}

    \index{peach \textit{(package)}!peach.ga \textit{(package)}!peach.ga.fitness \textit{(module)}!peach.ga.fitness.Ranking \textit{(class)}|)}
    \index{peach \textit{(package)}!peach.ga \textit{(package)}!peach.ga.fitness \textit{(module)}|)}

%
% API Documentation for Peach - Computational Intelligence for Python
% Module peach.ga.mutation
%
% Generated by epydoc 3.0.1
% [Mon Jan 24 15:39:51 2011]
%

%%%%%%%%%%%%%%%%%%%%%%%%%%%%%%%%%%%%%%%%%%%%%%%%%%%%%%%%%%%%%%%%%%%%%%%%%%%
%%                          Module Description                           %%
%%%%%%%%%%%%%%%%%%%%%%%%%%%%%%%%%%%%%%%%%%%%%%%%%%%%%%%%%%%%%%%%%%%%%%%%%%%

    \index{peach \textit{(package)}!peach.ga \textit{(package)}!peach.ga.mutation \textit{(module)}|(}
\section{Module peach.ga.mutation}

    \label{peach:ga:mutation}

Basic definitions and classes for operating mutation on chromosomes.

The mutation operator changes selected bits in the array corresponding to the
chromosome. This operation is not as common as the others, but some genetic
algorithms still implement it.

%%%%%%%%%%%%%%%%%%%%%%%%%%%%%%%%%%%%%%%%%%%%%%%%%%%%%%%%%%%%%%%%%%%%%%%%%%%
%%                               Variables                               %%
%%%%%%%%%%%%%%%%%%%%%%%%%%%%%%%%%%%%%%%%%%%%%%%%%%%%%%%%%%%%%%%%%%%%%%%%%%%

  \subsection{Variables}

    \vspace{-1cm}
\hspace{\varindent}\begin{longtable}{|p{\varnamewidth}|p{\vardescrwidth}|l}
\cline{1-2}
\cline{1-2} \centering \textbf{Name} & \centering \textbf{Description}& \\
\cline{1-2}
\endhead\cline{1-2}\multicolumn{3}{r}{\small\textit{continued on next page}}\\\endfoot\cline{1-2}
\endlastfoot\raggedright \_\-\_\-d\-o\-c\-\_\-\_\- & \raggedright \textbf{Value:} 
{\tt \texttt{...}}&\\
\cline{1-2}
\raggedright \_\-\_\-p\-a\-c\-k\-a\-g\-e\-\_\-\_\- & \raggedright \textbf{Value:} 
{\tt \texttt{'}\texttt{peach.ga}\texttt{'}}&\\
\cline{1-2}
\end{longtable}


%%%%%%%%%%%%%%%%%%%%%%%%%%%%%%%%%%%%%%%%%%%%%%%%%%%%%%%%%%%%%%%%%%%%%%%%%%%
%%                           Class Description                           %%
%%%%%%%%%%%%%%%%%%%%%%%%%%%%%%%%%%%%%%%%%%%%%%%%%%%%%%%%%%%%%%%%%%%%%%%%%%%

    \index{peach \textit{(package)}!peach.ga \textit{(package)}!peach.ga.mutation \textit{(module)}!peach.ga.mutation.Mutation \textit{(class)}|(}
\subsection{Class Mutation}

    \label{peach:ga:mutation:Mutation}
\begin{tabular}{cccccc}
% Line for object, linespec=[False]
\multicolumn{2}{r}{\settowidth{\BCL}{object}\multirow{2}{\BCL}{object}}
&&
  \\\cline{3-3}
  &&\multicolumn{1}{c|}{}
&&
  \\
&&\multicolumn{2}{l}{\textbf{peach.ga.mutation.Mutation}}
\end{tabular}

\textbf{Known Subclasses:} peach.ga.mutation.BitToBit


Base class for mutation operators.

This class should be subclassed if you want to create your own mutation
operator. The base class doesn't do much, it is only a prototype. As is done
with all the base classes within this library, use the \texttt{\_\_init\_\_} method
to configure your mutation behaviour -{}- if needed -{}- and the \texttt{\_\_call\_\_}
method to operate over a population.

A class derived from this one should implement at least 2 methods, defined
below:
%
\begin{quote}
%
\begin{description}
\item[{\_\_init\_\_(self, %
\raisebox{1em}{\hypertarget{id2}{}}\hyperlink{id1}{\textbf{\color{red}*}}cnf, %
\raisebox{1em}{\hypertarget{id4}{}}\hyperlink{id3}{\textbf{\color{red}**}}kw)}] \leavevmode 
Initializes the object. There is no mandatory arguments, but any
parameters can be used here to configure the operator. For example, a
class can define a mutation rate -{}- this should be defined here:
%
\begin{quote}{\ttfamily \raggedright \noindent
\_\_init\_\_(self,~rate=0.75)
}
\end{quote}

A default value should always be offered, if possible.

\item[{\_\_call\_\_(self, population)}] \leavevmode 
The \texttt{\_\_call\_\_} implementation should receive a population and operate
over it. Please, consult the \texttt{ga} module to see more information on
populations. It should return the processed population. No recomendation
on the internals of the method is made.

\end{description}

\end{quote}

Please, note that the GA implementations relies on this behaviour: it will
pass a population to your \texttt{\_\_call\_\_} method and expects to received the
result back.

%%%%%%%%%%%%%%%%%%%%%%%%%%%%%%%%%%%%%%%%%%%%%%%%%%%%%%%%%%%%%%%%%%%%%%%%%%%
%%                                Methods                                %%
%%%%%%%%%%%%%%%%%%%%%%%%%%%%%%%%%%%%%%%%%%%%%%%%%%%%%%%%%%%%%%%%%%%%%%%%%%%

  \subsubsection{Methods}


\large{\textbf{\textit{Inherited from object}}}

\begin{quote}
\_\_delattr\_\_(), \_\_format\_\_(), \_\_getattribute\_\_(), \_\_hash\_\_(), \_\_init\_\_(), \_\_new\_\_(), \_\_reduce\_\_(), \_\_reduce\_ex\_\_(), \_\_repr\_\_(), \_\_setattr\_\_(), \_\_sizeof\_\_(), \_\_str\_\_(), \_\_subclasshook\_\_()
\end{quote}

%%%%%%%%%%%%%%%%%%%%%%%%%%%%%%%%%%%%%%%%%%%%%%%%%%%%%%%%%%%%%%%%%%%%%%%%%%%
%%                              Properties                               %%
%%%%%%%%%%%%%%%%%%%%%%%%%%%%%%%%%%%%%%%%%%%%%%%%%%%%%%%%%%%%%%%%%%%%%%%%%%%

  \subsubsection{Properties}

    \vspace{-1cm}
\hspace{\varindent}\begin{longtable}{|p{\varnamewidth}|p{\vardescrwidth}|l}
\cline{1-2}
\cline{1-2} \centering \textbf{Name} & \centering \textbf{Description}& \\
\cline{1-2}
\endhead\cline{1-2}\multicolumn{3}{r}{\small\textit{continued on next page}}\\\endfoot\cline{1-2}
\endlastfoot\multicolumn{2}{|l|}{\textit{Inherited from object}}\\
\multicolumn{2}{|p{\varwidth}|}{\raggedright \_\_class\_\_}\\
\cline{1-2}
\end{longtable}

    \index{peach \textit{(package)}!peach.ga \textit{(package)}!peach.ga.mutation \textit{(module)}!peach.ga.mutation.Mutation \textit{(class)}|)}

%%%%%%%%%%%%%%%%%%%%%%%%%%%%%%%%%%%%%%%%%%%%%%%%%%%%%%%%%%%%%%%%%%%%%%%%%%%
%%                           Class Description                           %%
%%%%%%%%%%%%%%%%%%%%%%%%%%%%%%%%%%%%%%%%%%%%%%%%%%%%%%%%%%%%%%%%%%%%%%%%%%%

    \index{peach \textit{(package)}!peach.ga \textit{(package)}!peach.ga.mutation \textit{(module)}!peach.ga.mutation.BitToBit \textit{(class)}|(}
\subsection{Class BitToBit}

    \label{peach:ga:mutation:BitToBit}
\begin{tabular}{cccccccc}
% Line for object, linespec=[False, False]
\multicolumn{2}{r}{\settowidth{\BCL}{object}\multirow{2}{\BCL}{object}}
&&
&&
  \\\cline{3-3}
  &&\multicolumn{1}{c|}{}
&&
&&
  \\
% Line for peach.ga.mutation.Mutation, linespec=[False]
\multicolumn{4}{r}{\settowidth{\BCL}{peach.ga.mutation.Mutation}\multirow{2}{\BCL}{peach.ga.mutation.Mutation}}
&&
  \\\cline{5-5}
  &&&&\multicolumn{1}{c|}{}
&&
  \\
&&&&\multicolumn{2}{l}{\textbf{peach.ga.mutation.BitToBit}}
\end{tabular}


A simple bit-to-bit mutation operator.

This operator scans every individual in the population, in a bit-to-bit
fashion. If a uniformly random number is less than the mutation rate (see
below), then the bit is inverted. The mutation should be made very small,
since large populations will represent a big number of bits; it should never
be more than 0.5.

%%%%%%%%%%%%%%%%%%%%%%%%%%%%%%%%%%%%%%%%%%%%%%%%%%%%%%%%%%%%%%%%%%%%%%%%%%%
%%                                Methods                                %%
%%%%%%%%%%%%%%%%%%%%%%%%%%%%%%%%%%%%%%%%%%%%%%%%%%%%%%%%%%%%%%%%%%%%%%%%%%%

  \subsubsection{Methods}

    \vspace{0.5ex}

\hspace{.8\funcindent}\begin{boxedminipage}{\funcwidth}

    \raggedright \textbf{\_\_init\_\_}(\textit{self}, \textit{rate}={\tt 0.05})

    \vspace{-1.5ex}

    \rule{\textwidth}{0.5\fboxrule}
\setlength{\parskip}{2ex}

Initialize the mutation operator.
\setlength{\parskip}{1ex}
      \textbf{Parameters}
      \vspace{-1ex}

      \begin{quote}
        \begin{Ventry}{xxxx}

          \item[rate]


Probability that a single bit in an individual will be inverted.
        \end{Ventry}

      \end{quote}

      Overrides: object.\_\_init\_\_

    \end{boxedminipage}

    \label{peach:ga:mutation:BitToBit:__call__}
    \index{peach \textit{(package)}!peach.ga \textit{(package)}!peach.ga.mutation \textit{(module)}!peach.ga.mutation.BitToBit \textit{(class)}!peach.ga.mutation.BitToBit.\_\_call\_\_ \textit{(method)}}

    \vspace{0.5ex}

\hspace{.8\funcindent}\begin{boxedminipage}{\funcwidth}

    \raggedright \textbf{\_\_call\_\_}(\textit{self}, \textit{population})

    \vspace{-1.5ex}

    \rule{\textwidth}{0.5\fboxrule}
\setlength{\parskip}{2ex}

Applies the operator over a population.

The behaviour of this operator is as described above: it scans every bit
in every individual, and if a random number is less than the mutation
rate, the bit is inverted.
\setlength{\parskip}{1ex}
      \textbf{Parameters}
      \vspace{-1ex}

      \begin{quote}
        \begin{Ventry}{xxxxxxxxxx}

          \item[population]


A list of \texttt{Chromosomes} containing the present population of the
algorithm. It is processed and the results of the exchange are
returned to the caller.
        \end{Ventry}

      \end{quote}

      \textbf{Return Value}
    \vspace{-1ex}

      \begin{quote}

The processed population, a list of \texttt{Chromosomes}.
      \end{quote}

    \end{boxedminipage}


\large{\textbf{\textit{Inherited from object}}}

\begin{quote}
\_\_delattr\_\_(), \_\_format\_\_(), \_\_getattribute\_\_(), \_\_hash\_\_(), \_\_new\_\_(), \_\_reduce\_\_(), \_\_reduce\_ex\_\_(), \_\_repr\_\_(), \_\_setattr\_\_(), \_\_sizeof\_\_(), \_\_str\_\_(), \_\_subclasshook\_\_()
\end{quote}

%%%%%%%%%%%%%%%%%%%%%%%%%%%%%%%%%%%%%%%%%%%%%%%%%%%%%%%%%%%%%%%%%%%%%%%%%%%
%%                              Properties                               %%
%%%%%%%%%%%%%%%%%%%%%%%%%%%%%%%%%%%%%%%%%%%%%%%%%%%%%%%%%%%%%%%%%%%%%%%%%%%

  \subsubsection{Properties}

    \vspace{-1cm}
\hspace{\varindent}\begin{longtable}{|p{\varnamewidth}|p{\vardescrwidth}|l}
\cline{1-2}
\cline{1-2} \centering \textbf{Name} & \centering \textbf{Description}& \\
\cline{1-2}
\endhead\cline{1-2}\multicolumn{3}{r}{\small\textit{continued on next page}}\\\endfoot\cline{1-2}
\endlastfoot\multicolumn{2}{|l|}{\textit{Inherited from object}}\\
\multicolumn{2}{|p{\varwidth}|}{\raggedright \_\_class\_\_}\\
\cline{1-2}
\end{longtable}


%%%%%%%%%%%%%%%%%%%%%%%%%%%%%%%%%%%%%%%%%%%%%%%%%%%%%%%%%%%%%%%%%%%%%%%%%%%
%%                          Instance Variables                           %%
%%%%%%%%%%%%%%%%%%%%%%%%%%%%%%%%%%%%%%%%%%%%%%%%%%%%%%%%%%%%%%%%%%%%%%%%%%%

  \subsubsection{Instance Variables}

    \vspace{-1cm}
\hspace{\varindent}\begin{longtable}{|p{\varnamewidth}|p{\vardescrwidth}|l}
\cline{1-2}
\cline{1-2} \centering \textbf{Name} & \centering \textbf{Description}& \\
\cline{1-2}
\endhead\cline{1-2}\multicolumn{3}{r}{\small\textit{continued on next page}}\\\endfoot\cline{1-2}
\endlastfoot\raggedright r\-a\-t\-e\- & Property that contains the mutation rate.&\\
\cline{1-2}
\end{longtable}

    \index{peach \textit{(package)}!peach.ga \textit{(package)}!peach.ga.mutation \textit{(module)}!peach.ga.mutation.BitToBit \textit{(class)}|)}
    \index{peach \textit{(package)}!peach.ga \textit{(package)}!peach.ga.mutation \textit{(module)}|)}

%
% API Documentation for Peach - Computational Intelligence for Python
% Module peach.ga.selection
%
% Generated by epydoc 3.0.1
% [Sun Jul 31 17:00:40 2011]
%

%%%%%%%%%%%%%%%%%%%%%%%%%%%%%%%%%%%%%%%%%%%%%%%%%%%%%%%%%%%%%%%%%%%%%%%%%%%
%%                          Module Description                           %%
%%%%%%%%%%%%%%%%%%%%%%%%%%%%%%%%%%%%%%%%%%%%%%%%%%%%%%%%%%%%%%%%%%%%%%%%%%%

    \index{peach \textit{(package)}!peach.ga \textit{(package)}!peach.ga.selection \textit{(module)}|(}
\section{Module peach.ga.selection}

    \label{peach:ga:selection}

Basic classes and definitions for selection operator.

The first step in a genetic algorithm is the selection of the fittest
individuals. The selection method typically uses the fitness of the population
to compute which individuals are closer to the best solution. However, instead
of deterministically deciding which individuals continue to the next generation,
they are randomly choosen, the chances of an individual being choosen given by
its fitness value. This sub-module implements selection methods.

%%%%%%%%%%%%%%%%%%%%%%%%%%%%%%%%%%%%%%%%%%%%%%%%%%%%%%%%%%%%%%%%%%%%%%%%%%%
%%                               Variables                               %%
%%%%%%%%%%%%%%%%%%%%%%%%%%%%%%%%%%%%%%%%%%%%%%%%%%%%%%%%%%%%%%%%%%%%%%%%%%%

  \subsection{Variables}

    \vspace{-1cm}
\hspace{\varindent}\begin{longtable}{|p{\varnamewidth}|p{\vardescrwidth}|l}
\cline{1-2}
\cline{1-2} \centering \textbf{Name} & \centering \textbf{Description}& \\
\cline{1-2}
\endhead\cline{1-2}\multicolumn{3}{r}{\small\textit{continued on next page}}\\\endfoot\cline{1-2}
\endlastfoot\raggedright \_\-\_\-d\-o\-c\-\_\-\_\- & \raggedright \textbf{Value:} 
{\tt \texttt{...}}&\\
\cline{1-2}
\raggedright \_\-\_\-p\-a\-c\-k\-a\-g\-e\-\_\-\_\- & \raggedright \textbf{Value:} 
{\tt \texttt{'}\texttt{peach.ga}\texttt{'}}&\\
\cline{1-2}
\end{longtable}


%%%%%%%%%%%%%%%%%%%%%%%%%%%%%%%%%%%%%%%%%%%%%%%%%%%%%%%%%%%%%%%%%%%%%%%%%%%
%%                           Class Description                           %%
%%%%%%%%%%%%%%%%%%%%%%%%%%%%%%%%%%%%%%%%%%%%%%%%%%%%%%%%%%%%%%%%%%%%%%%%%%%

    \index{peach \textit{(package)}!peach.ga \textit{(package)}!peach.ga.selection \textit{(module)}!peach.ga.selection.Selection \textit{(class)}|(}
\subsection{Class Selection}

    \label{peach:ga:selection:Selection}
\begin{tabular}{cccccc}
% Line for object, linespec=[False]
\multicolumn{2}{r}{\settowidth{\BCL}{object}\multirow{2}{\BCL}{object}}
&&
  \\\cline{3-3}
  &&\multicolumn{1}{c|}{}
&&
  \\
&&\multicolumn{2}{l}{\textbf{peach.ga.selection.Selection}}
\end{tabular}

\textbf{Known Subclasses:}
peach.ga.selection.Baker,
    peach.ga.selection.BinaryTournament,
    peach.ga.selection.RouletteWheel


Base class for selection operators.

This class should be subclassed if you want to create your own selection
operator. The base class doesn't do much, it is only a prototype. As is done
with all the base classes within this library, use the \texttt{\_\_init\_\_} method
to configure your selection behaviour -{}- if needed -{}- and the \texttt{\_\_call\_\_}
method to operate over a population.

A class derived from this one should implement at least 2 methods, defined
below:
%
\begin{quote}
%
\begin{description}
\item[{\_\_init\_\_(self, %
\raisebox{1em}{\hypertarget{id2}{}}\hyperlink{id1}{\textbf{\color{red}*}}cnf, %
\raisebox{1em}{\hypertarget{id4}{}}\hyperlink{id3}{\textbf{\color{red}**}}kw)}] \leavevmode 
Initializes the object. There is no mandatory arguments, but any
parameters can be used here to configure the operator. A default value
should always be offered, if possible.

\item[{\_\_call\_\_(self, population)}] \leavevmode 
The \texttt{\_\_call\_\_} implementation should receive a population and operate
over it. Please, consult the \texttt{ga} module to see more information on
populations. It should return the processed population. No recomendation
on the internals of the method is made.

\end{description}

\end{quote}

Please, note that the GA implementations relies on this behaviour: it will
pass a population to your \texttt{\_\_call\_\_} method and expects to received the
result back.

%%%%%%%%%%%%%%%%%%%%%%%%%%%%%%%%%%%%%%%%%%%%%%%%%%%%%%%%%%%%%%%%%%%%%%%%%%%
%%                                Methods                                %%
%%%%%%%%%%%%%%%%%%%%%%%%%%%%%%%%%%%%%%%%%%%%%%%%%%%%%%%%%%%%%%%%%%%%%%%%%%%

  \subsubsection{Methods}


\large{\textbf{\textit{Inherited from object}}}

\begin{quote}
\_\_delattr\_\_(), \_\_format\_\_(), \_\_getattribute\_\_(), \_\_hash\_\_(), \_\_init\_\_(), \_\_new\_\_(), \_\_reduce\_\_(), \_\_reduce\_ex\_\_(), \_\_repr\_\_(), \_\_setattr\_\_(), \_\_sizeof\_\_(), \_\_str\_\_(), \_\_subclasshook\_\_()
\end{quote}

%%%%%%%%%%%%%%%%%%%%%%%%%%%%%%%%%%%%%%%%%%%%%%%%%%%%%%%%%%%%%%%%%%%%%%%%%%%
%%                              Properties                               %%
%%%%%%%%%%%%%%%%%%%%%%%%%%%%%%%%%%%%%%%%%%%%%%%%%%%%%%%%%%%%%%%%%%%%%%%%%%%

  \subsubsection{Properties}

    \vspace{-1cm}
\hspace{\varindent}\begin{longtable}{|p{\varnamewidth}|p{\vardescrwidth}|l}
\cline{1-2}
\cline{1-2} \centering \textbf{Name} & \centering \textbf{Description}& \\
\cline{1-2}
\endhead\cline{1-2}\multicolumn{3}{r}{\small\textit{continued on next page}}\\\endfoot\cline{1-2}
\endlastfoot\multicolumn{2}{|l|}{\textit{Inherited from object}}\\
\multicolumn{2}{|p{\varwidth}|}{\raggedright \_\_class\_\_}\\
\cline{1-2}
\end{longtable}

    \index{peach \textit{(package)}!peach.ga \textit{(package)}!peach.ga.selection \textit{(module)}!peach.ga.selection.Selection \textit{(class)}|)}

%%%%%%%%%%%%%%%%%%%%%%%%%%%%%%%%%%%%%%%%%%%%%%%%%%%%%%%%%%%%%%%%%%%%%%%%%%%
%%                           Class Description                           %%
%%%%%%%%%%%%%%%%%%%%%%%%%%%%%%%%%%%%%%%%%%%%%%%%%%%%%%%%%%%%%%%%%%%%%%%%%%%

    \index{peach \textit{(package)}!peach.ga \textit{(package)}!peach.ga.selection \textit{(module)}!peach.ga.selection.RouletteWheel \textit{(class)}|(}
\subsection{Class RouletteWheel}

    \label{peach:ga:selection:RouletteWheel}
\begin{tabular}{cccccccc}
% Line for object, linespec=[False, False]
\multicolumn{2}{r}{\settowidth{\BCL}{object}\multirow{2}{\BCL}{object}}
&&
&&
  \\\cline{3-3}
  &&\multicolumn{1}{c|}{}
&&
&&
  \\
% Line for peach.ga.selection.Selection, linespec=[False]
\multicolumn{4}{r}{\settowidth{\BCL}{peach.ga.selection.Selection}\multirow{2}{\BCL}{peach.ga.selection.Selection}}
&&
  \\\cline{5-5}
  &&&&\multicolumn{1}{c|}{}
&&
  \\
&&&&\multicolumn{2}{l}{\textbf{peach.ga.selection.RouletteWheel}}
\end{tabular}


The Roulette Wheel selection method.

This method randomly chooses a new population with the same size of the
original population. An individual is choosen with a probability
proportional to its fitness value, independent of what fitness method was
used. This is usually abstracted as a roulette wheel in texts about the
subject. Please, note that the selection is done \emph{in loco}, that is,
although the new population is returned, it is not a new list -{}- it is the
same list as before, but with values changed.

%%%%%%%%%%%%%%%%%%%%%%%%%%%%%%%%%%%%%%%%%%%%%%%%%%%%%%%%%%%%%%%%%%%%%%%%%%%
%%                                Methods                                %%
%%%%%%%%%%%%%%%%%%%%%%%%%%%%%%%%%%%%%%%%%%%%%%%%%%%%%%%%%%%%%%%%%%%%%%%%%%%

  \subsubsection{Methods}

    \label{peach:ga:selection:RouletteWheel:__call__}
    \index{peach \textit{(package)}!peach.ga \textit{(package)}!peach.ga.selection \textit{(module)}!peach.ga.selection.RouletteWheel \textit{(class)}!peach.ga.selection.RouletteWheel.\_\_call\_\_ \textit{(method)}}

    \vspace{0.5ex}

\hspace{.8\funcindent}\begin{boxedminipage}{\funcwidth}

    \raggedright \textbf{\_\_call\_\_}(\textit{self}, \textit{population})

    \vspace{-1.5ex}

    \rule{\textwidth}{0.5\fboxrule}
\setlength{\parskip}{2ex}

Selects the population.
\setlength{\parskip}{1ex}
      \textbf{Parameters}
      \vspace{-1ex}

      \begin{quote}
        \begin{Ventry}{xxxxxxxxxx}

          \item[population]


The list of chromosomes that should be operated over. The given list
is modified, so be aware that the old generation will not be
available after stepping the GA.
        \end{Ventry}

      \end{quote}

      \textbf{Return Value}
    \vspace{-1ex}

      \begin{quote}

The new population.
      \end{quote}

    \end{boxedminipage}


\large{\textbf{\textit{Inherited from object}}}

\begin{quote}
\_\_delattr\_\_(), \_\_format\_\_(), \_\_getattribute\_\_(), \_\_hash\_\_(), \_\_init\_\_(), \_\_new\_\_(), \_\_reduce\_\_(), \_\_reduce\_ex\_\_(), \_\_repr\_\_(), \_\_setattr\_\_(), \_\_sizeof\_\_(), \_\_str\_\_(), \_\_subclasshook\_\_()
\end{quote}

%%%%%%%%%%%%%%%%%%%%%%%%%%%%%%%%%%%%%%%%%%%%%%%%%%%%%%%%%%%%%%%%%%%%%%%%%%%
%%                              Properties                               %%
%%%%%%%%%%%%%%%%%%%%%%%%%%%%%%%%%%%%%%%%%%%%%%%%%%%%%%%%%%%%%%%%%%%%%%%%%%%

  \subsubsection{Properties}

    \vspace{-1cm}
\hspace{\varindent}\begin{longtable}{|p{\varnamewidth}|p{\vardescrwidth}|l}
\cline{1-2}
\cline{1-2} \centering \textbf{Name} & \centering \textbf{Description}& \\
\cline{1-2}
\endhead\cline{1-2}\multicolumn{3}{r}{\small\textit{continued on next page}}\\\endfoot\cline{1-2}
\endlastfoot\multicolumn{2}{|l|}{\textit{Inherited from object}}\\
\multicolumn{2}{|p{\varwidth}|}{\raggedright \_\_class\_\_}\\
\cline{1-2}
\end{longtable}

    \index{peach \textit{(package)}!peach.ga \textit{(package)}!peach.ga.selection \textit{(module)}!peach.ga.selection.RouletteWheel \textit{(class)}|)}

%%%%%%%%%%%%%%%%%%%%%%%%%%%%%%%%%%%%%%%%%%%%%%%%%%%%%%%%%%%%%%%%%%%%%%%%%%%
%%                           Class Description                           %%
%%%%%%%%%%%%%%%%%%%%%%%%%%%%%%%%%%%%%%%%%%%%%%%%%%%%%%%%%%%%%%%%%%%%%%%%%%%

    \index{peach \textit{(package)}!peach.ga \textit{(package)}!peach.ga.selection \textit{(module)}!peach.ga.selection.BinaryTournament \textit{(class)}|(}
\subsection{Class BinaryTournament}

    \label{peach:ga:selection:BinaryTournament}
\begin{tabular}{cccccccc}
% Line for object, linespec=[False, False]
\multicolumn{2}{r}{\settowidth{\BCL}{object}\multirow{2}{\BCL}{object}}
&&
&&
  \\\cline{3-3}
  &&\multicolumn{1}{c|}{}
&&
&&
  \\
% Line for peach.ga.selection.Selection, linespec=[False]
\multicolumn{4}{r}{\settowidth{\BCL}{peach.ga.selection.Selection}\multirow{2}{\BCL}{peach.ga.selection.Selection}}
&&
  \\\cline{5-5}
  &&&&\multicolumn{1}{c|}{}
&&
  \\
&&&&\multicolumn{2}{l}{\textbf{peach.ga.selection.BinaryTournament}}
\end{tabular}


The Binary Tournament selection method.

This method randomly chooses a new population with the same size of the
original population. Two individuals are choosen at random and they
``battle'', the fittest surviving for the next generation. Please, note that
the selection is done \emph{in loco}, that is, although the new population is
returned, it is not a new list -{}- it is the same list as before, but with
values changed.

%%%%%%%%%%%%%%%%%%%%%%%%%%%%%%%%%%%%%%%%%%%%%%%%%%%%%%%%%%%%%%%%%%%%%%%%%%%
%%                                Methods                                %%
%%%%%%%%%%%%%%%%%%%%%%%%%%%%%%%%%%%%%%%%%%%%%%%%%%%%%%%%%%%%%%%%%%%%%%%%%%%

  \subsubsection{Methods}

    \label{peach:ga:selection:BinaryTournament:__call__}
    \index{peach \textit{(package)}!peach.ga \textit{(package)}!peach.ga.selection \textit{(module)}!peach.ga.selection.BinaryTournament \textit{(class)}!peach.ga.selection.BinaryTournament.\_\_call\_\_ \textit{(method)}}

    \vspace{0.5ex}

\hspace{.8\funcindent}\begin{boxedminipage}{\funcwidth}

    \raggedright \textbf{\_\_call\_\_}(\textit{self}, \textit{population})

    \vspace{-1.5ex}

    \rule{\textwidth}{0.5\fboxrule}
\setlength{\parskip}{2ex}

Selects the population.
\setlength{\parskip}{1ex}
      \textbf{Parameters}
      \vspace{-1ex}

      \begin{quote}
        \begin{Ventry}{xxxxxxxxxx}

          \item[population]


The list of chromosomes that should be operated over. The given list
is modified, so be aware that the old generation will not be
available after stepping the GA.
        \end{Ventry}

      \end{quote}

      \textbf{Return Value}
    \vspace{-1ex}

      \begin{quote}

The new population.
      \end{quote}

    \end{boxedminipage}


\large{\textbf{\textit{Inherited from object}}}

\begin{quote}
\_\_delattr\_\_(), \_\_format\_\_(), \_\_getattribute\_\_(), \_\_hash\_\_(), \_\_init\_\_(), \_\_new\_\_(), \_\_reduce\_\_(), \_\_reduce\_ex\_\_(), \_\_repr\_\_(), \_\_setattr\_\_(), \_\_sizeof\_\_(), \_\_str\_\_(), \_\_subclasshook\_\_()
\end{quote}

%%%%%%%%%%%%%%%%%%%%%%%%%%%%%%%%%%%%%%%%%%%%%%%%%%%%%%%%%%%%%%%%%%%%%%%%%%%
%%                              Properties                               %%
%%%%%%%%%%%%%%%%%%%%%%%%%%%%%%%%%%%%%%%%%%%%%%%%%%%%%%%%%%%%%%%%%%%%%%%%%%%

  \subsubsection{Properties}

    \vspace{-1cm}
\hspace{\varindent}\begin{longtable}{|p{\varnamewidth}|p{\vardescrwidth}|l}
\cline{1-2}
\cline{1-2} \centering \textbf{Name} & \centering \textbf{Description}& \\
\cline{1-2}
\endhead\cline{1-2}\multicolumn{3}{r}{\small\textit{continued on next page}}\\\endfoot\cline{1-2}
\endlastfoot\multicolumn{2}{|l|}{\textit{Inherited from object}}\\
\multicolumn{2}{|p{\varwidth}|}{\raggedright \_\_class\_\_}\\
\cline{1-2}
\end{longtable}

    \index{peach \textit{(package)}!peach.ga \textit{(package)}!peach.ga.selection \textit{(module)}!peach.ga.selection.BinaryTournament \textit{(class)}|)}

%%%%%%%%%%%%%%%%%%%%%%%%%%%%%%%%%%%%%%%%%%%%%%%%%%%%%%%%%%%%%%%%%%%%%%%%%%%
%%                           Class Description                           %%
%%%%%%%%%%%%%%%%%%%%%%%%%%%%%%%%%%%%%%%%%%%%%%%%%%%%%%%%%%%%%%%%%%%%%%%%%%%

    \index{peach \textit{(package)}!peach.ga \textit{(package)}!peach.ga.selection \textit{(module)}!peach.ga.selection.Baker \textit{(class)}|(}
\subsection{Class Baker}

    \label{peach:ga:selection:Baker}
\begin{tabular}{cccccccc}
% Line for object, linespec=[False, False]
\multicolumn{2}{r}{\settowidth{\BCL}{object}\multirow{2}{\BCL}{object}}
&&
&&
  \\\cline{3-3}
  &&\multicolumn{1}{c|}{}
&&
&&
  \\
% Line for peach.ga.selection.Selection, linespec=[False]
\multicolumn{4}{r}{\settowidth{\BCL}{peach.ga.selection.Selection}\multirow{2}{\BCL}{peach.ga.selection.Selection}}
&&
  \\\cline{5-5}
  &&&&\multicolumn{1}{c|}{}
&&
  \\
&&&&\multicolumn{2}{l}{\textbf{peach.ga.selection.Baker}}
\end{tabular}


The Baker selection method.

This method is very similar to the Roulette Wheel, but instead or randomly
choosing every new member on the next generation, only the first probability
is randomized. The others are determined as equally spaced numbers from 0 to
1, from this number. Please, note that the selection is done \emph{in loco}, that
is, although the new population is returned, it is not a new list -{}- it is
the same list as before, but with values changed.

%%%%%%%%%%%%%%%%%%%%%%%%%%%%%%%%%%%%%%%%%%%%%%%%%%%%%%%%%%%%%%%%%%%%%%%%%%%
%%                                Methods                                %%
%%%%%%%%%%%%%%%%%%%%%%%%%%%%%%%%%%%%%%%%%%%%%%%%%%%%%%%%%%%%%%%%%%%%%%%%%%%

  \subsubsection{Methods}

    \label{peach:ga:selection:Baker:__call__}
    \index{peach \textit{(package)}!peach.ga \textit{(package)}!peach.ga.selection \textit{(module)}!peach.ga.selection.Baker \textit{(class)}!peach.ga.selection.Baker.\_\_call\_\_ \textit{(method)}}

    \vspace{0.5ex}

\hspace{.8\funcindent}\begin{boxedminipage}{\funcwidth}

    \raggedright \textbf{\_\_call\_\_}(\textit{self}, \textit{population})

    \vspace{-1.5ex}

    \rule{\textwidth}{0.5\fboxrule}
\setlength{\parskip}{2ex}

Selects the population.
\setlength{\parskip}{1ex}
      \textbf{Parameters}
      \vspace{-1ex}

      \begin{quote}
        \begin{Ventry}{xxxxxxxxxx}

          \item[population]


The list of chromosomes that should be operated over. The given list
is modified, so be aware that the old generation will not be
available after stepping the GA.
        \end{Ventry}

      \end{quote}

      \textbf{Return Value}
    \vspace{-1ex}

      \begin{quote}

The new population.
      \end{quote}

    \end{boxedminipage}


\large{\textbf{\textit{Inherited from object}}}

\begin{quote}
\_\_delattr\_\_(), \_\_format\_\_(), \_\_getattribute\_\_(), \_\_hash\_\_(), \_\_init\_\_(), \_\_new\_\_(), \_\_reduce\_\_(), \_\_reduce\_ex\_\_(), \_\_repr\_\_(), \_\_setattr\_\_(), \_\_sizeof\_\_(), \_\_str\_\_(), \_\_subclasshook\_\_()
\end{quote}

%%%%%%%%%%%%%%%%%%%%%%%%%%%%%%%%%%%%%%%%%%%%%%%%%%%%%%%%%%%%%%%%%%%%%%%%%%%
%%                              Properties                               %%
%%%%%%%%%%%%%%%%%%%%%%%%%%%%%%%%%%%%%%%%%%%%%%%%%%%%%%%%%%%%%%%%%%%%%%%%%%%

  \subsubsection{Properties}

    \vspace{-1cm}
\hspace{\varindent}\begin{longtable}{|p{\varnamewidth}|p{\vardescrwidth}|l}
\cline{1-2}
\cline{1-2} \centering \textbf{Name} & \centering \textbf{Description}& \\
\cline{1-2}
\endhead\cline{1-2}\multicolumn{3}{r}{\small\textit{continued on next page}}\\\endfoot\cline{1-2}
\endlastfoot\multicolumn{2}{|l|}{\textit{Inherited from object}}\\
\multicolumn{2}{|p{\varwidth}|}{\raggedright \_\_class\_\_}\\
\cline{1-2}
\end{longtable}

    \index{peach \textit{(package)}!peach.ga \textit{(package)}!peach.ga.selection \textit{(module)}!peach.ga.selection.Baker \textit{(class)}|)}
    \index{peach \textit{(package)}!peach.ga \textit{(package)}!peach.ga.selection \textit{(module)}|)}

%
% API Documentation for Peach - Computational Intelligence for Python
% Package peach.nn
%
% Generated by epydoc 3.0beta1
% [Mon Dec 21 08:51:37 2009]
%

%%%%%%%%%%%%%%%%%%%%%%%%%%%%%%%%%%%%%%%%%%%%%%%%%%%%%%%%%%%%%%%%%%%%%%%%%%%
%%                          Module Description                           %%
%%%%%%%%%%%%%%%%%%%%%%%%%%%%%%%%%%%%%%%%%%%%%%%%%%%%%%%%%%%%%%%%%%%%%%%%%%%

    \index{peach \textit{(package)}!peach.nn \textit{(package)}|(}
\section{Package peach.nn}

    \label{peach:nn}

This package implements support for neural networks. Consult:
\begin{quote}
\begin{description}
%[visit_definition_list_item]
\item[{af}] %[visit_definition]

A list of activation functions for use with neurons and a base class to
implement different activation functions;

%[depart_definition]
%[depart_definition_list_item]
%[visit_definition_list_item]
\item[{base}] %[visit_definition]

Basic definitions of the objects used with neural networks;

%[depart_definition]
%[depart_definition_list_item]
%[visit_definition_list_item]
\item[{lrule}] %[visit_definition]

Learning rules;

%[depart_definition]
%[depart_definition_list_item]
%[visit_definition_list_item]
\item[{nn}] %[visit_definition]

Implementation of different classes of neural networks;

%[depart_definition]
%[depart_definition_list_item]
\end{description}
\end{quote}

%%%%%%%%%%%%%%%%%%%%%%%%%%%%%%%%%%%%%%%%%%%%%%%%%%%%%%%%%%%%%%%%%%%%%%%%%%%
%%                                Modules                                %%
%%%%%%%%%%%%%%%%%%%%%%%%%%%%%%%%%%%%%%%%%%%%%%%%%%%%%%%%%%%%%%%%%%%%%%%%%%%

\subsection{Modules}

\begin{itemize}
\setlength{\parskip}{0ex}
\item \textbf{af}: 
Base activation functions and base class


  \textit{(Section \ref{peach:nn:af}, p.~\pageref{peach:nn:af})}

\item \textbf{base}: 
Basic definitions for layers of neurons.


  \textit{(Section \ref{peach:nn:base}, p.~\pageref{peach:nn:base})}

\item \textbf{lrules}: 
Learning rules for neural networks and base classes for custom learning.


  \textit{(Section \ref{peach:nn:lrules}, p.~\pageref{peach:nn:lrules})}

\item \textbf{nn}: 
Basic topologies of neural networks.


  \textit{(Section \ref{peach:nn:nn}, p.~\pageref{peach:nn:nn})}

\end{itemize}


%%%%%%%%%%%%%%%%%%%%%%%%%%%%%%%%%%%%%%%%%%%%%%%%%%%%%%%%%%%%%%%%%%%%%%%%%%%
%%                               Variables                               %%
%%%%%%%%%%%%%%%%%%%%%%%%%%%%%%%%%%%%%%%%%%%%%%%%%%%%%%%%%%%%%%%%%%%%%%%%%%%

  \subsection{Variables}

\begin{longtable}{|p{.30\textwidth}|p{.62\textwidth}|l}
\cline{1-2}
\cline{1-2} \centering \textbf{Name} & \centering \textbf{Description}& \\
\cline{1-2}
\endhead\cline{1-2}\multicolumn{3}{r}{\small\textit{continued on next page}}\\\endfoot\cline{1-2}
\endlastfoot\raggedright \_\-\_\-d\-o\-c\-\_\-\_\- & \raggedright \textbf{Value:} 
{\tt \texttt{...}}&\\
\cline{1-2}
\end{longtable}

    \index{peach \textit{(package)}!peach.nn \textit{(package)}|)}

%
% API Documentation for Peach - Computational Intelligence for Python
% Module peach.nn.af
%
% Generated by epydoc 3.0beta1
% [Mon Dec 21 08:51:37 2009]
%

%%%%%%%%%%%%%%%%%%%%%%%%%%%%%%%%%%%%%%%%%%%%%%%%%%%%%%%%%%%%%%%%%%%%%%%%%%%
%%                          Module Description                           %%
%%%%%%%%%%%%%%%%%%%%%%%%%%%%%%%%%%%%%%%%%%%%%%%%%%%%%%%%%%%%%%%%%%%%%%%%%%%

    \index{peach \textit{(package)}!peach.nn \textit{(package)}!peach.nn.af \textit{(module)}|(}
\section{Module peach.nn.af}

    \label{peach:nn:af}

Base activation functions and base class

Activation functions define if a neuron is activated or not. There are a lot of
different definitions for activation functions in the literature, and this
sub-package implements some of them. An activation function is defined by its
response and its derivative. Being conveniently defined as classes, it is
possible to define a custom derivative method.

In this package, also, there is a base class that should be subclassed if you
want to define your own activation function. This class, however, can be
instantiated with a standard Python function as an initialization parameter, and
it is adjusted to work with the internals of the package.

If the base class is instantiated, then the function should take a real number
as input, and return a real number. The response of the function determines if
the neuron is activated or not.

%%%%%%%%%%%%%%%%%%%%%%%%%%%%%%%%%%%%%%%%%%%%%%%%%%%%%%%%%%%%%%%%%%%%%%%%%%%
%%                               Variables                               %%
%%%%%%%%%%%%%%%%%%%%%%%%%%%%%%%%%%%%%%%%%%%%%%%%%%%%%%%%%%%%%%%%%%%%%%%%%%%

  \subsection{Variables}

\begin{longtable}{|p{.30\textwidth}|p{.62\textwidth}|l}
\cline{1-2}
\cline{1-2} \centering \textbf{Name} & \centering \textbf{Description}& \\
\cline{1-2}
\endhead\cline{1-2}\multicolumn{3}{r}{\small\textit{continued on next page}}\\\endfoot\cline{1-2}
\endlastfoot\raggedright \_\-\_\-d\-o\-c\-\_\-\_\- & \raggedright \textbf{Value:} 
{\tt \texttt{...}}&\\
\cline{1-2}
\end{longtable}


%%%%%%%%%%%%%%%%%%%%%%%%%%%%%%%%%%%%%%%%%%%%%%%%%%%%%%%%%%%%%%%%%%%%%%%%%%%
%%                           Class Description                           %%
%%%%%%%%%%%%%%%%%%%%%%%%%%%%%%%%%%%%%%%%%%%%%%%%%%%%%%%%%%%%%%%%%%%%%%%%%%%

    \index{peach \textit{(package)}!peach.nn \textit{(package)}!peach.nn.af \textit{(module)}!peach.nn.af.Activation \textit{(class)}|(}
\subsection{Class Activation}

    \label{peach:nn:af:Activation}
\begin{tabular}{cccccc}
% Line for object, linespec=[False]
\multicolumn{2}{r}{\settowidth{\BCL}{object}\multirow{2}{\BCL}{object}}
&&
  \\\cline{3-3}
  &&\multicolumn{1}{c|}{}
&&
  \\
&&\multicolumn{2}{l}{\textbf{peach.nn.af.Activation}}
\end{tabular}

\textbf{Known Subclasses:}
peach.nn.af.ArcTan,
    peach.nn.af.Linear,
    peach.nn.af.Sigmoid,
    peach.nn.af.Ramp,
    peach.nn.af.Signum,
    peach.nn.af.Threshold,
    peach.nn.af.TanH


Base class for activation functions.

This class can be used as base for activation functions. A subclass should
have at least three methods, described below:
\begin{quote}
\begin{description}
%[visit_definition_list_item]
\item[{{\_}{\_}init{\_}{\_}}] %[visit_definition]

This method should be used to configure the function. In general, some
parameters to change the behaviour of a simple function is passed. In a
subclass, the \texttt{{\_}{\_}init{\_}{\_}} method should call the mother class
initialization procedure.

%[depart_definition]
%[depart_definition_list_item]
%[visit_definition_list_item]
\item[{{\_}{\_}call{\_}{\_}}] %[visit_definition]

The \texttt{{\_}{\_}call{\_}{\_}} interface is the function call. It should receive a
\emph{vector} of real numbers and return a \emph{vector} of real numbers. Using
the capabilities of the \texttt{numpy} module will help a lot. In case you
don't know how to use, maybe instantiating this class instead will work
better (see below).

%[depart_definition]
%[depart_definition_list_item]
%[visit_definition_list_item]
\item[{derivative}] %[visit_definition]

This method implements the derivative of the activation function. It is
used in the learning methods. If one is not provided (but remember to
call the superclass \texttt{{\_}{\_}init{\_}{\_}} so that it is created).

%[depart_definition]
%[depart_definition_list_item]
\end{description}
\end{quote}

%%%%%%%%%%%%%%%%%%%%%%%%%%%%%%%%%%%%%%%%%%%%%%%%%%%%%%%%%%%%%%%%%%%%%%%%%%%
%%                                Methods                                %%
%%%%%%%%%%%%%%%%%%%%%%%%%%%%%%%%%%%%%%%%%%%%%%%%%%%%%%%%%%%%%%%%%%%%%%%%%%%

  \subsubsection{Methods}

    \vspace{0.5ex}

    \begin{boxedminipage}{\textwidth}

    \raggedright \textbf{\_\_init\_\_}(\textit{self}, \textit{f}=\texttt{None}, \textit{df}=\texttt{None})

    \vspace{-1.5ex}

    \rule{\textwidth}{0.5\fboxrule}

Initializes the activation function.

Instantiating this class creates and adjusts a standard Python function
to work with layers of neurons.
    \vspace{1ex}

      \textbf{Parameters}
      \begin{quote}
        \begin{Ventry}{xx}

          \item[f]


The activation function. It can be created as a lambda function or
any other method, but it should take a real value, corresponding to
the activation potential of a neuron, and return a real value,
corresponding to its activation. Defaults to \texttt{None}, if none is
given, the identity function is used.
          \item[df]


The derivative of the above function. It can be defined as above, or
not given. If not given, an estimate is calculated based on the
given function. Defaults to \texttt{None}.
        \end{Ventry}

      \end{quote}

    \vspace{1ex}

      Overrides: object.\_\_init\_\_

    \end{boxedminipage}

    \label{peach:nn:af:Activation:__call__}
    \index{peach \textit{(package)}!peach.nn \textit{(package)}!peach.nn.af \textit{(module)}!peach.nn.af.Activation \textit{(class)}!peach.nn.af.Activation.\_\_call\_\_ \textit{(method)}}

    \vspace{0.5ex}

    \begin{boxedminipage}{\textwidth}

    \raggedright \textbf{\_\_call\_\_}(\textit{self}, \textit{x})

    \vspace{-1.5ex}

    \rule{\textwidth}{0.5\fboxrule}

Call interface to the object.

This method applies the activation function over a vector of activation
potentials, and returns the results.
    \vspace{1ex}

      \textbf{Parameters}
      \begin{quote}
        \begin{Ventry}{x}

          \item[x]


A real number or a vector of real numbers representing the
activation potential of a neuron or a layer of neurons.
        \end{Ventry}

      \end{quote}

    \vspace{1ex}

      \textbf{Return Value}
      \begin{quote}

The activation function applied over the input vector.
      \end{quote}

    \vspace{1ex}

    \end{boxedminipage}

    \label{peach:nn:af:Activation:derivative}
    \index{peach \textit{(package)}!peach.nn \textit{(package)}!peach.nn.af \textit{(module)}!peach.nn.af.Activation \textit{(class)}!peach.nn.af.Activation.derivative \textit{(method)}}

    \vspace{0.5ex}

    \begin{boxedminipage}{\textwidth}

    \raggedright \textbf{derivative}(\textit{self}, \textit{x}, \textit{dx}=\texttt{5e-05})

    \vspace{-1.5ex}

    \rule{\textwidth}{0.5\fboxrule}

An estimate of the derivative of the activation function.

This method estimates the derivative using difference equations. This is
a simple estimate, but efficient nonetheless.
    \vspace{1ex}

      \textbf{Parameters}
      \begin{quote}
        \begin{Ventry}{xx}

          \item[x]


A real number or vector of real numbers representing the point over
which the derivative is to be calculated.
          \item[dx]


The value of the interval of the estimate. The smaller this number
is, the better. However, if made too small, the precision is not
enough to avoid errors. This defaults to 5e-5, which is the values
that gives the best results.
        \end{Ventry}

      \end{quote}

    \vspace{1ex}

      \textbf{Return Value}
      \begin{quote}

The value of the derivative over the given point.
      \end{quote}

    \vspace{1ex}

    \end{boxedminipage}

    \label{object:__delattr__}
    \index{object.\_\_delattr\_\_ \textit{(function)}}

    \vspace{0.5ex}

    \begin{boxedminipage}{\textwidth}

    \raggedright \textbf{\_\_delattr\_\_}(\textit{...})

    \vspace{-1.5ex}

    \rule{\textwidth}{0.5\fboxrule}

x.{\_}{\_}delattr{\_}{\_}('name') {\textless}=={\textgreater} del x.name
    \vspace{1ex}

    \end{boxedminipage}

    \label{object:__getattribute__}
    \index{object.\_\_getattribute\_\_ \textit{(function)}}

    \vspace{0.5ex}

    \begin{boxedminipage}{\textwidth}

    \raggedright \textbf{\_\_getattribute\_\_}(\textit{...})

    \vspace{-1.5ex}

    \rule{\textwidth}{0.5\fboxrule}

x.{\_}{\_}getattribute{\_}{\_}('name') {\textless}=={\textgreater} x.name
    \vspace{1ex}

    \end{boxedminipage}

    \label{object:__hash__}
    \index{object.\_\_hash\_\_ \textit{(function)}}

    \vspace{0.5ex}

    \begin{boxedminipage}{\textwidth}

    \raggedright \textbf{\_\_hash\_\_}(\textit{x})

    \vspace{-1.5ex}

    \rule{\textwidth}{0.5\fboxrule}

hash(x)
    \vspace{1ex}

    \end{boxedminipage}

    \label{object:__new__}
    \index{object.\_\_new\_\_ \textit{(function)}}

    \vspace{0.5ex}

    \begin{boxedminipage}{\textwidth}

    \raggedright \textbf{\_\_new\_\_}(\textit{T}, \textit{S}, \textit{...})

      \textbf{Return Value}
      \begin{quote}
\begin{alltt}
a new object with type S, a subtype of T
\end{alltt}

      \end{quote}

    \vspace{1ex}

    \end{boxedminipage}

    \label{object:__reduce__}
    \index{object.\_\_reduce\_\_ \textit{(function)}}

    \vspace{0.5ex}

    \begin{boxedminipage}{\textwidth}

    \raggedright \textbf{\_\_reduce\_\_}(\textit{...})

    \vspace{-1.5ex}

    \rule{\textwidth}{0.5\fboxrule}

helper for pickle
    \vspace{1ex}

    \end{boxedminipage}

    \label{object:__reduce_ex__}
    \index{object.\_\_reduce\_ex\_\_ \textit{(function)}}

    \vspace{0.5ex}

    \begin{boxedminipage}{\textwidth}

    \raggedright \textbf{\_\_reduce\_ex\_\_}(\textit{...})

    \vspace{-1.5ex}

    \rule{\textwidth}{0.5\fboxrule}

helper for pickle
    \vspace{1ex}

    \end{boxedminipage}

    \label{object:__repr__}
    \index{object.\_\_repr\_\_ \textit{(function)}}

    \vspace{0.5ex}

    \begin{boxedminipage}{\textwidth}

    \raggedright \textbf{\_\_repr\_\_}(\textit{x})

    \vspace{-1.5ex}

    \rule{\textwidth}{0.5\fboxrule}

repr(x)
    \vspace{1ex}

    \end{boxedminipage}

    \label{object:__setattr__}
    \index{object.\_\_setattr\_\_ \textit{(function)}}

    \vspace{0.5ex}

    \begin{boxedminipage}{\textwidth}

    \raggedright \textbf{\_\_setattr\_\_}(\textit{...})

    \vspace{-1.5ex}

    \rule{\textwidth}{0.5\fboxrule}

x.{\_}{\_}setattr{\_}{\_}('name', value) {\textless}=={\textgreater} x.name = value
    \vspace{1ex}

    \end{boxedminipage}

    \label{object:__str__}
    \index{object.\_\_str\_\_ \textit{(function)}}

    \vspace{0.5ex}

    \begin{boxedminipage}{\textwidth}

    \raggedright \textbf{\_\_str\_\_}(\textit{x})

    \vspace{-1.5ex}

    \rule{\textwidth}{0.5\fboxrule}

str(x)
    \vspace{1ex}

    \end{boxedminipage}


%%%%%%%%%%%%%%%%%%%%%%%%%%%%%%%%%%%%%%%%%%%%%%%%%%%%%%%%%%%%%%%%%%%%%%%%%%%
%%                              Properties                               %%
%%%%%%%%%%%%%%%%%%%%%%%%%%%%%%%%%%%%%%%%%%%%%%%%%%%%%%%%%%%%%%%%%%%%%%%%%%%

  \subsubsection{Properties}

\begin{longtable}{|p{.30\textwidth}|p{.62\textwidth}|l}
\cline{1-2}
\cline{1-2} \centering \textbf{Name} & \centering \textbf{Description}& \\
\cline{1-2}
\endhead\cline{1-2}\multicolumn{3}{r}{\small\textit{continued on next page}}\\\endfoot\cline{1-2}
\endlastfoot\raggedright \_\-\_\-c\-l\-a\-s\-s\-\_\-\_\- & \raggedright \textbf{Value:} 
{\tt {\textless}attribute '\_\_class\_\_' of 'object' objects{\textgreater}}&\\
\cline{1-2}
\end{longtable}


%%%%%%%%%%%%%%%%%%%%%%%%%%%%%%%%%%%%%%%%%%%%%%%%%%%%%%%%%%%%%%%%%%%%%%%%%%%
%%                          Instance Variables                           %%
%%%%%%%%%%%%%%%%%%%%%%%%%%%%%%%%%%%%%%%%%%%%%%%%%%%%%%%%%%%%%%%%%%%%%%%%%%%

  \subsubsection{Instance Variables}

\begin{longtable}{|p{.30\textwidth}|p{.62\textwidth}|l}
\cline{1-2}
\cline{1-2} \centering \textbf{Name} & \centering \textbf{Description}& \\
\cline{1-2}
\endhead\cline{1-2}\multicolumn{3}{r}{\small\textit{continued on next page}}\\\endfoot\cline{1-2}
\endlastfoot\raggedright d\- & An alias to the derivative of the function.&\\
\cline{1-2}
\end{longtable}

    \index{peach \textit{(package)}!peach.nn \textit{(package)}!peach.nn.af \textit{(module)}!peach.nn.af.Activation \textit{(class)}|)}

%%%%%%%%%%%%%%%%%%%%%%%%%%%%%%%%%%%%%%%%%%%%%%%%%%%%%%%%%%%%%%%%%%%%%%%%%%%
%%                           Class Description                           %%
%%%%%%%%%%%%%%%%%%%%%%%%%%%%%%%%%%%%%%%%%%%%%%%%%%%%%%%%%%%%%%%%%%%%%%%%%%%

    \index{peach \textit{(package)}!peach.nn \textit{(package)}!peach.nn.af \textit{(module)}!peach.nn.af.Threshold \textit{(class)}|(}
\subsection{Class Threshold}

    \label{peach:nn:af:Threshold}
\begin{tabular}{cccccccc}
% Line for object, linespec=[False, False]
\multicolumn{2}{r}{\settowidth{\BCL}{object}\multirow{2}{\BCL}{object}}
&&
&&
  \\\cline{3-3}
  &&\multicolumn{1}{c|}{}
&&
&&
  \\
% Line for peach.nn.af.Activation, linespec=[False]
\multicolumn{4}{r}{\settowidth{\BCL}{peach.nn.af.Activation}\multirow{2}{\BCL}{peach.nn.af.Activation}}
&&
  \\\cline{5-5}
  &&&&\multicolumn{1}{c|}{}
&&
  \\
&&&&\multicolumn{2}{l}{\textbf{peach.nn.af.Threshold}}
\end{tabular}


Threshold activation function.

%%%%%%%%%%%%%%%%%%%%%%%%%%%%%%%%%%%%%%%%%%%%%%%%%%%%%%%%%%%%%%%%%%%%%%%%%%%
%%                                Methods                                %%
%%%%%%%%%%%%%%%%%%%%%%%%%%%%%%%%%%%%%%%%%%%%%%%%%%%%%%%%%%%%%%%%%%%%%%%%%%%

  \subsubsection{Methods}

    \vspace{0.5ex}

    \begin{boxedminipage}{\textwidth}

    \raggedright \textbf{\_\_init\_\_}(\textit{self}, \textit{threshold}=\texttt{0.0}, \textit{amplitude}=\texttt{1.0})

    \vspace{-1.5ex}

    \rule{\textwidth}{0.5\fboxrule}

Initializes the object.
    \vspace{1ex}

      \textbf{Parameters}
      \begin{quote}
        \begin{Ventry}{xxxxxxxxx}

          \item[threshold]


The threshold value. If the value of the input is lower than this,
the function is 0, otherwise, it is the given \texttt{amplitude}.
          \item[amplitude]


The maximum value of the function.
        \end{Ventry}

      \end{quote}

    \vspace{1ex}

      Overrides: peach.nn.af.Activation.\_\_init\_\_

    \end{boxedminipage}

    \vspace{0.5ex}

    \begin{boxedminipage}{\textwidth}

    \raggedright \textbf{\_\_call\_\_}(\textit{self}, \textit{x})

    \vspace{-1.5ex}

    \rule{\textwidth}{0.5\fboxrule}

Call interface to the object.

This method applies the activation function over a vector of activation
potentials, and returns the results.
    \vspace{1ex}

      \textbf{Parameters}
      \begin{quote}
        \begin{Ventry}{x}

          \item[x]


A real number or a vector of real numbers representing the
activation potential of a neuron or a layer of neurons.
        \end{Ventry}

      \end{quote}

    \vspace{1ex}

      \textbf{Return Value}
      \begin{quote}

The activation function applied over the input vector.
      \end{quote}

    \vspace{1ex}

      Overrides: peach.nn.af.Activation.\_\_call\_\_

    \end{boxedminipage}

    \vspace{0.5ex}

    \begin{boxedminipage}{\textwidth}

    \raggedright \textbf{derivative}(\textit{self}, \textit{x})

    \vspace{-1.5ex}

    \rule{\textwidth}{0.5\fboxrule}

The function derivative. Technically, this function doesn't have a
derivative, but making it equals to 1, this can be used in learning
algorithms.
    \vspace{1ex}

      \textbf{Parameters}
      \begin{quote}
        \begin{Ventry}{x}

          \item[x]


A real number or a vector of real numbers representing the
activation potential of a neuron or a layer of neurons.
        \end{Ventry}

      \end{quote}

    \vspace{1ex}

      \textbf{Return Value}
      \begin{quote}

The derivative of the activation function applied over the input
vector.
      \end{quote}

    \vspace{1ex}

      Overrides: peach.nn.af.Activation.derivative

    \end{boxedminipage}

    \label{object:__delattr__}
    \index{object.\_\_delattr\_\_ \textit{(function)}}

    \vspace{0.5ex}

    \begin{boxedminipage}{\textwidth}

    \raggedright \textbf{\_\_delattr\_\_}(\textit{...})

    \vspace{-1.5ex}

    \rule{\textwidth}{0.5\fboxrule}

x.{\_}{\_}delattr{\_}{\_}('name') {\textless}=={\textgreater} del x.name
    \vspace{1ex}

    \end{boxedminipage}

    \label{object:__getattribute__}
    \index{object.\_\_getattribute\_\_ \textit{(function)}}

    \vspace{0.5ex}

    \begin{boxedminipage}{\textwidth}

    \raggedright \textbf{\_\_getattribute\_\_}(\textit{...})

    \vspace{-1.5ex}

    \rule{\textwidth}{0.5\fboxrule}

x.{\_}{\_}getattribute{\_}{\_}('name') {\textless}=={\textgreater} x.name
    \vspace{1ex}

    \end{boxedminipage}

    \label{object:__hash__}
    \index{object.\_\_hash\_\_ \textit{(function)}}

    \vspace{0.5ex}

    \begin{boxedminipage}{\textwidth}

    \raggedright \textbf{\_\_hash\_\_}(\textit{x})

    \vspace{-1.5ex}

    \rule{\textwidth}{0.5\fboxrule}

hash(x)
    \vspace{1ex}

    \end{boxedminipage}

    \label{object:__new__}
    \index{object.\_\_new\_\_ \textit{(function)}}

    \vspace{0.5ex}

    \begin{boxedminipage}{\textwidth}

    \raggedright \textbf{\_\_new\_\_}(\textit{T}, \textit{S}, \textit{...})

      \textbf{Return Value}
      \begin{quote}
\begin{alltt}
a new object with type S, a subtype of T
\end{alltt}

      \end{quote}

    \vspace{1ex}

    \end{boxedminipage}

    \label{object:__reduce__}
    \index{object.\_\_reduce\_\_ \textit{(function)}}

    \vspace{0.5ex}

    \begin{boxedminipage}{\textwidth}

    \raggedright \textbf{\_\_reduce\_\_}(\textit{...})

    \vspace{-1.5ex}

    \rule{\textwidth}{0.5\fboxrule}

helper for pickle
    \vspace{1ex}

    \end{boxedminipage}

    \label{object:__reduce_ex__}
    \index{object.\_\_reduce\_ex\_\_ \textit{(function)}}

    \vspace{0.5ex}

    \begin{boxedminipage}{\textwidth}

    \raggedright \textbf{\_\_reduce\_ex\_\_}(\textit{...})

    \vspace{-1.5ex}

    \rule{\textwidth}{0.5\fboxrule}

helper for pickle
    \vspace{1ex}

    \end{boxedminipage}

    \label{object:__repr__}
    \index{object.\_\_repr\_\_ \textit{(function)}}

    \vspace{0.5ex}

    \begin{boxedminipage}{\textwidth}

    \raggedright \textbf{\_\_repr\_\_}(\textit{x})

    \vspace{-1.5ex}

    \rule{\textwidth}{0.5\fboxrule}

repr(x)
    \vspace{1ex}

    \end{boxedminipage}

    \label{object:__setattr__}
    \index{object.\_\_setattr\_\_ \textit{(function)}}

    \vspace{0.5ex}

    \begin{boxedminipage}{\textwidth}

    \raggedright \textbf{\_\_setattr\_\_}(\textit{...})

    \vspace{-1.5ex}

    \rule{\textwidth}{0.5\fboxrule}

x.{\_}{\_}setattr{\_}{\_}('name', value) {\textless}=={\textgreater} x.name = value
    \vspace{1ex}

    \end{boxedminipage}

    \label{object:__str__}
    \index{object.\_\_str\_\_ \textit{(function)}}

    \vspace{0.5ex}

    \begin{boxedminipage}{\textwidth}

    \raggedright \textbf{\_\_str\_\_}(\textit{x})

    \vspace{-1.5ex}

    \rule{\textwidth}{0.5\fboxrule}

str(x)
    \vspace{1ex}

    \end{boxedminipage}


%%%%%%%%%%%%%%%%%%%%%%%%%%%%%%%%%%%%%%%%%%%%%%%%%%%%%%%%%%%%%%%%%%%%%%%%%%%
%%                              Properties                               %%
%%%%%%%%%%%%%%%%%%%%%%%%%%%%%%%%%%%%%%%%%%%%%%%%%%%%%%%%%%%%%%%%%%%%%%%%%%%

  \subsubsection{Properties}

\begin{longtable}{|p{.30\textwidth}|p{.62\textwidth}|l}
\cline{1-2}
\cline{1-2} \centering \textbf{Name} & \centering \textbf{Description}& \\
\cline{1-2}
\endhead\cline{1-2}\multicolumn{3}{r}{\small\textit{continued on next page}}\\\endfoot\cline{1-2}
\endlastfoot\raggedright \_\-\_\-c\-l\-a\-s\-s\-\_\-\_\- & \raggedright \textbf{Value:} 
{\tt {\textless}attribute '\_\_class\_\_' of 'object' objects{\textgreater}}&\\
\cline{1-2}
\end{longtable}


%%%%%%%%%%%%%%%%%%%%%%%%%%%%%%%%%%%%%%%%%%%%%%%%%%%%%%%%%%%%%%%%%%%%%%%%%%%
%%                          Instance Variables                           %%
%%%%%%%%%%%%%%%%%%%%%%%%%%%%%%%%%%%%%%%%%%%%%%%%%%%%%%%%%%%%%%%%%%%%%%%%%%%

  \subsubsection{Instance Variables}

\begin{longtable}{|p{.30\textwidth}|p{.62\textwidth}|l}
\cline{1-2}
\cline{1-2} \centering \textbf{Name} & \centering \textbf{Description}& \\
\cline{1-2}
\endhead\cline{1-2}\multicolumn{3}{r}{\small\textit{continued on next page}}\\\endfoot\cline{1-2}
\endlastfoot\raggedright d\- & An alias to the derivative of the function.&\\
\cline{1-2}
\end{longtable}

    \index{peach \textit{(package)}!peach.nn \textit{(package)}!peach.nn.af \textit{(module)}!peach.nn.af.Threshold \textit{(class)}|)}

%%%%%%%%%%%%%%%%%%%%%%%%%%%%%%%%%%%%%%%%%%%%%%%%%%%%%%%%%%%%%%%%%%%%%%%%%%%
%%                           Class Description                           %%
%%%%%%%%%%%%%%%%%%%%%%%%%%%%%%%%%%%%%%%%%%%%%%%%%%%%%%%%%%%%%%%%%%%%%%%%%%%

    \index{peach \textit{(package)}!peach.nn \textit{(package)}!peach.nn.af \textit{(module)}!peach.nn.af.Threshold \textit{(class)}|(}
\subsection{Class Threshold}

    \label{peach:nn:af:Threshold}
\begin{tabular}{cccccccc}
% Line for object, linespec=[False, False]
\multicolumn{2}{r}{\settowidth{\BCL}{object}\multirow{2}{\BCL}{object}}
&&
&&
  \\\cline{3-3}
  &&\multicolumn{1}{c|}{}
&&
&&
  \\
% Line for peach.nn.af.Activation, linespec=[False]
\multicolumn{4}{r}{\settowidth{\BCL}{peach.nn.af.Activation}\multirow{2}{\BCL}{peach.nn.af.Activation}}
&&
  \\\cline{5-5}
  &&&&\multicolumn{1}{c|}{}
&&
  \\
&&&&\multicolumn{2}{l}{\textbf{peach.nn.af.Threshold}}
\end{tabular}


Threshold activation function.

%%%%%%%%%%%%%%%%%%%%%%%%%%%%%%%%%%%%%%%%%%%%%%%%%%%%%%%%%%%%%%%%%%%%%%%%%%%
%%                                Methods                                %%
%%%%%%%%%%%%%%%%%%%%%%%%%%%%%%%%%%%%%%%%%%%%%%%%%%%%%%%%%%%%%%%%%%%%%%%%%%%

  \subsubsection{Methods}

    \vspace{0.5ex}

    \begin{boxedminipage}{\textwidth}

    \raggedright \textbf{\_\_init\_\_}(\textit{self}, \textit{threshold}=\texttt{0.0}, \textit{amplitude}=\texttt{1.0})

    \vspace{-1.5ex}

    \rule{\textwidth}{0.5\fboxrule}

Initializes the object.
    \vspace{1ex}

      \textbf{Parameters}
      \begin{quote}
        \begin{Ventry}{xxxxxxxxx}

          \item[threshold]


The threshold value. If the value of the input is lower than this,
the function is 0, otherwise, it is the given \texttt{amplitude}.
          \item[amplitude]


The maximum value of the function.
        \end{Ventry}

      \end{quote}

    \vspace{1ex}

      Overrides: peach.nn.af.Activation.\_\_init\_\_

    \end{boxedminipage}

    \vspace{0.5ex}

    \begin{boxedminipage}{\textwidth}

    \raggedright \textbf{\_\_call\_\_}(\textit{self}, \textit{x})

    \vspace{-1.5ex}

    \rule{\textwidth}{0.5\fboxrule}

Call interface to the object.

This method applies the activation function over a vector of activation
potentials, and returns the results.
    \vspace{1ex}

      \textbf{Parameters}
      \begin{quote}
        \begin{Ventry}{x}

          \item[x]


A real number or a vector of real numbers representing the
activation potential of a neuron or a layer of neurons.
        \end{Ventry}

      \end{quote}

    \vspace{1ex}

      \textbf{Return Value}
      \begin{quote}

The activation function applied over the input vector.
      \end{quote}

    \vspace{1ex}

      Overrides: peach.nn.af.Activation.\_\_call\_\_

    \end{boxedminipage}

    \vspace{0.5ex}

    \begin{boxedminipage}{\textwidth}

    \raggedright \textbf{derivative}(\textit{self}, \textit{x})

    \vspace{-1.5ex}

    \rule{\textwidth}{0.5\fboxrule}

The function derivative. Technically, this function doesn't have a
derivative, but making it equals to 1, this can be used in learning
algorithms.
    \vspace{1ex}

      \textbf{Parameters}
      \begin{quote}
        \begin{Ventry}{x}

          \item[x]


A real number or a vector of real numbers representing the
activation potential of a neuron or a layer of neurons.
        \end{Ventry}

      \end{quote}

    \vspace{1ex}

      \textbf{Return Value}
      \begin{quote}

The derivative of the activation function applied over the input
vector.
      \end{quote}

    \vspace{1ex}

      Overrides: peach.nn.af.Activation.derivative

    \end{boxedminipage}

    \label{object:__delattr__}
    \index{object.\_\_delattr\_\_ \textit{(function)}}

    \vspace{0.5ex}

    \begin{boxedminipage}{\textwidth}

    \raggedright \textbf{\_\_delattr\_\_}(\textit{...})

    \vspace{-1.5ex}

    \rule{\textwidth}{0.5\fboxrule}

x.{\_}{\_}delattr{\_}{\_}('name') {\textless}=={\textgreater} del x.name
    \vspace{1ex}

    \end{boxedminipage}

    \label{object:__getattribute__}
    \index{object.\_\_getattribute\_\_ \textit{(function)}}

    \vspace{0.5ex}

    \begin{boxedminipage}{\textwidth}

    \raggedright \textbf{\_\_getattribute\_\_}(\textit{...})

    \vspace{-1.5ex}

    \rule{\textwidth}{0.5\fboxrule}

x.{\_}{\_}getattribute{\_}{\_}('name') {\textless}=={\textgreater} x.name
    \vspace{1ex}

    \end{boxedminipage}

    \label{object:__hash__}
    \index{object.\_\_hash\_\_ \textit{(function)}}

    \vspace{0.5ex}

    \begin{boxedminipage}{\textwidth}

    \raggedright \textbf{\_\_hash\_\_}(\textit{x})

    \vspace{-1.5ex}

    \rule{\textwidth}{0.5\fboxrule}

hash(x)
    \vspace{1ex}

    \end{boxedminipage}

    \label{object:__new__}
    \index{object.\_\_new\_\_ \textit{(function)}}

    \vspace{0.5ex}

    \begin{boxedminipage}{\textwidth}

    \raggedright \textbf{\_\_new\_\_}(\textit{T}, \textit{S}, \textit{...})

      \textbf{Return Value}
      \begin{quote}
\begin{alltt}
a new object with type S, a subtype of T
\end{alltt}

      \end{quote}

    \vspace{1ex}

    \end{boxedminipage}

    \label{object:__reduce__}
    \index{object.\_\_reduce\_\_ \textit{(function)}}

    \vspace{0.5ex}

    \begin{boxedminipage}{\textwidth}

    \raggedright \textbf{\_\_reduce\_\_}(\textit{...})

    \vspace{-1.5ex}

    \rule{\textwidth}{0.5\fboxrule}

helper for pickle
    \vspace{1ex}

    \end{boxedminipage}

    \label{object:__reduce_ex__}
    \index{object.\_\_reduce\_ex\_\_ \textit{(function)}}

    \vspace{0.5ex}

    \begin{boxedminipage}{\textwidth}

    \raggedright \textbf{\_\_reduce\_ex\_\_}(\textit{...})

    \vspace{-1.5ex}

    \rule{\textwidth}{0.5\fboxrule}

helper for pickle
    \vspace{1ex}

    \end{boxedminipage}

    \label{object:__repr__}
    \index{object.\_\_repr\_\_ \textit{(function)}}

    \vspace{0.5ex}

    \begin{boxedminipage}{\textwidth}

    \raggedright \textbf{\_\_repr\_\_}(\textit{x})

    \vspace{-1.5ex}

    \rule{\textwidth}{0.5\fboxrule}

repr(x)
    \vspace{1ex}

    \end{boxedminipage}

    \label{object:__setattr__}
    \index{object.\_\_setattr\_\_ \textit{(function)}}

    \vspace{0.5ex}

    \begin{boxedminipage}{\textwidth}

    \raggedright \textbf{\_\_setattr\_\_}(\textit{...})

    \vspace{-1.5ex}

    \rule{\textwidth}{0.5\fboxrule}

x.{\_}{\_}setattr{\_}{\_}('name', value) {\textless}=={\textgreater} x.name = value
    \vspace{1ex}

    \end{boxedminipage}

    \label{object:__str__}
    \index{object.\_\_str\_\_ \textit{(function)}}

    \vspace{0.5ex}

    \begin{boxedminipage}{\textwidth}

    \raggedright \textbf{\_\_str\_\_}(\textit{x})

    \vspace{-1.5ex}

    \rule{\textwidth}{0.5\fboxrule}

str(x)
    \vspace{1ex}

    \end{boxedminipage}


%%%%%%%%%%%%%%%%%%%%%%%%%%%%%%%%%%%%%%%%%%%%%%%%%%%%%%%%%%%%%%%%%%%%%%%%%%%
%%                              Properties                               %%
%%%%%%%%%%%%%%%%%%%%%%%%%%%%%%%%%%%%%%%%%%%%%%%%%%%%%%%%%%%%%%%%%%%%%%%%%%%

  \subsubsection{Properties}

\begin{longtable}{|p{.30\textwidth}|p{.62\textwidth}|l}
\cline{1-2}
\cline{1-2} \centering \textbf{Name} & \centering \textbf{Description}& \\
\cline{1-2}
\endhead\cline{1-2}\multicolumn{3}{r}{\small\textit{continued on next page}}\\\endfoot\cline{1-2}
\endlastfoot\raggedright \_\-\_\-c\-l\-a\-s\-s\-\_\-\_\- & \raggedright \textbf{Value:} 
{\tt {\textless}attribute '\_\_class\_\_' of 'object' objects{\textgreater}}&\\
\cline{1-2}
\end{longtable}


%%%%%%%%%%%%%%%%%%%%%%%%%%%%%%%%%%%%%%%%%%%%%%%%%%%%%%%%%%%%%%%%%%%%%%%%%%%
%%                          Instance Variables                           %%
%%%%%%%%%%%%%%%%%%%%%%%%%%%%%%%%%%%%%%%%%%%%%%%%%%%%%%%%%%%%%%%%%%%%%%%%%%%

  \subsubsection{Instance Variables}

\begin{longtable}{|p{.30\textwidth}|p{.62\textwidth}|l}
\cline{1-2}
\cline{1-2} \centering \textbf{Name} & \centering \textbf{Description}& \\
\cline{1-2}
\endhead\cline{1-2}\multicolumn{3}{r}{\small\textit{continued on next page}}\\\endfoot\cline{1-2}
\endlastfoot\raggedright d\- & An alias to the derivative of the function.&\\
\cline{1-2}
\end{longtable}

    \index{peach \textit{(package)}!peach.nn \textit{(package)}!peach.nn.af \textit{(module)}!peach.nn.af.Threshold \textit{(class)}|)}

%%%%%%%%%%%%%%%%%%%%%%%%%%%%%%%%%%%%%%%%%%%%%%%%%%%%%%%%%%%%%%%%%%%%%%%%%%%
%%                           Class Description                           %%
%%%%%%%%%%%%%%%%%%%%%%%%%%%%%%%%%%%%%%%%%%%%%%%%%%%%%%%%%%%%%%%%%%%%%%%%%%%

    \index{peach \textit{(package)}!peach.nn \textit{(package)}!peach.nn.af \textit{(module)}!peach.nn.af.Linear \textit{(class)}|(}
\subsection{Class Linear}

    \label{peach:nn:af:Linear}
\begin{tabular}{cccccccc}
% Line for object, linespec=[False, False]
\multicolumn{2}{r}{\settowidth{\BCL}{object}\multirow{2}{\BCL}{object}}
&&
&&
  \\\cline{3-3}
  &&\multicolumn{1}{c|}{}
&&
&&
  \\
% Line for peach.nn.af.Activation, linespec=[False]
\multicolumn{4}{r}{\settowidth{\BCL}{peach.nn.af.Activation}\multirow{2}{\BCL}{peach.nn.af.Activation}}
&&
  \\\cline{5-5}
  &&&&\multicolumn{1}{c|}{}
&&
  \\
&&&&\multicolumn{2}{l}{\textbf{peach.nn.af.Linear}}
\end{tabular}


Identity activation function

%%%%%%%%%%%%%%%%%%%%%%%%%%%%%%%%%%%%%%%%%%%%%%%%%%%%%%%%%%%%%%%%%%%%%%%%%%%
%%                                Methods                                %%
%%%%%%%%%%%%%%%%%%%%%%%%%%%%%%%%%%%%%%%%%%%%%%%%%%%%%%%%%%%%%%%%%%%%%%%%%%%

  \subsubsection{Methods}

    \vspace{0.5ex}

    \begin{boxedminipage}{\textwidth}

    \raggedright \textbf{\_\_init\_\_}(\textit{self})

    \vspace{-1.5ex}

    \rule{\textwidth}{0.5\fboxrule}

Initializes the function
    \vspace{1ex}

      Overrides: peach.nn.af.Activation.\_\_init\_\_

    \end{boxedminipage}

    \vspace{0.5ex}

    \begin{boxedminipage}{\textwidth}

    \raggedright \textbf{\_\_call\_\_}(\textit{self}, \textit{x})

    \vspace{-1.5ex}

    \rule{\textwidth}{0.5\fboxrule}

Call interface to the object.

This method applies the activation function over a vector of activation
potentials, and returns the results.
    \vspace{1ex}

      \textbf{Parameters}
      \begin{quote}
        \begin{Ventry}{x}

          \item[x]


A real number or a vector of real numbers representing the
activation potential of a neuron or a layer of neurons.
        \end{Ventry}

      \end{quote}

    \vspace{1ex}

      \textbf{Return Value}
      \begin{quote}

The activation function applied over the input vector.
      \end{quote}

    \vspace{1ex}

      Overrides: peach.nn.af.Activation.\_\_call\_\_

    \end{boxedminipage}

    \vspace{0.5ex}

    \begin{boxedminipage}{\textwidth}

    \raggedright \textbf{derivative}(\textit{self}, \textit{x})

    \vspace{-1.5ex}

    \rule{\textwidth}{0.5\fboxrule}

The function derivative.
    \vspace{1ex}

      \textbf{Parameters}
      \begin{quote}
        \begin{Ventry}{x}

          \item[x]


A real number or a vector of real numbers representing the
activation potential of a neuron or a layer of neurons.
        \end{Ventry}

      \end{quote}

    \vspace{1ex}

      \textbf{Return Value}
      \begin{quote}

The derivative of the activation function applied over the input
vector.
      \end{quote}

    \vspace{1ex}

      Overrides: peach.nn.af.Activation.derivative

    \end{boxedminipage}

    \label{object:__delattr__}
    \index{object.\_\_delattr\_\_ \textit{(function)}}

    \vspace{0.5ex}

    \begin{boxedminipage}{\textwidth}

    \raggedright \textbf{\_\_delattr\_\_}(\textit{...})

    \vspace{-1.5ex}

    \rule{\textwidth}{0.5\fboxrule}

x.{\_}{\_}delattr{\_}{\_}('name') {\textless}=={\textgreater} del x.name
    \vspace{1ex}

    \end{boxedminipage}

    \label{object:__getattribute__}
    \index{object.\_\_getattribute\_\_ \textit{(function)}}

    \vspace{0.5ex}

    \begin{boxedminipage}{\textwidth}

    \raggedright \textbf{\_\_getattribute\_\_}(\textit{...})

    \vspace{-1.5ex}

    \rule{\textwidth}{0.5\fboxrule}

x.{\_}{\_}getattribute{\_}{\_}('name') {\textless}=={\textgreater} x.name
    \vspace{1ex}

    \end{boxedminipage}

    \label{object:__hash__}
    \index{object.\_\_hash\_\_ \textit{(function)}}

    \vspace{0.5ex}

    \begin{boxedminipage}{\textwidth}

    \raggedright \textbf{\_\_hash\_\_}(\textit{x})

    \vspace{-1.5ex}

    \rule{\textwidth}{0.5\fboxrule}

hash(x)
    \vspace{1ex}

    \end{boxedminipage}

    \label{object:__new__}
    \index{object.\_\_new\_\_ \textit{(function)}}

    \vspace{0.5ex}

    \begin{boxedminipage}{\textwidth}

    \raggedright \textbf{\_\_new\_\_}(\textit{T}, \textit{S}, \textit{...})

      \textbf{Return Value}
      \begin{quote}
\begin{alltt}
a new object with type S, a subtype of T
\end{alltt}

      \end{quote}

    \vspace{1ex}

    \end{boxedminipage}

    \label{object:__reduce__}
    \index{object.\_\_reduce\_\_ \textit{(function)}}

    \vspace{0.5ex}

    \begin{boxedminipage}{\textwidth}

    \raggedright \textbf{\_\_reduce\_\_}(\textit{...})

    \vspace{-1.5ex}

    \rule{\textwidth}{0.5\fboxrule}

helper for pickle
    \vspace{1ex}

    \end{boxedminipage}

    \label{object:__reduce_ex__}
    \index{object.\_\_reduce\_ex\_\_ \textit{(function)}}

    \vspace{0.5ex}

    \begin{boxedminipage}{\textwidth}

    \raggedright \textbf{\_\_reduce\_ex\_\_}(\textit{...})

    \vspace{-1.5ex}

    \rule{\textwidth}{0.5\fboxrule}

helper for pickle
    \vspace{1ex}

    \end{boxedminipage}

    \label{object:__repr__}
    \index{object.\_\_repr\_\_ \textit{(function)}}

    \vspace{0.5ex}

    \begin{boxedminipage}{\textwidth}

    \raggedright \textbf{\_\_repr\_\_}(\textit{x})

    \vspace{-1.5ex}

    \rule{\textwidth}{0.5\fboxrule}

repr(x)
    \vspace{1ex}

    \end{boxedminipage}

    \label{object:__setattr__}
    \index{object.\_\_setattr\_\_ \textit{(function)}}

    \vspace{0.5ex}

    \begin{boxedminipage}{\textwidth}

    \raggedright \textbf{\_\_setattr\_\_}(\textit{...})

    \vspace{-1.5ex}

    \rule{\textwidth}{0.5\fboxrule}

x.{\_}{\_}setattr{\_}{\_}('name', value) {\textless}=={\textgreater} x.name = value
    \vspace{1ex}

    \end{boxedminipage}

    \label{object:__str__}
    \index{object.\_\_str\_\_ \textit{(function)}}

    \vspace{0.5ex}

    \begin{boxedminipage}{\textwidth}

    \raggedright \textbf{\_\_str\_\_}(\textit{x})

    \vspace{-1.5ex}

    \rule{\textwidth}{0.5\fboxrule}

str(x)
    \vspace{1ex}

    \end{boxedminipage}


%%%%%%%%%%%%%%%%%%%%%%%%%%%%%%%%%%%%%%%%%%%%%%%%%%%%%%%%%%%%%%%%%%%%%%%%%%%
%%                              Properties                               %%
%%%%%%%%%%%%%%%%%%%%%%%%%%%%%%%%%%%%%%%%%%%%%%%%%%%%%%%%%%%%%%%%%%%%%%%%%%%

  \subsubsection{Properties}

\begin{longtable}{|p{.30\textwidth}|p{.62\textwidth}|l}
\cline{1-2}
\cline{1-2} \centering \textbf{Name} & \centering \textbf{Description}& \\
\cline{1-2}
\endhead\cline{1-2}\multicolumn{3}{r}{\small\textit{continued on next page}}\\\endfoot\cline{1-2}
\endlastfoot\raggedright \_\-\_\-c\-l\-a\-s\-s\-\_\-\_\- & \raggedright \textbf{Value:} 
{\tt {\textless}attribute '\_\_class\_\_' of 'object' objects{\textgreater}}&\\
\cline{1-2}
\end{longtable}


%%%%%%%%%%%%%%%%%%%%%%%%%%%%%%%%%%%%%%%%%%%%%%%%%%%%%%%%%%%%%%%%%%%%%%%%%%%
%%                          Instance Variables                           %%
%%%%%%%%%%%%%%%%%%%%%%%%%%%%%%%%%%%%%%%%%%%%%%%%%%%%%%%%%%%%%%%%%%%%%%%%%%%

  \subsubsection{Instance Variables}

\begin{longtable}{|p{.30\textwidth}|p{.62\textwidth}|l}
\cline{1-2}
\cline{1-2} \centering \textbf{Name} & \centering \textbf{Description}& \\
\cline{1-2}
\endhead\cline{1-2}\multicolumn{3}{r}{\small\textit{continued on next page}}\\\endfoot\cline{1-2}
\endlastfoot\raggedright d\- & An alias to the derivative of the function.&\\
\cline{1-2}
\end{longtable}

    \index{peach \textit{(package)}!peach.nn \textit{(package)}!peach.nn.af \textit{(module)}!peach.nn.af.Linear \textit{(class)}|)}

%%%%%%%%%%%%%%%%%%%%%%%%%%%%%%%%%%%%%%%%%%%%%%%%%%%%%%%%%%%%%%%%%%%%%%%%%%%
%%                           Class Description                           %%
%%%%%%%%%%%%%%%%%%%%%%%%%%%%%%%%%%%%%%%%%%%%%%%%%%%%%%%%%%%%%%%%%%%%%%%%%%%

    \index{peach \textit{(package)}!peach.nn \textit{(package)}!peach.nn.af \textit{(module)}!peach.nn.af.Linear \textit{(class)}|(}
\subsection{Class Linear}

    \label{peach:nn:af:Linear}
\begin{tabular}{cccccccc}
% Line for object, linespec=[False, False]
\multicolumn{2}{r}{\settowidth{\BCL}{object}\multirow{2}{\BCL}{object}}
&&
&&
  \\\cline{3-3}
  &&\multicolumn{1}{c|}{}
&&
&&
  \\
% Line for peach.nn.af.Activation, linespec=[False]
\multicolumn{4}{r}{\settowidth{\BCL}{peach.nn.af.Activation}\multirow{2}{\BCL}{peach.nn.af.Activation}}
&&
  \\\cline{5-5}
  &&&&\multicolumn{1}{c|}{}
&&
  \\
&&&&\multicolumn{2}{l}{\textbf{peach.nn.af.Linear}}
\end{tabular}


Identity activation function

%%%%%%%%%%%%%%%%%%%%%%%%%%%%%%%%%%%%%%%%%%%%%%%%%%%%%%%%%%%%%%%%%%%%%%%%%%%
%%                                Methods                                %%
%%%%%%%%%%%%%%%%%%%%%%%%%%%%%%%%%%%%%%%%%%%%%%%%%%%%%%%%%%%%%%%%%%%%%%%%%%%

  \subsubsection{Methods}

    \vspace{0.5ex}

    \begin{boxedminipage}{\textwidth}

    \raggedright \textbf{\_\_init\_\_}(\textit{self})

    \vspace{-1.5ex}

    \rule{\textwidth}{0.5\fboxrule}

Initializes the function
    \vspace{1ex}

      Overrides: peach.nn.af.Activation.\_\_init\_\_

    \end{boxedminipage}

    \vspace{0.5ex}

    \begin{boxedminipage}{\textwidth}

    \raggedright \textbf{\_\_call\_\_}(\textit{self}, \textit{x})

    \vspace{-1.5ex}

    \rule{\textwidth}{0.5\fboxrule}

Call interface to the object.

This method applies the activation function over a vector of activation
potentials, and returns the results.
    \vspace{1ex}

      \textbf{Parameters}
      \begin{quote}
        \begin{Ventry}{x}

          \item[x]


A real number or a vector of real numbers representing the
activation potential of a neuron or a layer of neurons.
        \end{Ventry}

      \end{quote}

    \vspace{1ex}

      \textbf{Return Value}
      \begin{quote}

The activation function applied over the input vector.
      \end{quote}

    \vspace{1ex}

      Overrides: peach.nn.af.Activation.\_\_call\_\_

    \end{boxedminipage}

    \vspace{0.5ex}

    \begin{boxedminipage}{\textwidth}

    \raggedright \textbf{derivative}(\textit{self}, \textit{x})

    \vspace{-1.5ex}

    \rule{\textwidth}{0.5\fboxrule}

The function derivative.
    \vspace{1ex}

      \textbf{Parameters}
      \begin{quote}
        \begin{Ventry}{x}

          \item[x]


A real number or a vector of real numbers representing the
activation potential of a neuron or a layer of neurons.
        \end{Ventry}

      \end{quote}

    \vspace{1ex}

      \textbf{Return Value}
      \begin{quote}

The derivative of the activation function applied over the input
vector.
      \end{quote}

    \vspace{1ex}

      Overrides: peach.nn.af.Activation.derivative

    \end{boxedminipage}

    \label{object:__delattr__}
    \index{object.\_\_delattr\_\_ \textit{(function)}}

    \vspace{0.5ex}

    \begin{boxedminipage}{\textwidth}

    \raggedright \textbf{\_\_delattr\_\_}(\textit{...})

    \vspace{-1.5ex}

    \rule{\textwidth}{0.5\fboxrule}

x.{\_}{\_}delattr{\_}{\_}('name') {\textless}=={\textgreater} del x.name
    \vspace{1ex}

    \end{boxedminipage}

    \label{object:__getattribute__}
    \index{object.\_\_getattribute\_\_ \textit{(function)}}

    \vspace{0.5ex}

    \begin{boxedminipage}{\textwidth}

    \raggedright \textbf{\_\_getattribute\_\_}(\textit{...})

    \vspace{-1.5ex}

    \rule{\textwidth}{0.5\fboxrule}

x.{\_}{\_}getattribute{\_}{\_}('name') {\textless}=={\textgreater} x.name
    \vspace{1ex}

    \end{boxedminipage}

    \label{object:__hash__}
    \index{object.\_\_hash\_\_ \textit{(function)}}

    \vspace{0.5ex}

    \begin{boxedminipage}{\textwidth}

    \raggedright \textbf{\_\_hash\_\_}(\textit{x})

    \vspace{-1.5ex}

    \rule{\textwidth}{0.5\fboxrule}

hash(x)
    \vspace{1ex}

    \end{boxedminipage}

    \label{object:__new__}
    \index{object.\_\_new\_\_ \textit{(function)}}

    \vspace{0.5ex}

    \begin{boxedminipage}{\textwidth}

    \raggedright \textbf{\_\_new\_\_}(\textit{T}, \textit{S}, \textit{...})

      \textbf{Return Value}
      \begin{quote}
\begin{alltt}
a new object with type S, a subtype of T
\end{alltt}

      \end{quote}

    \vspace{1ex}

    \end{boxedminipage}

    \label{object:__reduce__}
    \index{object.\_\_reduce\_\_ \textit{(function)}}

    \vspace{0.5ex}

    \begin{boxedminipage}{\textwidth}

    \raggedright \textbf{\_\_reduce\_\_}(\textit{...})

    \vspace{-1.5ex}

    \rule{\textwidth}{0.5\fboxrule}

helper for pickle
    \vspace{1ex}

    \end{boxedminipage}

    \label{object:__reduce_ex__}
    \index{object.\_\_reduce\_ex\_\_ \textit{(function)}}

    \vspace{0.5ex}

    \begin{boxedminipage}{\textwidth}

    \raggedright \textbf{\_\_reduce\_ex\_\_}(\textit{...})

    \vspace{-1.5ex}

    \rule{\textwidth}{0.5\fboxrule}

helper for pickle
    \vspace{1ex}

    \end{boxedminipage}

    \label{object:__repr__}
    \index{object.\_\_repr\_\_ \textit{(function)}}

    \vspace{0.5ex}

    \begin{boxedminipage}{\textwidth}

    \raggedright \textbf{\_\_repr\_\_}(\textit{x})

    \vspace{-1.5ex}

    \rule{\textwidth}{0.5\fboxrule}

repr(x)
    \vspace{1ex}

    \end{boxedminipage}

    \label{object:__setattr__}
    \index{object.\_\_setattr\_\_ \textit{(function)}}

    \vspace{0.5ex}

    \begin{boxedminipage}{\textwidth}

    \raggedright \textbf{\_\_setattr\_\_}(\textit{...})

    \vspace{-1.5ex}

    \rule{\textwidth}{0.5\fboxrule}

x.{\_}{\_}setattr{\_}{\_}('name', value) {\textless}=={\textgreater} x.name = value
    \vspace{1ex}

    \end{boxedminipage}

    \label{object:__str__}
    \index{object.\_\_str\_\_ \textit{(function)}}

    \vspace{0.5ex}

    \begin{boxedminipage}{\textwidth}

    \raggedright \textbf{\_\_str\_\_}(\textit{x})

    \vspace{-1.5ex}

    \rule{\textwidth}{0.5\fboxrule}

str(x)
    \vspace{1ex}

    \end{boxedminipage}


%%%%%%%%%%%%%%%%%%%%%%%%%%%%%%%%%%%%%%%%%%%%%%%%%%%%%%%%%%%%%%%%%%%%%%%%%%%
%%                              Properties                               %%
%%%%%%%%%%%%%%%%%%%%%%%%%%%%%%%%%%%%%%%%%%%%%%%%%%%%%%%%%%%%%%%%%%%%%%%%%%%

  \subsubsection{Properties}

\begin{longtable}{|p{.30\textwidth}|p{.62\textwidth}|l}
\cline{1-2}
\cline{1-2} \centering \textbf{Name} & \centering \textbf{Description}& \\
\cline{1-2}
\endhead\cline{1-2}\multicolumn{3}{r}{\small\textit{continued on next page}}\\\endfoot\cline{1-2}
\endlastfoot\raggedright \_\-\_\-c\-l\-a\-s\-s\-\_\-\_\- & \raggedright \textbf{Value:} 
{\tt {\textless}attribute '\_\_class\_\_' of 'object' objects{\textgreater}}&\\
\cline{1-2}
\end{longtable}


%%%%%%%%%%%%%%%%%%%%%%%%%%%%%%%%%%%%%%%%%%%%%%%%%%%%%%%%%%%%%%%%%%%%%%%%%%%
%%                          Instance Variables                           %%
%%%%%%%%%%%%%%%%%%%%%%%%%%%%%%%%%%%%%%%%%%%%%%%%%%%%%%%%%%%%%%%%%%%%%%%%%%%

  \subsubsection{Instance Variables}

\begin{longtable}{|p{.30\textwidth}|p{.62\textwidth}|l}
\cline{1-2}
\cline{1-2} \centering \textbf{Name} & \centering \textbf{Description}& \\
\cline{1-2}
\endhead\cline{1-2}\multicolumn{3}{r}{\small\textit{continued on next page}}\\\endfoot\cline{1-2}
\endlastfoot\raggedright d\- & An alias to the derivative of the function.&\\
\cline{1-2}
\end{longtable}

    \index{peach \textit{(package)}!peach.nn \textit{(package)}!peach.nn.af \textit{(module)}!peach.nn.af.Linear \textit{(class)}|)}

%%%%%%%%%%%%%%%%%%%%%%%%%%%%%%%%%%%%%%%%%%%%%%%%%%%%%%%%%%%%%%%%%%%%%%%%%%%
%%                           Class Description                           %%
%%%%%%%%%%%%%%%%%%%%%%%%%%%%%%%%%%%%%%%%%%%%%%%%%%%%%%%%%%%%%%%%%%%%%%%%%%%

    \index{peach \textit{(package)}!peach.nn \textit{(package)}!peach.nn.af \textit{(module)}!peach.nn.af.Ramp \textit{(class)}|(}
\subsection{Class Ramp}

    \label{peach:nn:af:Ramp}
\begin{tabular}{cccccccc}
% Line for object, linespec=[False, False]
\multicolumn{2}{r}{\settowidth{\BCL}{object}\multirow{2}{\BCL}{object}}
&&
&&
  \\\cline{3-3}
  &&\multicolumn{1}{c|}{}
&&
&&
  \\
% Line for peach.nn.af.Activation, linespec=[False]
\multicolumn{4}{r}{\settowidth{\BCL}{peach.nn.af.Activation}\multirow{2}{\BCL}{peach.nn.af.Activation}}
&&
  \\\cline{5-5}
  &&&&\multicolumn{1}{c|}{}
&&
  \\
&&&&\multicolumn{2}{l}{\textbf{peach.nn.af.Ramp}}
\end{tabular}


Ramp activation function

%%%%%%%%%%%%%%%%%%%%%%%%%%%%%%%%%%%%%%%%%%%%%%%%%%%%%%%%%%%%%%%%%%%%%%%%%%%
%%                                Methods                                %%
%%%%%%%%%%%%%%%%%%%%%%%%%%%%%%%%%%%%%%%%%%%%%%%%%%%%%%%%%%%%%%%%%%%%%%%%%%%

  \subsubsection{Methods}

    \vspace{0.5ex}

    \begin{boxedminipage}{\textwidth}

    \raggedright \textbf{\_\_init\_\_}(\textit{self}, \textit{p0}=\texttt{\texttt{(}-0.5\texttt{, }0.0\texttt{)}}, \textit{p1}=\texttt{\texttt{(}0.5\texttt{, }1.0\texttt{)}})

    \vspace{-1.5ex}

    \rule{\textwidth}{0.5\fboxrule}

Initializes the object.

Two points are needed to set this function. They are used to determine
where the ramp begins and where it ends.
    \vspace{1ex}

      \textbf{Parameters}
      \begin{quote}
        \begin{Ventry}{xx}

          \item[p0]


The starting point, given as a tuple \texttt{(x0, y0)}. For values of the
input below \texttt{x0}, the function returns \texttt{y0}. Defaults to
\texttt{(-0.5, 0.0)}.
          \item[p1]


The ending point, given as a tuple \texttt{(x1, y1)}. For values of the
input above \texttt{x1}, the function returns \texttt{y1}. Defaults to
\texttt{(0.5, 1.0)}.
        \end{Ventry}

      \end{quote}

    \vspace{1ex}

      Overrides: peach.nn.af.Activation.\_\_init\_\_

    \end{boxedminipage}

    \vspace{0.5ex}

    \begin{boxedminipage}{\textwidth}

    \raggedright \textbf{\_\_call\_\_}(\textit{self}, \textit{x})

    \vspace{-1.5ex}

    \rule{\textwidth}{0.5\fboxrule}

Call interface to the object.

This method applies the activation function over a vector of activation
potentials, and returns the results.
    \vspace{1ex}

      \textbf{Parameters}
      \begin{quote}
        \begin{Ventry}{x}

          \item[x]


A real number or a vector of real numbers representing the
activation potential of a neuron or a layer of neurons.
        \end{Ventry}

      \end{quote}

    \vspace{1ex}

      \textbf{Return Value}
      \begin{quote}

The activation function applied over the input vector.
      \end{quote}

    \vspace{1ex}

      Overrides: peach.nn.af.Activation.\_\_call\_\_

    \end{boxedminipage}

    \vspace{0.5ex}

    \begin{boxedminipage}{\textwidth}

    \raggedright \textbf{derivative}(\textit{self}, \textit{x})

    \vspace{-1.5ex}

    \rule{\textwidth}{0.5\fboxrule}

The function derivative.
    \vspace{1ex}

      \textbf{Parameters}
      \begin{quote}
        \begin{Ventry}{x}

          \item[x]


A real number or a vector of real numbers representing the
activation potential of a neuron or a layer of neurons.
        \end{Ventry}

      \end{quote}

    \vspace{1ex}

      \textbf{Return Value}
      \begin{quote}

The derivative of the activation function applied over the input
vector.
      \end{quote}

    \vspace{1ex}

      Overrides: peach.nn.af.Activation.derivative

    \end{boxedminipage}

    \label{object:__delattr__}
    \index{object.\_\_delattr\_\_ \textit{(function)}}

    \vspace{0.5ex}

    \begin{boxedminipage}{\textwidth}

    \raggedright \textbf{\_\_delattr\_\_}(\textit{...})

    \vspace{-1.5ex}

    \rule{\textwidth}{0.5\fboxrule}

x.{\_}{\_}delattr{\_}{\_}('name') {\textless}=={\textgreater} del x.name
    \vspace{1ex}

    \end{boxedminipage}

    \label{object:__getattribute__}
    \index{object.\_\_getattribute\_\_ \textit{(function)}}

    \vspace{0.5ex}

    \begin{boxedminipage}{\textwidth}

    \raggedright \textbf{\_\_getattribute\_\_}(\textit{...})

    \vspace{-1.5ex}

    \rule{\textwidth}{0.5\fboxrule}

x.{\_}{\_}getattribute{\_}{\_}('name') {\textless}=={\textgreater} x.name
    \vspace{1ex}

    \end{boxedminipage}

    \label{object:__hash__}
    \index{object.\_\_hash\_\_ \textit{(function)}}

    \vspace{0.5ex}

    \begin{boxedminipage}{\textwidth}

    \raggedright \textbf{\_\_hash\_\_}(\textit{x})

    \vspace{-1.5ex}

    \rule{\textwidth}{0.5\fboxrule}

hash(x)
    \vspace{1ex}

    \end{boxedminipage}

    \label{object:__new__}
    \index{object.\_\_new\_\_ \textit{(function)}}

    \vspace{0.5ex}

    \begin{boxedminipage}{\textwidth}

    \raggedright \textbf{\_\_new\_\_}(\textit{T}, \textit{S}, \textit{...})

      \textbf{Return Value}
      \begin{quote}
\begin{alltt}
a new object with type S, a subtype of T
\end{alltt}

      \end{quote}

    \vspace{1ex}

    \end{boxedminipage}

    \label{object:__reduce__}
    \index{object.\_\_reduce\_\_ \textit{(function)}}

    \vspace{0.5ex}

    \begin{boxedminipage}{\textwidth}

    \raggedright \textbf{\_\_reduce\_\_}(\textit{...})

    \vspace{-1.5ex}

    \rule{\textwidth}{0.5\fboxrule}

helper for pickle
    \vspace{1ex}

    \end{boxedminipage}

    \label{object:__reduce_ex__}
    \index{object.\_\_reduce\_ex\_\_ \textit{(function)}}

    \vspace{0.5ex}

    \begin{boxedminipage}{\textwidth}

    \raggedright \textbf{\_\_reduce\_ex\_\_}(\textit{...})

    \vspace{-1.5ex}

    \rule{\textwidth}{0.5\fboxrule}

helper for pickle
    \vspace{1ex}

    \end{boxedminipage}

    \label{object:__repr__}
    \index{object.\_\_repr\_\_ \textit{(function)}}

    \vspace{0.5ex}

    \begin{boxedminipage}{\textwidth}

    \raggedright \textbf{\_\_repr\_\_}(\textit{x})

    \vspace{-1.5ex}

    \rule{\textwidth}{0.5\fboxrule}

repr(x)
    \vspace{1ex}

    \end{boxedminipage}

    \label{object:__setattr__}
    \index{object.\_\_setattr\_\_ \textit{(function)}}

    \vspace{0.5ex}

    \begin{boxedminipage}{\textwidth}

    \raggedright \textbf{\_\_setattr\_\_}(\textit{...})

    \vspace{-1.5ex}

    \rule{\textwidth}{0.5\fboxrule}

x.{\_}{\_}setattr{\_}{\_}('name', value) {\textless}=={\textgreater} x.name = value
    \vspace{1ex}

    \end{boxedminipage}

    \label{object:__str__}
    \index{object.\_\_str\_\_ \textit{(function)}}

    \vspace{0.5ex}

    \begin{boxedminipage}{\textwidth}

    \raggedright \textbf{\_\_str\_\_}(\textit{x})

    \vspace{-1.5ex}

    \rule{\textwidth}{0.5\fboxrule}

str(x)
    \vspace{1ex}

    \end{boxedminipage}


%%%%%%%%%%%%%%%%%%%%%%%%%%%%%%%%%%%%%%%%%%%%%%%%%%%%%%%%%%%%%%%%%%%%%%%%%%%
%%                              Properties                               %%
%%%%%%%%%%%%%%%%%%%%%%%%%%%%%%%%%%%%%%%%%%%%%%%%%%%%%%%%%%%%%%%%%%%%%%%%%%%

  \subsubsection{Properties}

\begin{longtable}{|p{.30\textwidth}|p{.62\textwidth}|l}
\cline{1-2}
\cline{1-2} \centering \textbf{Name} & \centering \textbf{Description}& \\
\cline{1-2}
\endhead\cline{1-2}\multicolumn{3}{r}{\small\textit{continued on next page}}\\\endfoot\cline{1-2}
\endlastfoot\raggedright \_\-\_\-c\-l\-a\-s\-s\-\_\-\_\- & \raggedright \textbf{Value:} 
{\tt {\textless}attribute '\_\_class\_\_' of 'object' objects{\textgreater}}&\\
\cline{1-2}
\end{longtable}


%%%%%%%%%%%%%%%%%%%%%%%%%%%%%%%%%%%%%%%%%%%%%%%%%%%%%%%%%%%%%%%%%%%%%%%%%%%
%%                          Instance Variables                           %%
%%%%%%%%%%%%%%%%%%%%%%%%%%%%%%%%%%%%%%%%%%%%%%%%%%%%%%%%%%%%%%%%%%%%%%%%%%%

  \subsubsection{Instance Variables}

\begin{longtable}{|p{.30\textwidth}|p{.62\textwidth}|l}
\cline{1-2}
\cline{1-2} \centering \textbf{Name} & \centering \textbf{Description}& \\
\cline{1-2}
\endhead\cline{1-2}\multicolumn{3}{r}{\small\textit{continued on next page}}\\\endfoot\cline{1-2}
\endlastfoot\raggedright d\- & An alias to the derivative of the function.&\\
\cline{1-2}
\end{longtable}

    \index{peach \textit{(package)}!peach.nn \textit{(package)}!peach.nn.af \textit{(module)}!peach.nn.af.Ramp \textit{(class)}|)}

%%%%%%%%%%%%%%%%%%%%%%%%%%%%%%%%%%%%%%%%%%%%%%%%%%%%%%%%%%%%%%%%%%%%%%%%%%%
%%                           Class Description                           %%
%%%%%%%%%%%%%%%%%%%%%%%%%%%%%%%%%%%%%%%%%%%%%%%%%%%%%%%%%%%%%%%%%%%%%%%%%%%

    \index{peach \textit{(package)}!peach.nn \textit{(package)}!peach.nn.af \textit{(module)}!peach.nn.af.Sigmoid \textit{(class)}|(}
\subsection{Class Sigmoid}

    \label{peach:nn:af:Sigmoid}
\begin{tabular}{cccccccc}
% Line for object, linespec=[False, False]
\multicolumn{2}{r}{\settowidth{\BCL}{object}\multirow{2}{\BCL}{object}}
&&
&&
  \\\cline{3-3}
  &&\multicolumn{1}{c|}{}
&&
&&
  \\
% Line for peach.nn.af.Activation, linespec=[False]
\multicolumn{4}{r}{\settowidth{\BCL}{peach.nn.af.Activation}\multirow{2}{\BCL}{peach.nn.af.Activation}}
&&
  \\\cline{5-5}
  &&&&\multicolumn{1}{c|}{}
&&
  \\
&&&&\multicolumn{2}{l}{\textbf{peach.nn.af.Sigmoid}}
\end{tabular}


Sigmoid activation function

%%%%%%%%%%%%%%%%%%%%%%%%%%%%%%%%%%%%%%%%%%%%%%%%%%%%%%%%%%%%%%%%%%%%%%%%%%%
%%                                Methods                                %%
%%%%%%%%%%%%%%%%%%%%%%%%%%%%%%%%%%%%%%%%%%%%%%%%%%%%%%%%%%%%%%%%%%%%%%%%%%%

  \subsubsection{Methods}

    \vspace{0.5ex}

    \begin{boxedminipage}{\textwidth}

    \raggedright \textbf{\_\_init\_\_}(\textit{self}, \textit{a}=\texttt{1.0}, \textit{x0}=\texttt{0.0})

    \vspace{-1.5ex}

    \rule{\textwidth}{0.5\fboxrule}

Initializes the object.
    \vspace{1ex}

      \textbf{Parameters}
      \begin{quote}
        \begin{Ventry}{xx}

          \item[a]


The slope of the function in the center \texttt{x0}. Defaults to 1.0.
          \item[x0]


The center of the sigmoid. Defaults to 0.0.
        \end{Ventry}

      \end{quote}

    \vspace{1ex}

      Overrides: peach.nn.af.Activation.\_\_init\_\_

    \end{boxedminipage}

    \vspace{0.5ex}

    \begin{boxedminipage}{\textwidth}

    \raggedright \textbf{\_\_call\_\_}(\textit{self}, \textit{x})

    \vspace{-1.5ex}

    \rule{\textwidth}{0.5\fboxrule}

Call interface to the object.

This method applies the activation function over a vector of activation
potentials, and returns the results.
    \vspace{1ex}

      \textbf{Parameters}
      \begin{quote}
        \begin{Ventry}{x}

          \item[x]


A real number or a vector of real numbers representing the
activation potential of a neuron or a layer of neurons.
        \end{Ventry}

      \end{quote}

    \vspace{1ex}

      \textbf{Return Value}
      \begin{quote}

The activation function applied over the input vector.
      \end{quote}

    \vspace{1ex}

      Overrides: peach.nn.af.Activation.\_\_call\_\_

    \end{boxedminipage}

    \vspace{0.5ex}

    \begin{boxedminipage}{\textwidth}

    \raggedright \textbf{derivative}(\textit{self}, \textit{x})

    \vspace{-1.5ex}

    \rule{\textwidth}{0.5\fboxrule}

The function derivative.
    \vspace{1ex}

      \textbf{Parameters}
      \begin{quote}
        \begin{Ventry}{x}

          \item[x]


A real number or a vector of real numbers representing the
activation potential of a neuron or a layer of neurons.
        \end{Ventry}

      \end{quote}

    \vspace{1ex}

      \textbf{Return Value}
      \begin{quote}

The derivative of the activation function applied over the input
vector.
      \end{quote}

    \vspace{1ex}

      Overrides: peach.nn.af.Activation.derivative

    \end{boxedminipage}

    \label{object:__delattr__}
    \index{object.\_\_delattr\_\_ \textit{(function)}}

    \vspace{0.5ex}

    \begin{boxedminipage}{\textwidth}

    \raggedright \textbf{\_\_delattr\_\_}(\textit{...})

    \vspace{-1.5ex}

    \rule{\textwidth}{0.5\fboxrule}

x.{\_}{\_}delattr{\_}{\_}('name') {\textless}=={\textgreater} del x.name
    \vspace{1ex}

    \end{boxedminipage}

    \label{object:__getattribute__}
    \index{object.\_\_getattribute\_\_ \textit{(function)}}

    \vspace{0.5ex}

    \begin{boxedminipage}{\textwidth}

    \raggedright \textbf{\_\_getattribute\_\_}(\textit{...})

    \vspace{-1.5ex}

    \rule{\textwidth}{0.5\fboxrule}

x.{\_}{\_}getattribute{\_}{\_}('name') {\textless}=={\textgreater} x.name
    \vspace{1ex}

    \end{boxedminipage}

    \label{object:__hash__}
    \index{object.\_\_hash\_\_ \textit{(function)}}

    \vspace{0.5ex}

    \begin{boxedminipage}{\textwidth}

    \raggedright \textbf{\_\_hash\_\_}(\textit{x})

    \vspace{-1.5ex}

    \rule{\textwidth}{0.5\fboxrule}

hash(x)
    \vspace{1ex}

    \end{boxedminipage}

    \label{object:__new__}
    \index{object.\_\_new\_\_ \textit{(function)}}

    \vspace{0.5ex}

    \begin{boxedminipage}{\textwidth}

    \raggedright \textbf{\_\_new\_\_}(\textit{T}, \textit{S}, \textit{...})

      \textbf{Return Value}
      \begin{quote}
\begin{alltt}
a new object with type S, a subtype of T
\end{alltt}

      \end{quote}

    \vspace{1ex}

    \end{boxedminipage}

    \label{object:__reduce__}
    \index{object.\_\_reduce\_\_ \textit{(function)}}

    \vspace{0.5ex}

    \begin{boxedminipage}{\textwidth}

    \raggedright \textbf{\_\_reduce\_\_}(\textit{...})

    \vspace{-1.5ex}

    \rule{\textwidth}{0.5\fboxrule}

helper for pickle
    \vspace{1ex}

    \end{boxedminipage}

    \label{object:__reduce_ex__}
    \index{object.\_\_reduce\_ex\_\_ \textit{(function)}}

    \vspace{0.5ex}

    \begin{boxedminipage}{\textwidth}

    \raggedright \textbf{\_\_reduce\_ex\_\_}(\textit{...})

    \vspace{-1.5ex}

    \rule{\textwidth}{0.5\fboxrule}

helper for pickle
    \vspace{1ex}

    \end{boxedminipage}

    \label{object:__repr__}
    \index{object.\_\_repr\_\_ \textit{(function)}}

    \vspace{0.5ex}

    \begin{boxedminipage}{\textwidth}

    \raggedright \textbf{\_\_repr\_\_}(\textit{x})

    \vspace{-1.5ex}

    \rule{\textwidth}{0.5\fboxrule}

repr(x)
    \vspace{1ex}

    \end{boxedminipage}

    \label{object:__setattr__}
    \index{object.\_\_setattr\_\_ \textit{(function)}}

    \vspace{0.5ex}

    \begin{boxedminipage}{\textwidth}

    \raggedright \textbf{\_\_setattr\_\_}(\textit{...})

    \vspace{-1.5ex}

    \rule{\textwidth}{0.5\fboxrule}

x.{\_}{\_}setattr{\_}{\_}('name', value) {\textless}=={\textgreater} x.name = value
    \vspace{1ex}

    \end{boxedminipage}

    \label{object:__str__}
    \index{object.\_\_str\_\_ \textit{(function)}}

    \vspace{0.5ex}

    \begin{boxedminipage}{\textwidth}

    \raggedright \textbf{\_\_str\_\_}(\textit{x})

    \vspace{-1.5ex}

    \rule{\textwidth}{0.5\fboxrule}

str(x)
    \vspace{1ex}

    \end{boxedminipage}


%%%%%%%%%%%%%%%%%%%%%%%%%%%%%%%%%%%%%%%%%%%%%%%%%%%%%%%%%%%%%%%%%%%%%%%%%%%
%%                              Properties                               %%
%%%%%%%%%%%%%%%%%%%%%%%%%%%%%%%%%%%%%%%%%%%%%%%%%%%%%%%%%%%%%%%%%%%%%%%%%%%

  \subsubsection{Properties}

\begin{longtable}{|p{.30\textwidth}|p{.62\textwidth}|l}
\cline{1-2}
\cline{1-2} \centering \textbf{Name} & \centering \textbf{Description}& \\
\cline{1-2}
\endhead\cline{1-2}\multicolumn{3}{r}{\small\textit{continued on next page}}\\\endfoot\cline{1-2}
\endlastfoot\raggedright \_\-\_\-c\-l\-a\-s\-s\-\_\-\_\- & \raggedright \textbf{Value:} 
{\tt {\textless}attribute '\_\_class\_\_' of 'object' objects{\textgreater}}&\\
\cline{1-2}
\end{longtable}


%%%%%%%%%%%%%%%%%%%%%%%%%%%%%%%%%%%%%%%%%%%%%%%%%%%%%%%%%%%%%%%%%%%%%%%%%%%
%%                          Instance Variables                           %%
%%%%%%%%%%%%%%%%%%%%%%%%%%%%%%%%%%%%%%%%%%%%%%%%%%%%%%%%%%%%%%%%%%%%%%%%%%%

  \subsubsection{Instance Variables}

\begin{longtable}{|p{.30\textwidth}|p{.62\textwidth}|l}
\cline{1-2}
\cline{1-2} \centering \textbf{Name} & \centering \textbf{Description}& \\
\cline{1-2}
\endhead\cline{1-2}\multicolumn{3}{r}{\small\textit{continued on next page}}\\\endfoot\cline{1-2}
\endlastfoot\raggedright d\- & An alias to the derivative of the function.&\\
\cline{1-2}
\end{longtable}

    \index{peach \textit{(package)}!peach.nn \textit{(package)}!peach.nn.af \textit{(module)}!peach.nn.af.Sigmoid \textit{(class)}|)}

%%%%%%%%%%%%%%%%%%%%%%%%%%%%%%%%%%%%%%%%%%%%%%%%%%%%%%%%%%%%%%%%%%%%%%%%%%%
%%                           Class Description                           %%
%%%%%%%%%%%%%%%%%%%%%%%%%%%%%%%%%%%%%%%%%%%%%%%%%%%%%%%%%%%%%%%%%%%%%%%%%%%

    \index{peach \textit{(package)}!peach.nn \textit{(package)}!peach.nn.af \textit{(module)}!peach.nn.af.Sigmoid \textit{(class)}|(}
\subsection{Class Sigmoid}

    \label{peach:nn:af:Sigmoid}
\begin{tabular}{cccccccc}
% Line for object, linespec=[False, False]
\multicolumn{2}{r}{\settowidth{\BCL}{object}\multirow{2}{\BCL}{object}}
&&
&&
  \\\cline{3-3}
  &&\multicolumn{1}{c|}{}
&&
&&
  \\
% Line for peach.nn.af.Activation, linespec=[False]
\multicolumn{4}{r}{\settowidth{\BCL}{peach.nn.af.Activation}\multirow{2}{\BCL}{peach.nn.af.Activation}}
&&
  \\\cline{5-5}
  &&&&\multicolumn{1}{c|}{}
&&
  \\
&&&&\multicolumn{2}{l}{\textbf{peach.nn.af.Sigmoid}}
\end{tabular}


Sigmoid activation function

%%%%%%%%%%%%%%%%%%%%%%%%%%%%%%%%%%%%%%%%%%%%%%%%%%%%%%%%%%%%%%%%%%%%%%%%%%%
%%                                Methods                                %%
%%%%%%%%%%%%%%%%%%%%%%%%%%%%%%%%%%%%%%%%%%%%%%%%%%%%%%%%%%%%%%%%%%%%%%%%%%%

  \subsubsection{Methods}

    \vspace{0.5ex}

    \begin{boxedminipage}{\textwidth}

    \raggedright \textbf{\_\_init\_\_}(\textit{self}, \textit{a}=\texttt{1.0}, \textit{x0}=\texttt{0.0})

    \vspace{-1.5ex}

    \rule{\textwidth}{0.5\fboxrule}

Initializes the object.
    \vspace{1ex}

      \textbf{Parameters}
      \begin{quote}
        \begin{Ventry}{xx}

          \item[a]


The slope of the function in the center \texttt{x0}. Defaults to 1.0.
          \item[x0]


The center of the sigmoid. Defaults to 0.0.
        \end{Ventry}

      \end{quote}

    \vspace{1ex}

      Overrides: peach.nn.af.Activation.\_\_init\_\_

    \end{boxedminipage}

    \vspace{0.5ex}

    \begin{boxedminipage}{\textwidth}

    \raggedright \textbf{\_\_call\_\_}(\textit{self}, \textit{x})

    \vspace{-1.5ex}

    \rule{\textwidth}{0.5\fboxrule}

Call interface to the object.

This method applies the activation function over a vector of activation
potentials, and returns the results.
    \vspace{1ex}

      \textbf{Parameters}
      \begin{quote}
        \begin{Ventry}{x}

          \item[x]


A real number or a vector of real numbers representing the
activation potential of a neuron or a layer of neurons.
        \end{Ventry}

      \end{quote}

    \vspace{1ex}

      \textbf{Return Value}
      \begin{quote}

The activation function applied over the input vector.
      \end{quote}

    \vspace{1ex}

      Overrides: peach.nn.af.Activation.\_\_call\_\_

    \end{boxedminipage}

    \vspace{0.5ex}

    \begin{boxedminipage}{\textwidth}

    \raggedright \textbf{derivative}(\textit{self}, \textit{x})

    \vspace{-1.5ex}

    \rule{\textwidth}{0.5\fboxrule}

The function derivative.
    \vspace{1ex}

      \textbf{Parameters}
      \begin{quote}
        \begin{Ventry}{x}

          \item[x]


A real number or a vector of real numbers representing the
activation potential of a neuron or a layer of neurons.
        \end{Ventry}

      \end{quote}

    \vspace{1ex}

      \textbf{Return Value}
      \begin{quote}

The derivative of the activation function applied over the input
vector.
      \end{quote}

    \vspace{1ex}

      Overrides: peach.nn.af.Activation.derivative

    \end{boxedminipage}

    \label{object:__delattr__}
    \index{object.\_\_delattr\_\_ \textit{(function)}}

    \vspace{0.5ex}

    \begin{boxedminipage}{\textwidth}

    \raggedright \textbf{\_\_delattr\_\_}(\textit{...})

    \vspace{-1.5ex}

    \rule{\textwidth}{0.5\fboxrule}

x.{\_}{\_}delattr{\_}{\_}('name') {\textless}=={\textgreater} del x.name
    \vspace{1ex}

    \end{boxedminipage}

    \label{object:__getattribute__}
    \index{object.\_\_getattribute\_\_ \textit{(function)}}

    \vspace{0.5ex}

    \begin{boxedminipage}{\textwidth}

    \raggedright \textbf{\_\_getattribute\_\_}(\textit{...})

    \vspace{-1.5ex}

    \rule{\textwidth}{0.5\fboxrule}

x.{\_}{\_}getattribute{\_}{\_}('name') {\textless}=={\textgreater} x.name
    \vspace{1ex}

    \end{boxedminipage}

    \label{object:__hash__}
    \index{object.\_\_hash\_\_ \textit{(function)}}

    \vspace{0.5ex}

    \begin{boxedminipage}{\textwidth}

    \raggedright \textbf{\_\_hash\_\_}(\textit{x})

    \vspace{-1.5ex}

    \rule{\textwidth}{0.5\fboxrule}

hash(x)
    \vspace{1ex}

    \end{boxedminipage}

    \label{object:__new__}
    \index{object.\_\_new\_\_ \textit{(function)}}

    \vspace{0.5ex}

    \begin{boxedminipage}{\textwidth}

    \raggedright \textbf{\_\_new\_\_}(\textit{T}, \textit{S}, \textit{...})

      \textbf{Return Value}
      \begin{quote}
\begin{alltt}
a new object with type S, a subtype of T
\end{alltt}

      \end{quote}

    \vspace{1ex}

    \end{boxedminipage}

    \label{object:__reduce__}
    \index{object.\_\_reduce\_\_ \textit{(function)}}

    \vspace{0.5ex}

    \begin{boxedminipage}{\textwidth}

    \raggedright \textbf{\_\_reduce\_\_}(\textit{...})

    \vspace{-1.5ex}

    \rule{\textwidth}{0.5\fboxrule}

helper for pickle
    \vspace{1ex}

    \end{boxedminipage}

    \label{object:__reduce_ex__}
    \index{object.\_\_reduce\_ex\_\_ \textit{(function)}}

    \vspace{0.5ex}

    \begin{boxedminipage}{\textwidth}

    \raggedright \textbf{\_\_reduce\_ex\_\_}(\textit{...})

    \vspace{-1.5ex}

    \rule{\textwidth}{0.5\fboxrule}

helper for pickle
    \vspace{1ex}

    \end{boxedminipage}

    \label{object:__repr__}
    \index{object.\_\_repr\_\_ \textit{(function)}}

    \vspace{0.5ex}

    \begin{boxedminipage}{\textwidth}

    \raggedright \textbf{\_\_repr\_\_}(\textit{x})

    \vspace{-1.5ex}

    \rule{\textwidth}{0.5\fboxrule}

repr(x)
    \vspace{1ex}

    \end{boxedminipage}

    \label{object:__setattr__}
    \index{object.\_\_setattr\_\_ \textit{(function)}}

    \vspace{0.5ex}

    \begin{boxedminipage}{\textwidth}

    \raggedright \textbf{\_\_setattr\_\_}(\textit{...})

    \vspace{-1.5ex}

    \rule{\textwidth}{0.5\fboxrule}

x.{\_}{\_}setattr{\_}{\_}('name', value) {\textless}=={\textgreater} x.name = value
    \vspace{1ex}

    \end{boxedminipage}

    \label{object:__str__}
    \index{object.\_\_str\_\_ \textit{(function)}}

    \vspace{0.5ex}

    \begin{boxedminipage}{\textwidth}

    \raggedright \textbf{\_\_str\_\_}(\textit{x})

    \vspace{-1.5ex}

    \rule{\textwidth}{0.5\fboxrule}

str(x)
    \vspace{1ex}

    \end{boxedminipage}


%%%%%%%%%%%%%%%%%%%%%%%%%%%%%%%%%%%%%%%%%%%%%%%%%%%%%%%%%%%%%%%%%%%%%%%%%%%
%%                              Properties                               %%
%%%%%%%%%%%%%%%%%%%%%%%%%%%%%%%%%%%%%%%%%%%%%%%%%%%%%%%%%%%%%%%%%%%%%%%%%%%

  \subsubsection{Properties}

\begin{longtable}{|p{.30\textwidth}|p{.62\textwidth}|l}
\cline{1-2}
\cline{1-2} \centering \textbf{Name} & \centering \textbf{Description}& \\
\cline{1-2}
\endhead\cline{1-2}\multicolumn{3}{r}{\small\textit{continued on next page}}\\\endfoot\cline{1-2}
\endlastfoot\raggedright \_\-\_\-c\-l\-a\-s\-s\-\_\-\_\- & \raggedright \textbf{Value:} 
{\tt {\textless}attribute '\_\_class\_\_' of 'object' objects{\textgreater}}&\\
\cline{1-2}
\end{longtable}


%%%%%%%%%%%%%%%%%%%%%%%%%%%%%%%%%%%%%%%%%%%%%%%%%%%%%%%%%%%%%%%%%%%%%%%%%%%
%%                          Instance Variables                           %%
%%%%%%%%%%%%%%%%%%%%%%%%%%%%%%%%%%%%%%%%%%%%%%%%%%%%%%%%%%%%%%%%%%%%%%%%%%%

  \subsubsection{Instance Variables}

\begin{longtable}{|p{.30\textwidth}|p{.62\textwidth}|l}
\cline{1-2}
\cline{1-2} \centering \textbf{Name} & \centering \textbf{Description}& \\
\cline{1-2}
\endhead\cline{1-2}\multicolumn{3}{r}{\small\textit{continued on next page}}\\\endfoot\cline{1-2}
\endlastfoot\raggedright d\- & An alias to the derivative of the function.&\\
\cline{1-2}
\end{longtable}

    \index{peach \textit{(package)}!peach.nn \textit{(package)}!peach.nn.af \textit{(module)}!peach.nn.af.Sigmoid \textit{(class)}|)}

%%%%%%%%%%%%%%%%%%%%%%%%%%%%%%%%%%%%%%%%%%%%%%%%%%%%%%%%%%%%%%%%%%%%%%%%%%%
%%                           Class Description                           %%
%%%%%%%%%%%%%%%%%%%%%%%%%%%%%%%%%%%%%%%%%%%%%%%%%%%%%%%%%%%%%%%%%%%%%%%%%%%

    \index{peach \textit{(package)}!peach.nn \textit{(package)}!peach.nn.af \textit{(module)}!peach.nn.af.Signum \textit{(class)}|(}
\subsection{Class Signum}

    \label{peach:nn:af:Signum}
\begin{tabular}{cccccccc}
% Line for object, linespec=[False, False]
\multicolumn{2}{r}{\settowidth{\BCL}{object}\multirow{2}{\BCL}{object}}
&&
&&
  \\\cline{3-3}
  &&\multicolumn{1}{c|}{}
&&
&&
  \\
% Line for peach.nn.af.Activation, linespec=[False]
\multicolumn{4}{r}{\settowidth{\BCL}{peach.nn.af.Activation}\multirow{2}{\BCL}{peach.nn.af.Activation}}
&&
  \\\cline{5-5}
  &&&&\multicolumn{1}{c|}{}
&&
  \\
&&&&\multicolumn{2}{l}{\textbf{peach.nn.af.Signum}}
\end{tabular}


Signum activation function

%%%%%%%%%%%%%%%%%%%%%%%%%%%%%%%%%%%%%%%%%%%%%%%%%%%%%%%%%%%%%%%%%%%%%%%%%%%
%%                                Methods                                %%
%%%%%%%%%%%%%%%%%%%%%%%%%%%%%%%%%%%%%%%%%%%%%%%%%%%%%%%%%%%%%%%%%%%%%%%%%%%

  \subsubsection{Methods}

    \vspace{0.5ex}

    \begin{boxedminipage}{\textwidth}

    \raggedright \textbf{\_\_init\_\_}(\textit{self})

    \vspace{-1.5ex}

    \rule{\textwidth}{0.5\fboxrule}

Initializes the object.
    \vspace{1ex}

      Overrides: peach.nn.af.Activation.\_\_init\_\_

    \end{boxedminipage}

    \vspace{0.5ex}

    \begin{boxedminipage}{\textwidth}

    \raggedright \textbf{\_\_call\_\_}(\textit{self}, \textit{x})

    \vspace{-1.5ex}

    \rule{\textwidth}{0.5\fboxrule}

Call interface to the object.

This method applies the activation function over a vector of activation
potentials, and returns the results.
    \vspace{1ex}

      \textbf{Parameters}
      \begin{quote}
        \begin{Ventry}{x}

          \item[x]


A real number or a vector of real numbers representing the
activation potential of a neuron or a layer of neurons.
        \end{Ventry}

      \end{quote}

    \vspace{1ex}

      \textbf{Return Value}
      \begin{quote}

The activation function applied over the input vector.
      \end{quote}

    \vspace{1ex}

      Overrides: peach.nn.af.Activation.\_\_call\_\_

    \end{boxedminipage}

    \vspace{0.5ex}

    \begin{boxedminipage}{\textwidth}

    \raggedright \textbf{derivative}(\textit{self}, \textit{x})

    \vspace{-1.5ex}

    \rule{\textwidth}{0.5\fboxrule}

The function derivative. Technically, this function doesn't have a
derivative, but making it equals to 1, this can be used in learning
algorithms.
    \vspace{1ex}

      \textbf{Parameters}
      \begin{quote}
        \begin{Ventry}{x}

          \item[x]


A real number or a vector of real numbers representing the
activation potential of a neuron or a layer of neurons.
        \end{Ventry}

      \end{quote}

    \vspace{1ex}

      \textbf{Return Value}
      \begin{quote}

The derivative of the activation function applied over the input
vector.
      \end{quote}

    \vspace{1ex}

      Overrides: peach.nn.af.Activation.derivative

    \end{boxedminipage}

    \label{object:__delattr__}
    \index{object.\_\_delattr\_\_ \textit{(function)}}

    \vspace{0.5ex}

    \begin{boxedminipage}{\textwidth}

    \raggedright \textbf{\_\_delattr\_\_}(\textit{...})

    \vspace{-1.5ex}

    \rule{\textwidth}{0.5\fboxrule}

x.{\_}{\_}delattr{\_}{\_}('name') {\textless}=={\textgreater} del x.name
    \vspace{1ex}

    \end{boxedminipage}

    \label{object:__getattribute__}
    \index{object.\_\_getattribute\_\_ \textit{(function)}}

    \vspace{0.5ex}

    \begin{boxedminipage}{\textwidth}

    \raggedright \textbf{\_\_getattribute\_\_}(\textit{...})

    \vspace{-1.5ex}

    \rule{\textwidth}{0.5\fboxrule}

x.{\_}{\_}getattribute{\_}{\_}('name') {\textless}=={\textgreater} x.name
    \vspace{1ex}

    \end{boxedminipage}

    \label{object:__hash__}
    \index{object.\_\_hash\_\_ \textit{(function)}}

    \vspace{0.5ex}

    \begin{boxedminipage}{\textwidth}

    \raggedright \textbf{\_\_hash\_\_}(\textit{x})

    \vspace{-1.5ex}

    \rule{\textwidth}{0.5\fboxrule}

hash(x)
    \vspace{1ex}

    \end{boxedminipage}

    \label{object:__new__}
    \index{object.\_\_new\_\_ \textit{(function)}}

    \vspace{0.5ex}

    \begin{boxedminipage}{\textwidth}

    \raggedright \textbf{\_\_new\_\_}(\textit{T}, \textit{S}, \textit{...})

      \textbf{Return Value}
      \begin{quote}
\begin{alltt}
a new object with type S, a subtype of T
\end{alltt}

      \end{quote}

    \vspace{1ex}

    \end{boxedminipage}

    \label{object:__reduce__}
    \index{object.\_\_reduce\_\_ \textit{(function)}}

    \vspace{0.5ex}

    \begin{boxedminipage}{\textwidth}

    \raggedright \textbf{\_\_reduce\_\_}(\textit{...})

    \vspace{-1.5ex}

    \rule{\textwidth}{0.5\fboxrule}

helper for pickle
    \vspace{1ex}

    \end{boxedminipage}

    \label{object:__reduce_ex__}
    \index{object.\_\_reduce\_ex\_\_ \textit{(function)}}

    \vspace{0.5ex}

    \begin{boxedminipage}{\textwidth}

    \raggedright \textbf{\_\_reduce\_ex\_\_}(\textit{...})

    \vspace{-1.5ex}

    \rule{\textwidth}{0.5\fboxrule}

helper for pickle
    \vspace{1ex}

    \end{boxedminipage}

    \label{object:__repr__}
    \index{object.\_\_repr\_\_ \textit{(function)}}

    \vspace{0.5ex}

    \begin{boxedminipage}{\textwidth}

    \raggedright \textbf{\_\_repr\_\_}(\textit{x})

    \vspace{-1.5ex}

    \rule{\textwidth}{0.5\fboxrule}

repr(x)
    \vspace{1ex}

    \end{boxedminipage}

    \label{object:__setattr__}
    \index{object.\_\_setattr\_\_ \textit{(function)}}

    \vspace{0.5ex}

    \begin{boxedminipage}{\textwidth}

    \raggedright \textbf{\_\_setattr\_\_}(\textit{...})

    \vspace{-1.5ex}

    \rule{\textwidth}{0.5\fboxrule}

x.{\_}{\_}setattr{\_}{\_}('name', value) {\textless}=={\textgreater} x.name = value
    \vspace{1ex}

    \end{boxedminipage}

    \label{object:__str__}
    \index{object.\_\_str\_\_ \textit{(function)}}

    \vspace{0.5ex}

    \begin{boxedminipage}{\textwidth}

    \raggedright \textbf{\_\_str\_\_}(\textit{x})

    \vspace{-1.5ex}

    \rule{\textwidth}{0.5\fboxrule}

str(x)
    \vspace{1ex}

    \end{boxedminipage}


%%%%%%%%%%%%%%%%%%%%%%%%%%%%%%%%%%%%%%%%%%%%%%%%%%%%%%%%%%%%%%%%%%%%%%%%%%%
%%                              Properties                               %%
%%%%%%%%%%%%%%%%%%%%%%%%%%%%%%%%%%%%%%%%%%%%%%%%%%%%%%%%%%%%%%%%%%%%%%%%%%%

  \subsubsection{Properties}

\begin{longtable}{|p{.30\textwidth}|p{.62\textwidth}|l}
\cline{1-2}
\cline{1-2} \centering \textbf{Name} & \centering \textbf{Description}& \\
\cline{1-2}
\endhead\cline{1-2}\multicolumn{3}{r}{\small\textit{continued on next page}}\\\endfoot\cline{1-2}
\endlastfoot\raggedright \_\-\_\-c\-l\-a\-s\-s\-\_\-\_\- & \raggedright \textbf{Value:} 
{\tt {\textless}attribute '\_\_class\_\_' of 'object' objects{\textgreater}}&\\
\cline{1-2}
\end{longtable}


%%%%%%%%%%%%%%%%%%%%%%%%%%%%%%%%%%%%%%%%%%%%%%%%%%%%%%%%%%%%%%%%%%%%%%%%%%%
%%                          Instance Variables                           %%
%%%%%%%%%%%%%%%%%%%%%%%%%%%%%%%%%%%%%%%%%%%%%%%%%%%%%%%%%%%%%%%%%%%%%%%%%%%

  \subsubsection{Instance Variables}

\begin{longtable}{|p{.30\textwidth}|p{.62\textwidth}|l}
\cline{1-2}
\cline{1-2} \centering \textbf{Name} & \centering \textbf{Description}& \\
\cline{1-2}
\endhead\cline{1-2}\multicolumn{3}{r}{\small\textit{continued on next page}}\\\endfoot\cline{1-2}
\endlastfoot\raggedright d\- & An alias to the derivative of the function.&\\
\cline{1-2}
\end{longtable}

    \index{peach \textit{(package)}!peach.nn \textit{(package)}!peach.nn.af \textit{(module)}!peach.nn.af.Signum \textit{(class)}|)}

%%%%%%%%%%%%%%%%%%%%%%%%%%%%%%%%%%%%%%%%%%%%%%%%%%%%%%%%%%%%%%%%%%%%%%%%%%%
%%                           Class Description                           %%
%%%%%%%%%%%%%%%%%%%%%%%%%%%%%%%%%%%%%%%%%%%%%%%%%%%%%%%%%%%%%%%%%%%%%%%%%%%

    \index{peach \textit{(package)}!peach.nn \textit{(package)}!peach.nn.af \textit{(module)}!peach.nn.af.ArcTan \textit{(class)}|(}
\subsection{Class ArcTan}

    \label{peach:nn:af:ArcTan}
\begin{tabular}{cccccccc}
% Line for object, linespec=[False, False]
\multicolumn{2}{r}{\settowidth{\BCL}{object}\multirow{2}{\BCL}{object}}
&&
&&
  \\\cline{3-3}
  &&\multicolumn{1}{c|}{}
&&
&&
  \\
% Line for peach.nn.af.Activation, linespec=[False]
\multicolumn{4}{r}{\settowidth{\BCL}{peach.nn.af.Activation}\multirow{2}{\BCL}{peach.nn.af.Activation}}
&&
  \\\cline{5-5}
  &&&&\multicolumn{1}{c|}{}
&&
  \\
&&&&\multicolumn{2}{l}{\textbf{peach.nn.af.ArcTan}}
\end{tabular}


Inverse tangent activation function

%%%%%%%%%%%%%%%%%%%%%%%%%%%%%%%%%%%%%%%%%%%%%%%%%%%%%%%%%%%%%%%%%%%%%%%%%%%
%%                                Methods                                %%
%%%%%%%%%%%%%%%%%%%%%%%%%%%%%%%%%%%%%%%%%%%%%%%%%%%%%%%%%%%%%%%%%%%%%%%%%%%

  \subsubsection{Methods}

    \vspace{0.5ex}

    \begin{boxedminipage}{\textwidth}

    \raggedright \textbf{\_\_init\_\_}(\textit{self}, \textit{a}=\texttt{1.0}, \textit{x0}=\texttt{0.0})

    \vspace{-1.5ex}

    \rule{\textwidth}{0.5\fboxrule}

Initializes the object
    \vspace{1ex}

      \textbf{Parameters}
      \begin{quote}
        \begin{Ventry}{xx}

          \item[a]


The slope of the function in the center \texttt{x0}. Defaults to 1.0.
          \item[x0]


The center of the sigmoid. Defaults to 0.0.
        \end{Ventry}

      \end{quote}

    \vspace{1ex}

      Overrides: peach.nn.af.Activation.\_\_init\_\_

    \end{boxedminipage}

    \vspace{0.5ex}

    \begin{boxedminipage}{\textwidth}

    \raggedright \textbf{\_\_call\_\_}(\textit{self}, \textit{x})

    \vspace{-1.5ex}

    \rule{\textwidth}{0.5\fboxrule}

Call interface to the object.

This method applies the activation function over a vector of activation
potentials, and returns the results.
    \vspace{1ex}

      \textbf{Parameters}
      \begin{quote}
        \begin{Ventry}{x}

          \item[x]


A real number or a vector of real numbers representing the
activation potential of a neuron or a layer of neurons.
        \end{Ventry}

      \end{quote}

    \vspace{1ex}

      \textbf{Return Value}
      \begin{quote}

The activation function applied over the input vector.
      \end{quote}

    \vspace{1ex}

      Overrides: peach.nn.af.Activation.\_\_call\_\_

    \end{boxedminipage}

    \vspace{0.5ex}

    \begin{boxedminipage}{\textwidth}

    \raggedright \textbf{derivative}(\textit{self}, \textit{x})

    \vspace{-1.5ex}

    \rule{\textwidth}{0.5\fboxrule}

The function derivative.
    \vspace{1ex}

      \textbf{Parameters}
      \begin{quote}
        \begin{Ventry}{x}

          \item[x]


A real number or a vector of real numbers representing the
activation potential of a neuron or a layer of neurons.
        \end{Ventry}

      \end{quote}

    \vspace{1ex}

      \textbf{Return Value}
      \begin{quote}

The derivative of the activation function applied over the input
vector.
      \end{quote}

    \vspace{1ex}

      Overrides: peach.nn.af.Activation.derivative

    \end{boxedminipage}

    \label{object:__delattr__}
    \index{object.\_\_delattr\_\_ \textit{(function)}}

    \vspace{0.5ex}

    \begin{boxedminipage}{\textwidth}

    \raggedright \textbf{\_\_delattr\_\_}(\textit{...})

    \vspace{-1.5ex}

    \rule{\textwidth}{0.5\fboxrule}

x.{\_}{\_}delattr{\_}{\_}('name') {\textless}=={\textgreater} del x.name
    \vspace{1ex}

    \end{boxedminipage}

    \label{object:__getattribute__}
    \index{object.\_\_getattribute\_\_ \textit{(function)}}

    \vspace{0.5ex}

    \begin{boxedminipage}{\textwidth}

    \raggedright \textbf{\_\_getattribute\_\_}(\textit{...})

    \vspace{-1.5ex}

    \rule{\textwidth}{0.5\fboxrule}

x.{\_}{\_}getattribute{\_}{\_}('name') {\textless}=={\textgreater} x.name
    \vspace{1ex}

    \end{boxedminipage}

    \label{object:__hash__}
    \index{object.\_\_hash\_\_ \textit{(function)}}

    \vspace{0.5ex}

    \begin{boxedminipage}{\textwidth}

    \raggedright \textbf{\_\_hash\_\_}(\textit{x})

    \vspace{-1.5ex}

    \rule{\textwidth}{0.5\fboxrule}

hash(x)
    \vspace{1ex}

    \end{boxedminipage}

    \label{object:__new__}
    \index{object.\_\_new\_\_ \textit{(function)}}

    \vspace{0.5ex}

    \begin{boxedminipage}{\textwidth}

    \raggedright \textbf{\_\_new\_\_}(\textit{T}, \textit{S}, \textit{...})

      \textbf{Return Value}
      \begin{quote}
\begin{alltt}
a new object with type S, a subtype of T
\end{alltt}

      \end{quote}

    \vspace{1ex}

    \end{boxedminipage}

    \label{object:__reduce__}
    \index{object.\_\_reduce\_\_ \textit{(function)}}

    \vspace{0.5ex}

    \begin{boxedminipage}{\textwidth}

    \raggedright \textbf{\_\_reduce\_\_}(\textit{...})

    \vspace{-1.5ex}

    \rule{\textwidth}{0.5\fboxrule}

helper for pickle
    \vspace{1ex}

    \end{boxedminipage}

    \label{object:__reduce_ex__}
    \index{object.\_\_reduce\_ex\_\_ \textit{(function)}}

    \vspace{0.5ex}

    \begin{boxedminipage}{\textwidth}

    \raggedright \textbf{\_\_reduce\_ex\_\_}(\textit{...})

    \vspace{-1.5ex}

    \rule{\textwidth}{0.5\fboxrule}

helper for pickle
    \vspace{1ex}

    \end{boxedminipage}

    \label{object:__repr__}
    \index{object.\_\_repr\_\_ \textit{(function)}}

    \vspace{0.5ex}

    \begin{boxedminipage}{\textwidth}

    \raggedright \textbf{\_\_repr\_\_}(\textit{x})

    \vspace{-1.5ex}

    \rule{\textwidth}{0.5\fboxrule}

repr(x)
    \vspace{1ex}

    \end{boxedminipage}

    \label{object:__setattr__}
    \index{object.\_\_setattr\_\_ \textit{(function)}}

    \vspace{0.5ex}

    \begin{boxedminipage}{\textwidth}

    \raggedright \textbf{\_\_setattr\_\_}(\textit{...})

    \vspace{-1.5ex}

    \rule{\textwidth}{0.5\fboxrule}

x.{\_}{\_}setattr{\_}{\_}('name', value) {\textless}=={\textgreater} x.name = value
    \vspace{1ex}

    \end{boxedminipage}

    \label{object:__str__}
    \index{object.\_\_str\_\_ \textit{(function)}}

    \vspace{0.5ex}

    \begin{boxedminipage}{\textwidth}

    \raggedright \textbf{\_\_str\_\_}(\textit{x})

    \vspace{-1.5ex}

    \rule{\textwidth}{0.5\fboxrule}

str(x)
    \vspace{1ex}

    \end{boxedminipage}


%%%%%%%%%%%%%%%%%%%%%%%%%%%%%%%%%%%%%%%%%%%%%%%%%%%%%%%%%%%%%%%%%%%%%%%%%%%
%%                              Properties                               %%
%%%%%%%%%%%%%%%%%%%%%%%%%%%%%%%%%%%%%%%%%%%%%%%%%%%%%%%%%%%%%%%%%%%%%%%%%%%

  \subsubsection{Properties}

\begin{longtable}{|p{.30\textwidth}|p{.62\textwidth}|l}
\cline{1-2}
\cline{1-2} \centering \textbf{Name} & \centering \textbf{Description}& \\
\cline{1-2}
\endhead\cline{1-2}\multicolumn{3}{r}{\small\textit{continued on next page}}\\\endfoot\cline{1-2}
\endlastfoot\raggedright \_\-\_\-c\-l\-a\-s\-s\-\_\-\_\- & \raggedright \textbf{Value:} 
{\tt {\textless}attribute '\_\_class\_\_' of 'object' objects{\textgreater}}&\\
\cline{1-2}
\end{longtable}


%%%%%%%%%%%%%%%%%%%%%%%%%%%%%%%%%%%%%%%%%%%%%%%%%%%%%%%%%%%%%%%%%%%%%%%%%%%
%%                          Instance Variables                           %%
%%%%%%%%%%%%%%%%%%%%%%%%%%%%%%%%%%%%%%%%%%%%%%%%%%%%%%%%%%%%%%%%%%%%%%%%%%%

  \subsubsection{Instance Variables}

\begin{longtable}{|p{.30\textwidth}|p{.62\textwidth}|l}
\cline{1-2}
\cline{1-2} \centering \textbf{Name} & \centering \textbf{Description}& \\
\cline{1-2}
\endhead\cline{1-2}\multicolumn{3}{r}{\small\textit{continued on next page}}\\\endfoot\cline{1-2}
\endlastfoot\raggedright d\- & An alias to the derivative of the function.&\\
\cline{1-2}
\end{longtable}

    \index{peach \textit{(package)}!peach.nn \textit{(package)}!peach.nn.af \textit{(module)}!peach.nn.af.ArcTan \textit{(class)}|)}

%%%%%%%%%%%%%%%%%%%%%%%%%%%%%%%%%%%%%%%%%%%%%%%%%%%%%%%%%%%%%%%%%%%%%%%%%%%
%%                           Class Description                           %%
%%%%%%%%%%%%%%%%%%%%%%%%%%%%%%%%%%%%%%%%%%%%%%%%%%%%%%%%%%%%%%%%%%%%%%%%%%%

    \index{peach \textit{(package)}!peach.nn \textit{(package)}!peach.nn.af \textit{(module)}!peach.nn.af.TanH \textit{(class)}|(}
\subsection{Class TanH}

    \label{peach:nn:af:TanH}
\begin{tabular}{cccccccc}
% Line for object, linespec=[False, False]
\multicolumn{2}{r}{\settowidth{\BCL}{object}\multirow{2}{\BCL}{object}}
&&
&&
  \\\cline{3-3}
  &&\multicolumn{1}{c|}{}
&&
&&
  \\
% Line for peach.nn.af.Activation, linespec=[False]
\multicolumn{4}{r}{\settowidth{\BCL}{peach.nn.af.Activation}\multirow{2}{\BCL}{peach.nn.af.Activation}}
&&
  \\\cline{5-5}
  &&&&\multicolumn{1}{c|}{}
&&
  \\
&&&&\multicolumn{2}{l}{\textbf{peach.nn.af.TanH}}
\end{tabular}


Hyperbolic tangent activation function

%%%%%%%%%%%%%%%%%%%%%%%%%%%%%%%%%%%%%%%%%%%%%%%%%%%%%%%%%%%%%%%%%%%%%%%%%%%
%%                                Methods                                %%
%%%%%%%%%%%%%%%%%%%%%%%%%%%%%%%%%%%%%%%%%%%%%%%%%%%%%%%%%%%%%%%%%%%%%%%%%%%

  \subsubsection{Methods}

    \vspace{0.5ex}

    \begin{boxedminipage}{\textwidth}

    \raggedright \textbf{\_\_init\_\_}(\textit{self}, \textit{a}=\texttt{1.0}, \textit{x0}=\texttt{0.0})

    \vspace{-1.5ex}

    \rule{\textwidth}{0.5\fboxrule}

Initializes the object
    \vspace{1ex}

      \textbf{Parameters}
      \begin{quote}
        \begin{Ventry}{xx}

          \item[a]


The slope of the function in the center \texttt{x0}. Defaults to 1.0.
          \item[x0]


The center of the sigmoid. Defaults to 0.0.
        \end{Ventry}

      \end{quote}

    \vspace{1ex}

      Overrides: peach.nn.af.Activation.\_\_init\_\_

    \end{boxedminipage}

    \vspace{0.5ex}

    \begin{boxedminipage}{\textwidth}

    \raggedright \textbf{\_\_call\_\_}(\textit{self}, \textit{x})

    \vspace{-1.5ex}

    \rule{\textwidth}{0.5\fboxrule}

Call interface to the object.

This method applies the activation function over a vector of activation
potentials, and returns the results.
    \vspace{1ex}

      \textbf{Parameters}
      \begin{quote}
        \begin{Ventry}{x}

          \item[x]


A real number or a vector of real numbers representing the
activation potential of a neuron or a layer of neurons.
        \end{Ventry}

      \end{quote}

    \vspace{1ex}

      \textbf{Return Value}
      \begin{quote}

The activation function applied over the input vector.
      \end{quote}

    \vspace{1ex}

      Overrides: peach.nn.af.Activation.\_\_call\_\_

    \end{boxedminipage}

    \vspace{0.5ex}

    \begin{boxedminipage}{\textwidth}

    \raggedright \textbf{derivative}(\textit{self}, \textit{x})

    \vspace{-1.5ex}

    \rule{\textwidth}{0.5\fboxrule}

The function derivative.
    \vspace{1ex}

      \textbf{Parameters}
      \begin{quote}
        \begin{Ventry}{x}

          \item[x]


A real number or a vector of real numbers representing the
activation potential of a neuron or a layer of neurons.
        \end{Ventry}

      \end{quote}

    \vspace{1ex}

      \textbf{Return Value}
      \begin{quote}

The derivative of the activation function applied over the input
vector.
      \end{quote}

    \vspace{1ex}

      Overrides: peach.nn.af.Activation.derivative

    \end{boxedminipage}

    \label{object:__delattr__}
    \index{object.\_\_delattr\_\_ \textit{(function)}}

    \vspace{0.5ex}

    \begin{boxedminipage}{\textwidth}

    \raggedright \textbf{\_\_delattr\_\_}(\textit{...})

    \vspace{-1.5ex}

    \rule{\textwidth}{0.5\fboxrule}

x.{\_}{\_}delattr{\_}{\_}('name') {\textless}=={\textgreater} del x.name
    \vspace{1ex}

    \end{boxedminipage}

    \label{object:__getattribute__}
    \index{object.\_\_getattribute\_\_ \textit{(function)}}

    \vspace{0.5ex}

    \begin{boxedminipage}{\textwidth}

    \raggedright \textbf{\_\_getattribute\_\_}(\textit{...})

    \vspace{-1.5ex}

    \rule{\textwidth}{0.5\fboxrule}

x.{\_}{\_}getattribute{\_}{\_}('name') {\textless}=={\textgreater} x.name
    \vspace{1ex}

    \end{boxedminipage}

    \label{object:__hash__}
    \index{object.\_\_hash\_\_ \textit{(function)}}

    \vspace{0.5ex}

    \begin{boxedminipage}{\textwidth}

    \raggedright \textbf{\_\_hash\_\_}(\textit{x})

    \vspace{-1.5ex}

    \rule{\textwidth}{0.5\fboxrule}

hash(x)
    \vspace{1ex}

    \end{boxedminipage}

    \label{object:__new__}
    \index{object.\_\_new\_\_ \textit{(function)}}

    \vspace{0.5ex}

    \begin{boxedminipage}{\textwidth}

    \raggedright \textbf{\_\_new\_\_}(\textit{T}, \textit{S}, \textit{...})

      \textbf{Return Value}
      \begin{quote}
\begin{alltt}
a new object with type S, a subtype of T
\end{alltt}

      \end{quote}

    \vspace{1ex}

    \end{boxedminipage}

    \label{object:__reduce__}
    \index{object.\_\_reduce\_\_ \textit{(function)}}

    \vspace{0.5ex}

    \begin{boxedminipage}{\textwidth}

    \raggedright \textbf{\_\_reduce\_\_}(\textit{...})

    \vspace{-1.5ex}

    \rule{\textwidth}{0.5\fboxrule}

helper for pickle
    \vspace{1ex}

    \end{boxedminipage}

    \label{object:__reduce_ex__}
    \index{object.\_\_reduce\_ex\_\_ \textit{(function)}}

    \vspace{0.5ex}

    \begin{boxedminipage}{\textwidth}

    \raggedright \textbf{\_\_reduce\_ex\_\_}(\textit{...})

    \vspace{-1.5ex}

    \rule{\textwidth}{0.5\fboxrule}

helper for pickle
    \vspace{1ex}

    \end{boxedminipage}

    \label{object:__repr__}
    \index{object.\_\_repr\_\_ \textit{(function)}}

    \vspace{0.5ex}

    \begin{boxedminipage}{\textwidth}

    \raggedright \textbf{\_\_repr\_\_}(\textit{x})

    \vspace{-1.5ex}

    \rule{\textwidth}{0.5\fboxrule}

repr(x)
    \vspace{1ex}

    \end{boxedminipage}

    \label{object:__setattr__}
    \index{object.\_\_setattr\_\_ \textit{(function)}}

    \vspace{0.5ex}

    \begin{boxedminipage}{\textwidth}

    \raggedright \textbf{\_\_setattr\_\_}(\textit{...})

    \vspace{-1.5ex}

    \rule{\textwidth}{0.5\fboxrule}

x.{\_}{\_}setattr{\_}{\_}('name', value) {\textless}=={\textgreater} x.name = value
    \vspace{1ex}

    \end{boxedminipage}

    \label{object:__str__}
    \index{object.\_\_str\_\_ \textit{(function)}}

    \vspace{0.5ex}

    \begin{boxedminipage}{\textwidth}

    \raggedright \textbf{\_\_str\_\_}(\textit{x})

    \vspace{-1.5ex}

    \rule{\textwidth}{0.5\fboxrule}

str(x)
    \vspace{1ex}

    \end{boxedminipage}


%%%%%%%%%%%%%%%%%%%%%%%%%%%%%%%%%%%%%%%%%%%%%%%%%%%%%%%%%%%%%%%%%%%%%%%%%%%
%%                              Properties                               %%
%%%%%%%%%%%%%%%%%%%%%%%%%%%%%%%%%%%%%%%%%%%%%%%%%%%%%%%%%%%%%%%%%%%%%%%%%%%

  \subsubsection{Properties}

\begin{longtable}{|p{.30\textwidth}|p{.62\textwidth}|l}
\cline{1-2}
\cline{1-2} \centering \textbf{Name} & \centering \textbf{Description}& \\
\cline{1-2}
\endhead\cline{1-2}\multicolumn{3}{r}{\small\textit{continued on next page}}\\\endfoot\cline{1-2}
\endlastfoot\raggedright \_\-\_\-c\-l\-a\-s\-s\-\_\-\_\- & \raggedright \textbf{Value:} 
{\tt {\textless}attribute '\_\_class\_\_' of 'object' objects{\textgreater}}&\\
\cline{1-2}
\end{longtable}


%%%%%%%%%%%%%%%%%%%%%%%%%%%%%%%%%%%%%%%%%%%%%%%%%%%%%%%%%%%%%%%%%%%%%%%%%%%
%%                          Instance Variables                           %%
%%%%%%%%%%%%%%%%%%%%%%%%%%%%%%%%%%%%%%%%%%%%%%%%%%%%%%%%%%%%%%%%%%%%%%%%%%%

  \subsubsection{Instance Variables}

\begin{longtable}{|p{.30\textwidth}|p{.62\textwidth}|l}
\cline{1-2}
\cline{1-2} \centering \textbf{Name} & \centering \textbf{Description}& \\
\cline{1-2}
\endhead\cline{1-2}\multicolumn{3}{r}{\small\textit{continued on next page}}\\\endfoot\cline{1-2}
\endlastfoot\raggedright d\- & An alias to the derivative of the function.&\\
\cline{1-2}
\end{longtable}

    \index{peach \textit{(package)}!peach.nn \textit{(package)}!peach.nn.af \textit{(module)}!peach.nn.af.TanH \textit{(class)}|)}
    \index{peach \textit{(package)}!peach.nn \textit{(package)}!peach.nn.af \textit{(module)}|)}

%
% API Documentation for Peach - Computational Intelligence for Python
% Module peach.nn.base
%
% Generated by epydoc 3.0.1
% [Mon Jan 24 15:39:51 2011]
%

%%%%%%%%%%%%%%%%%%%%%%%%%%%%%%%%%%%%%%%%%%%%%%%%%%%%%%%%%%%%%%%%%%%%%%%%%%%
%%                          Module Description                           %%
%%%%%%%%%%%%%%%%%%%%%%%%%%%%%%%%%%%%%%%%%%%%%%%%%%%%%%%%%%%%%%%%%%%%%%%%%%%

    \index{peach \textit{(package)}!peach.nn \textit{(package)}!peach.nn.base \textit{(module)}|(}
\section{Module peach.nn.base}

    \label{peach:nn:base}

Basic definitions for layers of neurons.

This subpackage implements the basic classes used with neural networks. A neural
network is basically implemented as a layer of neurons. To speed things up, a
layer is implemented as a array, where each line represents the weight vector
of a neuron. Further definitions and algorithms are based on this definition.

%%%%%%%%%%%%%%%%%%%%%%%%%%%%%%%%%%%%%%%%%%%%%%%%%%%%%%%%%%%%%%%%%%%%%%%%%%%
%%                               Variables                               %%
%%%%%%%%%%%%%%%%%%%%%%%%%%%%%%%%%%%%%%%%%%%%%%%%%%%%%%%%%%%%%%%%%%%%%%%%%%%

  \subsection{Variables}

    \vspace{-1cm}
\hspace{\varindent}\begin{longtable}{|p{\varnamewidth}|p{\vardescrwidth}|l}
\cline{1-2}
\cline{1-2} \centering \textbf{Name} & \centering \textbf{Description}& \\
\cline{1-2}
\endhead\cline{1-2}\multicolumn{3}{r}{\small\textit{continued on next page}}\\\endfoot\cline{1-2}
\endlastfoot\raggedright \_\-\_\-d\-o\-c\-\_\-\_\- & \raggedright \textbf{Value:} 
{\tt \texttt{...}}&\\
\cline{1-2}
\raggedright \_\-\_\-p\-a\-c\-k\-a\-g\-e\-\_\-\_\- & \raggedright \textbf{Value:} 
{\tt \texttt{'}\texttt{peach.nn}\texttt{'}}&\\
\cline{1-2}
\end{longtable}


%%%%%%%%%%%%%%%%%%%%%%%%%%%%%%%%%%%%%%%%%%%%%%%%%%%%%%%%%%%%%%%%%%%%%%%%%%%
%%                           Class Description                           %%
%%%%%%%%%%%%%%%%%%%%%%%%%%%%%%%%%%%%%%%%%%%%%%%%%%%%%%%%%%%%%%%%%%%%%%%%%%%

    \index{peach \textit{(package)}!peach.nn \textit{(package)}!peach.nn.base \textit{(module)}!peach.nn.base.Layer \textit{(class)}|(}
\subsection{Class Layer}

    \label{peach:nn:base:Layer}
\begin{tabular}{cccccc}
% Line for object, linespec=[False]
\multicolumn{2}{r}{\settowidth{\BCL}{object}\multirow{2}{\BCL}{object}}
&&
  \\\cline{3-3}
  &&\multicolumn{1}{c|}{}
&&
  \\
&&\multicolumn{2}{l}{\textbf{peach.nn.base.Layer}}
\end{tabular}

\textbf{Known Subclasses:} peach.nn.nnet.SOM


Base class for neural networks.

This class implements a layer of neurons. It is represented by a array of
real values. Each line of the array represents the weight vector of a
single neuron. If the neurons on the layer are biased, then the first
element of the weight vector is the bias weight, and the bias input is
always valued 1. Also, to each layer is associated an activation function,
that determines if the neuron is fired or not. Please, consult the module
\texttt{af} to see more about activation functions.

In general, this class shoulb be subclassed if you want to use neural nets.
But, as neural nets are very different one from the other, check carefully
the documentation to see if the attributes, properties and methods are
suited to your task.

%%%%%%%%%%%%%%%%%%%%%%%%%%%%%%%%%%%%%%%%%%%%%%%%%%%%%%%%%%%%%%%%%%%%%%%%%%%
%%                                Methods                                %%
%%%%%%%%%%%%%%%%%%%%%%%%%%%%%%%%%%%%%%%%%%%%%%%%%%%%%%%%%%%%%%%%%%%%%%%%%%%

  \subsubsection{Methods}

    \vspace{0.5ex}

\hspace{.8\funcindent}\begin{boxedminipage}{\funcwidth}

    \raggedright \textbf{\_\_init\_\_}(\textit{self}, \textit{shape}, \textit{phi}={\tt {\textless}class 'peach.nn.af.Linear'{\textgreater}}, \textit{bias}={\tt False})

    \vspace{-1.5ex}

    \rule{\textwidth}{0.5\fboxrule}
\setlength{\parskip}{2ex}

Initializes the layer.

A layer is represented by a array where each line is the weight vector
of a single neuron. The first element of the vector is the bias weight,
in case the neuron is biased. Associated with the layer is an activation
function defined in an appropriate way.
\setlength{\parskip}{1ex}
      \textbf{Parameters}
      \vspace{-1ex}

      \begin{quote}
        \begin{Ventry}{xxxxx}

          \item[shape]


Stablishes the size of the layer. It must be a two-tuple of the
format \texttt{(m, n)}, where \texttt{m} is the number of neurons in the
layer, and \texttt{n} is the number of inputs of each neuron. The neurons
in the layer all have the same number of inputs.
          \item[phi]


The activation function. It can be an \texttt{Activation} object (please,
consult the \texttt{af} module) or a standard Python function. In this
case, it must receive a single real value and return a single real
value which determines if the neuron is activated or not. Defaults
to \texttt{Linear}.
          \item[bias]


If \texttt{True}, then the neurons on the layer are biased. That means
that an additional weight is added to each neuron to represent the
bias. If \texttt{False}, no modification is made.
        \end{Ventry}

      \end{quote}

      Overrides: object.\_\_init\_\_

    \end{boxedminipage}

    \label{peach:nn:base:Layer:__getitem__}
    \index{peach \textit{(package)}!peach.nn \textit{(package)}!peach.nn.base \textit{(module)}!peach.nn.base.Layer \textit{(class)}!peach.nn.base.Layer.\_\_getitem\_\_ \textit{(method)}}

    \vspace{0.5ex}

\hspace{.8\funcindent}\begin{boxedminipage}{\funcwidth}

    \raggedright \textbf{\_\_getitem\_\_}(\textit{self}, \textit{n})

    \vspace{-1.5ex}

    \rule{\textwidth}{0.5\fboxrule}
\setlength{\parskip}{2ex}

The \texttt{{[} {]}} get interface.

The input to this method is forwarded to the \texttt{weights} property. That
means that it will return the respective line/element of the weight
array.
\setlength{\parskip}{1ex}
      \textbf{Parameters}
      \vspace{-1ex}

      \begin{quote}
        \begin{Ventry}{x}

          \item[n]


A slice object containing the elements referenced. Since it is
forwarded to an array, it behaves exactly as one.
        \end{Ventry}

      \end{quote}

      \textbf{Return Value}
    \vspace{-1ex}

      \begin{quote}

The element or elements in the referenced indices.
      \end{quote}

    \end{boxedminipage}

    \label{peach:nn:base:Layer:__setitem__}
    \index{peach \textit{(package)}!peach.nn \textit{(package)}!peach.nn.base \textit{(module)}!peach.nn.base.Layer \textit{(class)}!peach.nn.base.Layer.\_\_setitem\_\_ \textit{(method)}}

    \vspace{0.5ex}

\hspace{.8\funcindent}\begin{boxedminipage}{\funcwidth}

    \raggedright \textbf{\_\_setitem\_\_}(\textit{self}, \textit{n}, \textit{w})

    \vspace{-1.5ex}

    \rule{\textwidth}{0.5\fboxrule}
\setlength{\parskip}{2ex}

The \texttt{{[} {]}} set interface.

The inputs to this method are forwarded to the \texttt{weights} property.
That means that it will set the respective line/element of the weight
array.
\setlength{\parskip}{1ex}
      \textbf{Parameters}
      \vspace{-1ex}

      \begin{quote}
        \begin{Ventry}{x}

          \item[n]


A slice object containing the elements referenced. Since it is
forwarded to an array, it behaves exactly as one.
          \item[w]


A value or array of values to be set in the given indices.
        \end{Ventry}

      \end{quote}

    \end{boxedminipage}

    \label{peach:nn:base:Layer:__call__}
    \index{peach \textit{(package)}!peach.nn \textit{(package)}!peach.nn.base \textit{(module)}!peach.nn.base.Layer \textit{(class)}!peach.nn.base.Layer.\_\_call\_\_ \textit{(method)}}

    \vspace{0.5ex}

\hspace{.8\funcindent}\begin{boxedminipage}{\funcwidth}

    \raggedright \textbf{\_\_call\_\_}(\textit{self}, \textit{x})

    \vspace{-1.5ex}

    \rule{\textwidth}{0.5\fboxrule}
\setlength{\parskip}{2ex}

The feedforward method to the layer.

The \texttt{\_\_call\_\_} interface should be called if the answer of the neuron
to a given input vector \texttt{x} is desired. \emph{This method has collateral
effects}, so beware. After the calling of this method, the \texttt{v} and
\texttt{y} properties are set with the activation potential and the answer of
the neurons, respectivelly.
\setlength{\parskip}{1ex}
      \textbf{Parameters}
      \vspace{-1ex}

      \begin{quote}
        \begin{Ventry}{x}

          \item[x]


The input vector to the layer.
        \end{Ventry}

      \end{quote}

      \textbf{Return Value}
    \vspace{-1ex}

      \begin{quote}

The vector containing the answer of every neuron in the layer, in the
respective order.
      \end{quote}

    \end{boxedminipage}


\large{\textbf{\textit{Inherited from object}}}

\begin{quote}
\_\_delattr\_\_(), \_\_format\_\_(), \_\_getattribute\_\_(), \_\_hash\_\_(), \_\_new\_\_(), \_\_reduce\_\_(), \_\_reduce\_ex\_\_(), \_\_repr\_\_(), \_\_setattr\_\_(), \_\_sizeof\_\_(), \_\_str\_\_(), \_\_subclasshook\_\_()
\end{quote}

%%%%%%%%%%%%%%%%%%%%%%%%%%%%%%%%%%%%%%%%%%%%%%%%%%%%%%%%%%%%%%%%%%%%%%%%%%%
%%                              Properties                               %%
%%%%%%%%%%%%%%%%%%%%%%%%%%%%%%%%%%%%%%%%%%%%%%%%%%%%%%%%%%%%%%%%%%%%%%%%%%%

  \subsubsection{Properties}

    \vspace{-1cm}
\hspace{\varindent}\begin{longtable}{|p{\varnamewidth}|p{\vardescrwidth}|l}
\cline{1-2}
\cline{1-2} \centering \textbf{Name} & \centering \textbf{Description}& \\
\cline{1-2}
\endhead\cline{1-2}\multicolumn{3}{r}{\small\textit{continued on next page}}\\\endfoot\cline{1-2}
\endlastfoot\raggedright s\-i\-z\-e\- & &\\
\cline{1-2}
\raggedright i\-n\-p\-u\-t\-s\- & &\\
\cline{1-2}
\raggedright s\-h\-a\-p\-e\- & &\\
\cline{1-2}
\raggedright b\-i\-a\-s\- & &\\
\cline{1-2}
\raggedright w\-e\-i\-g\-h\-t\-s\- & &\\
\cline{1-2}
\raggedright p\-h\-i\- & &\\
\cline{1-2}
\raggedright v\- & &\\
\cline{1-2}
\raggedright y\- & &\\
\cline{1-2}
\multicolumn{2}{|l|}{\textit{Inherited from object}}\\
\multicolumn{2}{|p{\varwidth}|}{\raggedright \_\_class\_\_}\\
\cline{1-2}
\end{longtable}

    \index{peach \textit{(package)}!peach.nn \textit{(package)}!peach.nn.base \textit{(module)}!peach.nn.base.Layer \textit{(class)}|)}
    \index{peach \textit{(package)}!peach.nn \textit{(package)}!peach.nn.base \textit{(module)}|)}

%
% API Documentation for Peach - Computational Intelligence for Python
% Module peach.nn.kmeans
%
% Generated by epydoc 3.0.1
% [Thu Jul 28 16:37:48 2011]
%

%%%%%%%%%%%%%%%%%%%%%%%%%%%%%%%%%%%%%%%%%%%%%%%%%%%%%%%%%%%%%%%%%%%%%%%%%%%
%%                          Module Description                           %%
%%%%%%%%%%%%%%%%%%%%%%%%%%%%%%%%%%%%%%%%%%%%%%%%%%%%%%%%%%%%%%%%%%%%%%%%%%%

    \index{peach \textit{(package)}!peach.nn \textit{(package)}!peach.nn.kmeans \textit{(module)}|(}
\section{Module peach.nn.kmeans}

    \label{peach:nn:kmeans}

K-Means clustering algorithm

This sub-package implements the K-Means clustering algorithm. This algorithm,
given a set of points, finds a set of vectors that best represents a partition
for these points. These vectors represent the center of a cloud of points that
are nearest to them.

This algorithm is one that can be used with radial basis function (RBF) networks
to find the centers of the RBFs. Usually, training a RBFN in two passes -{}- first
positioning them, and then computing their variance.

%%%%%%%%%%%%%%%%%%%%%%%%%%%%%%%%%%%%%%%%%%%%%%%%%%%%%%%%%%%%%%%%%%%%%%%%%%%
%%                               Functions                               %%
%%%%%%%%%%%%%%%%%%%%%%%%%%%%%%%%%%%%%%%%%%%%%%%%%%%%%%%%%%%%%%%%%%%%%%%%%%%

  \subsection{Functions}

    \label{peach:nn:kmeans:ClassByDistance}
    \index{peach \textit{(package)}!peach.nn \textit{(package)}!peach.nn.kmeans \textit{(module)}!peach.nn.kmeans.ClassByDistance \textit{(function)}}

    \vspace{0.5ex}

\hspace{.8\funcindent}\begin{boxedminipage}{\funcwidth}

    \raggedright \textbf{ClassByDistance}(\textit{xs}, \textit{c})

    \vspace{-1.5ex}

    \rule{\textwidth}{0.5\fboxrule}
\setlength{\parskip}{2ex}

Given a set of points and a list of centers, classify the points according
to their euclidian distance to the centers.
\setlength{\parskip}{1ex}
      \textbf{Parameters}
      \vspace{-1ex}

      \begin{quote}
        \begin{Ventry}{xx}

          \item[xs]


Set of points to be classified. They must be given as a list or array of
one-dimensional vectors, one per line.
          \item[c]


Set of centers. Must also be given as a lista or array of
one-dimensional vectors, one per line.
        \end{Ventry}

      \end{quote}

      \textbf{Return Value}
    \vspace{-1ex}

      \begin{quote}

A list of index of the classification. The indices are the position of the
cluster in the given parameters \texttt{c}.
      \end{quote}

    \end{boxedminipage}

    \label{peach:nn:kmeans:ClusterByMean}
    \index{peach \textit{(package)}!peach.nn \textit{(package)}!peach.nn.kmeans \textit{(module)}!peach.nn.kmeans.ClusterByMean \textit{(function)}}

    \vspace{0.5ex}

\hspace{.8\funcindent}\begin{boxedminipage}{\funcwidth}

    \raggedright \textbf{ClusterByMean}(\textit{x})

    \vspace{-1.5ex}

    \rule{\textwidth}{0.5\fboxrule}
\setlength{\parskip}{2ex}

This function computes the center of a cluster by averaging the vectors in
the input set by simply averaging each component.
\setlength{\parskip}{1ex}
      \textbf{Parameters}
      \vspace{-1ex}

      \begin{quote}
        \begin{Ventry}{x}

          \item[x]


Set of points to be clustered. They must be given in the form of a list
or array of one-dimensional points.
        \end{Ventry}

      \end{quote}

      \textbf{Return Value}
    \vspace{-1ex}

      \begin{quote}

A one-dimensional array representing the center of the cluster.
      \end{quote}

    \end{boxedminipage}


%%%%%%%%%%%%%%%%%%%%%%%%%%%%%%%%%%%%%%%%%%%%%%%%%%%%%%%%%%%%%%%%%%%%%%%%%%%
%%                               Variables                               %%
%%%%%%%%%%%%%%%%%%%%%%%%%%%%%%%%%%%%%%%%%%%%%%%%%%%%%%%%%%%%%%%%%%%%%%%%%%%

  \subsection{Variables}

    \vspace{-1cm}
\hspace{\varindent}\begin{longtable}{|p{\varnamewidth}|p{\vardescrwidth}|l}
\cline{1-2}
\cline{1-2} \centering \textbf{Name} & \centering \textbf{Description}& \\
\cline{1-2}
\endhead\cline{1-2}\multicolumn{3}{r}{\small\textit{continued on next page}}\\\endfoot\cline{1-2}
\endlastfoot\raggedright \_\-\_\-d\-o\-c\-\_\-\_\- & \raggedright \textbf{Value:} 
{\tt \texttt{...}}&\\
\cline{1-2}
\raggedright \_\-\_\-p\-a\-c\-k\-a\-g\-e\-\_\-\_\- & \raggedright \textbf{Value:} 
{\tt \texttt{'}\texttt{peach.nn}\texttt{'}}&\\
\cline{1-2}
\end{longtable}


%%%%%%%%%%%%%%%%%%%%%%%%%%%%%%%%%%%%%%%%%%%%%%%%%%%%%%%%%%%%%%%%%%%%%%%%%%%
%%                           Class Description                           %%
%%%%%%%%%%%%%%%%%%%%%%%%%%%%%%%%%%%%%%%%%%%%%%%%%%%%%%%%%%%%%%%%%%%%%%%%%%%

    \index{peach \textit{(package)}!peach.nn \textit{(package)}!peach.nn.kmeans \textit{(module)}!peach.nn.kmeans.KMeans \textit{(class)}|(}
\subsection{Class KMeans}

    \label{peach:nn:kmeans:KMeans}
\begin{tabular}{cccccc}
% Line for object, linespec=[False]
\multicolumn{2}{r}{\settowidth{\BCL}{object}\multirow{2}{\BCL}{object}}
&&
  \\\cline{3-3}
  &&\multicolumn{1}{c|}{}
&&
  \\
&&\multicolumn{2}{l}{\textbf{peach.nn.kmeans.KMeans}}
\end{tabular}


K-Means clustering algorithm

This class implements the known and very used K-Means clustering algorithm.
In this algorithm, the centers of the clusters are selected randomly. The
points on the training set are classified in accord to their closeness to
the cluster centers. This changes the positions of the centers, which
changes the classification of the points. This iteration is repeated until
no changes occur.

Traditional K-Means implementations classify the points in the training set
according to the euclidian distance to the centers, and centers are computed
as the average of the points associated to it. This is the default behaviour
of this implementation, but it is configurable. Please, read below for more
detail.

%%%%%%%%%%%%%%%%%%%%%%%%%%%%%%%%%%%%%%%%%%%%%%%%%%%%%%%%%%%%%%%%%%%%%%%%%%%
%%                                Methods                                %%
%%%%%%%%%%%%%%%%%%%%%%%%%%%%%%%%%%%%%%%%%%%%%%%%%%%%%%%%%%%%%%%%%%%%%%%%%%%

  \subsubsection{Methods}

    \vspace{0.5ex}

\hspace{.8\funcindent}\begin{boxedminipage}{\funcwidth}

    \raggedright \textbf{\_\_init\_\_}(\textit{self}, \textit{training\_set}, \textit{nclusters}, \textit{classifier}={\tt {\textless}function ClassByDistance at 0x973d10c{\textgreater}}, \textit{clusterer}={\tt {\textless}function ClusterByMean at 0x973d294{\textgreater}})

    \vspace{-1.5ex}

    \rule{\textwidth}{0.5\fboxrule}
\setlength{\parskip}{2ex}

Initializes the algorithm.
\setlength{\parskip}{1ex}
      \textbf{Parameters}
      \vspace{-1ex}

      \begin{quote}
        \begin{Ventry}{xxxxxxxxxxxx}

          \item[training\_set]


A list or array of vectors containing the data to be classified.
Each of the vectors in this list \emph{must} have the same dimension, or
the algorithm won't behave correctly. Notice that each vector can be
given as a tuple -{}- internally, everything is converted to arrays.
          \item[nclusters]


The number of clusters to be found. This must be, of course, bigger
than 1. These represent the number of centers found once the
algorithm terminates.
          \item[classifier]


A function that classifies each of the points in the training set.
This function receives the training set and a list of centers, and
classify each of the points according to the given metric. Please,
look at the documentation on these functions for more information.
Its default value is %
\raisebox{1em}{\hypertarget{id2}{}}\hyperlink{id1}{\textbf{\color{red}``}}ClassByDistance` , which uses euclidian
distance as metric.
          \item[clusterer]


A function that computes the center of the cluster, given a set of
points. This function receives a list of points and returns the
vector representing the cluster. For more information, look at the
documentation for these functions. Its default value is
\texttt{ClusterByMean}, in which the cluster is represented by the mean
value of the vectors.
        \end{Ventry}

      \end{quote}

      Overrides: object.\_\_init\_\_

    \end{boxedminipage}

    \label{peach:nn:kmeans:KMeans:step}
    \index{peach \textit{(package)}!peach.nn \textit{(package)}!peach.nn.kmeans \textit{(module)}!peach.nn.kmeans.KMeans \textit{(class)}!peach.nn.kmeans.KMeans.step \textit{(method)}}

    \vspace{0.5ex}

\hspace{.8\funcindent}\begin{boxedminipage}{\funcwidth}

    \raggedright \textbf{step}(\textit{self})

    \vspace{-1.5ex}

    \rule{\textwidth}{0.5\fboxrule}
\setlength{\parskip}{2ex}

This method runs one step of the algorithm. It might be useful to track
the changes in the parameters.
\setlength{\parskip}{1ex}
      \textbf{Return Value}
    \vspace{-1ex}

      \begin{quote}

The computed centers for this iteration.
      \end{quote}

    \end{boxedminipage}

    \label{peach:nn:kmeans:KMeans:__call__}
    \index{peach \textit{(package)}!peach.nn \textit{(package)}!peach.nn.kmeans \textit{(module)}!peach.nn.kmeans.KMeans \textit{(class)}!peach.nn.kmeans.KMeans.\_\_call\_\_ \textit{(method)}}

    \vspace{0.5ex}

\hspace{.8\funcindent}\begin{boxedminipage}{\funcwidth}

    \raggedright \textbf{\_\_call\_\_}(\textit{self}, \textit{imax}={\tt 20})

    \vspace{-1.5ex}

    \rule{\textwidth}{0.5\fboxrule}
\setlength{\parskip}{2ex}

The \texttt{\_\_call\_\_} interface is used to run the algorithm until
convergence is found.
\setlength{\parskip}{1ex}
      \textbf{Parameters}
      \vspace{-1ex}

      \begin{quote}
        \begin{Ventry}{xxxx}

          \item[imax]


Specifies the maximum number of iterations admitted in the execution
of the algorithm. It defaults to 20.
        \end{Ventry}

      \end{quote}

      \textbf{Return Value}
    \vspace{-1ex}

      \begin{quote}

An array containing, at each line, the vectors representing the
centers of the clustered regions.
      \end{quote}

    \end{boxedminipage}


\large{\textbf{\textit{Inherited from object}}}

\begin{quote}
\_\_delattr\_\_(), \_\_format\_\_(), \_\_getattribute\_\_(), \_\_hash\_\_(), \_\_new\_\_(), \_\_reduce\_\_(), \_\_reduce\_ex\_\_(), \_\_repr\_\_(), \_\_setattr\_\_(), \_\_sizeof\_\_(), \_\_str\_\_(), \_\_subclasshook\_\_()
\end{quote}

%%%%%%%%%%%%%%%%%%%%%%%%%%%%%%%%%%%%%%%%%%%%%%%%%%%%%%%%%%%%%%%%%%%%%%%%%%%
%%                              Properties                               %%
%%%%%%%%%%%%%%%%%%%%%%%%%%%%%%%%%%%%%%%%%%%%%%%%%%%%%%%%%%%%%%%%%%%%%%%%%%%

  \subsubsection{Properties}

    \vspace{-1cm}
\hspace{\varindent}\begin{longtable}{|p{\varnamewidth}|p{\vardescrwidth}|l}
\cline{1-2}
\cline{1-2} \centering \textbf{Name} & \centering \textbf{Description}& \\
\cline{1-2}
\endhead\cline{1-2}\multicolumn{3}{r}{\small\textit{continued on next page}}\\\endfoot\cline{1-2}
\endlastfoot\raggedright c\- & &\\
\cline{1-2}
\multicolumn{2}{|l|}{\textit{Inherited from object}}\\
\multicolumn{2}{|p{\varwidth}|}{\raggedright \_\_class\_\_}\\
\cline{1-2}
\end{longtable}

    \index{peach \textit{(package)}!peach.nn \textit{(package)}!peach.nn.kmeans \textit{(module)}!peach.nn.kmeans.KMeans \textit{(class)}|)}
    \index{peach \textit{(package)}!peach.nn \textit{(package)}!peach.nn.kmeans \textit{(module)}|)}

%
% API Documentation for Peach - Computational Intelligence for Python
% Module peach.nn.lrules
%
% Generated by epydoc 3.0beta1
% [Mon Dec 21 08:51:37 2009]
%

%%%%%%%%%%%%%%%%%%%%%%%%%%%%%%%%%%%%%%%%%%%%%%%%%%%%%%%%%%%%%%%%%%%%%%%%%%%
%%                          Module Description                           %%
%%%%%%%%%%%%%%%%%%%%%%%%%%%%%%%%%%%%%%%%%%%%%%%%%%%%%%%%%%%%%%%%%%%%%%%%%%%

    \index{peach \textit{(package)}!peach.nn \textit{(package)}!peach.nn.lrules \textit{(module)}|(}
\section{Module peach.nn.lrules}

    \label{peach:nn:lrules}

Learning rules for neural networks and base classes for custom learning.

This sub-package implements learning methods commonly used with neural networks.
There are a lot of different topologies and different learning methods for each
one. It is very difficult to find a consistent framework for defining learning
methods, in consequence. This method defines some base classes that are coupled
with the neural networks that they are supposed to work with. Also, based on
these classes, some of the traditional methods are implemented.

If you want to implement a different learning method, you must subclass the
correct base class. Consult the classes below. Also, pay attention to how the
implementation is expected to behave. Since learning algorithms are usually
somewhat complex, care should be taken to make everything work accordingly.

%%%%%%%%%%%%%%%%%%%%%%%%%%%%%%%%%%%%%%%%%%%%%%%%%%%%%%%%%%%%%%%%%%%%%%%%%%%
%%                               Variables                               %%
%%%%%%%%%%%%%%%%%%%%%%%%%%%%%%%%%%%%%%%%%%%%%%%%%%%%%%%%%%%%%%%%%%%%%%%%%%%

  \subsection{Variables}

\begin{longtable}{|p{.30\textwidth}|p{.62\textwidth}|l}
\cline{1-2}
\cline{1-2} \centering \textbf{Name} & \centering \textbf{Description}& \\
\cline{1-2}
\endhead\cline{1-2}\multicolumn{3}{r}{\small\textit{continued on next page}}\\\endfoot\cline{1-2}
\endlastfoot\raggedright \_\-\_\-d\-o\-c\-\_\-\_\- & \raggedright \textbf{Value:} 
{\tt \texttt{...}}&\\
\cline{1-2}
\raggedright \_\-B\-I\-A\-S\- & \raggedright This constant vector is defined to implement in a fast way the bias of a
neuron, as an input of value 1, stacked over the real input to the neuron.

\textbf{Value:} 
{\tt array([[ 1.]])}&\\
\cline{1-2}
\end{longtable}


%%%%%%%%%%%%%%%%%%%%%%%%%%%%%%%%%%%%%%%%%%%%%%%%%%%%%%%%%%%%%%%%%%%%%%%%%%%
%%                           Class Description                           %%
%%%%%%%%%%%%%%%%%%%%%%%%%%%%%%%%%%%%%%%%%%%%%%%%%%%%%%%%%%%%%%%%%%%%%%%%%%%

    \index{peach \textit{(package)}!peach.nn \textit{(package)}!peach.nn.lrules \textit{(module)}!peach.nn.lrules.FFLearning \textit{(class)}|(}
\subsection{Class FFLearning}

    \label{peach:nn:lrules:FFLearning}
\begin{tabular}{cccccc}
% Line for object, linespec=[False]
\multicolumn{2}{r}{\settowidth{\BCL}{object}\multirow{2}{\BCL}{object}}
&&
  \\\cline{3-3}
  &&\multicolumn{1}{c|}{}
&&
  \\
&&\multicolumn{2}{l}{\textbf{peach.nn.lrules.FFLearning}}
\end{tabular}

\textbf{Known Subclasses:}
peach.nn.lrules.BackPropagation,
    peach.nn.lrules.LMS


Base class for FeedForwarding Multilayer neural networks.

As a base class, this class doesn't do anything. You should subclass this
class if you want to implement a learning method for multilayer networks.

A learning method for a neural net of this kind must deal with a
\texttt{FeedForward} instance. A \texttt{FeedForward} object is a list of \texttt{Layers}
(consulting the documentation of these classes is important!). Each layer is
a bidimensional array, where each line represents the synaptic weights of a
single neuron. So, a multilayer network is actually a three-dimensional
array, if you will. Usually, though, learning methods for this kind of net
propagate some measure of the error from the output back to the input (the
\texttt{BackPropagation} method, for instance).

A class implementing a learning method should have at least two methods:
\begin{quote}
\begin{description}
%[visit_definition_list_item]
\item[{{\_}{\_}init{\_}{\_}}] %[visit_definition]

The \texttt{{\_}{\_}init{\_}{\_}} method should initialize the object. It is in general
used to configure some property of the learning algorithm, such as the
learning rate.

%[depart_definition]
%[depart_definition_list_item]
%[visit_definition_list_item]
\item[{{\_}{\_}call{\_}{\_}}] %[visit_definition]

The \texttt{{\_}{\_}call{\_}{\_}} interface is how the method should interact with the
neural network. It should have the following signature:
\begin{quote}{\ttfamily \raggedright \noindent
{\_}{\_}call{\_}{\_}(self,~nn,~x,~d)
}\end{quote}

where \texttt{nn} is the \texttt{FeedForward} instance to be modified \emph{in loco},
\texttt{x} is the input vector and \texttt{d} is the desired response of the net
for that particular input vector. It should return nothing.

%[depart_definition]
%[depart_definition_list_item]
\end{description}
\end{quote}

%%%%%%%%%%%%%%%%%%%%%%%%%%%%%%%%%%%%%%%%%%%%%%%%%%%%%%%%%%%%%%%%%%%%%%%%%%%
%%                                Methods                                %%
%%%%%%%%%%%%%%%%%%%%%%%%%%%%%%%%%%%%%%%%%%%%%%%%%%%%%%%%%%%%%%%%%%%%%%%%%%%

  \subsubsection{Methods}

    \label{peach:nn:lrules:FFLearning:__call__}
    \index{peach \textit{(package)}!peach.nn \textit{(package)}!peach.nn.lrules \textit{(module)}!peach.nn.lrules.FFLearning \textit{(class)}!peach.nn.lrules.FFLearning.\_\_call\_\_ \textit{(method)}}

    \vspace{0.5ex}

    \begin{boxedminipage}{\textwidth}

    \raggedright \textbf{\_\_call\_\_}(\textit{self}, \textit{nn}, \textit{x}, \textit{d})

    \vspace{-1.5ex}

    \rule{\textwidth}{0.5\fboxrule}

The \texttt{{\_}{\_}call{\_}{\_}} interface.

Read the documentation for this class for more information. A call to
the class should have the following parameters:
    \vspace{1ex}

      \textbf{Parameters}
      \begin{quote}
        \begin{Ventry}{xx}

          \item[nn]


A \texttt{FeedForward} neural network instance that is going to be
modified by the learning algorithm. The modification is made \emph{in
loco}, that is, the synaptic weights of \texttt{nn} should be modified
in place, and not returned from this function.
          \item[x]


The input vector from the training set.
          \item[d]


The desired response for the given input vector.
        \end{Ventry}

      \end{quote}

    \vspace{1ex}

    \end{boxedminipage}

    \label{object:__delattr__}
    \index{object.\_\_delattr\_\_ \textit{(function)}}

    \vspace{0.5ex}

    \begin{boxedminipage}{\textwidth}

    \raggedright \textbf{\_\_delattr\_\_}(\textit{...})

    \vspace{-1.5ex}

    \rule{\textwidth}{0.5\fboxrule}

x.{\_}{\_}delattr{\_}{\_}('name') {\textless}=={\textgreater} del x.name
    \vspace{1ex}

    \end{boxedminipage}

    \label{object:__getattribute__}
    \index{object.\_\_getattribute\_\_ \textit{(function)}}

    \vspace{0.5ex}

    \begin{boxedminipage}{\textwidth}

    \raggedright \textbf{\_\_getattribute\_\_}(\textit{...})

    \vspace{-1.5ex}

    \rule{\textwidth}{0.5\fboxrule}

x.{\_}{\_}getattribute{\_}{\_}('name') {\textless}=={\textgreater} x.name
    \vspace{1ex}

    \end{boxedminipage}

    \label{object:__hash__}
    \index{object.\_\_hash\_\_ \textit{(function)}}

    \vspace{0.5ex}

    \begin{boxedminipage}{\textwidth}

    \raggedright \textbf{\_\_hash\_\_}(\textit{x})

    \vspace{-1.5ex}

    \rule{\textwidth}{0.5\fboxrule}

hash(x)
    \vspace{1ex}

    \end{boxedminipage}

    \label{object:__init__}
    \index{object.\_\_init\_\_ \textit{(function)}}

    \vspace{0.5ex}

    \begin{boxedminipage}{\textwidth}

    \raggedright \textbf{\_\_init\_\_}(\textit{...})

    \vspace{-1.5ex}

    \rule{\textwidth}{0.5\fboxrule}

x.{\_}{\_}init{\_}{\_}(...) initializes x; see x.{\_}{\_}class{\_}{\_}.{\_}{\_}doc{\_}{\_} for signature
    \vspace{1ex}

    \end{boxedminipage}

    \label{object:__new__}
    \index{object.\_\_new\_\_ \textit{(function)}}

    \vspace{0.5ex}

    \begin{boxedminipage}{\textwidth}

    \raggedright \textbf{\_\_new\_\_}(\textit{T}, \textit{S}, \textit{...})

      \textbf{Return Value}
      \begin{quote}
\begin{alltt}
a new object with type S, a subtype of T
\end{alltt}

      \end{quote}

    \vspace{1ex}

    \end{boxedminipage}

    \label{object:__reduce__}
    \index{object.\_\_reduce\_\_ \textit{(function)}}

    \vspace{0.5ex}

    \begin{boxedminipage}{\textwidth}

    \raggedright \textbf{\_\_reduce\_\_}(\textit{...})

    \vspace{-1.5ex}

    \rule{\textwidth}{0.5\fboxrule}

helper for pickle
    \vspace{1ex}

    \end{boxedminipage}

    \label{object:__reduce_ex__}
    \index{object.\_\_reduce\_ex\_\_ \textit{(function)}}

    \vspace{0.5ex}

    \begin{boxedminipage}{\textwidth}

    \raggedright \textbf{\_\_reduce\_ex\_\_}(\textit{...})

    \vspace{-1.5ex}

    \rule{\textwidth}{0.5\fboxrule}

helper for pickle
    \vspace{1ex}

    \end{boxedminipage}

    \label{object:__repr__}
    \index{object.\_\_repr\_\_ \textit{(function)}}

    \vspace{0.5ex}

    \begin{boxedminipage}{\textwidth}

    \raggedright \textbf{\_\_repr\_\_}(\textit{x})

    \vspace{-1.5ex}

    \rule{\textwidth}{0.5\fboxrule}

repr(x)
    \vspace{1ex}

    \end{boxedminipage}

    \label{object:__setattr__}
    \index{object.\_\_setattr\_\_ \textit{(function)}}

    \vspace{0.5ex}

    \begin{boxedminipage}{\textwidth}

    \raggedright \textbf{\_\_setattr\_\_}(\textit{...})

    \vspace{-1.5ex}

    \rule{\textwidth}{0.5\fboxrule}

x.{\_}{\_}setattr{\_}{\_}('name', value) {\textless}=={\textgreater} x.name = value
    \vspace{1ex}

    \end{boxedminipage}

    \label{object:__str__}
    \index{object.\_\_str\_\_ \textit{(function)}}

    \vspace{0.5ex}

    \begin{boxedminipage}{\textwidth}

    \raggedright \textbf{\_\_str\_\_}(\textit{x})

    \vspace{-1.5ex}

    \rule{\textwidth}{0.5\fboxrule}

str(x)
    \vspace{1ex}

    \end{boxedminipage}


%%%%%%%%%%%%%%%%%%%%%%%%%%%%%%%%%%%%%%%%%%%%%%%%%%%%%%%%%%%%%%%%%%%%%%%%%%%
%%                              Properties                               %%
%%%%%%%%%%%%%%%%%%%%%%%%%%%%%%%%%%%%%%%%%%%%%%%%%%%%%%%%%%%%%%%%%%%%%%%%%%%

  \subsubsection{Properties}

\begin{longtable}{|p{.30\textwidth}|p{.62\textwidth}|l}
\cline{1-2}
\cline{1-2} \centering \textbf{Name} & \centering \textbf{Description}& \\
\cline{1-2}
\endhead\cline{1-2}\multicolumn{3}{r}{\small\textit{continued on next page}}\\\endfoot\cline{1-2}
\endlastfoot\raggedright \_\-\_\-c\-l\-a\-s\-s\-\_\-\_\- & \raggedright \textbf{Value:} 
{\tt {\textless}attribute '\_\_class\_\_' of 'object' objects{\textgreater}}&\\
\cline{1-2}
\end{longtable}

    \index{peach \textit{(package)}!peach.nn \textit{(package)}!peach.nn.lrules \textit{(module)}!peach.nn.lrules.FFLearning \textit{(class)}|)}

%%%%%%%%%%%%%%%%%%%%%%%%%%%%%%%%%%%%%%%%%%%%%%%%%%%%%%%%%%%%%%%%%%%%%%%%%%%
%%                           Class Description                           %%
%%%%%%%%%%%%%%%%%%%%%%%%%%%%%%%%%%%%%%%%%%%%%%%%%%%%%%%%%%%%%%%%%%%%%%%%%%%

    \index{peach \textit{(package)}!peach.nn \textit{(package)}!peach.nn.lrules \textit{(module)}!peach.nn.lrules.LMS \textit{(class)}|(}
\subsection{Class LMS}

    \label{peach:nn:lrules:LMS}
\begin{tabular}{cccccccc}
% Line for object, linespec=[False, False]
\multicolumn{2}{r}{\settowidth{\BCL}{object}\multirow{2}{\BCL}{object}}
&&
&&
  \\\cline{3-3}
  &&\multicolumn{1}{c|}{}
&&
&&
  \\
% Line for peach.nn.lrules.FFLearning, linespec=[False]
\multicolumn{4}{r}{\settowidth{\BCL}{peach.nn.lrules.FFLearning}\multirow{2}{\BCL}{peach.nn.lrules.FFLearning}}
&&
  \\\cline{5-5}
  &&&&\multicolumn{1}{c|}{}
&&
  \\
&&&&\multicolumn{2}{l}{\textbf{peach.nn.lrules.LMS}}
\end{tabular}


The Least-Mean-Square (LMS) learning method.

The LMS method is a very simple method of learning, thoroughly described in
virtually every book about the subject. Please, consult a good book on
neural networks for more information. This implementation tries to use the
\texttt{numpy} routines as much as possible for better efficiency.

%%%%%%%%%%%%%%%%%%%%%%%%%%%%%%%%%%%%%%%%%%%%%%%%%%%%%%%%%%%%%%%%%%%%%%%%%%%
%%                                Methods                                %%
%%%%%%%%%%%%%%%%%%%%%%%%%%%%%%%%%%%%%%%%%%%%%%%%%%%%%%%%%%%%%%%%%%%%%%%%%%%

  \subsubsection{Methods}

    \vspace{0.5ex}

    \begin{boxedminipage}{\textwidth}

    \raggedright \textbf{\_\_init\_\_}(\textit{self}, \textit{lrate}=\texttt{0.05})

    \vspace{-1.5ex}

    \rule{\textwidth}{0.5\fboxrule}

Initializes the object.
    \vspace{1ex}

      \textbf{Parameters}
      \begin{quote}
        \begin{Ventry}{xxxxx}

          \item[lrate]


Learning rate to be used in the algorithm. Defaults to 0.05.
        \end{Ventry}

      \end{quote}

    \vspace{1ex}

      Overrides: object.\_\_init\_\_

    \end{boxedminipage}

    \vspace{0.5ex}

    \begin{boxedminipage}{\textwidth}

    \raggedright \textbf{\_\_call\_\_}(\textit{self}, \textit{nn}, \textit{x}, \textit{d})

    \vspace{-1.5ex}

    \rule{\textwidth}{0.5\fboxrule}

The \texttt{{\_}{\_}call{\_}{\_}} interface.

The learning implementation. Read the documentation for the base class
for more information. A call to the class should have the following
parameters:
    \vspace{1ex}

      \textbf{Parameters}
      \begin{quote}
        \begin{Ventry}{xx}

          \item[nn]


A \texttt{FeedForward} neural network instance that is going to be
modified by the learning algorithm. The modification is made \emph{in
loco}, that is, the synaptic weights of \texttt{nn} should be modified
in place, and not returned from this function.
          \item[x]


The input vector from the training set.
          \item[d]


The desired response for the given input vector.
        \end{Ventry}

      \end{quote}

    \vspace{1ex}

      Overrides: peach.nn.lrules.FFLearning.\_\_call\_\_

    \end{boxedminipage}

    \label{object:__delattr__}
    \index{object.\_\_delattr\_\_ \textit{(function)}}

    \vspace{0.5ex}

    \begin{boxedminipage}{\textwidth}

    \raggedright \textbf{\_\_delattr\_\_}(\textit{...})

    \vspace{-1.5ex}

    \rule{\textwidth}{0.5\fboxrule}

x.{\_}{\_}delattr{\_}{\_}('name') {\textless}=={\textgreater} del x.name
    \vspace{1ex}

    \end{boxedminipage}

    \label{object:__getattribute__}
    \index{object.\_\_getattribute\_\_ \textit{(function)}}

    \vspace{0.5ex}

    \begin{boxedminipage}{\textwidth}

    \raggedright \textbf{\_\_getattribute\_\_}(\textit{...})

    \vspace{-1.5ex}

    \rule{\textwidth}{0.5\fboxrule}

x.{\_}{\_}getattribute{\_}{\_}('name') {\textless}=={\textgreater} x.name
    \vspace{1ex}

    \end{boxedminipage}

    \label{object:__hash__}
    \index{object.\_\_hash\_\_ \textit{(function)}}

    \vspace{0.5ex}

    \begin{boxedminipage}{\textwidth}

    \raggedright \textbf{\_\_hash\_\_}(\textit{x})

    \vspace{-1.5ex}

    \rule{\textwidth}{0.5\fboxrule}

hash(x)
    \vspace{1ex}

    \end{boxedminipage}

    \label{object:__new__}
    \index{object.\_\_new\_\_ \textit{(function)}}

    \vspace{0.5ex}

    \begin{boxedminipage}{\textwidth}

    \raggedright \textbf{\_\_new\_\_}(\textit{T}, \textit{S}, \textit{...})

      \textbf{Return Value}
      \begin{quote}
\begin{alltt}
a new object with type S, a subtype of T
\end{alltt}

      \end{quote}

    \vspace{1ex}

    \end{boxedminipage}

    \label{object:__reduce__}
    \index{object.\_\_reduce\_\_ \textit{(function)}}

    \vspace{0.5ex}

    \begin{boxedminipage}{\textwidth}

    \raggedright \textbf{\_\_reduce\_\_}(\textit{...})

    \vspace{-1.5ex}

    \rule{\textwidth}{0.5\fboxrule}

helper for pickle
    \vspace{1ex}

    \end{boxedminipage}

    \label{object:__reduce_ex__}
    \index{object.\_\_reduce\_ex\_\_ \textit{(function)}}

    \vspace{0.5ex}

    \begin{boxedminipage}{\textwidth}

    \raggedright \textbf{\_\_reduce\_ex\_\_}(\textit{...})

    \vspace{-1.5ex}

    \rule{\textwidth}{0.5\fboxrule}

helper for pickle
    \vspace{1ex}

    \end{boxedminipage}

    \label{object:__repr__}
    \index{object.\_\_repr\_\_ \textit{(function)}}

    \vspace{0.5ex}

    \begin{boxedminipage}{\textwidth}

    \raggedright \textbf{\_\_repr\_\_}(\textit{x})

    \vspace{-1.5ex}

    \rule{\textwidth}{0.5\fboxrule}

repr(x)
    \vspace{1ex}

    \end{boxedminipage}

    \label{object:__setattr__}
    \index{object.\_\_setattr\_\_ \textit{(function)}}

    \vspace{0.5ex}

    \begin{boxedminipage}{\textwidth}

    \raggedright \textbf{\_\_setattr\_\_}(\textit{...})

    \vspace{-1.5ex}

    \rule{\textwidth}{0.5\fboxrule}

x.{\_}{\_}setattr{\_}{\_}('name', value) {\textless}=={\textgreater} x.name = value
    \vspace{1ex}

    \end{boxedminipage}

    \label{object:__str__}
    \index{object.\_\_str\_\_ \textit{(function)}}

    \vspace{0.5ex}

    \begin{boxedminipage}{\textwidth}

    \raggedright \textbf{\_\_str\_\_}(\textit{x})

    \vspace{-1.5ex}

    \rule{\textwidth}{0.5\fboxrule}

str(x)
    \vspace{1ex}

    \end{boxedminipage}


%%%%%%%%%%%%%%%%%%%%%%%%%%%%%%%%%%%%%%%%%%%%%%%%%%%%%%%%%%%%%%%%%%%%%%%%%%%
%%                              Properties                               %%
%%%%%%%%%%%%%%%%%%%%%%%%%%%%%%%%%%%%%%%%%%%%%%%%%%%%%%%%%%%%%%%%%%%%%%%%%%%

  \subsubsection{Properties}

\begin{longtable}{|p{.30\textwidth}|p{.62\textwidth}|l}
\cline{1-2}
\cline{1-2} \centering \textbf{Name} & \centering \textbf{Description}& \\
\cline{1-2}
\endhead\cline{1-2}\multicolumn{3}{r}{\small\textit{continued on next page}}\\\endfoot\cline{1-2}
\endlastfoot\raggedright \_\-\_\-c\-l\-a\-s\-s\-\_\-\_\- & \raggedright \textbf{Value:} 
{\tt {\textless}attribute '\_\_class\_\_' of 'object' objects{\textgreater}}&\\
\cline{1-2}
\end{longtable}


%%%%%%%%%%%%%%%%%%%%%%%%%%%%%%%%%%%%%%%%%%%%%%%%%%%%%%%%%%%%%%%%%%%%%%%%%%%
%%                          Instance Variables                           %%
%%%%%%%%%%%%%%%%%%%%%%%%%%%%%%%%%%%%%%%%%%%%%%%%%%%%%%%%%%%%%%%%%%%%%%%%%%%

  \subsubsection{Instance Variables}

\begin{longtable}{|p{.30\textwidth}|p{.62\textwidth}|l}
\cline{1-2}
\cline{1-2} \centering \textbf{Name} & \centering \textbf{Description}& \\
\cline{1-2}
\endhead\cline{1-2}\multicolumn{3}{r}{\small\textit{continued on next page}}\\\endfoot\cline{1-2}
\endlastfoot\raggedright l\-r\-a\-t\-e\- & Learning rate used in the algorithm.&\\
\cline{1-2}
\end{longtable}

    \index{peach \textit{(package)}!peach.nn \textit{(package)}!peach.nn.lrules \textit{(module)}!peach.nn.lrules.LMS \textit{(class)}|)}

%%%%%%%%%%%%%%%%%%%%%%%%%%%%%%%%%%%%%%%%%%%%%%%%%%%%%%%%%%%%%%%%%%%%%%%%%%%
%%                           Class Description                           %%
%%%%%%%%%%%%%%%%%%%%%%%%%%%%%%%%%%%%%%%%%%%%%%%%%%%%%%%%%%%%%%%%%%%%%%%%%%%

    \index{peach \textit{(package)}!peach.nn \textit{(package)}!peach.nn.lrules \textit{(module)}!peach.nn.lrules.LMS \textit{(class)}|(}
\subsection{Class LMS}

    \label{peach:nn:lrules:LMS}
\begin{tabular}{cccccccc}
% Line for object, linespec=[False, False]
\multicolumn{2}{r}{\settowidth{\BCL}{object}\multirow{2}{\BCL}{object}}
&&
&&
  \\\cline{3-3}
  &&\multicolumn{1}{c|}{}
&&
&&
  \\
% Line for peach.nn.lrules.FFLearning, linespec=[False]
\multicolumn{4}{r}{\settowidth{\BCL}{peach.nn.lrules.FFLearning}\multirow{2}{\BCL}{peach.nn.lrules.FFLearning}}
&&
  \\\cline{5-5}
  &&&&\multicolumn{1}{c|}{}
&&
  \\
&&&&\multicolumn{2}{l}{\textbf{peach.nn.lrules.LMS}}
\end{tabular}


The Least-Mean-Square (LMS) learning method.

The LMS method is a very simple method of learning, thoroughly described in
virtually every book about the subject. Please, consult a good book on
neural networks for more information. This implementation tries to use the
\texttt{numpy} routines as much as possible for better efficiency.

%%%%%%%%%%%%%%%%%%%%%%%%%%%%%%%%%%%%%%%%%%%%%%%%%%%%%%%%%%%%%%%%%%%%%%%%%%%
%%                                Methods                                %%
%%%%%%%%%%%%%%%%%%%%%%%%%%%%%%%%%%%%%%%%%%%%%%%%%%%%%%%%%%%%%%%%%%%%%%%%%%%

  \subsubsection{Methods}

    \vspace{0.5ex}

    \begin{boxedminipage}{\textwidth}

    \raggedright \textbf{\_\_init\_\_}(\textit{self}, \textit{lrate}=\texttt{0.05})

    \vspace{-1.5ex}

    \rule{\textwidth}{0.5\fboxrule}

Initializes the object.
    \vspace{1ex}

      \textbf{Parameters}
      \begin{quote}
        \begin{Ventry}{xxxxx}

          \item[lrate]


Learning rate to be used in the algorithm. Defaults to 0.05.
        \end{Ventry}

      \end{quote}

    \vspace{1ex}

      Overrides: object.\_\_init\_\_

    \end{boxedminipage}

    \vspace{0.5ex}

    \begin{boxedminipage}{\textwidth}

    \raggedright \textbf{\_\_call\_\_}(\textit{self}, \textit{nn}, \textit{x}, \textit{d})

    \vspace{-1.5ex}

    \rule{\textwidth}{0.5\fboxrule}

The \texttt{{\_}{\_}call{\_}{\_}} interface.

The learning implementation. Read the documentation for the base class
for more information. A call to the class should have the following
parameters:
    \vspace{1ex}

      \textbf{Parameters}
      \begin{quote}
        \begin{Ventry}{xx}

          \item[nn]


A \texttt{FeedForward} neural network instance that is going to be
modified by the learning algorithm. The modification is made \emph{in
loco}, that is, the synaptic weights of \texttt{nn} should be modified
in place, and not returned from this function.
          \item[x]


The input vector from the training set.
          \item[d]


The desired response for the given input vector.
        \end{Ventry}

      \end{quote}

    \vspace{1ex}

      Overrides: peach.nn.lrules.FFLearning.\_\_call\_\_

    \end{boxedminipage}

    \label{object:__delattr__}
    \index{object.\_\_delattr\_\_ \textit{(function)}}

    \vspace{0.5ex}

    \begin{boxedminipage}{\textwidth}

    \raggedright \textbf{\_\_delattr\_\_}(\textit{...})

    \vspace{-1.5ex}

    \rule{\textwidth}{0.5\fboxrule}

x.{\_}{\_}delattr{\_}{\_}('name') {\textless}=={\textgreater} del x.name
    \vspace{1ex}

    \end{boxedminipage}

    \label{object:__getattribute__}
    \index{object.\_\_getattribute\_\_ \textit{(function)}}

    \vspace{0.5ex}

    \begin{boxedminipage}{\textwidth}

    \raggedright \textbf{\_\_getattribute\_\_}(\textit{...})

    \vspace{-1.5ex}

    \rule{\textwidth}{0.5\fboxrule}

x.{\_}{\_}getattribute{\_}{\_}('name') {\textless}=={\textgreater} x.name
    \vspace{1ex}

    \end{boxedminipage}

    \label{object:__hash__}
    \index{object.\_\_hash\_\_ \textit{(function)}}

    \vspace{0.5ex}

    \begin{boxedminipage}{\textwidth}

    \raggedright \textbf{\_\_hash\_\_}(\textit{x})

    \vspace{-1.5ex}

    \rule{\textwidth}{0.5\fboxrule}

hash(x)
    \vspace{1ex}

    \end{boxedminipage}

    \label{object:__new__}
    \index{object.\_\_new\_\_ \textit{(function)}}

    \vspace{0.5ex}

    \begin{boxedminipage}{\textwidth}

    \raggedright \textbf{\_\_new\_\_}(\textit{T}, \textit{S}, \textit{...})

      \textbf{Return Value}
      \begin{quote}
\begin{alltt}
a new object with type S, a subtype of T
\end{alltt}

      \end{quote}

    \vspace{1ex}

    \end{boxedminipage}

    \label{object:__reduce__}
    \index{object.\_\_reduce\_\_ \textit{(function)}}

    \vspace{0.5ex}

    \begin{boxedminipage}{\textwidth}

    \raggedright \textbf{\_\_reduce\_\_}(\textit{...})

    \vspace{-1.5ex}

    \rule{\textwidth}{0.5\fboxrule}

helper for pickle
    \vspace{1ex}

    \end{boxedminipage}

    \label{object:__reduce_ex__}
    \index{object.\_\_reduce\_ex\_\_ \textit{(function)}}

    \vspace{0.5ex}

    \begin{boxedminipage}{\textwidth}

    \raggedright \textbf{\_\_reduce\_ex\_\_}(\textit{...})

    \vspace{-1.5ex}

    \rule{\textwidth}{0.5\fboxrule}

helper for pickle
    \vspace{1ex}

    \end{boxedminipage}

    \label{object:__repr__}
    \index{object.\_\_repr\_\_ \textit{(function)}}

    \vspace{0.5ex}

    \begin{boxedminipage}{\textwidth}

    \raggedright \textbf{\_\_repr\_\_}(\textit{x})

    \vspace{-1.5ex}

    \rule{\textwidth}{0.5\fboxrule}

repr(x)
    \vspace{1ex}

    \end{boxedminipage}

    \label{object:__setattr__}
    \index{object.\_\_setattr\_\_ \textit{(function)}}

    \vspace{0.5ex}

    \begin{boxedminipage}{\textwidth}

    \raggedright \textbf{\_\_setattr\_\_}(\textit{...})

    \vspace{-1.5ex}

    \rule{\textwidth}{0.5\fboxrule}

x.{\_}{\_}setattr{\_}{\_}('name', value) {\textless}=={\textgreater} x.name = value
    \vspace{1ex}

    \end{boxedminipage}

    \label{object:__str__}
    \index{object.\_\_str\_\_ \textit{(function)}}

    \vspace{0.5ex}

    \begin{boxedminipage}{\textwidth}

    \raggedright \textbf{\_\_str\_\_}(\textit{x})

    \vspace{-1.5ex}

    \rule{\textwidth}{0.5\fboxrule}

str(x)
    \vspace{1ex}

    \end{boxedminipage}


%%%%%%%%%%%%%%%%%%%%%%%%%%%%%%%%%%%%%%%%%%%%%%%%%%%%%%%%%%%%%%%%%%%%%%%%%%%
%%                              Properties                               %%
%%%%%%%%%%%%%%%%%%%%%%%%%%%%%%%%%%%%%%%%%%%%%%%%%%%%%%%%%%%%%%%%%%%%%%%%%%%

  \subsubsection{Properties}

\begin{longtable}{|p{.30\textwidth}|p{.62\textwidth}|l}
\cline{1-2}
\cline{1-2} \centering \textbf{Name} & \centering \textbf{Description}& \\
\cline{1-2}
\endhead\cline{1-2}\multicolumn{3}{r}{\small\textit{continued on next page}}\\\endfoot\cline{1-2}
\endlastfoot\raggedright \_\-\_\-c\-l\-a\-s\-s\-\_\-\_\- & \raggedright \textbf{Value:} 
{\tt {\textless}attribute '\_\_class\_\_' of 'object' objects{\textgreater}}&\\
\cline{1-2}
\end{longtable}


%%%%%%%%%%%%%%%%%%%%%%%%%%%%%%%%%%%%%%%%%%%%%%%%%%%%%%%%%%%%%%%%%%%%%%%%%%%
%%                          Instance Variables                           %%
%%%%%%%%%%%%%%%%%%%%%%%%%%%%%%%%%%%%%%%%%%%%%%%%%%%%%%%%%%%%%%%%%%%%%%%%%%%

  \subsubsection{Instance Variables}

\begin{longtable}{|p{.30\textwidth}|p{.62\textwidth}|l}
\cline{1-2}
\cline{1-2} \centering \textbf{Name} & \centering \textbf{Description}& \\
\cline{1-2}
\endhead\cline{1-2}\multicolumn{3}{r}{\small\textit{continued on next page}}\\\endfoot\cline{1-2}
\endlastfoot\raggedright l\-r\-a\-t\-e\- & Learning rate used in the algorithm.&\\
\cline{1-2}
\end{longtable}

    \index{peach \textit{(package)}!peach.nn \textit{(package)}!peach.nn.lrules \textit{(module)}!peach.nn.lrules.LMS \textit{(class)}|)}

%%%%%%%%%%%%%%%%%%%%%%%%%%%%%%%%%%%%%%%%%%%%%%%%%%%%%%%%%%%%%%%%%%%%%%%%%%%
%%                           Class Description                           %%
%%%%%%%%%%%%%%%%%%%%%%%%%%%%%%%%%%%%%%%%%%%%%%%%%%%%%%%%%%%%%%%%%%%%%%%%%%%

    \index{peach \textit{(package)}!peach.nn \textit{(package)}!peach.nn.lrules \textit{(module)}!peach.nn.lrules.LMS \textit{(class)}|(}
\subsection{Class LMS}

    \label{peach:nn:lrules:LMS}
\begin{tabular}{cccccccc}
% Line for object, linespec=[False, False]
\multicolumn{2}{r}{\settowidth{\BCL}{object}\multirow{2}{\BCL}{object}}
&&
&&
  \\\cline{3-3}
  &&\multicolumn{1}{c|}{}
&&
&&
  \\
% Line for peach.nn.lrules.FFLearning, linespec=[False]
\multicolumn{4}{r}{\settowidth{\BCL}{peach.nn.lrules.FFLearning}\multirow{2}{\BCL}{peach.nn.lrules.FFLearning}}
&&
  \\\cline{5-5}
  &&&&\multicolumn{1}{c|}{}
&&
  \\
&&&&\multicolumn{2}{l}{\textbf{peach.nn.lrules.LMS}}
\end{tabular}


The Least-Mean-Square (LMS) learning method.

The LMS method is a very simple method of learning, thoroughly described in
virtually every book about the subject. Please, consult a good book on
neural networks for more information. This implementation tries to use the
\texttt{numpy} routines as much as possible for better efficiency.

%%%%%%%%%%%%%%%%%%%%%%%%%%%%%%%%%%%%%%%%%%%%%%%%%%%%%%%%%%%%%%%%%%%%%%%%%%%
%%                                Methods                                %%
%%%%%%%%%%%%%%%%%%%%%%%%%%%%%%%%%%%%%%%%%%%%%%%%%%%%%%%%%%%%%%%%%%%%%%%%%%%

  \subsubsection{Methods}

    \vspace{0.5ex}

    \begin{boxedminipage}{\textwidth}

    \raggedright \textbf{\_\_init\_\_}(\textit{self}, \textit{lrate}=\texttt{0.05})

    \vspace{-1.5ex}

    \rule{\textwidth}{0.5\fboxrule}

Initializes the object.
    \vspace{1ex}

      \textbf{Parameters}
      \begin{quote}
        \begin{Ventry}{xxxxx}

          \item[lrate]


Learning rate to be used in the algorithm. Defaults to 0.05.
        \end{Ventry}

      \end{quote}

    \vspace{1ex}

      Overrides: object.\_\_init\_\_

    \end{boxedminipage}

    \vspace{0.5ex}

    \begin{boxedminipage}{\textwidth}

    \raggedright \textbf{\_\_call\_\_}(\textit{self}, \textit{nn}, \textit{x}, \textit{d})

    \vspace{-1.5ex}

    \rule{\textwidth}{0.5\fboxrule}

The \texttt{{\_}{\_}call{\_}{\_}} interface.

The learning implementation. Read the documentation for the base class
for more information. A call to the class should have the following
parameters:
    \vspace{1ex}

      \textbf{Parameters}
      \begin{quote}
        \begin{Ventry}{xx}

          \item[nn]


A \texttt{FeedForward} neural network instance that is going to be
modified by the learning algorithm. The modification is made \emph{in
loco}, that is, the synaptic weights of \texttt{nn} should be modified
in place, and not returned from this function.
          \item[x]


The input vector from the training set.
          \item[d]


The desired response for the given input vector.
        \end{Ventry}

      \end{quote}

    \vspace{1ex}

      Overrides: peach.nn.lrules.FFLearning.\_\_call\_\_

    \end{boxedminipage}

    \label{object:__delattr__}
    \index{object.\_\_delattr\_\_ \textit{(function)}}

    \vspace{0.5ex}

    \begin{boxedminipage}{\textwidth}

    \raggedright \textbf{\_\_delattr\_\_}(\textit{...})

    \vspace{-1.5ex}

    \rule{\textwidth}{0.5\fboxrule}

x.{\_}{\_}delattr{\_}{\_}('name') {\textless}=={\textgreater} del x.name
    \vspace{1ex}

    \end{boxedminipage}

    \label{object:__getattribute__}
    \index{object.\_\_getattribute\_\_ \textit{(function)}}

    \vspace{0.5ex}

    \begin{boxedminipage}{\textwidth}

    \raggedright \textbf{\_\_getattribute\_\_}(\textit{...})

    \vspace{-1.5ex}

    \rule{\textwidth}{0.5\fboxrule}

x.{\_}{\_}getattribute{\_}{\_}('name') {\textless}=={\textgreater} x.name
    \vspace{1ex}

    \end{boxedminipage}

    \label{object:__hash__}
    \index{object.\_\_hash\_\_ \textit{(function)}}

    \vspace{0.5ex}

    \begin{boxedminipage}{\textwidth}

    \raggedright \textbf{\_\_hash\_\_}(\textit{x})

    \vspace{-1.5ex}

    \rule{\textwidth}{0.5\fboxrule}

hash(x)
    \vspace{1ex}

    \end{boxedminipage}

    \label{object:__new__}
    \index{object.\_\_new\_\_ \textit{(function)}}

    \vspace{0.5ex}

    \begin{boxedminipage}{\textwidth}

    \raggedright \textbf{\_\_new\_\_}(\textit{T}, \textit{S}, \textit{...})

      \textbf{Return Value}
      \begin{quote}
\begin{alltt}
a new object with type S, a subtype of T
\end{alltt}

      \end{quote}

    \vspace{1ex}

    \end{boxedminipage}

    \label{object:__reduce__}
    \index{object.\_\_reduce\_\_ \textit{(function)}}

    \vspace{0.5ex}

    \begin{boxedminipage}{\textwidth}

    \raggedright \textbf{\_\_reduce\_\_}(\textit{...})

    \vspace{-1.5ex}

    \rule{\textwidth}{0.5\fboxrule}

helper for pickle
    \vspace{1ex}

    \end{boxedminipage}

    \label{object:__reduce_ex__}
    \index{object.\_\_reduce\_ex\_\_ \textit{(function)}}

    \vspace{0.5ex}

    \begin{boxedminipage}{\textwidth}

    \raggedright \textbf{\_\_reduce\_ex\_\_}(\textit{...})

    \vspace{-1.5ex}

    \rule{\textwidth}{0.5\fboxrule}

helper for pickle
    \vspace{1ex}

    \end{boxedminipage}

    \label{object:__repr__}
    \index{object.\_\_repr\_\_ \textit{(function)}}

    \vspace{0.5ex}

    \begin{boxedminipage}{\textwidth}

    \raggedright \textbf{\_\_repr\_\_}(\textit{x})

    \vspace{-1.5ex}

    \rule{\textwidth}{0.5\fboxrule}

repr(x)
    \vspace{1ex}

    \end{boxedminipage}

    \label{object:__setattr__}
    \index{object.\_\_setattr\_\_ \textit{(function)}}

    \vspace{0.5ex}

    \begin{boxedminipage}{\textwidth}

    \raggedright \textbf{\_\_setattr\_\_}(\textit{...})

    \vspace{-1.5ex}

    \rule{\textwidth}{0.5\fboxrule}

x.{\_}{\_}setattr{\_}{\_}('name', value) {\textless}=={\textgreater} x.name = value
    \vspace{1ex}

    \end{boxedminipage}

    \label{object:__str__}
    \index{object.\_\_str\_\_ \textit{(function)}}

    \vspace{0.5ex}

    \begin{boxedminipage}{\textwidth}

    \raggedright \textbf{\_\_str\_\_}(\textit{x})

    \vspace{-1.5ex}

    \rule{\textwidth}{0.5\fboxrule}

str(x)
    \vspace{1ex}

    \end{boxedminipage}


%%%%%%%%%%%%%%%%%%%%%%%%%%%%%%%%%%%%%%%%%%%%%%%%%%%%%%%%%%%%%%%%%%%%%%%%%%%
%%                              Properties                               %%
%%%%%%%%%%%%%%%%%%%%%%%%%%%%%%%%%%%%%%%%%%%%%%%%%%%%%%%%%%%%%%%%%%%%%%%%%%%

  \subsubsection{Properties}

\begin{longtable}{|p{.30\textwidth}|p{.62\textwidth}|l}
\cline{1-2}
\cline{1-2} \centering \textbf{Name} & \centering \textbf{Description}& \\
\cline{1-2}
\endhead\cline{1-2}\multicolumn{3}{r}{\small\textit{continued on next page}}\\\endfoot\cline{1-2}
\endlastfoot\raggedright \_\-\_\-c\-l\-a\-s\-s\-\_\-\_\- & \raggedright \textbf{Value:} 
{\tt {\textless}attribute '\_\_class\_\_' of 'object' objects{\textgreater}}&\\
\cline{1-2}
\end{longtable}


%%%%%%%%%%%%%%%%%%%%%%%%%%%%%%%%%%%%%%%%%%%%%%%%%%%%%%%%%%%%%%%%%%%%%%%%%%%
%%                          Instance Variables                           %%
%%%%%%%%%%%%%%%%%%%%%%%%%%%%%%%%%%%%%%%%%%%%%%%%%%%%%%%%%%%%%%%%%%%%%%%%%%%

  \subsubsection{Instance Variables}

\begin{longtable}{|p{.30\textwidth}|p{.62\textwidth}|l}
\cline{1-2}
\cline{1-2} \centering \textbf{Name} & \centering \textbf{Description}& \\
\cline{1-2}
\endhead\cline{1-2}\multicolumn{3}{r}{\small\textit{continued on next page}}\\\endfoot\cline{1-2}
\endlastfoot\raggedright l\-r\-a\-t\-e\- & Learning rate used in the algorithm.&\\
\cline{1-2}
\end{longtable}

    \index{peach \textit{(package)}!peach.nn \textit{(package)}!peach.nn.lrules \textit{(module)}!peach.nn.lrules.LMS \textit{(class)}|)}

%%%%%%%%%%%%%%%%%%%%%%%%%%%%%%%%%%%%%%%%%%%%%%%%%%%%%%%%%%%%%%%%%%%%%%%%%%%
%%                           Class Description                           %%
%%%%%%%%%%%%%%%%%%%%%%%%%%%%%%%%%%%%%%%%%%%%%%%%%%%%%%%%%%%%%%%%%%%%%%%%%%%

    \index{peach \textit{(package)}!peach.nn \textit{(package)}!peach.nn.lrules \textit{(module)}!peach.nn.lrules.BackPropagation \textit{(class)}|(}
\subsection{Class BackPropagation}

    \label{peach:nn:lrules:BackPropagation}
\begin{tabular}{cccccccc}
% Line for object, linespec=[False, False]
\multicolumn{2}{r}{\settowidth{\BCL}{object}\multirow{2}{\BCL}{object}}
&&
&&
  \\\cline{3-3}
  &&\multicolumn{1}{c|}{}
&&
&&
  \\
% Line for peach.nn.lrules.FFLearning, linespec=[False]
\multicolumn{4}{r}{\settowidth{\BCL}{peach.nn.lrules.FFLearning}\multirow{2}{\BCL}{peach.nn.lrules.FFLearning}}
&&
  \\\cline{5-5}
  &&&&\multicolumn{1}{c|}{}
&&
  \\
&&&&\multicolumn{2}{l}{\textbf{peach.nn.lrules.BackPropagation}}
\end{tabular}


The BackPropagation learning method.

The backpropagation method is a very simple method of learning, thoroughly
described in virtually every book about the subject. Please, consult a good
book on neural networks for more information. This implementation tries to
use the \texttt{numpy} routines as much as possible for better efficiency.

%%%%%%%%%%%%%%%%%%%%%%%%%%%%%%%%%%%%%%%%%%%%%%%%%%%%%%%%%%%%%%%%%%%%%%%%%%%
%%                                Methods                                %%
%%%%%%%%%%%%%%%%%%%%%%%%%%%%%%%%%%%%%%%%%%%%%%%%%%%%%%%%%%%%%%%%%%%%%%%%%%%

  \subsubsection{Methods}

    \vspace{0.5ex}

    \begin{boxedminipage}{\textwidth}

    \raggedright \textbf{\_\_init\_\_}(\textit{self}, \textit{lrate}=\texttt{0.05})

    \vspace{-1.5ex}

    \rule{\textwidth}{0.5\fboxrule}

Initializes the object.
    \vspace{1ex}

      \textbf{Parameters}
      \begin{quote}
        \begin{Ventry}{xxxxx}

          \item[lrate]


Learning rate to be used in the algorithm. Defaults to 0.05.
        \end{Ventry}

      \end{quote}

    \vspace{1ex}

      Overrides: object.\_\_init\_\_

    \end{boxedminipage}

    \vspace{0.5ex}

    \begin{boxedminipage}{\textwidth}

    \raggedright \textbf{\_\_call\_\_}(\textit{self}, \textit{nn}, \textit{x}, \textit{d})

    \vspace{-1.5ex}

    \rule{\textwidth}{0.5\fboxrule}

The \texttt{{\_}{\_}call{\_}{\_}} interface.

The learning implementation. Read the documentation for the base class
for more information. A call to the class should have the following
parameters:
    \vspace{1ex}

      \textbf{Parameters}
      \begin{quote}
        \begin{Ventry}{xx}

          \item[nn]


A \texttt{FeedForward} neural network instance that is going to be
modified by the learning algorithm. The modification is made \emph{in
loco}, that is, the synaptic weights of \texttt{nn} should be modified
in place, and not returned from this function.
          \item[x]


The input vector from the training set.
          \item[d]


The desired response for the given input vector.
        \end{Ventry}

      \end{quote}

    \vspace{1ex}

      Overrides: peach.nn.lrules.FFLearning.\_\_call\_\_

    \end{boxedminipage}

    \label{object:__delattr__}
    \index{object.\_\_delattr\_\_ \textit{(function)}}

    \vspace{0.5ex}

    \begin{boxedminipage}{\textwidth}

    \raggedright \textbf{\_\_delattr\_\_}(\textit{...})

    \vspace{-1.5ex}

    \rule{\textwidth}{0.5\fboxrule}

x.{\_}{\_}delattr{\_}{\_}('name') {\textless}=={\textgreater} del x.name
    \vspace{1ex}

    \end{boxedminipage}

    \label{object:__getattribute__}
    \index{object.\_\_getattribute\_\_ \textit{(function)}}

    \vspace{0.5ex}

    \begin{boxedminipage}{\textwidth}

    \raggedright \textbf{\_\_getattribute\_\_}(\textit{...})

    \vspace{-1.5ex}

    \rule{\textwidth}{0.5\fboxrule}

x.{\_}{\_}getattribute{\_}{\_}('name') {\textless}=={\textgreater} x.name
    \vspace{1ex}

    \end{boxedminipage}

    \label{object:__hash__}
    \index{object.\_\_hash\_\_ \textit{(function)}}

    \vspace{0.5ex}

    \begin{boxedminipage}{\textwidth}

    \raggedright \textbf{\_\_hash\_\_}(\textit{x})

    \vspace{-1.5ex}

    \rule{\textwidth}{0.5\fboxrule}

hash(x)
    \vspace{1ex}

    \end{boxedminipage}

    \label{object:__new__}
    \index{object.\_\_new\_\_ \textit{(function)}}

    \vspace{0.5ex}

    \begin{boxedminipage}{\textwidth}

    \raggedright \textbf{\_\_new\_\_}(\textit{T}, \textit{S}, \textit{...})

      \textbf{Return Value}
      \begin{quote}
\begin{alltt}
a new object with type S, a subtype of T
\end{alltt}

      \end{quote}

    \vspace{1ex}

    \end{boxedminipage}

    \label{object:__reduce__}
    \index{object.\_\_reduce\_\_ \textit{(function)}}

    \vspace{0.5ex}

    \begin{boxedminipage}{\textwidth}

    \raggedright \textbf{\_\_reduce\_\_}(\textit{...})

    \vspace{-1.5ex}

    \rule{\textwidth}{0.5\fboxrule}

helper for pickle
    \vspace{1ex}

    \end{boxedminipage}

    \label{object:__reduce_ex__}
    \index{object.\_\_reduce\_ex\_\_ \textit{(function)}}

    \vspace{0.5ex}

    \begin{boxedminipage}{\textwidth}

    \raggedright \textbf{\_\_reduce\_ex\_\_}(\textit{...})

    \vspace{-1.5ex}

    \rule{\textwidth}{0.5\fboxrule}

helper for pickle
    \vspace{1ex}

    \end{boxedminipage}

    \label{object:__repr__}
    \index{object.\_\_repr\_\_ \textit{(function)}}

    \vspace{0.5ex}

    \begin{boxedminipage}{\textwidth}

    \raggedright \textbf{\_\_repr\_\_}(\textit{x})

    \vspace{-1.5ex}

    \rule{\textwidth}{0.5\fboxrule}

repr(x)
    \vspace{1ex}

    \end{boxedminipage}

    \label{object:__setattr__}
    \index{object.\_\_setattr\_\_ \textit{(function)}}

    \vspace{0.5ex}

    \begin{boxedminipage}{\textwidth}

    \raggedright \textbf{\_\_setattr\_\_}(\textit{...})

    \vspace{-1.5ex}

    \rule{\textwidth}{0.5\fboxrule}

x.{\_}{\_}setattr{\_}{\_}('name', value) {\textless}=={\textgreater} x.name = value
    \vspace{1ex}

    \end{boxedminipage}

    \label{object:__str__}
    \index{object.\_\_str\_\_ \textit{(function)}}

    \vspace{0.5ex}

    \begin{boxedminipage}{\textwidth}

    \raggedright \textbf{\_\_str\_\_}(\textit{x})

    \vspace{-1.5ex}

    \rule{\textwidth}{0.5\fboxrule}

str(x)
    \vspace{1ex}

    \end{boxedminipage}


%%%%%%%%%%%%%%%%%%%%%%%%%%%%%%%%%%%%%%%%%%%%%%%%%%%%%%%%%%%%%%%%%%%%%%%%%%%
%%                              Properties                               %%
%%%%%%%%%%%%%%%%%%%%%%%%%%%%%%%%%%%%%%%%%%%%%%%%%%%%%%%%%%%%%%%%%%%%%%%%%%%

  \subsubsection{Properties}

\begin{longtable}{|p{.30\textwidth}|p{.62\textwidth}|l}
\cline{1-2}
\cline{1-2} \centering \textbf{Name} & \centering \textbf{Description}& \\
\cline{1-2}
\endhead\cline{1-2}\multicolumn{3}{r}{\small\textit{continued on next page}}\\\endfoot\cline{1-2}
\endlastfoot\raggedright \_\-\_\-c\-l\-a\-s\-s\-\_\-\_\- & \raggedright \textbf{Value:} 
{\tt {\textless}attribute '\_\_class\_\_' of 'object' objects{\textgreater}}&\\
\cline{1-2}
\end{longtable}


%%%%%%%%%%%%%%%%%%%%%%%%%%%%%%%%%%%%%%%%%%%%%%%%%%%%%%%%%%%%%%%%%%%%%%%%%%%
%%                          Instance Variables                           %%
%%%%%%%%%%%%%%%%%%%%%%%%%%%%%%%%%%%%%%%%%%%%%%%%%%%%%%%%%%%%%%%%%%%%%%%%%%%

  \subsubsection{Instance Variables}

\begin{longtable}{|p{.30\textwidth}|p{.62\textwidth}|l}
\cline{1-2}
\cline{1-2} \centering \textbf{Name} & \centering \textbf{Description}& \\
\cline{1-2}
\endhead\cline{1-2}\multicolumn{3}{r}{\small\textit{continued on next page}}\\\endfoot\cline{1-2}
\endlastfoot\raggedright l\-r\-a\-t\-e\- & Learning rate used in the algorithm.&\\
\cline{1-2}
\end{longtable}

    \index{peach \textit{(package)}!peach.nn \textit{(package)}!peach.nn.lrules \textit{(module)}!peach.nn.lrules.BackPropagation \textit{(class)}|)}

%%%%%%%%%%%%%%%%%%%%%%%%%%%%%%%%%%%%%%%%%%%%%%%%%%%%%%%%%%%%%%%%%%%%%%%%%%%
%%                           Class Description                           %%
%%%%%%%%%%%%%%%%%%%%%%%%%%%%%%%%%%%%%%%%%%%%%%%%%%%%%%%%%%%%%%%%%%%%%%%%%%%

    \index{peach \textit{(package)}!peach.nn \textit{(package)}!peach.nn.lrules \textit{(module)}!peach.nn.lrules.SOMLearning \textit{(class)}|(}
\subsection{Class SOMLearning}

    \label{peach:nn:lrules:SOMLearning}
\begin{tabular}{cccccc}
% Line for object, linespec=[False]
\multicolumn{2}{r}{\settowidth{\BCL}{object}\multirow{2}{\BCL}{object}}
&&
  \\\cline{3-3}
  &&\multicolumn{1}{c|}{}
&&
  \\
&&\multicolumn{2}{l}{\textbf{peach.nn.lrules.SOMLearning}}
\end{tabular}

\textbf{Known Subclasses:}
peach.nn.lrules.Competitive,
    peach.nn.lrules.Cooperative,
    peach.nn.lrules.WinnerTakesAll


Base class for Self-Organizing Maps.

As a base class, this class doesn't do anything. You should subclass this
class if you want to implement a learning method for self-organizing maps.

A learning method for a neural net of this kind must deal with a \texttt{SOM}
instance. A \texttt{SOM} object is a \texttt{Layer} (consulting the documentation of
these classes is important!).

A class implementing a learning method should have at least two methods:
\begin{quote}
\begin{description}
%[visit_definition_list_item]
\item[{{\_}{\_}init{\_}{\_}}] %[visit_definition]

The \texttt{{\_}{\_}init{\_}{\_}} method should initialize the object. It is in general
used to configure some property of the learning algorithm, such as the
learning rate.

%[depart_definition]
%[depart_definition_list_item]
%[visit_definition_list_item]
\item[{{\_}{\_}call{\_}{\_}}] %[visit_definition]

The \texttt{{\_}{\_}call{\_}{\_}} interface is how the method should interact with the
neural network. It should have the following signature:
\begin{quote}{\ttfamily \raggedright \noindent
{\_}{\_}call{\_}{\_}(self,~nn,~x)
}\end{quote}

where \texttt{nn} is the \texttt{SOM} instance to be modified \emph{in loco}, and \texttt{x}
is the input vector. It should return nothing.

%[depart_definition]
%[depart_definition_list_item]
\end{description}
\end{quote}

%%%%%%%%%%%%%%%%%%%%%%%%%%%%%%%%%%%%%%%%%%%%%%%%%%%%%%%%%%%%%%%%%%%%%%%%%%%
%%                                Methods                                %%
%%%%%%%%%%%%%%%%%%%%%%%%%%%%%%%%%%%%%%%%%%%%%%%%%%%%%%%%%%%%%%%%%%%%%%%%%%%

  \subsubsection{Methods}

    \label{peach:nn:lrules:SOMLearning:__call__}
    \index{peach \textit{(package)}!peach.nn \textit{(package)}!peach.nn.lrules \textit{(module)}!peach.nn.lrules.SOMLearning \textit{(class)}!peach.nn.lrules.SOMLearning.\_\_call\_\_ \textit{(method)}}

    \vspace{0.5ex}

    \begin{boxedminipage}{\textwidth}

    \raggedright \textbf{\_\_call\_\_}(\textit{self}, \textit{nn}, \textit{x}, \textit{d})

    \vspace{-1.5ex}

    \rule{\textwidth}{0.5\fboxrule}

The \texttt{{\_}{\_}call{\_}{\_}} interface.

Read the documentation for this class for more information. A call to
the class should have the following parameters:
    \vspace{1ex}

      \textbf{Parameters}
      \begin{quote}
        \begin{Ventry}{xx}

          \item[nn]


A \texttt{SOM} neural network instance that is going to be modified by
the learning algorithm. The modification is made \emph{in loco}, that is,
the synaptic weights of \texttt{nn} should be modified in place, and not
returned from this function.
          \item[x]


The input vector from the training set.
        \end{Ventry}

      \end{quote}

    \vspace{1ex}

    \end{boxedminipage}

    \label{object:__delattr__}
    \index{object.\_\_delattr\_\_ \textit{(function)}}

    \vspace{0.5ex}

    \begin{boxedminipage}{\textwidth}

    \raggedright \textbf{\_\_delattr\_\_}(\textit{...})

    \vspace{-1.5ex}

    \rule{\textwidth}{0.5\fboxrule}

x.{\_}{\_}delattr{\_}{\_}('name') {\textless}=={\textgreater} del x.name
    \vspace{1ex}

    \end{boxedminipage}

    \label{object:__getattribute__}
    \index{object.\_\_getattribute\_\_ \textit{(function)}}

    \vspace{0.5ex}

    \begin{boxedminipage}{\textwidth}

    \raggedright \textbf{\_\_getattribute\_\_}(\textit{...})

    \vspace{-1.5ex}

    \rule{\textwidth}{0.5\fboxrule}

x.{\_}{\_}getattribute{\_}{\_}('name') {\textless}=={\textgreater} x.name
    \vspace{1ex}

    \end{boxedminipage}

    \label{object:__hash__}
    \index{object.\_\_hash\_\_ \textit{(function)}}

    \vspace{0.5ex}

    \begin{boxedminipage}{\textwidth}

    \raggedright \textbf{\_\_hash\_\_}(\textit{x})

    \vspace{-1.5ex}

    \rule{\textwidth}{0.5\fboxrule}

hash(x)
    \vspace{1ex}

    \end{boxedminipage}

    \label{object:__init__}
    \index{object.\_\_init\_\_ \textit{(function)}}

    \vspace{0.5ex}

    \begin{boxedminipage}{\textwidth}

    \raggedright \textbf{\_\_init\_\_}(\textit{...})

    \vspace{-1.5ex}

    \rule{\textwidth}{0.5\fboxrule}

x.{\_}{\_}init{\_}{\_}(...) initializes x; see x.{\_}{\_}class{\_}{\_}.{\_}{\_}doc{\_}{\_} for signature
    \vspace{1ex}

    \end{boxedminipage}

    \label{object:__new__}
    \index{object.\_\_new\_\_ \textit{(function)}}

    \vspace{0.5ex}

    \begin{boxedminipage}{\textwidth}

    \raggedright \textbf{\_\_new\_\_}(\textit{T}, \textit{S}, \textit{...})

      \textbf{Return Value}
      \begin{quote}
\begin{alltt}
a new object with type S, a subtype of T
\end{alltt}

      \end{quote}

    \vspace{1ex}

    \end{boxedminipage}

    \label{object:__reduce__}
    \index{object.\_\_reduce\_\_ \textit{(function)}}

    \vspace{0.5ex}

    \begin{boxedminipage}{\textwidth}

    \raggedright \textbf{\_\_reduce\_\_}(\textit{...})

    \vspace{-1.5ex}

    \rule{\textwidth}{0.5\fboxrule}

helper for pickle
    \vspace{1ex}

    \end{boxedminipage}

    \label{object:__reduce_ex__}
    \index{object.\_\_reduce\_ex\_\_ \textit{(function)}}

    \vspace{0.5ex}

    \begin{boxedminipage}{\textwidth}

    \raggedright \textbf{\_\_reduce\_ex\_\_}(\textit{...})

    \vspace{-1.5ex}

    \rule{\textwidth}{0.5\fboxrule}

helper for pickle
    \vspace{1ex}

    \end{boxedminipage}

    \label{object:__repr__}
    \index{object.\_\_repr\_\_ \textit{(function)}}

    \vspace{0.5ex}

    \begin{boxedminipage}{\textwidth}

    \raggedright \textbf{\_\_repr\_\_}(\textit{x})

    \vspace{-1.5ex}

    \rule{\textwidth}{0.5\fboxrule}

repr(x)
    \vspace{1ex}

    \end{boxedminipage}

    \label{object:__setattr__}
    \index{object.\_\_setattr\_\_ \textit{(function)}}

    \vspace{0.5ex}

    \begin{boxedminipage}{\textwidth}

    \raggedright \textbf{\_\_setattr\_\_}(\textit{...})

    \vspace{-1.5ex}

    \rule{\textwidth}{0.5\fboxrule}

x.{\_}{\_}setattr{\_}{\_}('name', value) {\textless}=={\textgreater} x.name = value
    \vspace{1ex}

    \end{boxedminipage}

    \label{object:__str__}
    \index{object.\_\_str\_\_ \textit{(function)}}

    \vspace{0.5ex}

    \begin{boxedminipage}{\textwidth}

    \raggedright \textbf{\_\_str\_\_}(\textit{x})

    \vspace{-1.5ex}

    \rule{\textwidth}{0.5\fboxrule}

str(x)
    \vspace{1ex}

    \end{boxedminipage}


%%%%%%%%%%%%%%%%%%%%%%%%%%%%%%%%%%%%%%%%%%%%%%%%%%%%%%%%%%%%%%%%%%%%%%%%%%%
%%                              Properties                               %%
%%%%%%%%%%%%%%%%%%%%%%%%%%%%%%%%%%%%%%%%%%%%%%%%%%%%%%%%%%%%%%%%%%%%%%%%%%%

  \subsubsection{Properties}

\begin{longtable}{|p{.30\textwidth}|p{.62\textwidth}|l}
\cline{1-2}
\cline{1-2} \centering \textbf{Name} & \centering \textbf{Description}& \\
\cline{1-2}
\endhead\cline{1-2}\multicolumn{3}{r}{\small\textit{continued on next page}}\\\endfoot\cline{1-2}
\endlastfoot\raggedright \_\-\_\-c\-l\-a\-s\-s\-\_\-\_\- & \raggedright \textbf{Value:} 
{\tt {\textless}attribute '\_\_class\_\_' of 'object' objects{\textgreater}}&\\
\cline{1-2}
\end{longtable}

    \index{peach \textit{(package)}!peach.nn \textit{(package)}!peach.nn.lrules \textit{(module)}!peach.nn.lrules.SOMLearning \textit{(class)}|)}

%%%%%%%%%%%%%%%%%%%%%%%%%%%%%%%%%%%%%%%%%%%%%%%%%%%%%%%%%%%%%%%%%%%%%%%%%%%
%%                           Class Description                           %%
%%%%%%%%%%%%%%%%%%%%%%%%%%%%%%%%%%%%%%%%%%%%%%%%%%%%%%%%%%%%%%%%%%%%%%%%%%%

    \index{peach \textit{(package)}!peach.nn \textit{(package)}!peach.nn.lrules \textit{(module)}!peach.nn.lrules.WinnerTakesAll \textit{(class)}|(}
\subsection{Class WinnerTakesAll}

    \label{peach:nn:lrules:WinnerTakesAll}
\begin{tabular}{cccccccc}
% Line for object, linespec=[False, False]
\multicolumn{2}{r}{\settowidth{\BCL}{object}\multirow{2}{\BCL}{object}}
&&
&&
  \\\cline{3-3}
  &&\multicolumn{1}{c|}{}
&&
&&
  \\
% Line for peach.nn.lrules.SOMLearning, linespec=[False]
\multicolumn{4}{r}{\settowidth{\BCL}{peach.nn.lrules.SOMLearning}\multirow{2}{\BCL}{peach.nn.lrules.SOMLearning}}
&&
  \\\cline{5-5}
  &&&&\multicolumn{1}{c|}{}
&&
  \\
&&&&\multicolumn{2}{l}{\textbf{peach.nn.lrules.WinnerTakesAll}}
\end{tabular}


Purely competitive learning method without learning rate adjust.

A winner-takes-all strategy detects the winner on the self-organizing map
and adjusts it in the direction of the input vector, scaled by the learning
rate. Its tendency is to cluster around the gravity center of the points in
the training set.

%%%%%%%%%%%%%%%%%%%%%%%%%%%%%%%%%%%%%%%%%%%%%%%%%%%%%%%%%%%%%%%%%%%%%%%%%%%
%%                                Methods                                %%
%%%%%%%%%%%%%%%%%%%%%%%%%%%%%%%%%%%%%%%%%%%%%%%%%%%%%%%%%%%%%%%%%%%%%%%%%%%

  \subsubsection{Methods}

    \vspace{0.5ex}

    \begin{boxedminipage}{\textwidth}

    \raggedright \textbf{\_\_init\_\_}(\textit{self}, \textit{lrate}=\texttt{0.05})

    \vspace{-1.5ex}

    \rule{\textwidth}{0.5\fboxrule}

Initializes the object.
    \vspace{1ex}

      \textbf{Parameters}
      \begin{quote}
        \begin{Ventry}{xxxxx}

          \item[lrate]


Learning rate to be used in the algorithm. Defaults to 0.05.
        \end{Ventry}

      \end{quote}

    \vspace{1ex}

      Overrides: object.\_\_init\_\_

    \end{boxedminipage}

    \vspace{0.5ex}

    \begin{boxedminipage}{\textwidth}

    \raggedright \textbf{\_\_call\_\_}(\textit{self}, \textit{nn}, \textit{x})

    \vspace{-1.5ex}

    \rule{\textwidth}{0.5\fboxrule}

The \texttt{{\_}{\_}call{\_}{\_}} interface.

The learning implementation. Read the documentation for the base class
for more information. A call to the class should have the following
parameters:
    \vspace{1ex}

      \textbf{Parameters}
      \begin{quote}
        \begin{Ventry}{xx}

          \item[nn]


A \texttt{SOM} neural network instance that is going to be modified by
the learning algorithm. The modification is made \emph{in loco}, that is,
the synaptic weights of \texttt{nn} should be modified in place, and not
returned from this function.
          \item[x]


The input vector from the training set.
        \end{Ventry}

      \end{quote}

    \vspace{1ex}

      Overrides: peach.nn.lrules.SOMLearning.\_\_call\_\_

    \end{boxedminipage}

    \label{object:__delattr__}
    \index{object.\_\_delattr\_\_ \textit{(function)}}

    \vspace{0.5ex}

    \begin{boxedminipage}{\textwidth}

    \raggedright \textbf{\_\_delattr\_\_}(\textit{...})

    \vspace{-1.5ex}

    \rule{\textwidth}{0.5\fboxrule}

x.{\_}{\_}delattr{\_}{\_}('name') {\textless}=={\textgreater} del x.name
    \vspace{1ex}

    \end{boxedminipage}

    \label{object:__getattribute__}
    \index{object.\_\_getattribute\_\_ \textit{(function)}}

    \vspace{0.5ex}

    \begin{boxedminipage}{\textwidth}

    \raggedright \textbf{\_\_getattribute\_\_}(\textit{...})

    \vspace{-1.5ex}

    \rule{\textwidth}{0.5\fboxrule}

x.{\_}{\_}getattribute{\_}{\_}('name') {\textless}=={\textgreater} x.name
    \vspace{1ex}

    \end{boxedminipage}

    \label{object:__hash__}
    \index{object.\_\_hash\_\_ \textit{(function)}}

    \vspace{0.5ex}

    \begin{boxedminipage}{\textwidth}

    \raggedright \textbf{\_\_hash\_\_}(\textit{x})

    \vspace{-1.5ex}

    \rule{\textwidth}{0.5\fboxrule}

hash(x)
    \vspace{1ex}

    \end{boxedminipage}

    \label{object:__new__}
    \index{object.\_\_new\_\_ \textit{(function)}}

    \vspace{0.5ex}

    \begin{boxedminipage}{\textwidth}

    \raggedright \textbf{\_\_new\_\_}(\textit{T}, \textit{S}, \textit{...})

      \textbf{Return Value}
      \begin{quote}
\begin{alltt}
a new object with type S, a subtype of T
\end{alltt}

      \end{quote}

    \vspace{1ex}

    \end{boxedminipage}

    \label{object:__reduce__}
    \index{object.\_\_reduce\_\_ \textit{(function)}}

    \vspace{0.5ex}

    \begin{boxedminipage}{\textwidth}

    \raggedright \textbf{\_\_reduce\_\_}(\textit{...})

    \vspace{-1.5ex}

    \rule{\textwidth}{0.5\fboxrule}

helper for pickle
    \vspace{1ex}

    \end{boxedminipage}

    \label{object:__reduce_ex__}
    \index{object.\_\_reduce\_ex\_\_ \textit{(function)}}

    \vspace{0.5ex}

    \begin{boxedminipage}{\textwidth}

    \raggedright \textbf{\_\_reduce\_ex\_\_}(\textit{...})

    \vspace{-1.5ex}

    \rule{\textwidth}{0.5\fboxrule}

helper for pickle
    \vspace{1ex}

    \end{boxedminipage}

    \label{object:__repr__}
    \index{object.\_\_repr\_\_ \textit{(function)}}

    \vspace{0.5ex}

    \begin{boxedminipage}{\textwidth}

    \raggedright \textbf{\_\_repr\_\_}(\textit{x})

    \vspace{-1.5ex}

    \rule{\textwidth}{0.5\fboxrule}

repr(x)
    \vspace{1ex}

    \end{boxedminipage}

    \label{object:__setattr__}
    \index{object.\_\_setattr\_\_ \textit{(function)}}

    \vspace{0.5ex}

    \begin{boxedminipage}{\textwidth}

    \raggedright \textbf{\_\_setattr\_\_}(\textit{...})

    \vspace{-1.5ex}

    \rule{\textwidth}{0.5\fboxrule}

x.{\_}{\_}setattr{\_}{\_}('name', value) {\textless}=={\textgreater} x.name = value
    \vspace{1ex}

    \end{boxedminipage}

    \label{object:__str__}
    \index{object.\_\_str\_\_ \textit{(function)}}

    \vspace{0.5ex}

    \begin{boxedminipage}{\textwidth}

    \raggedright \textbf{\_\_str\_\_}(\textit{x})

    \vspace{-1.5ex}

    \rule{\textwidth}{0.5\fboxrule}

str(x)
    \vspace{1ex}

    \end{boxedminipage}


%%%%%%%%%%%%%%%%%%%%%%%%%%%%%%%%%%%%%%%%%%%%%%%%%%%%%%%%%%%%%%%%%%%%%%%%%%%
%%                              Properties                               %%
%%%%%%%%%%%%%%%%%%%%%%%%%%%%%%%%%%%%%%%%%%%%%%%%%%%%%%%%%%%%%%%%%%%%%%%%%%%

  \subsubsection{Properties}

\begin{longtable}{|p{.30\textwidth}|p{.62\textwidth}|l}
\cline{1-2}
\cline{1-2} \centering \textbf{Name} & \centering \textbf{Description}& \\
\cline{1-2}
\endhead\cline{1-2}\multicolumn{3}{r}{\small\textit{continued on next page}}\\\endfoot\cline{1-2}
\endlastfoot\raggedright \_\-\_\-c\-l\-a\-s\-s\-\_\-\_\- & \raggedright \textbf{Value:} 
{\tt {\textless}attribute '\_\_class\_\_' of 'object' objects{\textgreater}}&\\
\cline{1-2}
\end{longtable}


%%%%%%%%%%%%%%%%%%%%%%%%%%%%%%%%%%%%%%%%%%%%%%%%%%%%%%%%%%%%%%%%%%%%%%%%%%%
%%                          Instance Variables                           %%
%%%%%%%%%%%%%%%%%%%%%%%%%%%%%%%%%%%%%%%%%%%%%%%%%%%%%%%%%%%%%%%%%%%%%%%%%%%

  \subsubsection{Instance Variables}

\begin{longtable}{|p{.30\textwidth}|p{.62\textwidth}|l}
\cline{1-2}
\cline{1-2} \centering \textbf{Name} & \centering \textbf{Description}& \\
\cline{1-2}
\endhead\cline{1-2}\multicolumn{3}{r}{\small\textit{continued on next page}}\\\endfoot\cline{1-2}
\endlastfoot\raggedright l\-r\-a\-t\-e\- & Learning rate used with the algorithm.&\\
\cline{1-2}
\end{longtable}

    \index{peach \textit{(package)}!peach.nn \textit{(package)}!peach.nn.lrules \textit{(module)}!peach.nn.lrules.WinnerTakesAll \textit{(class)}|)}

%%%%%%%%%%%%%%%%%%%%%%%%%%%%%%%%%%%%%%%%%%%%%%%%%%%%%%%%%%%%%%%%%%%%%%%%%%%
%%                           Class Description                           %%
%%%%%%%%%%%%%%%%%%%%%%%%%%%%%%%%%%%%%%%%%%%%%%%%%%%%%%%%%%%%%%%%%%%%%%%%%%%

    \index{peach \textit{(package)}!peach.nn \textit{(package)}!peach.nn.lrules \textit{(module)}!peach.nn.lrules.WinnerTakesAll \textit{(class)}|(}
\subsection{Class WinnerTakesAll}

    \label{peach:nn:lrules:WinnerTakesAll}
\begin{tabular}{cccccccc}
% Line for object, linespec=[False, False]
\multicolumn{2}{r}{\settowidth{\BCL}{object}\multirow{2}{\BCL}{object}}
&&
&&
  \\\cline{3-3}
  &&\multicolumn{1}{c|}{}
&&
&&
  \\
% Line for peach.nn.lrules.SOMLearning, linespec=[False]
\multicolumn{4}{r}{\settowidth{\BCL}{peach.nn.lrules.SOMLearning}\multirow{2}{\BCL}{peach.nn.lrules.SOMLearning}}
&&
  \\\cline{5-5}
  &&&&\multicolumn{1}{c|}{}
&&
  \\
&&&&\multicolumn{2}{l}{\textbf{peach.nn.lrules.WinnerTakesAll}}
\end{tabular}


Purely competitive learning method without learning rate adjust.

A winner-takes-all strategy detects the winner on the self-organizing map
and adjusts it in the direction of the input vector, scaled by the learning
rate. Its tendency is to cluster around the gravity center of the points in
the training set.

%%%%%%%%%%%%%%%%%%%%%%%%%%%%%%%%%%%%%%%%%%%%%%%%%%%%%%%%%%%%%%%%%%%%%%%%%%%
%%                                Methods                                %%
%%%%%%%%%%%%%%%%%%%%%%%%%%%%%%%%%%%%%%%%%%%%%%%%%%%%%%%%%%%%%%%%%%%%%%%%%%%

  \subsubsection{Methods}

    \vspace{0.5ex}

    \begin{boxedminipage}{\textwidth}

    \raggedright \textbf{\_\_init\_\_}(\textit{self}, \textit{lrate}=\texttt{0.05})

    \vspace{-1.5ex}

    \rule{\textwidth}{0.5\fboxrule}

Initializes the object.
    \vspace{1ex}

      \textbf{Parameters}
      \begin{quote}
        \begin{Ventry}{xxxxx}

          \item[lrate]


Learning rate to be used in the algorithm. Defaults to 0.05.
        \end{Ventry}

      \end{quote}

    \vspace{1ex}

      Overrides: object.\_\_init\_\_

    \end{boxedminipage}

    \vspace{0.5ex}

    \begin{boxedminipage}{\textwidth}

    \raggedright \textbf{\_\_call\_\_}(\textit{self}, \textit{nn}, \textit{x})

    \vspace{-1.5ex}

    \rule{\textwidth}{0.5\fboxrule}

The \texttt{{\_}{\_}call{\_}{\_}} interface.

The learning implementation. Read the documentation for the base class
for more information. A call to the class should have the following
parameters:
    \vspace{1ex}

      \textbf{Parameters}
      \begin{quote}
        \begin{Ventry}{xx}

          \item[nn]


A \texttt{SOM} neural network instance that is going to be modified by
the learning algorithm. The modification is made \emph{in loco}, that is,
the synaptic weights of \texttt{nn} should be modified in place, and not
returned from this function.
          \item[x]


The input vector from the training set.
        \end{Ventry}

      \end{quote}

    \vspace{1ex}

      Overrides: peach.nn.lrules.SOMLearning.\_\_call\_\_

    \end{boxedminipage}

    \label{object:__delattr__}
    \index{object.\_\_delattr\_\_ \textit{(function)}}

    \vspace{0.5ex}

    \begin{boxedminipage}{\textwidth}

    \raggedright \textbf{\_\_delattr\_\_}(\textit{...})

    \vspace{-1.5ex}

    \rule{\textwidth}{0.5\fboxrule}

x.{\_}{\_}delattr{\_}{\_}('name') {\textless}=={\textgreater} del x.name
    \vspace{1ex}

    \end{boxedminipage}

    \label{object:__getattribute__}
    \index{object.\_\_getattribute\_\_ \textit{(function)}}

    \vspace{0.5ex}

    \begin{boxedminipage}{\textwidth}

    \raggedright \textbf{\_\_getattribute\_\_}(\textit{...})

    \vspace{-1.5ex}

    \rule{\textwidth}{0.5\fboxrule}

x.{\_}{\_}getattribute{\_}{\_}('name') {\textless}=={\textgreater} x.name
    \vspace{1ex}

    \end{boxedminipage}

    \label{object:__hash__}
    \index{object.\_\_hash\_\_ \textit{(function)}}

    \vspace{0.5ex}

    \begin{boxedminipage}{\textwidth}

    \raggedright \textbf{\_\_hash\_\_}(\textit{x})

    \vspace{-1.5ex}

    \rule{\textwidth}{0.5\fboxrule}

hash(x)
    \vspace{1ex}

    \end{boxedminipage}

    \label{object:__new__}
    \index{object.\_\_new\_\_ \textit{(function)}}

    \vspace{0.5ex}

    \begin{boxedminipage}{\textwidth}

    \raggedright \textbf{\_\_new\_\_}(\textit{T}, \textit{S}, \textit{...})

      \textbf{Return Value}
      \begin{quote}
\begin{alltt}
a new object with type S, a subtype of T
\end{alltt}

      \end{quote}

    \vspace{1ex}

    \end{boxedminipage}

    \label{object:__reduce__}
    \index{object.\_\_reduce\_\_ \textit{(function)}}

    \vspace{0.5ex}

    \begin{boxedminipage}{\textwidth}

    \raggedright \textbf{\_\_reduce\_\_}(\textit{...})

    \vspace{-1.5ex}

    \rule{\textwidth}{0.5\fboxrule}

helper for pickle
    \vspace{1ex}

    \end{boxedminipage}

    \label{object:__reduce_ex__}
    \index{object.\_\_reduce\_ex\_\_ \textit{(function)}}

    \vspace{0.5ex}

    \begin{boxedminipage}{\textwidth}

    \raggedright \textbf{\_\_reduce\_ex\_\_}(\textit{...})

    \vspace{-1.5ex}

    \rule{\textwidth}{0.5\fboxrule}

helper for pickle
    \vspace{1ex}

    \end{boxedminipage}

    \label{object:__repr__}
    \index{object.\_\_repr\_\_ \textit{(function)}}

    \vspace{0.5ex}

    \begin{boxedminipage}{\textwidth}

    \raggedright \textbf{\_\_repr\_\_}(\textit{x})

    \vspace{-1.5ex}

    \rule{\textwidth}{0.5\fboxrule}

repr(x)
    \vspace{1ex}

    \end{boxedminipage}

    \label{object:__setattr__}
    \index{object.\_\_setattr\_\_ \textit{(function)}}

    \vspace{0.5ex}

    \begin{boxedminipage}{\textwidth}

    \raggedright \textbf{\_\_setattr\_\_}(\textit{...})

    \vspace{-1.5ex}

    \rule{\textwidth}{0.5\fboxrule}

x.{\_}{\_}setattr{\_}{\_}('name', value) {\textless}=={\textgreater} x.name = value
    \vspace{1ex}

    \end{boxedminipage}

    \label{object:__str__}
    \index{object.\_\_str\_\_ \textit{(function)}}

    \vspace{0.5ex}

    \begin{boxedminipage}{\textwidth}

    \raggedright \textbf{\_\_str\_\_}(\textit{x})

    \vspace{-1.5ex}

    \rule{\textwidth}{0.5\fboxrule}

str(x)
    \vspace{1ex}

    \end{boxedminipage}


%%%%%%%%%%%%%%%%%%%%%%%%%%%%%%%%%%%%%%%%%%%%%%%%%%%%%%%%%%%%%%%%%%%%%%%%%%%
%%                              Properties                               %%
%%%%%%%%%%%%%%%%%%%%%%%%%%%%%%%%%%%%%%%%%%%%%%%%%%%%%%%%%%%%%%%%%%%%%%%%%%%

  \subsubsection{Properties}

\begin{longtable}{|p{.30\textwidth}|p{.62\textwidth}|l}
\cline{1-2}
\cline{1-2} \centering \textbf{Name} & \centering \textbf{Description}& \\
\cline{1-2}
\endhead\cline{1-2}\multicolumn{3}{r}{\small\textit{continued on next page}}\\\endfoot\cline{1-2}
\endlastfoot\raggedright \_\-\_\-c\-l\-a\-s\-s\-\_\-\_\- & \raggedright \textbf{Value:} 
{\tt {\textless}attribute '\_\_class\_\_' of 'object' objects{\textgreater}}&\\
\cline{1-2}
\end{longtable}


%%%%%%%%%%%%%%%%%%%%%%%%%%%%%%%%%%%%%%%%%%%%%%%%%%%%%%%%%%%%%%%%%%%%%%%%%%%
%%                          Instance Variables                           %%
%%%%%%%%%%%%%%%%%%%%%%%%%%%%%%%%%%%%%%%%%%%%%%%%%%%%%%%%%%%%%%%%%%%%%%%%%%%

  \subsubsection{Instance Variables}

\begin{longtable}{|p{.30\textwidth}|p{.62\textwidth}|l}
\cline{1-2}
\cline{1-2} \centering \textbf{Name} & \centering \textbf{Description}& \\
\cline{1-2}
\endhead\cline{1-2}\multicolumn{3}{r}{\small\textit{continued on next page}}\\\endfoot\cline{1-2}
\endlastfoot\raggedright l\-r\-a\-t\-e\- & Learning rate used with the algorithm.&\\
\cline{1-2}
\end{longtable}

    \index{peach \textit{(package)}!peach.nn \textit{(package)}!peach.nn.lrules \textit{(module)}!peach.nn.lrules.WinnerTakesAll \textit{(class)}|)}

%%%%%%%%%%%%%%%%%%%%%%%%%%%%%%%%%%%%%%%%%%%%%%%%%%%%%%%%%%%%%%%%%%%%%%%%%%%
%%                           Class Description                           %%
%%%%%%%%%%%%%%%%%%%%%%%%%%%%%%%%%%%%%%%%%%%%%%%%%%%%%%%%%%%%%%%%%%%%%%%%%%%

    \index{peach \textit{(package)}!peach.nn \textit{(package)}!peach.nn.lrules \textit{(module)}!peach.nn.lrules.Competitive \textit{(class)}|(}
\subsection{Class Competitive}

    \label{peach:nn:lrules:Competitive}
\begin{tabular}{cccccccc}
% Line for object, linespec=[False, False]
\multicolumn{2}{r}{\settowidth{\BCL}{object}\multirow{2}{\BCL}{object}}
&&
&&
  \\\cline{3-3}
  &&\multicolumn{1}{c|}{}
&&
&&
  \\
% Line for peach.nn.lrules.SOMLearning, linespec=[False]
\multicolumn{4}{r}{\settowidth{\BCL}{peach.nn.lrules.SOMLearning}\multirow{2}{\BCL}{peach.nn.lrules.SOMLearning}}
&&
  \\\cline{5-5}
  &&&&\multicolumn{1}{c|}{}
&&
  \\
&&&&\multicolumn{2}{l}{\textbf{peach.nn.lrules.Competitive}}
\end{tabular}


Competitive learning with time adjust of the learning rate.

A competitive strategy detects the winner on the self-organizing map and
adjusts it in the direction of the input vector, scaled by the learning
rate. Its tendency is to cluster around the gravity center of the points in
the training set. As time passes, the learning rate grows smaller, this
allows for better adjustment of the synaptic weights.

%%%%%%%%%%%%%%%%%%%%%%%%%%%%%%%%%%%%%%%%%%%%%%%%%%%%%%%%%%%%%%%%%%%%%%%%%%%
%%                                Methods                                %%
%%%%%%%%%%%%%%%%%%%%%%%%%%%%%%%%%%%%%%%%%%%%%%%%%%%%%%%%%%%%%%%%%%%%%%%%%%%

  \subsubsection{Methods}

    \vspace{0.5ex}

    \begin{boxedminipage}{\textwidth}

    \raggedright \textbf{\_\_init\_\_}(\textit{self}, \textit{lrate}=\texttt{0.05}, \textit{tl}=\texttt{1000.0})

    \vspace{-1.5ex}

    \rule{\textwidth}{0.5\fboxrule}

Initializes the object.
    \vspace{1ex}

      \textbf{Parameters}
      \begin{quote}
        \begin{Ventry}{xxxxx}

          \item[lrate]


Learning rate to be used in the algorithm. Defaults to 0.05.
          \item[tl]


Time constant that measures how many iterations will be needed to
reduce the learning rate to a small value. Defaults to 1000.
        \end{Ventry}

      \end{quote}

    \vspace{1ex}

      Overrides: object.\_\_init\_\_

    \end{boxedminipage}

    \vspace{0.5ex}

    \begin{boxedminipage}{\textwidth}

    \raggedright \textbf{\_\_call\_\_}(\textit{self}, \textit{nn}, \textit{x})

    \vspace{-1.5ex}

    \rule{\textwidth}{0.5\fboxrule}

The \texttt{{\_}{\_}call{\_}{\_}} interface.

The learning implementation. Read the documentation for the base class
for more information. A call to the class should have the following
parameters:
    \vspace{1ex}

      \textbf{Parameters}
      \begin{quote}
        \begin{Ventry}{xx}

          \item[nn]


A \texttt{SOM} neural network instance that is going to be modified by
the learning algorithm. The modification is made \emph{in loco}, that is,
the synaptic weights of \texttt{nn} should be modified in place, and not
returned from this function.
          \item[x]


The input vector from the training set.
        \end{Ventry}

      \end{quote}

    \vspace{1ex}

      Overrides: peach.nn.lrules.SOMLearning.\_\_call\_\_

    \end{boxedminipage}

    \label{object:__delattr__}
    \index{object.\_\_delattr\_\_ \textit{(function)}}

    \vspace{0.5ex}

    \begin{boxedminipage}{\textwidth}

    \raggedright \textbf{\_\_delattr\_\_}(\textit{...})

    \vspace{-1.5ex}

    \rule{\textwidth}{0.5\fboxrule}

x.{\_}{\_}delattr{\_}{\_}('name') {\textless}=={\textgreater} del x.name
    \vspace{1ex}

    \end{boxedminipage}

    \label{object:__getattribute__}
    \index{object.\_\_getattribute\_\_ \textit{(function)}}

    \vspace{0.5ex}

    \begin{boxedminipage}{\textwidth}

    \raggedright \textbf{\_\_getattribute\_\_}(\textit{...})

    \vspace{-1.5ex}

    \rule{\textwidth}{0.5\fboxrule}

x.{\_}{\_}getattribute{\_}{\_}('name') {\textless}=={\textgreater} x.name
    \vspace{1ex}

    \end{boxedminipage}

    \label{object:__hash__}
    \index{object.\_\_hash\_\_ \textit{(function)}}

    \vspace{0.5ex}

    \begin{boxedminipage}{\textwidth}

    \raggedright \textbf{\_\_hash\_\_}(\textit{x})

    \vspace{-1.5ex}

    \rule{\textwidth}{0.5\fboxrule}

hash(x)
    \vspace{1ex}

    \end{boxedminipage}

    \label{object:__new__}
    \index{object.\_\_new\_\_ \textit{(function)}}

    \vspace{0.5ex}

    \begin{boxedminipage}{\textwidth}

    \raggedright \textbf{\_\_new\_\_}(\textit{T}, \textit{S}, \textit{...})

      \textbf{Return Value}
      \begin{quote}
\begin{alltt}
a new object with type S, a subtype of T
\end{alltt}

      \end{quote}

    \vspace{1ex}

    \end{boxedminipage}

    \label{object:__reduce__}
    \index{object.\_\_reduce\_\_ \textit{(function)}}

    \vspace{0.5ex}

    \begin{boxedminipage}{\textwidth}

    \raggedright \textbf{\_\_reduce\_\_}(\textit{...})

    \vspace{-1.5ex}

    \rule{\textwidth}{0.5\fboxrule}

helper for pickle
    \vspace{1ex}

    \end{boxedminipage}

    \label{object:__reduce_ex__}
    \index{object.\_\_reduce\_ex\_\_ \textit{(function)}}

    \vspace{0.5ex}

    \begin{boxedminipage}{\textwidth}

    \raggedright \textbf{\_\_reduce\_ex\_\_}(\textit{...})

    \vspace{-1.5ex}

    \rule{\textwidth}{0.5\fboxrule}

helper for pickle
    \vspace{1ex}

    \end{boxedminipage}

    \label{object:__repr__}
    \index{object.\_\_repr\_\_ \textit{(function)}}

    \vspace{0.5ex}

    \begin{boxedminipage}{\textwidth}

    \raggedright \textbf{\_\_repr\_\_}(\textit{x})

    \vspace{-1.5ex}

    \rule{\textwidth}{0.5\fboxrule}

repr(x)
    \vspace{1ex}

    \end{boxedminipage}

    \label{object:__setattr__}
    \index{object.\_\_setattr\_\_ \textit{(function)}}

    \vspace{0.5ex}

    \begin{boxedminipage}{\textwidth}

    \raggedright \textbf{\_\_setattr\_\_}(\textit{...})

    \vspace{-1.5ex}

    \rule{\textwidth}{0.5\fboxrule}

x.{\_}{\_}setattr{\_}{\_}('name', value) {\textless}=={\textgreater} x.name = value
    \vspace{1ex}

    \end{boxedminipage}

    \label{object:__str__}
    \index{object.\_\_str\_\_ \textit{(function)}}

    \vspace{0.5ex}

    \begin{boxedminipage}{\textwidth}

    \raggedright \textbf{\_\_str\_\_}(\textit{x})

    \vspace{-1.5ex}

    \rule{\textwidth}{0.5\fboxrule}

str(x)
    \vspace{1ex}

    \end{boxedminipage}


%%%%%%%%%%%%%%%%%%%%%%%%%%%%%%%%%%%%%%%%%%%%%%%%%%%%%%%%%%%%%%%%%%%%%%%%%%%
%%                              Properties                               %%
%%%%%%%%%%%%%%%%%%%%%%%%%%%%%%%%%%%%%%%%%%%%%%%%%%%%%%%%%%%%%%%%%%%%%%%%%%%

  \subsubsection{Properties}

\begin{longtable}{|p{.30\textwidth}|p{.62\textwidth}|l}
\cline{1-2}
\cline{1-2} \centering \textbf{Name} & \centering \textbf{Description}& \\
\cline{1-2}
\endhead\cline{1-2}\multicolumn{3}{r}{\small\textit{continued on next page}}\\\endfoot\cline{1-2}
\endlastfoot\raggedright \_\-\_\-c\-l\-a\-s\-s\-\_\-\_\- & \raggedright \textbf{Value:} 
{\tt {\textless}attribute '\_\_class\_\_' of 'object' objects{\textgreater}}&\\
\cline{1-2}
\end{longtable}

    \index{peach \textit{(package)}!peach.nn \textit{(package)}!peach.nn.lrules \textit{(module)}!peach.nn.lrules.Competitive \textit{(class)}|)}

%%%%%%%%%%%%%%%%%%%%%%%%%%%%%%%%%%%%%%%%%%%%%%%%%%%%%%%%%%%%%%%%%%%%%%%%%%%
%%                           Class Description                           %%
%%%%%%%%%%%%%%%%%%%%%%%%%%%%%%%%%%%%%%%%%%%%%%%%%%%%%%%%%%%%%%%%%%%%%%%%%%%

    \index{peach \textit{(package)}!peach.nn \textit{(package)}!peach.nn.lrules \textit{(module)}!peach.nn.lrules.Cooperative \textit{(class)}|(}
\subsection{Class Cooperative}

    \label{peach:nn:lrules:Cooperative}
\begin{tabular}{cccccccc}
% Line for object, linespec=[False, False]
\multicolumn{2}{r}{\settowidth{\BCL}{object}\multirow{2}{\BCL}{object}}
&&
&&
  \\\cline{3-3}
  &&\multicolumn{1}{c|}{}
&&
&&
  \\
% Line for peach.nn.lrules.SOMLearning, linespec=[False]
\multicolumn{4}{r}{\settowidth{\BCL}{peach.nn.lrules.SOMLearning}\multirow{2}{\BCL}{peach.nn.lrules.SOMLearning}}
&&
  \\\cline{5-5}
  &&&&\multicolumn{1}{c|}{}
&&
  \\
&&&&\multicolumn{2}{l}{\textbf{peach.nn.lrules.Cooperative}}
\end{tabular}


Cooperative learning with time adjust of the learning rate and neighborhood
function to propagate cooperation

A cooperative strategy detects the winner on the self-organizing map and
adjusts it in the direction of the input vector, scaled by the learning
rate. Its tendency is to cluster around the gravity center of the points in
the training set. As time passes, the learning rate grows smaller, this
allows for better adjustment of the synaptic weights.

Also, a neighborhood is defined on the winner. Neurons close to the winner
are also updated in the direction of the input vector, although with a
smaller scale determined by the neighborhood function. A neighborhood
function is 1. at 0., and decreases monotonically as the distance increases.

\emph{There are issues with this class!} -{}- some of the class capabilities are
yet to be developed.

%%%%%%%%%%%%%%%%%%%%%%%%%%%%%%%%%%%%%%%%%%%%%%%%%%%%%%%%%%%%%%%%%%%%%%%%%%%
%%                                Methods                                %%
%%%%%%%%%%%%%%%%%%%%%%%%%%%%%%%%%%%%%%%%%%%%%%%%%%%%%%%%%%%%%%%%%%%%%%%%%%%

  \subsubsection{Methods}

    \vspace{0.5ex}

    \begin{boxedminipage}{\textwidth}

    \raggedright \textbf{\_\_init\_\_}(\textit{self}, \textit{lrate}=\texttt{0.05}, \textit{tl}=\texttt{1000}, \textit{tn}=\texttt{1000})

    \vspace{-1.5ex}

    \rule{\textwidth}{0.5\fboxrule}

Initializes the object.
    \vspace{1ex}

      \textbf{Parameters}
      \begin{quote}
        \begin{Ventry}{xxxxx}

          \item[lrate]


Learning rate to be used in the algorithm. Defaults to 0.05.
          \item[tl]


Time constant that measures how many iterations will be needed to
reduce the learning rate to a small value. Defaults to 1000.
          \item[tn]


Time constant that measures how many iterations will be needed to
shrink the neighborhood. Defaults to 1000.
        \end{Ventry}

      \end{quote}

    \vspace{1ex}

      Overrides: object.\_\_init\_\_

    \end{boxedminipage}

    \vspace{0.5ex}

    \begin{boxedminipage}{\textwidth}

    \raggedright \textbf{\_\_call\_\_}(\textit{self}, \textit{nn}, \textit{x})

    \vspace{-1.5ex}

    \rule{\textwidth}{0.5\fboxrule}

The \texttt{{\_}{\_}call{\_}{\_}} interface.

The learning implementation. Read the documentation for the base class
for more information. A call to the class should have the following
parameters:
    \vspace{1ex}

      \textbf{Parameters}
      \begin{quote}
        \begin{Ventry}{xx}

          \item[nn]


A \texttt{SOM} neural network instance that is going to be modified by
the learning algorithm. The modification is made \emph{in loco}, that is,
the synaptic weights of \texttt{nn} should be modified in place, and not
returned from this function.
          \item[x]


The input vector from the training set.
        \end{Ventry}

      \end{quote}

    \vspace{1ex}

      Overrides: peach.nn.lrules.SOMLearning.\_\_call\_\_

    \end{boxedminipage}

    \label{object:__delattr__}
    \index{object.\_\_delattr\_\_ \textit{(function)}}

    \vspace{0.5ex}

    \begin{boxedminipage}{\textwidth}

    \raggedright \textbf{\_\_delattr\_\_}(\textit{...})

    \vspace{-1.5ex}

    \rule{\textwidth}{0.5\fboxrule}

x.{\_}{\_}delattr{\_}{\_}('name') {\textless}=={\textgreater} del x.name
    \vspace{1ex}

    \end{boxedminipage}

    \label{object:__getattribute__}
    \index{object.\_\_getattribute\_\_ \textit{(function)}}

    \vspace{0.5ex}

    \begin{boxedminipage}{\textwidth}

    \raggedright \textbf{\_\_getattribute\_\_}(\textit{...})

    \vspace{-1.5ex}

    \rule{\textwidth}{0.5\fboxrule}

x.{\_}{\_}getattribute{\_}{\_}('name') {\textless}=={\textgreater} x.name
    \vspace{1ex}

    \end{boxedminipage}

    \label{object:__hash__}
    \index{object.\_\_hash\_\_ \textit{(function)}}

    \vspace{0.5ex}

    \begin{boxedminipage}{\textwidth}

    \raggedright \textbf{\_\_hash\_\_}(\textit{x})

    \vspace{-1.5ex}

    \rule{\textwidth}{0.5\fboxrule}

hash(x)
    \vspace{1ex}

    \end{boxedminipage}

    \label{object:__new__}
    \index{object.\_\_new\_\_ \textit{(function)}}

    \vspace{0.5ex}

    \begin{boxedminipage}{\textwidth}

    \raggedright \textbf{\_\_new\_\_}(\textit{T}, \textit{S}, \textit{...})

      \textbf{Return Value}
      \begin{quote}
\begin{alltt}
a new object with type S, a subtype of T
\end{alltt}

      \end{quote}

    \vspace{1ex}

    \end{boxedminipage}

    \label{object:__reduce__}
    \index{object.\_\_reduce\_\_ \textit{(function)}}

    \vspace{0.5ex}

    \begin{boxedminipage}{\textwidth}

    \raggedright \textbf{\_\_reduce\_\_}(\textit{...})

    \vspace{-1.5ex}

    \rule{\textwidth}{0.5\fboxrule}

helper for pickle
    \vspace{1ex}

    \end{boxedminipage}

    \label{object:__reduce_ex__}
    \index{object.\_\_reduce\_ex\_\_ \textit{(function)}}

    \vspace{0.5ex}

    \begin{boxedminipage}{\textwidth}

    \raggedright \textbf{\_\_reduce\_ex\_\_}(\textit{...})

    \vspace{-1.5ex}

    \rule{\textwidth}{0.5\fboxrule}

helper for pickle
    \vspace{1ex}

    \end{boxedminipage}

    \label{object:__repr__}
    \index{object.\_\_repr\_\_ \textit{(function)}}

    \vspace{0.5ex}

    \begin{boxedminipage}{\textwidth}

    \raggedright \textbf{\_\_repr\_\_}(\textit{x})

    \vspace{-1.5ex}

    \rule{\textwidth}{0.5\fboxrule}

repr(x)
    \vspace{1ex}

    \end{boxedminipage}

    \label{object:__setattr__}
    \index{object.\_\_setattr\_\_ \textit{(function)}}

    \vspace{0.5ex}

    \begin{boxedminipage}{\textwidth}

    \raggedright \textbf{\_\_setattr\_\_}(\textit{...})

    \vspace{-1.5ex}

    \rule{\textwidth}{0.5\fboxrule}

x.{\_}{\_}setattr{\_}{\_}('name', value) {\textless}=={\textgreater} x.name = value
    \vspace{1ex}

    \end{boxedminipage}

    \label{object:__str__}
    \index{object.\_\_str\_\_ \textit{(function)}}

    \vspace{0.5ex}

    \begin{boxedminipage}{\textwidth}

    \raggedright \textbf{\_\_str\_\_}(\textit{x})

    \vspace{-1.5ex}

    \rule{\textwidth}{0.5\fboxrule}

str(x)
    \vspace{1ex}

    \end{boxedminipage}


%%%%%%%%%%%%%%%%%%%%%%%%%%%%%%%%%%%%%%%%%%%%%%%%%%%%%%%%%%%%%%%%%%%%%%%%%%%
%%                              Properties                               %%
%%%%%%%%%%%%%%%%%%%%%%%%%%%%%%%%%%%%%%%%%%%%%%%%%%%%%%%%%%%%%%%%%%%%%%%%%%%

  \subsubsection{Properties}

\begin{longtable}{|p{.30\textwidth}|p{.62\textwidth}|l}
\cline{1-2}
\cline{1-2} \centering \textbf{Name} & \centering \textbf{Description}& \\
\cline{1-2}
\endhead\cline{1-2}\multicolumn{3}{r}{\small\textit{continued on next page}}\\\endfoot\cline{1-2}
\endlastfoot\raggedright \_\-\_\-c\-l\-a\-s\-s\-\_\-\_\- & \raggedright \textbf{Value:} 
{\tt {\textless}attribute '\_\_class\_\_' of 'object' objects{\textgreater}}&\\
\cline{1-2}
\end{longtable}

    \index{peach \textit{(package)}!peach.nn \textit{(package)}!peach.nn.lrules \textit{(module)}!peach.nn.lrules.Cooperative \textit{(class)}|)}
    \index{peach \textit{(package)}!peach.nn \textit{(package)}!peach.nn.lrules \textit{(module)}|)}

%
% API Documentation for Peach - Computational Intelligence for Python
% Module peach.nn.mem
%
% Generated by epydoc 3.0.1
% [Thu Jul 28 16:37:48 2011]
%

%%%%%%%%%%%%%%%%%%%%%%%%%%%%%%%%%%%%%%%%%%%%%%%%%%%%%%%%%%%%%%%%%%%%%%%%%%%
%%                          Module Description                           %%
%%%%%%%%%%%%%%%%%%%%%%%%%%%%%%%%%%%%%%%%%%%%%%%%%%%%%%%%%%%%%%%%%%%%%%%%%%%

    \index{peach \textit{(package)}!peach.nn \textit{(package)}!peach.nn.mem \textit{(module)}|(}
\section{Module peach.nn.mem}

    \label{peach:nn:mem}

Associative memories and Hopfield network model.

This sub-package implements associative memories. In associative memories,
information is recovered by supplying not an exact index (such as in their
usual counterparts), but supplying an index simmilar enough that the information
can be deduced from what is stored in its synaptic weights. There are a number
of different memories of this kind.

The most common type is the Hopfield network. A Hopfield network is a recurrent
self-associative memory. Although there are real-valued versions of the network,
the binary type is more common. In it, patterns are recovered from an initial
estimate through an iterative process.

%%%%%%%%%%%%%%%%%%%%%%%%%%%%%%%%%%%%%%%%%%%%%%%%%%%%%%%%%%%%%%%%%%%%%%%%%%%
%%                               Functions                               %%
%%%%%%%%%%%%%%%%%%%%%%%%%%%%%%%%%%%%%%%%%%%%%%%%%%%%%%%%%%%%%%%%%%%%%%%%%%%

  \subsection{Functions}

    \label{peach:nn:nnet:randn}
    \index{peach \textit{(package)}!peach.nn \textit{(package)}!peach.nn.nnet \textit{(module)}!peach.nn.nnet.randn \textit{(function)}}

    \vspace{0.5ex}

\hspace{.8\funcindent}\begin{boxedminipage}{\funcwidth}

    \raggedright \textbf{randn}(\textit{d0}, \textit{d1}, \textit{dn}, \textit{...})

    \vspace{-1.5ex}

    \rule{\textwidth}{0.5\fboxrule}
\setlength{\parskip}{2ex}

Returns zero-mean, unit-variance Gaussian random numbers in an
array of shape (d0, d1, ..., dn).
%
\begin{description}
\item[{Note:  This is a convenience function. If you want an}] \leavevmode 
interface that takes a tuple as the first argument
use numpy.random.standard\_normal(shape\_tuple).

\end{description}
\setlength{\parskip}{1ex}
    \end{boxedminipage}


%%%%%%%%%%%%%%%%%%%%%%%%%%%%%%%%%%%%%%%%%%%%%%%%%%%%%%%%%%%%%%%%%%%%%%%%%%%
%%                               Variables                               %%
%%%%%%%%%%%%%%%%%%%%%%%%%%%%%%%%%%%%%%%%%%%%%%%%%%%%%%%%%%%%%%%%%%%%%%%%%%%

  \subsection{Variables}

    \vspace{-1cm}
\hspace{\varindent}\begin{longtable}{|p{\varnamewidth}|p{\vardescrwidth}|l}
\cline{1-2}
\cline{1-2} \centering \textbf{Name} & \centering \textbf{Description}& \\
\cline{1-2}
\endhead\cline{1-2}\multicolumn{3}{r}{\small\textit{continued on next page}}\\\endfoot\cline{1-2}
\endlastfoot\raggedright \_\-\_\-d\-o\-c\-\_\-\_\- & \raggedright \textbf{Value:} 
{\tt \texttt{...}}&\\
\cline{1-2}
\raggedright \_\-\_\-p\-a\-c\-k\-a\-g\-e\-\_\-\_\- & \raggedright \textbf{Value:} 
{\tt \texttt{'}\texttt{peach.nn}\texttt{'}}&\\
\cline{1-2}
\raggedright a\-r\-c\-t\-a\-n\- & \raggedright \textbf{Value:} 
{\tt {\textless}ufunc 'arctan'{\textgreater}}&\\
\cline{1-2}
\raggedright c\-o\-s\-h\- & \raggedright \textbf{Value:} 
{\tt {\textless}ufunc 'cosh'{\textgreater}}&\\
\cline{1-2}
\raggedright e\-x\-p\- & \raggedright \textbf{Value:} 
{\tt {\textless}ufunc 'exp'{\textgreater}}&\\
\cline{1-2}
\raggedright p\-i\- & \raggedright \textbf{Value:} 
{\tt 3.14159265359}&\\
\cline{1-2}
\raggedright s\-i\-g\-n\- & \raggedright \textbf{Value:} 
{\tt {\textless}ufunc 'sign'{\textgreater}}&\\
\cline{1-2}
\raggedright t\-a\-n\-h\- & \raggedright \textbf{Value:} 
{\tt {\textless}ufunc 'tanh'{\textgreater}}&\\
\cline{1-2}
\end{longtable}


%%%%%%%%%%%%%%%%%%%%%%%%%%%%%%%%%%%%%%%%%%%%%%%%%%%%%%%%%%%%%%%%%%%%%%%%%%%
%%                           Class Description                           %%
%%%%%%%%%%%%%%%%%%%%%%%%%%%%%%%%%%%%%%%%%%%%%%%%%%%%%%%%%%%%%%%%%%%%%%%%%%%

    \index{peach \textit{(package)}!peach.nn \textit{(package)}!peach.nn.mem \textit{(module)}!peach.nn.mem.Hopfield \textit{(class)}|(}
\subsection{Class Hopfield}

    \label{peach:nn:mem:Hopfield}
\begin{tabular}{cccccccc}
% Line for object, linespec=[False, False]
\multicolumn{2}{r}{\settowidth{\BCL}{object}\multirow{2}{\BCL}{object}}
&&
&&
  \\\cline{3-3}
  &&\multicolumn{1}{c|}{}
&&
&&
  \\
% Line for peach.nn.base.Layer, linespec=[False]
\multicolumn{4}{r}{\settowidth{\BCL}{peach.nn.base.Layer}\multirow{2}{\BCL}{peach.nn.base.Layer}}
&&
  \\\cline{5-5}
  &&&&\multicolumn{1}{c|}{}
&&
  \\
&&&&\multicolumn{2}{l}{\textbf{peach.nn.mem.Hopfield}}
\end{tabular}


Hopfield neural network model

A Hopfield network is a recurrent network of one layer of neurons. There
output of every neuron is conected to the inputs of every other neuron, but
not to itself. This kind of network is autoassociative, or content-based
memory. That means that, given a noisy version of a pattern stored in it,
the network is capable of, through an iterative algorithm, recover the
original pattern, removing the noise. There is a limit in the quantity of
patterns that can be stored without causing error, and if a pattern is
stored, its negated form is also stored.

This is the binary form of the Hopfield network, which is the most common
form. It implements a \texttt{Layer} of neurons, without bias, and with the
Signum as the activation function.

%%%%%%%%%%%%%%%%%%%%%%%%%%%%%%%%%%%%%%%%%%%%%%%%%%%%%%%%%%%%%%%%%%%%%%%%%%%
%%                                Methods                                %%
%%%%%%%%%%%%%%%%%%%%%%%%%%%%%%%%%%%%%%%%%%%%%%%%%%%%%%%%%%%%%%%%%%%%%%%%%%%

  \subsubsection{Methods}

    \vspace{0.5ex}

\hspace{.8\funcindent}\begin{boxedminipage}{\funcwidth}

    \raggedright \textbf{\_\_init\_\_}(\textit{self}, \textit{size}, \textit{phi}={\tt {\textless}class 'peach.nn.af.Signum'{\textgreater}})

    \vspace{-1.5ex}

    \rule{\textwidth}{0.5\fboxrule}
\setlength{\parskip}{2ex}

Initializes the Hopfield network.

The Hopfield network is implemented as a layer of neurons.
\setlength{\parskip}{1ex}
      \textbf{Parameters}
      \vspace{-1ex}

      \begin{quote}
        \begin{Ventry}{xxxx}

          \item[size]


The number of neurons in the network. In a Hopfield network, the
number of neurons is also the number of inputs in each neuron, and
the dimensionality of the patterns to be stored and recovered.
          \item[phi]


The activation function. Traditionally, the Hopfield network uses
the signum function as activation. This is the default value.
        \end{Ventry}

      \end{quote}

      Overrides: object.\_\_init\_\_

    \end{boxedminipage}

    \label{peach:nn:mem:Hopfield:learn}
    \index{peach \textit{(package)}!peach.nn \textit{(package)}!peach.nn.mem \textit{(module)}!peach.nn.mem.Hopfield \textit{(class)}!peach.nn.mem.Hopfield.learn \textit{(method)}}

    \vspace{0.5ex}

\hspace{.8\funcindent}\begin{boxedminipage}{\funcwidth}

    \raggedright \textbf{learn}(\textit{self}, \textit{x})

    \vspace{-1.5ex}

    \rule{\textwidth}{0.5\fboxrule}
\setlength{\parskip}{2ex}

Applies one example of the training set to the network.

Training a Hopfield network is not exactly an iterative procedure. The
network usually stores a small number of patterns, and the learning
procedure consists only in computing the synaptic weight matrix, which
can be done in very few steps (in fact, just the number of patterns).
This method is here for consistency with the rest of the library, but
it works, anyway.
\setlength{\parskip}{1ex}
      \textbf{Parameters}
      \vspace{-1ex}

      \begin{quote}
        \begin{Ventry}{x}

          \item[x]


The pattern to be stored. It must be a vector with the same size as
the network, or else an exception will be raised. The pattern can be
of any dimensionality, but it will internally be converted to a
column vector.
        \end{Ventry}

      \end{quote}

    \end{boxedminipage}

    \label{peach:nn:mem:Hopfield:train}
    \index{peach \textit{(package)}!peach.nn \textit{(package)}!peach.nn.mem \textit{(module)}!peach.nn.mem.Hopfield \textit{(class)}!peach.nn.mem.Hopfield.train \textit{(method)}}

    \vspace{0.5ex}

\hspace{.8\funcindent}\begin{boxedminipage}{\funcwidth}

    \raggedright \textbf{train}(\textit{self}, \textit{train\_set})

    \vspace{-1.5ex}

    \rule{\textwidth}{0.5\fboxrule}
\setlength{\parskip}{2ex}

Presents a training set to the network

This method stores all the patterns of the training set in the weight
matrix. It calls the \texttt{learn} method for every pattern in the set.
\setlength{\parskip}{1ex}
      \textbf{Parameters}
      \vspace{-1ex}

      \begin{quote}
        \begin{Ventry}{xxxxxxxxx}

          \item[train\_set]


A list containing all the patterns to be stored in the network. Each
pattern is a vector of any dimensions, which are converted
internally to a column vector.
        \end{Ventry}

      \end{quote}

    \end{boxedminipage}

    \label{peach:nn:mem:Hopfield:step}
    \index{peach \textit{(package)}!peach.nn \textit{(package)}!peach.nn.mem \textit{(module)}!peach.nn.mem.Hopfield \textit{(class)}!peach.nn.mem.Hopfield.step \textit{(method)}}

    \vspace{0.5ex}

\hspace{.8\funcindent}\begin{boxedminipage}{\funcwidth}

    \raggedright \textbf{step}(\textit{self}, \textit{x})

    \vspace{-1.5ex}

    \rule{\textwidth}{0.5\fboxrule}
\setlength{\parskip}{2ex}

Performs a step in the recovering procedure

The algorithm for recovering the patterns stored in a Hopfield network
is an iterative algorithm which goes from a starting test pattern (a
stored pattern with noise) and recovers the noiseless version -{}- if
possible. This method takes the test pattern and performs one step of
the convergence
\setlength{\parskip}{1ex}
      \textbf{Parameters}
      \vspace{-1ex}

      \begin{quote}
        \begin{Ventry}{x}

          \item[x]


The noisy pattern.
        \end{Ventry}

      \end{quote}

      \textbf{Return Value}
    \vspace{-1ex}

      \begin{quote}

The result of one step of the convergence. This might be the same as
the input pattern, or the pattern with one component inverted.
      \end{quote}

    \end{boxedminipage}

    \vspace{0.5ex}

\hspace{.8\funcindent}\begin{boxedminipage}{\funcwidth}

    \raggedright \textbf{\_\_call\_\_}(\textit{self}, \textit{x}, \textit{imax}={\tt 2000}, \textit{eqmax}={\tt 100})

    \vspace{-1.5ex}

    \rule{\textwidth}{0.5\fboxrule}
\setlength{\parskip}{2ex}

Recovers a stored pattern

The \texttt{\_\_call\_\_} interface should be called if a memory needs to be
recovered from the network. Given a noisy pattern \texttt{x}, the algorithm
will be executed until convergence or a maximum number of iterations
occur. This method repeatedly calls the \texttt{step} method until a stop
condition is reached. The stop condition is the maximum number of
iterations, or a number of iterations where no changes are found in the
retrieved pattern.
\setlength{\parskip}{1ex}
      \textbf{Parameters}
      \vspace{-1ex}

      \begin{quote}
        \begin{Ventry}{xxxxx}

          \item[x]


The noisy pattern vector presented to the network.
          \item[imax]


The maximum number of iterations the algorithm is to be repeated.
When this number of iterations is reached, the algorithm will stop,
whether the pattern was found or not. Defaults to 2000.
          \item[eqmax]


The maximum number of iterations the algorithm will be repeated if
no changes occur in the retrieval of the pattern. At each iteration
of the algorithm, a component might change. It is considered that,
if a number of iterations are performed and no changes are found in
the pattern, then the algorithm converged, and it stops. Defaults to
100.
        \end{Ventry}

      \end{quote}

      \textbf{Return Value}
    \vspace{-1ex}

      \begin{quote}

The vector containing the recovered pattern from the stored memories.
      \end{quote}

      Overrides: peach.nn.base.Layer.\_\_call\_\_

    \end{boxedminipage}


\large{\textbf{\textit{Inherited from peach.nn.base.Layer\textit{(Section \ref{peach:nn:base:Layer})}}}}

\begin{quote}
\_\_getitem\_\_(), \_\_setitem\_\_()
\end{quote}

\large{\textbf{\textit{Inherited from object}}}

\begin{quote}
\_\_delattr\_\_(), \_\_format\_\_(), \_\_getattribute\_\_(), \_\_hash\_\_(), \_\_new\_\_(), \_\_reduce\_\_(), \_\_reduce\_ex\_\_(), \_\_repr\_\_(), \_\_setattr\_\_(), \_\_sizeof\_\_(), \_\_str\_\_(), \_\_subclasshook\_\_()
\end{quote}

%%%%%%%%%%%%%%%%%%%%%%%%%%%%%%%%%%%%%%%%%%%%%%%%%%%%%%%%%%%%%%%%%%%%%%%%%%%
%%                              Properties                               %%
%%%%%%%%%%%%%%%%%%%%%%%%%%%%%%%%%%%%%%%%%%%%%%%%%%%%%%%%%%%%%%%%%%%%%%%%%%%

  \subsubsection{Properties}

    \vspace{-1cm}
\hspace{\varindent}\begin{longtable}{|p{\varnamewidth}|p{\vardescrwidth}|l}
\cline{1-2}
\cline{1-2} \centering \textbf{Name} & \centering \textbf{Description}& \\
\cline{1-2}
\endhead\cline{1-2}\multicolumn{3}{r}{\small\textit{continued on next page}}\\\endfoot\cline{1-2}
\endlastfoot\raggedright i\-n\-p\-u\-t\-s\- & &\\
\cline{1-2}
\raggedright w\-e\-i\-g\-h\-t\-s\- & &\\
\cline{1-2}
\multicolumn{2}{|l|}{\textit{Inherited from peach.nn.base.Layer \textit{(Section \ref{peach:nn:base:Layer})}}}\\
\multicolumn{2}{|p{\varwidth}|}{\raggedright bias, phi, shape, size, v, y}\\
\cline{1-2}
\multicolumn{2}{|l|}{\textit{Inherited from object}}\\
\multicolumn{2}{|p{\varwidth}|}{\raggedright \_\_class\_\_}\\
\cline{1-2}
\end{longtable}

    \index{peach \textit{(package)}!peach.nn \textit{(package)}!peach.nn.mem \textit{(module)}!peach.nn.mem.Hopfield \textit{(class)}|)}
    \index{peach \textit{(package)}!peach.nn \textit{(package)}!peach.nn.mem \textit{(module)}|)}

%
% API Documentation for Peach - Computational Intelligence for Python
% Module peach.nn.nnet
%
% Generated by epydoc 3.0.1
% [Mon Jan 24 15:39:52 2011]
%

%%%%%%%%%%%%%%%%%%%%%%%%%%%%%%%%%%%%%%%%%%%%%%%%%%%%%%%%%%%%%%%%%%%%%%%%%%%
%%                          Module Description                           %%
%%%%%%%%%%%%%%%%%%%%%%%%%%%%%%%%%%%%%%%%%%%%%%%%%%%%%%%%%%%%%%%%%%%%%%%%%%%

    \index{peach \textit{(package)}!peach.nn \textit{(package)}!peach.nn.nnet \textit{(module)}|(}
\section{Module peach.nn.nnet}

    \label{peach:nn:nnet}

Basic topologies of neural networks.

This sub-package implements various neural network topologies, see the complete
list below. These topologies are implemented using the \texttt{Layer} class of the
\texttt{base} sub-package. Please, consult the documentation of that module for more
information on layers of neurons. The neural nets implemented here don't derive
from the \texttt{Layer} class, instead, they have instance variables to take control
of them. Thus, there is no base class for networks. While subclassing the
classes of this module is usually safe, it is recomended that a new kind of
net is developed from the ground up.

%%%%%%%%%%%%%%%%%%%%%%%%%%%%%%%%%%%%%%%%%%%%%%%%%%%%%%%%%%%%%%%%%%%%%%%%%%%
%%                               Functions                               %%
%%%%%%%%%%%%%%%%%%%%%%%%%%%%%%%%%%%%%%%%%%%%%%%%%%%%%%%%%%%%%%%%%%%%%%%%%%%

  \subsection{Functions}

    \label{peach:nn:nnet:randn}
    \index{peach \textit{(package)}!peach.nn \textit{(package)}!peach.nn.nnet \textit{(module)}!peach.nn.nnet.randn \textit{(function)}}

    \vspace{0.5ex}

\hspace{.8\funcindent}\begin{boxedminipage}{\funcwidth}

    \raggedright \textbf{randn}(\textit{d0}, \textit{d1}, \textit{dn}, \textit{...})

    \vspace{-1.5ex}

    \rule{\textwidth}{0.5\fboxrule}
\setlength{\parskip}{2ex}

Returns zero-mean, unit-variance Gaussian random numbers in an
array of shape (d0, d1, ..., dn).
%
\begin{description}
\item[{Note:  This is a convenience function. If you want an}] \leavevmode 
interface that takes a tuple as the first argument
use numpy.random.standard\_normal(shape\_tuple).

\end{description}
\setlength{\parskip}{1ex}
    \end{boxedminipage}


%%%%%%%%%%%%%%%%%%%%%%%%%%%%%%%%%%%%%%%%%%%%%%%%%%%%%%%%%%%%%%%%%%%%%%%%%%%
%%                               Variables                               %%
%%%%%%%%%%%%%%%%%%%%%%%%%%%%%%%%%%%%%%%%%%%%%%%%%%%%%%%%%%%%%%%%%%%%%%%%%%%

  \subsection{Variables}

    \vspace{-1cm}
\hspace{\varindent}\begin{longtable}{|p{\varnamewidth}|p{\vardescrwidth}|l}
\cline{1-2}
\cline{1-2} \centering \textbf{Name} & \centering \textbf{Description}& \\
\cline{1-2}
\endhead\cline{1-2}\multicolumn{3}{r}{\small\textit{continued on next page}}\\\endfoot\cline{1-2}
\endlastfoot\raggedright \_\-\_\-d\-o\-c\-\_\-\_\- & \raggedright \textbf{Value:} 
{\tt \texttt{...}}&\\
\cline{1-2}
\raggedright \_\-\_\-p\-a\-c\-k\-a\-g\-e\-\_\-\_\- & \raggedright \textbf{Value:} 
{\tt \texttt{'}\texttt{peach.nn}\texttt{'}}&\\
\cline{1-2}
\raggedright a\-r\-c\-t\-a\-n\- & \raggedright \textbf{Value:} 
{\tt {\textless}ufunc 'arctan'{\textgreater}}&\\
\cline{1-2}
\raggedright c\-o\-s\-h\- & \raggedright \textbf{Value:} 
{\tt {\textless}ufunc 'cosh'{\textgreater}}&\\
\cline{1-2}
\raggedright e\-x\-p\- & \raggedright \textbf{Value:} 
{\tt {\textless}ufunc 'exp'{\textgreater}}&\\
\cline{1-2}
\raggedright p\-i\- & \raggedright \textbf{Value:} 
{\tt 3.14159265359}&\\
\cline{1-2}
\raggedright t\-a\-n\-h\- & \raggedright \textbf{Value:} 
{\tt {\textless}ufunc 'tanh'{\textgreater}}&\\
\cline{1-2}
\end{longtable}


%%%%%%%%%%%%%%%%%%%%%%%%%%%%%%%%%%%%%%%%%%%%%%%%%%%%%%%%%%%%%%%%%%%%%%%%%%%
%%                           Class Description                           %%
%%%%%%%%%%%%%%%%%%%%%%%%%%%%%%%%%%%%%%%%%%%%%%%%%%%%%%%%%%%%%%%%%%%%%%%%%%%

    \index{peach \textit{(package)}!peach.nn \textit{(package)}!peach.nn.nnet \textit{(module)}!peach.nn.nnet.FeedForward \textit{(class)}|(}
\subsection{Class FeedForward}

    \label{peach:nn:nnet:FeedForward}
\begin{tabular}{cccccccc}
% Line for object, linespec=[False, False]
\multicolumn{2}{r}{\settowidth{\BCL}{object}\multirow{2}{\BCL}{object}}
&&
&&
  \\\cline{3-3}
  &&\multicolumn{1}{c|}{}
&&
&&
  \\
% Line for list, linespec=[False]
\multicolumn{4}{r}{\settowidth{\BCL}{list}\multirow{2}{\BCL}{list}}
&&
  \\\cline{5-5}
  &&&&\multicolumn{1}{c|}{}
&&
  \\
&&&&\multicolumn{2}{l}{\textbf{peach.nn.nnet.FeedForward}}
\end{tabular}


Classic completely connected neural network.

A feedforward neural network is implemented as a list of layers, each layer
being a \texttt{Layer} object (please consult the documentation on the \texttt{base}
module for more information on layers). The layers are completely connected,
which means that every neuron in one layers is connected to every other
neuron in the following layer.

There is a number of learning methods that are already implemented, but in
general, any learning class derived from \texttt{FFLearning} can be used. No
other kind of learning can be used. Please, consult the documentation on the
\texttt{lrules} (\emph{learning rules}) module.

%%%%%%%%%%%%%%%%%%%%%%%%%%%%%%%%%%%%%%%%%%%%%%%%%%%%%%%%%%%%%%%%%%%%%%%%%%%
%%                                Methods                                %%
%%%%%%%%%%%%%%%%%%%%%%%%%%%%%%%%%%%%%%%%%%%%%%%%%%%%%%%%%%%%%%%%%%%%%%%%%%%

  \subsubsection{Methods}

    \vspace{0.5ex}

\hspace{.8\funcindent}\begin{boxedminipage}{\funcwidth}

    \raggedright \textbf{\_\_init\_\_}(\textit{self}, \textit{layers}, \textit{phi}={\tt {\textless}class 'peach.nn.af.Linear'{\textgreater}}, \textit{lrule}={\tt {\textless}class 'peach.nn.lrules.BackPropagation'{\textgreater}}, \textit{bias}={\tt False})

    \vspace{-1.5ex}

    \rule{\textwidth}{0.5\fboxrule}
\setlength{\parskip}{2ex}

Initializes a feedforward neural network.

A feedforward network is implemented as a list of layers, completely
connected.
\setlength{\parskip}{1ex}
      \textbf{Parameters}
      \vspace{-1ex}

      \begin{quote}
        \begin{Ventry}{xxxxxx}

          \item[layers]


A list of integers containing the shape of the network. The first
element of the list is the number of inputs of the network (or, as
somebody prefer, the number of input neurons); the number of outputs
is the number of neurons in the last layer. Thus, at least two
numbers should be given.
          \item[phi]


The activation functions to be used with each layer of the network.
Please consult the \texttt{Layer} documentation in the \texttt{base} module
for more information. This parameter can be a single function or a
list of functions. If only one function is given, then the same
function is used in every layer. If a list of functions is given,
then the layers use the functions in the sequence given. Note that
heterogeneous networks can be created that way. Defaults to
\texttt{Linear}.
          \item[lrule]


The learning rule used. Only \texttt{FFLearning} objects (instances of
the class or of the subclasses) are allowed. Defaults to
\texttt{BackPropagation}. Check the \texttt{lrules} documentation for more
information.
          \item[bias]


If \texttt{True}, then the neurons are biased.
        \end{Ventry}

      \end{quote}

      \textbf{Return Value}
    \vspace{-1ex}

      \begin{quote}

new empty list
      \end{quote}

      Overrides: object.\_\_init\_\_

    \end{boxedminipage}

    \label{peach:nn:nnet:FeedForward:__call__}
    \index{peach \textit{(package)}!peach.nn \textit{(package)}!peach.nn.nnet \textit{(module)}!peach.nn.nnet.FeedForward \textit{(class)}!peach.nn.nnet.FeedForward.\_\_call\_\_ \textit{(method)}}

    \vspace{0.5ex}

\hspace{.8\funcindent}\begin{boxedminipage}{\funcwidth}

    \raggedright \textbf{\_\_call\_\_}(\textit{self}, \textit{x})

    \vspace{-1.5ex}

    \rule{\textwidth}{0.5\fboxrule}
\setlength{\parskip}{2ex}

The feedforward method of the network.

The \texttt{\_\_call\_\_} interface should be called if the answer of the neuron
network to a given input vector \texttt{x} is desired. \emph{This method has
collateral effects}, so beware. After the calling of this method, the
\texttt{y} property is set with the activation potential and the answer of
the neurons, respectivelly.
\setlength{\parskip}{1ex}
      \textbf{Parameters}
      \vspace{-1ex}

      \begin{quote}
        \begin{Ventry}{x}

          \item[x]


The input vector to the network.
        \end{Ventry}

      \end{quote}

      \textbf{Return Value}
    \vspace{-1ex}

      \begin{quote}

The vector containing the answer of every neuron in the last layer, in
the respective order.
      \end{quote}

    \end{boxedminipage}

    \label{peach:nn:nnet:FeedForward:learn}
    \index{peach \textit{(package)}!peach.nn \textit{(package)}!peach.nn.nnet \textit{(module)}!peach.nn.nnet.FeedForward \textit{(class)}!peach.nn.nnet.FeedForward.learn \textit{(method)}}

    \vspace{0.5ex}

\hspace{.8\funcindent}\begin{boxedminipage}{\funcwidth}

    \raggedright \textbf{learn}(\textit{self}, \textit{x}, \textit{d})

    \vspace{-1.5ex}

    \rule{\textwidth}{0.5\fboxrule}
\setlength{\parskip}{2ex}

Applies one example of the training set to the network.

Using this method, one iteration of the learning procedure is made with
the neurons of this network. This method presents one example (not
necessarilly of a training set) and applies the learning rule over the
network. The learning rule is defined in the initialization of the
network, and some are implemented on the \texttt{lrules} method. New methods
can be created, consult the \texttt{lrules} documentation but, for
\texttt{FeedForward} instances, only \texttt{FFLearning} learning is allowed.

Also, notice that \emph{this method only applies the learning method!} The
network should be fed with the same input vector before trying to learn
anything first. Consult the \texttt{feed} and \texttt{train} methods below for
more ways to train a network.
\setlength{\parskip}{1ex}
      \textbf{Parameters}
      \vspace{-1ex}

      \begin{quote}
        \begin{Ventry}{x}

          \item[x]


Input vector of the example. It should be a column vector of the
correct dimension, that is, the number of input neurons.
          \item[d]


The desired answer of the network for this particular input vector.
Notice that the desired answer should have the same dimension of the
last layer of the network. This means that a desired answer should
be given for every output of the network.
        \end{Ventry}

      \end{quote}

      \textbf{Return Value}
    \vspace{-1ex}

      \begin{quote}

The error obtained by the network.
      \end{quote}

    \end{boxedminipage}

    \label{peach:nn:nnet:FeedForward:feed}
    \index{peach \textit{(package)}!peach.nn \textit{(package)}!peach.nn.nnet \textit{(module)}!peach.nn.nnet.FeedForward \textit{(class)}!peach.nn.nnet.FeedForward.feed \textit{(method)}}

    \vspace{0.5ex}

\hspace{.8\funcindent}\begin{boxedminipage}{\funcwidth}

    \raggedright \textbf{feed}(\textit{self}, \textit{x}, \textit{d})

    \vspace{-1.5ex}

    \rule{\textwidth}{0.5\fboxrule}
\setlength{\parskip}{2ex}

Feed the network and applies one example of the training set to the
network.

Using this method, one iteration of the learning procedure is made with
the neurons of this network. This method presents one example (not
necessarilly of a training set) and applies the learning rule over the
network. The learning rule is defined in the initialization of the
network, and some are implemented on the \texttt{lrules} method. New methods
can be created, consult the \texttt{lrules} documentation but, for
\texttt{FeedForward} instances, only \texttt{FFLearning} learning is allowed.

Also, notice that \emph{this method feeds the network} before applying the
learning rule. Feeding the network has collateral effects, and some
properties change when this happens. Namely, the \texttt{y} property is set.
Please consult the \texttt{\_\_call\_\_} interface.
\setlength{\parskip}{1ex}
      \textbf{Parameters}
      \vspace{-1ex}

      \begin{quote}
        \begin{Ventry}{x}

          \item[x]


Input vector of the example. It should be a column vector of the
correct dimension, that is, the number of input neurons.
          \item[d]


The desired answer of the network for this particular input vector.
Notice that the desired answer should have the same dimension of the
last layer of the network. This means that a desired answer should
be given for every output of the network.
        \end{Ventry}

      \end{quote}

      \textbf{Return Value}
    \vspace{-1ex}

      \begin{quote}

The error obtained by the network.
      \end{quote}

    \end{boxedminipage}

    \label{peach:nn:nnet:FeedForward:train}
    \index{peach \textit{(package)}!peach.nn \textit{(package)}!peach.nn.nnet \textit{(module)}!peach.nn.nnet.FeedForward \textit{(class)}!peach.nn.nnet.FeedForward.train \textit{(method)}}

    \vspace{0.5ex}

\hspace{.8\funcindent}\begin{boxedminipage}{\funcwidth}

    \raggedright \textbf{train}(\textit{self}, \textit{train\_set}, \textit{imax}={\tt 2000}, \textit{emax}={\tt 1e-05}, \textit{randomize}={\tt False})

    \vspace{-1.5ex}

    \rule{\textwidth}{0.5\fboxrule}
\setlength{\parskip}{2ex}

Presents a training set to the network.

This method automatizes the training of the network. Given a training
set, the examples are shown to the network (possibly in a randomized
way). A maximum number of iterations or a maximum admitted error should
be given as a stop condition.
\setlength{\parskip}{1ex}
      \textbf{Parameters}
      \vspace{-1ex}

      \begin{quote}
        \begin{Ventry}{xxxxxxxxx}

          \item[train\_set]


The training set is a list of examples. It can have any size and can
contain repeated examples. In fact, the definition of the training
set is open. Each element of the training set, however, should be a
two-tuple \texttt{(x, d)}, where \texttt{x} is the input vector, and \texttt{d} is
the desired response of the network for this particular input. See
the \texttt{learn} and \texttt{feed} for more information.
          \item[imax]


The maximum number of iterations. Examples from the training set
will be presented to the network while this limit is not reached.
Defaults to 2000.
          \item[emax]


The maximum admitted error. Examples from the training set will be
presented to the network until the error obtained is lower than this
limit. Defaults to 1e-5.
          \item[randomize]


If this is \texttt{True}, then the examples are shown in a randomized
order. If \texttt{False}, then the examples are shown in the same order
that they appear in the \texttt{train\_set} list. Defaults to \texttt{False}.
        \end{Ventry}

      \end{quote}

    \end{boxedminipage}


\large{\textbf{\textit{Inherited from list}}}

\begin{quote}
\_\_add\_\_(), \_\_contains\_\_(), \_\_delitem\_\_(), \_\_delslice\_\_(), \_\_eq\_\_(), \_\_ge\_\_(), \_\_getattribute\_\_(), \_\_getitem\_\_(), \_\_getslice\_\_(), \_\_gt\_\_(), \_\_iadd\_\_(), \_\_imul\_\_(), \_\_iter\_\_(), \_\_le\_\_(), \_\_len\_\_(), \_\_lt\_\_(), \_\_mul\_\_(), \_\_ne\_\_(), \_\_new\_\_(), \_\_repr\_\_(), \_\_reversed\_\_(), \_\_rmul\_\_(), \_\_setitem\_\_(), \_\_setslice\_\_(), \_\_sizeof\_\_(), append(), count(), extend(), index(), insert(), pop(), remove(), reverse(), sort()
\end{quote}

\large{\textbf{\textit{Inherited from object}}}

\begin{quote}
\_\_delattr\_\_(), \_\_format\_\_(), \_\_reduce\_\_(), \_\_reduce\_ex\_\_(), \_\_setattr\_\_(), \_\_str\_\_(), \_\_subclasshook\_\_()
\end{quote}

%%%%%%%%%%%%%%%%%%%%%%%%%%%%%%%%%%%%%%%%%%%%%%%%%%%%%%%%%%%%%%%%%%%%%%%%%%%
%%                              Properties                               %%
%%%%%%%%%%%%%%%%%%%%%%%%%%%%%%%%%%%%%%%%%%%%%%%%%%%%%%%%%%%%%%%%%%%%%%%%%%%

  \subsubsection{Properties}

    \vspace{-1cm}
\hspace{\varindent}\begin{longtable}{|p{\varnamewidth}|p{\vardescrwidth}|l}
\cline{1-2}
\cline{1-2} \centering \textbf{Name} & \centering \textbf{Description}& \\
\cline{1-2}
\endhead\cline{1-2}\multicolumn{3}{r}{\small\textit{continued on next page}}\\\endfoot\cline{1-2}
\endlastfoot\raggedright n\-l\-a\-y\-e\-r\-s\- & &\\
\cline{1-2}
\raggedright b\-i\-a\-s\- & &\\
\cline{1-2}
\raggedright y\- & &\\
\cline{1-2}
\raggedright p\-h\-i\- & &\\
\cline{1-2}
\multicolumn{2}{|l|}{\textit{Inherited from object}}\\
\multicolumn{2}{|p{\varwidth}|}{\raggedright \_\_class\_\_}\\
\cline{1-2}
\end{longtable}


%%%%%%%%%%%%%%%%%%%%%%%%%%%%%%%%%%%%%%%%%%%%%%%%%%%%%%%%%%%%%%%%%%%%%%%%%%%
%%                            Class Variables                            %%
%%%%%%%%%%%%%%%%%%%%%%%%%%%%%%%%%%%%%%%%%%%%%%%%%%%%%%%%%%%%%%%%%%%%%%%%%%%

  \subsubsection{Class Variables}

    \vspace{-1cm}
\hspace{\varindent}\begin{longtable}{|p{\varnamewidth}|p{\vardescrwidth}|l}
\cline{1-2}
\cline{1-2} \centering \textbf{Name} & \centering \textbf{Description}& \\
\cline{1-2}
\endhead\cline{1-2}\multicolumn{3}{r}{\small\textit{continued on next page}}\\\endfoot\cline{1-2}
\endlastfoot\multicolumn{2}{|l|}{\textit{Inherited from list}}\\
\multicolumn{2}{|p{\varwidth}|}{\raggedright \_\_hash\_\_}\\
\cline{1-2}
\end{longtable}

    \index{peach \textit{(package)}!peach.nn \textit{(package)}!peach.nn.nnet \textit{(module)}!peach.nn.nnet.FeedForward \textit{(class)}|)}

%%%%%%%%%%%%%%%%%%%%%%%%%%%%%%%%%%%%%%%%%%%%%%%%%%%%%%%%%%%%%%%%%%%%%%%%%%%
%%                           Class Description                           %%
%%%%%%%%%%%%%%%%%%%%%%%%%%%%%%%%%%%%%%%%%%%%%%%%%%%%%%%%%%%%%%%%%%%%%%%%%%%

    \index{peach \textit{(package)}!peach.nn \textit{(package)}!peach.nn.nnet \textit{(module)}!peach.nn.nnet.SOM \textit{(class)}|(}
\subsection{Class SOM}

    \label{peach:nn:nnet:SOM}
\begin{tabular}{cccccccc}
% Line for object, linespec=[False, False]
\multicolumn{2}{r}{\settowidth{\BCL}{object}\multirow{2}{\BCL}{object}}
&&
&&
  \\\cline{3-3}
  &&\multicolumn{1}{c|}{}
&&
&&
  \\
% Line for peach.nn.base.Layer, linespec=[False]
\multicolumn{4}{r}{\settowidth{\BCL}{peach.nn.base.Layer}\multirow{2}{\BCL}{peach.nn.base.Layer}}
&&
  \\\cline{5-5}
  &&&&\multicolumn{1}{c|}{}
&&
  \\
&&&&\multicolumn{2}{l}{\textbf{peach.nn.nnet.SOM}}
\end{tabular}


A Self-Organizing Map (SOM).

A self-organizing map is a type of neural network that is trained via
unsupervised learning. In particular, the self-organizing map finds the
neuron closest to an input vector -{}- this neuron is the winning neuron, and
it is the answer of the network. Thus, the SOM is usually used for
classification and pattern recognition.

The SOM is a single-layer network, so this class subclasses the \texttt{Layer}
class. But some of the properties of a \texttt{Layer} object are not available or
make no sense in this context.

%%%%%%%%%%%%%%%%%%%%%%%%%%%%%%%%%%%%%%%%%%%%%%%%%%%%%%%%%%%%%%%%%%%%%%%%%%%
%%                                Methods                                %%
%%%%%%%%%%%%%%%%%%%%%%%%%%%%%%%%%%%%%%%%%%%%%%%%%%%%%%%%%%%%%%%%%%%%%%%%%%%

  \subsubsection{Methods}

    \vspace{0.5ex}

\hspace{.8\funcindent}\begin{boxedminipage}{\funcwidth}

    \raggedright \textbf{\_\_init\_\_}(\textit{self}, \textit{shape}, \textit{lrule}={\tt {\textless}class 'peach.nn.lrules.Competitive'{\textgreater}})

    \vspace{-1.5ex}

    \rule{\textwidth}{0.5\fboxrule}
\setlength{\parskip}{2ex}

Initializes a self-organizing map.

A self-organizing map is implemented as a layer of neurons. There is no
connection among the neurons. The answer to a given input is the neuron
closer to the given input. \texttt{phi} (the activation function) \texttt{v} (the
activation potential) and \texttt{bias} are not used.
\setlength{\parskip}{1ex}
      \textbf{Parameters}
      \vspace{-1ex}

      \begin{quote}
        \begin{Ventry}{xxxxx}

          \item[shape]


Stablishes the size of the SOM. It must be a two-tuple of the
format \texttt{(m, n)}, where \texttt{m} is the number of neurons in the
layer, and \texttt{n} is the number of inputs of each neuron. The neurons
in the layer all have the same number of inputs.
          \item[lrule]


The learning rule used. Only \texttt{SOMLearning} objects (instances of
the class or of the subclasses) are allowed. Defaults to
\texttt{Competitive}. Check the \texttt{lrules} documentation for more
information.
        \end{Ventry}

      \end{quote}

      Overrides: object.\_\_init\_\_

    \end{boxedminipage}

    \vspace{0.5ex}

\hspace{.8\funcindent}\begin{boxedminipage}{\funcwidth}

    \raggedright \textbf{\_\_call\_\_}(\textit{self}, \textit{x})

    \vspace{-1.5ex}

    \rule{\textwidth}{0.5\fboxrule}
\setlength{\parskip}{2ex}

The response of the network to a given input.

The \texttt{\_\_call\_\_} interface should be called if the answer of the neuron
network to a given input vector \texttt{x} is desired. \emph{This method has
collateral effects}, so beware. After the calling of this method, the
\texttt{y} property is set with the activation potential and the answer of
the neurons, respectivelly.
\setlength{\parskip}{1ex}
      \textbf{Parameters}
      \vspace{-1ex}

      \begin{quote}
        \begin{Ventry}{x}

          \item[x]


The input vector to the network.
        \end{Ventry}

      \end{quote}

      \textbf{Return Value}
    \vspace{-1ex}

      \begin{quote}

The winning neuron.
      \end{quote}

      Overrides: peach.nn.base.Layer.\_\_call\_\_

    \end{boxedminipage}

    \label{peach:nn:nnet:SOM:learn}
    \index{peach \textit{(package)}!peach.nn \textit{(package)}!peach.nn.nnet \textit{(module)}!peach.nn.nnet.SOM \textit{(class)}!peach.nn.nnet.SOM.learn \textit{(method)}}

    \vspace{0.5ex}

\hspace{.8\funcindent}\begin{boxedminipage}{\funcwidth}

    \raggedright \textbf{learn}(\textit{self}, \textit{x})

    \vspace{-1.5ex}

    \rule{\textwidth}{0.5\fboxrule}
\setlength{\parskip}{2ex}

Applies one example of the training set to the network.

Using this method, one iteration of the learning procedure is made with
the neurons of this network. This method presents one example (not
necessarilly of a training set) and applies the learning rule over the
network. The learning rule is defined in the initialization of the
network, and some are implemented on the \texttt{lrules} method. New methods
can be created, consult the \texttt{lrules} documentation but, for
\texttt{SOM} instances, only \texttt{SOMLearning} learning is allowed.

Also, notice that \emph{this method only applies the learning method!} The
network should be fed with the same input vector before trying to learn
anything first. Consult the \texttt{feed} and \texttt{train} methods below for
more ways to train a network.
\setlength{\parskip}{1ex}
      \textbf{Parameters}
      \vspace{-1ex}

      \begin{quote}
        \begin{Ventry}{x}

          \item[x]


Input vector of the example. It should be a column vector of the
correct dimension, that is, the number of input neurons.
        \end{Ventry}

      \end{quote}

      \textbf{Return Value}
    \vspace{-1ex}

      \begin{quote}

The error obtained by the network.
      \end{quote}

    \end{boxedminipage}

    \label{peach:nn:nnet:SOM:feed}
    \index{peach \textit{(package)}!peach.nn \textit{(package)}!peach.nn.nnet \textit{(module)}!peach.nn.nnet.SOM \textit{(class)}!peach.nn.nnet.SOM.feed \textit{(method)}}

    \vspace{0.5ex}

\hspace{.8\funcindent}\begin{boxedminipage}{\funcwidth}

    \raggedright \textbf{feed}(\textit{self}, \textit{x})

    \vspace{-1.5ex}

    \rule{\textwidth}{0.5\fboxrule}
\setlength{\parskip}{2ex}

Feed the network and applies one example of the training set to the
network.

Using this method, one iteration of the learning procedure is made with
the neurons of this network. This method presents one example (not
necessarilly of a training set) and applies the learning rule over the
network. The learning rule is defined in the initialization of the
network, and some are implemented on the \texttt{lrules} method. New methods
can be created, consult the \texttt{lrules} documentation but, for
\texttt{SOM} instances, only \texttt{SOMLearning} learning is allowed.

Also, notice that \emph{this method feeds the network} before applying the
learning rule. Feeding the network has collateral effects, and some
properties change when this happens. Namely, the \texttt{y} property is set.
Please consult the \texttt{\_\_call\_\_} interface.
\setlength{\parskip}{1ex}
      \textbf{Parameters}
      \vspace{-1ex}

      \begin{quote}
        \begin{Ventry}{x}

          \item[x]


Input vector of the example. It should be a column vector of the
correct dimension, that is, the number of input neurons.
        \end{Ventry}

      \end{quote}

      \textbf{Return Value}
    \vspace{-1ex}

      \begin{quote}

The error obtained by the network.
      \end{quote}

    \end{boxedminipage}

    \label{peach:nn:nnet:SOM:train}
    \index{peach \textit{(package)}!peach.nn \textit{(package)}!peach.nn.nnet \textit{(module)}!peach.nn.nnet.SOM \textit{(class)}!peach.nn.nnet.SOM.train \textit{(method)}}

    \vspace{0.5ex}

\hspace{.8\funcindent}\begin{boxedminipage}{\funcwidth}

    \raggedright \textbf{train}(\textit{self}, \textit{train\_set}, \textit{imax}={\tt 2000}, \textit{emax}={\tt 1e-05}, \textit{randomize}={\tt False})

    \vspace{-1.5ex}

    \rule{\textwidth}{0.5\fboxrule}
\setlength{\parskip}{2ex}

Presents a training set to the network.

This method automatizes the training of the network. Given a training
set, the examples are shown to the network (possibly in a randomized
way). A maximum number of iterations or a maximum admitted error should
be given as a stop condition.
\setlength{\parskip}{1ex}
      \textbf{Parameters}
      \vspace{-1ex}

      \begin{quote}
        \begin{Ventry}{xxxxxxxxx}

          \item[train\_set]


The training set is a list of examples. It can have any size and can
contain repeated examples. In fact, the definition of the training
set is open. Each element of the training set, however, should be a
input vector of the correct dimensions, See the \texttt{learn} and
\texttt{feed} for more information.
          \item[imax]


The maximum number of iterations. Examples from the training set
will be presented to the network while this limit is not reached.
Defaults to 2000.
          \item[emax]


The maximum admitted error. Examples from the training set will be
presented to the network until the error obtained is lower than this
limit. Defaults to 1e-5.
          \item[randomize]


If this is \texttt{True}, then the examples are shown in a randomized
order. If \texttt{False}, then the examples are shown in the same order
that they appear in the \texttt{train\_set} list. Defaults to \texttt{False}.
        \end{Ventry}

      \end{quote}

    \end{boxedminipage}


\large{\textbf{\textit{Inherited from peach.nn.base.Layer\textit{(Section \ref{peach:nn:base:Layer})}}}}

\begin{quote}
\_\_getitem\_\_(), \_\_setitem\_\_()
\end{quote}

\large{\textbf{\textit{Inherited from object}}}

\begin{quote}
\_\_delattr\_\_(), \_\_format\_\_(), \_\_getattribute\_\_(), \_\_hash\_\_(), \_\_new\_\_(), \_\_reduce\_\_(), \_\_reduce\_ex\_\_(), \_\_repr\_\_(), \_\_setattr\_\_(), \_\_sizeof\_\_(), \_\_str\_\_(), \_\_subclasshook\_\_()
\end{quote}

%%%%%%%%%%%%%%%%%%%%%%%%%%%%%%%%%%%%%%%%%%%%%%%%%%%%%%%%%%%%%%%%%%%%%%%%%%%
%%                              Properties                               %%
%%%%%%%%%%%%%%%%%%%%%%%%%%%%%%%%%%%%%%%%%%%%%%%%%%%%%%%%%%%%%%%%%%%%%%%%%%%

  \subsubsection{Properties}

    \vspace{-1cm}
\hspace{\varindent}\begin{longtable}{|p{\varnamewidth}|p{\vardescrwidth}|l}
\cline{1-2}
\cline{1-2} \centering \textbf{Name} & \centering \textbf{Description}& \\
\cline{1-2}
\endhead\cline{1-2}\multicolumn{3}{r}{\small\textit{continued on next page}}\\\endfoot\cline{1-2}
\endlastfoot\raggedright y\- & &\\
\cline{1-2}
\multicolumn{2}{|l|}{\textit{Inherited from peach.nn.base.Layer \textit{(Section \ref{peach:nn:base:Layer})}}}\\
\multicolumn{2}{|p{\varwidth}|}{\raggedright bias, inputs, phi, shape, size, v, weights}\\
\cline{1-2}
\multicolumn{2}{|l|}{\textit{Inherited from object}}\\
\multicolumn{2}{|p{\varwidth}|}{\raggedright \_\_class\_\_}\\
\cline{1-2}
\end{longtable}

    \index{peach \textit{(package)}!peach.nn \textit{(package)}!peach.nn.nnet \textit{(module)}!peach.nn.nnet.SOM \textit{(class)}|)}
    \index{peach \textit{(package)}!peach.nn \textit{(package)}!peach.nn.nnet \textit{(module)}|)}

%
% API Documentation for Peach - Computational Intelligence for Python
% Module peach.nn.rbfn
%
% Generated by epydoc 3.0.1
% [Sun Jul 31 17:00:41 2011]
%

%%%%%%%%%%%%%%%%%%%%%%%%%%%%%%%%%%%%%%%%%%%%%%%%%%%%%%%%%%%%%%%%%%%%%%%%%%%
%%                          Module Description                           %%
%%%%%%%%%%%%%%%%%%%%%%%%%%%%%%%%%%%%%%%%%%%%%%%%%%%%%%%%%%%%%%%%%%%%%%%%%%%

    \index{peach \textit{(package)}!peach.nn \textit{(package)}!peach.nn.rbfn \textit{(module)}|(}
\section{Module peach.nn.rbfn}

    \label{peach:nn:rbfn}

Radial Basis Function Networks

This sub-package implements the basic behaviour of radial basis function
networks. This is a two-layer neural network that works as a universal function
approximator. The activation functions of the first layer are radial basis
functions (RBFs), that are symmetric around the origin, that is, the value of
this kind of function depends only on the distance of the evaluated point to the
origin. The second layer has only one neuron with linear activation, that is, it
only combines the inputs of the first layer.

The training of this kind of network, while it can be done using a traditional
optimization technique such as gradient descent, is usually made in two steps.
In the first step, the position of the centers and the width of the RBFs are
computed. In the second step, the weights of the second layer are adapted. In
this module, the RBFN architecture is implemented, allowing training of the
second layer. Centers must be supplied, but they can be easily found from the
training set using algorithms such as K-Means (the one traditionally used),
SOMs or Fuzzy C-Means.

%%%%%%%%%%%%%%%%%%%%%%%%%%%%%%%%%%%%%%%%%%%%%%%%%%%%%%%%%%%%%%%%%%%%%%%%%%%
%%                               Functions                               %%
%%%%%%%%%%%%%%%%%%%%%%%%%%%%%%%%%%%%%%%%%%%%%%%%%%%%%%%%%%%%%%%%%%%%%%%%%%%

  \subsection{Functions}

    \label{peach:nn:rbfn:randn}
    \index{peach \textit{(package)}!peach.nn \textit{(package)}!peach.nn.rbfn \textit{(module)}!peach.nn.rbfn.randn \textit{(function)}}

    \vspace{0.5ex}

\hspace{.8\funcindent}\begin{boxedminipage}{\funcwidth}

    \raggedright \textbf{randn}(\textit{d0}, \textit{d1}, \textit{dn}, \textit{...})

    \vspace{-1.5ex}

    \rule{\textwidth}{0.5\fboxrule}
\setlength{\parskip}{2ex}

Returns zero-mean, unit-variance Gaussian random numbers in an
array of shape (d0, d1, ..., dn).
%
\begin{description}
\item[{Note:  This is a convenience function. If you want an}] \leavevmode 
interface that takes a tuple as the first argument
use numpy.random.standard\_normal(shape\_tuple).

\end{description}
\setlength{\parskip}{1ex}
    \end{boxedminipage}


%%%%%%%%%%%%%%%%%%%%%%%%%%%%%%%%%%%%%%%%%%%%%%%%%%%%%%%%%%%%%%%%%%%%%%%%%%%
%%                               Variables                               %%
%%%%%%%%%%%%%%%%%%%%%%%%%%%%%%%%%%%%%%%%%%%%%%%%%%%%%%%%%%%%%%%%%%%%%%%%%%%

  \subsection{Variables}

    \vspace{-1cm}
\hspace{\varindent}\begin{longtable}{|p{\varnamewidth}|p{\vardescrwidth}|l}
\cline{1-2}
\cline{1-2} \centering \textbf{Name} & \centering \textbf{Description}& \\
\cline{1-2}
\endhead\cline{1-2}\multicolumn{3}{r}{\small\textit{continued on next page}}\\\endfoot\cline{1-2}
\endlastfoot\raggedright \_\-\_\-d\-o\-c\-\_\-\_\- & \raggedright \textbf{Value:} 
{\tt \texttt{...}}&\\
\cline{1-2}
\raggedright \_\-\_\-p\-a\-c\-k\-a\-g\-e\-\_\-\_\- & \raggedright \textbf{Value:} 
{\tt \texttt{'}\texttt{peach.nn}\texttt{'}}&\\
\cline{1-2}
\raggedright a\-b\-s\- & \raggedright \textbf{Value:} 
{\tt {\textless}ufunc 'absolute'{\textgreater}}&\\
\cline{1-2}
\raggedright a\-r\-c\-t\-a\-n\- & \raggedright \textbf{Value:} 
{\tt {\textless}ufunc 'arctan'{\textgreater}}&\\
\cline{1-2}
\raggedright c\-o\-s\-h\- & \raggedright \textbf{Value:} 
{\tt {\textless}ufunc 'cosh'{\textgreater}}&\\
\cline{1-2}
\raggedright e\-x\-p\- & \raggedright \textbf{Value:} 
{\tt {\textless}ufunc 'exp'{\textgreater}}&\\
\cline{1-2}
\raggedright p\-i\- & \raggedright \textbf{Value:} 
{\tt 3.14159265359}&\\
\cline{1-2}
\raggedright s\-i\-g\-n\- & \raggedright \textbf{Value:} 
{\tt {\textless}ufunc 'sign'{\textgreater}}&\\
\cline{1-2}
\raggedright s\-q\-r\-t\- & \raggedright \textbf{Value:} 
{\tt {\textless}ufunc 'sqrt'{\textgreater}}&\\
\cline{1-2}
\raggedright t\-a\-n\-h\- & \raggedright \textbf{Value:} 
{\tt {\textless}ufunc 'tanh'{\textgreater}}&\\
\cline{1-2}
\end{longtable}


%%%%%%%%%%%%%%%%%%%%%%%%%%%%%%%%%%%%%%%%%%%%%%%%%%%%%%%%%%%%%%%%%%%%%%%%%%%
%%                           Class Description                           %%
%%%%%%%%%%%%%%%%%%%%%%%%%%%%%%%%%%%%%%%%%%%%%%%%%%%%%%%%%%%%%%%%%%%%%%%%%%%

    \index{peach \textit{(package)}!peach.nn \textit{(package)}!peach.nn.rbfn \textit{(module)}!peach.nn.rbfn.RBFN \textit{(class)}|(}
\subsection{Class RBFN}

    \label{peach:nn:rbfn:RBFN}
\begin{tabular}{cccccc}
% Line for object, linespec=[False]
\multicolumn{2}{r}{\settowidth{\BCL}{object}\multirow{2}{\BCL}{object}}
&&
  \\\cline{3-3}
  &&\multicolumn{1}{c|}{}
&&
  \\
&&\multicolumn{2}{l}{\textbf{peach.nn.rbfn.RBFN}}
\end{tabular}


%%%%%%%%%%%%%%%%%%%%%%%%%%%%%%%%%%%%%%%%%%%%%%%%%%%%%%%%%%%%%%%%%%%%%%%%%%%
%%                                Methods                                %%
%%%%%%%%%%%%%%%%%%%%%%%%%%%%%%%%%%%%%%%%%%%%%%%%%%%%%%%%%%%%%%%%%%%%%%%%%%%

  \subsubsection{Methods}

    \vspace{0.5ex}

\hspace{.8\funcindent}\begin{boxedminipage}{\funcwidth}

    \raggedright \textbf{\_\_init\_\_}(\textit{self}, \textit{c}, \textit{phi}={\tt {\textless}class 'peach.nn.af.Gaussian'{\textgreater}}, \textit{phi2}={\tt {\textless}class 'peach.nn.af.Linear'{\textgreater}})

    \vspace{-1.5ex}

    \rule{\textwidth}{0.5\fboxrule}
\setlength{\parskip}{2ex}

Initializes the radial basis function network.

A radial basis function is implemented as two layers of neurons, the
first one with the RBFs, the second one a linear combinator.
\setlength{\parskip}{1ex}
      \textbf{Parameters}
      \vspace{-1ex}

      \begin{quote}
        \begin{Ventry}{xxxx}

          \item[c]


Two-dimensional array containing the centers of the radial basis
functions, where each line is a vector with the components of the
center. Thus, the number of lines in this array is the number of
centers of the network.
          \item[phi]


The radial basis function to be used in the first layer. Defaults to
the gaussian.
          \item[phi2]


The activation function of the second layer. If the network is being
used to approximate functions, this should be Linear. Since this is
the most commom situation, it is the default value. In occasions,
this can be made (say) a sigmoid, for pattern recognition.
        \end{Ventry}

      \end{quote}

      Overrides: object.\_\_init\_\_

    \end{boxedminipage}

    \label{peach:nn:rbfn:RBFN:__call__}
    \index{peach \textit{(package)}!peach.nn \textit{(package)}!peach.nn.rbfn \textit{(module)}!peach.nn.rbfn.RBFN \textit{(class)}!peach.nn.rbfn.RBFN.\_\_call\_\_ \textit{(method)}}

    \vspace{0.5ex}

\hspace{.8\funcindent}\begin{boxedminipage}{\funcwidth}

    \raggedright \textbf{\_\_call\_\_}(\textit{self}, \textit{x})

    \vspace{-1.5ex}

    \rule{\textwidth}{0.5\fboxrule}
\setlength{\parskip}{2ex}

Feeds the network and return the result.

The \texttt{\_\_call\_\_} interface should be called if the answer of the neuron
network to a given input vector \texttt{x} is desired. \emph{This method has
collateral effects}, so beware. After the calling of this method, the
\texttt{y} property is set with the activation potential and the answer of
the neurons, respectivelly.
\setlength{\parskip}{1ex}
      \textbf{Parameters}
      \vspace{-1ex}

      \begin{quote}
        \begin{Ventry}{x}

          \item[x]


The input vector to the network.
        \end{Ventry}

      \end{quote}

      \textbf{Return Value}
    \vspace{-1ex}

      \begin{quote}

The vector containing the answer of every neuron in the last layer, in
the respective order.
      \end{quote}

    \end{boxedminipage}

    \label{peach:nn:rbfn:RBFN:learn}
    \index{peach \textit{(package)}!peach.nn \textit{(package)}!peach.nn.rbfn \textit{(module)}!peach.nn.rbfn.RBFN \textit{(class)}!peach.nn.rbfn.RBFN.learn \textit{(method)}}

    \vspace{0.5ex}

\hspace{.8\funcindent}\begin{boxedminipage}{\funcwidth}

    \raggedright \textbf{learn}(\textit{self}, \textit{x}, \textit{d})

    \vspace{-1.5ex}

    \rule{\textwidth}{0.5\fboxrule}
\setlength{\parskip}{2ex}

Applies one example of the training set to the network.

Using this method, one iteration of the learning procedure is executed
for the second layer of the network. This method presents one example
(not necessarilly from a training set) and applies the learning rule
over the layer. The learning rule is defined in the initialization of
the network, and some are implemented on the \texttt{lrules} method. New
methods can be created, consult the \texttt{lrules} documentation but, for
the second layer of a \texttt{RBFN'{}' instance, only `{}`FFLearning} learning is
allowed.

Also, notice that \emph{this method only applies the learning method!} The
network should be fed with the same input vector before trying to learn
anything first. Consult the \texttt{feed} and \texttt{train} methods below for
more ways to train a network.
\setlength{\parskip}{1ex}
      \textbf{Parameters}
      \vspace{-1ex}

      \begin{quote}
        \begin{Ventry}{x}

          \item[x]


Input vector of the example. It should be a column vector of the
correct dimension, that is, the number of input neurons.
          \item[d]


The desired answer of the network for this particular input vector.
Notice that the desired answer should have the same dimension of the
last layer of the network. This means that a desired answer should
be given for every output of the network.
        \end{Ventry}

      \end{quote}

      \textbf{Return Value}
    \vspace{-1ex}

      \begin{quote}

The error obtained by the network.
      \end{quote}

    \end{boxedminipage}

    \label{peach:nn:rbfn:RBFN:feed}
    \index{peach \textit{(package)}!peach.nn \textit{(package)}!peach.nn.rbfn \textit{(module)}!peach.nn.rbfn.RBFN \textit{(class)}!peach.nn.rbfn.RBFN.feed \textit{(method)}}

    \vspace{0.5ex}

\hspace{.8\funcindent}\begin{boxedminipage}{\funcwidth}

    \raggedright \textbf{feed}(\textit{self}, \textit{x}, \textit{d})

    \vspace{-1.5ex}

    \rule{\textwidth}{0.5\fboxrule}
\setlength{\parskip}{2ex}

Feed the network and applies one example of the training set to the
network. This adapts only the synaptic weights in the second layer of
the RBFN.

Using this method, one iteration of the learning procedure is made with
the neurons of this network. This method presents one example (not
necessarilly from a training set) and applies the learning rule over the
network. The learning rule is defined in the initialization of the
network, and some are implemented on the \texttt{lrules} method. New methods
can be created, consult the \texttt{lrules} documentation but, for the second
layer of a \texttt{RBFN}, only \texttt{FFLearning} learning is allowed.

Also, notice that \emph{this method feeds the network} before applying the
learning rule. Feeding the network has collateral effects, and some
properties change when this happens. Namely, the \texttt{y} property is set.
Please consult the \texttt{\_\_call\_\_} interface.
\setlength{\parskip}{1ex}
      \textbf{Parameters}
      \vspace{-1ex}

      \begin{quote}
        \begin{Ventry}{x}

          \item[x]


Input vector of the example. It should be a column vector of the
correct dimension, that is, the number of input neurons.
          \item[d]


The desired answer of the network for this particular input vector.
Notice that the desired answer should have the same dimension of the
last layer of the network. This means that a desired answer should
be given for every output of the network.
        \end{Ventry}

      \end{quote}

      \textbf{Return Value}
    \vspace{-1ex}

      \begin{quote}

The error obtained by the network.
      \end{quote}

    \end{boxedminipage}

    \label{peach:nn:rbfn:RBFN:train}
    \index{peach \textit{(package)}!peach.nn \textit{(package)}!peach.nn.rbfn \textit{(module)}!peach.nn.rbfn.RBFN \textit{(class)}!peach.nn.rbfn.RBFN.train \textit{(method)}}

    \vspace{0.5ex}

\hspace{.8\funcindent}\begin{boxedminipage}{\funcwidth}

    \raggedright \textbf{train}(\textit{self}, \textit{train\_set}, \textit{imax}={\tt 2000}, \textit{emax}={\tt 1e-05}, \textit{randomize}={\tt False})

    \vspace{-1.5ex}

    \rule{\textwidth}{0.5\fboxrule}
\setlength{\parskip}{2ex}

Presents a training set to the network.

This method automatizes the training of the network. Given a training
set, the examples are shown to the network (possibly in a randomized
way). A maximum number of iterations or a maximum admitted error should
be given as a stop condition.
\setlength{\parskip}{1ex}
      \textbf{Parameters}
      \vspace{-1ex}

      \begin{quote}
        \begin{Ventry}{xxxxxxxxx}

          \item[train\_set]


The training set is a list of examples. It can have any size and can
contain repeated examples. In fact, the definition of the training
set is open. Each element of the training set, however, should be a
two-tuple \texttt{(x, d)}, where \texttt{x} is the input vector, and \texttt{d} is
the desired response of the network for this particular input. See
the \texttt{learn} and \texttt{feed} for more information.
          \item[imax]


The maximum number of iterations. Examples from the training set
will be presented to the network while this limit is not reached.
Defaults to 2000.
          \item[emax]


The maximum admitted error. Examples from the training set will be
presented to the network until the error obtained is lower than this
limit. Defaults to 1e-5.
          \item[randomize]


If this is \texttt{True}, then the examples are shown in a randomized
order. If \texttt{False}, then the examples are shown in the same order
that they appear in the \texttt{train\_set} list. Defaults to \texttt{False}.
        \end{Ventry}

      \end{quote}

    \end{boxedminipage}


\large{\textbf{\textit{Inherited from object}}}

\begin{quote}
\_\_delattr\_\_(), \_\_format\_\_(), \_\_getattribute\_\_(), \_\_hash\_\_(), \_\_new\_\_(), \_\_reduce\_\_(), \_\_reduce\_ex\_\_(), \_\_repr\_\_(), \_\_setattr\_\_(), \_\_sizeof\_\_(), \_\_str\_\_(), \_\_subclasshook\_\_()
\end{quote}

%%%%%%%%%%%%%%%%%%%%%%%%%%%%%%%%%%%%%%%%%%%%%%%%%%%%%%%%%%%%%%%%%%%%%%%%%%%
%%                              Properties                               %%
%%%%%%%%%%%%%%%%%%%%%%%%%%%%%%%%%%%%%%%%%%%%%%%%%%%%%%%%%%%%%%%%%%%%%%%%%%%

  \subsubsection{Properties}

    \vspace{-1cm}
\hspace{\varindent}\begin{longtable}{|p{\varnamewidth}|p{\vardescrwidth}|l}
\cline{1-2}
\cline{1-2} \centering \textbf{Name} & \centering \textbf{Description}& \\
\cline{1-2}
\endhead\cline{1-2}\multicolumn{3}{r}{\small\textit{continued on next page}}\\\endfoot\cline{1-2}
\endlastfoot\raggedright w\-i\-d\-t\-h\- & &\\
\cline{1-2}
\raggedright w\-e\-i\-g\-h\-t\-s\- & &\\
\cline{1-2}
\raggedright y\- & &\\
\cline{1-2}
\raggedright p\-h\-i\- & &\\
\cline{1-2}
\raggedright p\-h\-i\-2\- & &\\
\cline{1-2}
\multicolumn{2}{|l|}{\textit{Inherited from object}}\\
\multicolumn{2}{|p{\varwidth}|}{\raggedright \_\_class\_\_}\\
\cline{1-2}
\end{longtable}

    \index{peach \textit{(package)}!peach.nn \textit{(package)}!peach.nn.rbfn \textit{(module)}!peach.nn.rbfn.RBFN \textit{(class)}|)}
    \index{peach \textit{(package)}!peach.nn \textit{(package)}!peach.nn.rbfn \textit{(module)}|)}

%
% API Documentation for Peach - Computational Intelligence for Python
% Package peach.optm
%
% Generated by epydoc 3.0.1
% [Fri Feb  4 17:21:21 2011]
%

%%%%%%%%%%%%%%%%%%%%%%%%%%%%%%%%%%%%%%%%%%%%%%%%%%%%%%%%%%%%%%%%%%%%%%%%%%%
%%                          Module Description                           %%
%%%%%%%%%%%%%%%%%%%%%%%%%%%%%%%%%%%%%%%%%%%%%%%%%%%%%%%%%%%%%%%%%%%%%%%%%%%

    \index{peach \textit{(package)}!peach.optm \textit{(package)}|(}
\section{Package peach.optm}

    \label{peach:optm}

This package implements deterministic optimization methods. Consult:
%
\begin{quote}
%
\begin{description}
\item[{base}] \leavevmode 
Basic definitions and interface with the optimization methods;

\item[{linear}] \leavevmode 
Basic methods for one variable optimization;

\item[{multivar}] \leavevmode 
Gradient, Newton and othe multivariable optimization methods;

\item[{quasinewton}] \leavevmode 
Quasi-Newton methods;

\end{description}

\end{quote}

Every optimizer works in pretty much the same way. Instantiate the respective
class, using as parameter the cost function to be optimized, the first estimate
(a scalar in case of a single variable optimization, and a one-dimensional array
in case of multivariable optimization) and some other parameters. Use \texttt{step()}
to perform one iteration of the method, use the \texttt{\_\_call\_\_()} method to perform
the search until the stop conditions are met. See each method for details.

%%%%%%%%%%%%%%%%%%%%%%%%%%%%%%%%%%%%%%%%%%%%%%%%%%%%%%%%%%%%%%%%%%%%%%%%%%%
%%                                Modules                                %%
%%%%%%%%%%%%%%%%%%%%%%%%%%%%%%%%%%%%%%%%%%%%%%%%%%%%%%%%%%%%%%%%%%%%%%%%%%%

\subsection{Modules}

\begin{itemize}
\setlength{\parskip}{0ex}
\item \textbf{base}: 
Basic definitons and base class for optimizers


  \textit{(Section \ref{peach:optm:base}, p.~\pageref{peach:optm:base})}

\item \textbf{linear}: 
This package implements basic one variable only optimizers.


  \textit{(Section \ref{peach:optm:linear}, p.~\pageref{peach:optm:linear})}

\item \textbf{multivar}: 
This package implements basic multivariable optimizers, including gradient and
Newton searches.


  \textit{(Section \ref{peach:optm:multivar}, p.~\pageref{peach:optm:multivar})}

\item \textbf{quasinewton}: 
This package implements basic quasi-Newton optimizers. Newton optimizer is very
efficient, except that inverse matrices need to be calculated at each
convergence step. These methods try to estimate the hessian inverse iteratively,
thus increasing performance.


  \textit{(Section \ref{peach:optm:quasinewton}, p.~\pageref{peach:optm:quasinewton})}

\item \textbf{stochastic}
  \textit{(Section \ref{peach:optm:stochastic}, p.~\pageref{peach:optm:stochastic})}

\end{itemize}


%%%%%%%%%%%%%%%%%%%%%%%%%%%%%%%%%%%%%%%%%%%%%%%%%%%%%%%%%%%%%%%%%%%%%%%%%%%
%%                               Variables                               %%
%%%%%%%%%%%%%%%%%%%%%%%%%%%%%%%%%%%%%%%%%%%%%%%%%%%%%%%%%%%%%%%%%%%%%%%%%%%

  \subsection{Variables}

    \vspace{-1cm}
\hspace{\varindent}\begin{longtable}{|p{\varnamewidth}|p{\vardescrwidth}|l}
\cline{1-2}
\cline{1-2} \centering \textbf{Name} & \centering \textbf{Description}& \\
\cline{1-2}
\endhead\cline{1-2}\multicolumn{3}{r}{\small\textit{continued on next page}}\\\endfoot\cline{1-2}
\endlastfoot\raggedright \_\-\_\-d\-o\-c\-\_\-\_\- & \raggedright \textbf{Value:} 
{\tt \texttt{...}}&\\
\cline{1-2}
\raggedright \_\-\_\-p\-a\-c\-k\-a\-g\-e\-\_\-\_\- & \raggedright \textbf{Value:} 
{\tt \texttt{'}\texttt{peach.optm}\texttt{'}}&\\
\cline{1-2}
\raggedright a\-b\-s\- & \raggedright \textbf{Value:} 
{\tt {\textless}ufunc 'absolute'{\textgreater}}&\\
\cline{1-2}
\end{longtable}

    \index{peach \textit{(package)}!peach.optm \textit{(package)}|)}

%
% API Documentation for Peach - Computational Intelligence for Python
% Module peach.optm.base
%
% Generated by epydoc 3.0.1
% [Mon Jan 24 15:39:52 2011]
%

%%%%%%%%%%%%%%%%%%%%%%%%%%%%%%%%%%%%%%%%%%%%%%%%%%%%%%%%%%%%%%%%%%%%%%%%%%%
%%                          Module Description                           %%
%%%%%%%%%%%%%%%%%%%%%%%%%%%%%%%%%%%%%%%%%%%%%%%%%%%%%%%%%%%%%%%%%%%%%%%%%%%

    \index{peach \textit{(package)}!peach.optm \textit{(package)}!peach.optm.base \textit{(module)}|(}
\section{Module peach.optm.base}

    \label{peach:optm:base}

Basic definitons and base class for optimizers

This sub-package exports some auxiliary functions to work with cost functions,
namely, a function to calculate gradient vectors and hessian matrices, which are
extremely important in optimization.

Also, a base class, \texttt{Optimizer}, for all optimizers. Sub-class this class if
you want to create your own optmizer, and follow the interface. This will allow
easy configuration of your own scripts and comparison between methods.

%%%%%%%%%%%%%%%%%%%%%%%%%%%%%%%%%%%%%%%%%%%%%%%%%%%%%%%%%%%%%%%%%%%%%%%%%%%
%%                               Functions                               %%
%%%%%%%%%%%%%%%%%%%%%%%%%%%%%%%%%%%%%%%%%%%%%%%%%%%%%%%%%%%%%%%%%%%%%%%%%%%

  \subsection{Functions}

    \label{peach:optm:base:gradient}
    \index{peach \textit{(package)}!peach.optm \textit{(package)}!peach.optm.base \textit{(module)}!peach.optm.base.gradient \textit{(function)}}

    \vspace{0.5ex}

\hspace{.8\funcindent}\begin{boxedminipage}{\funcwidth}

    \raggedright \textbf{gradient}(\textit{f}, \textit{dx}={\tt 1e-05})

    \vspace{-1.5ex}

    \rule{\textwidth}{0.5\fboxrule}
\setlength{\parskip}{2ex}

Creates a function that computes the gradient vector of a scalar field.

This function takes as a parameter a scalar function and creates a new
function that is able to compute the derivative (in case of single variable
functions) or the gradient vector (in case of multivariable functions.
Please, note that this function takes as a parameter a \emph{function}, and
returns as a result \emph{another function}. Calling the returned function on a
point will give the gradient vector of the original function at that point:
%
\begin{quote}{\ttfamily \raggedright \noindent
>{}>{}>~def~f(x):\\
~~~~~~~~return~x\textasciicircum{}2\\
~\\
>{}>{}>~df~=~gradient(f)\\
>{}>{}>~df(1)\\
2
}
\end{quote}

In the above example, \texttt{df} is a generated function which will return the
result of the expression \texttt{2*x}, the derivative of the original function.
In the case \texttt{f} is a multivariable function, it is assumed that its
argument is a line vector.
\setlength{\parskip}{1ex}
      \textbf{Parameters}
      \vspace{-1ex}

      \begin{quote}
        \begin{Ventry}{xx}

          \item[f]


Any function, one- or multivariable. The function must be an scalar
function, though there is no checking at the moment the function is
created. If \texttt{f} is not an scalar function, an exception will be
raised at the moment the returned function is used.
          \item[dx]


Optional argument that gives the precision of the calculation. It is
recommended that \texttt{dx = sqrt(D)}, where \texttt{D} is the machine precision.
It defaults to \texttt{1e-5}, which usually gives a good estimate.
        \end{Ventry}

      \end{quote}

      \textbf{Return Value}
    \vspace{-1ex}

      \begin{quote}

A new function which, upon calling, gives the derivative or gradient
vector of the original function on the analised point. The parameter of
the returned function is a real number or a line vector where the gradient
should be calculated.
      \end{quote}

    \end{boxedminipage}

    \label{peach:optm:base:hessian}
    \index{peach \textit{(package)}!peach.optm \textit{(package)}!peach.optm.base \textit{(module)}!peach.optm.base.hessian \textit{(function)}}

    \vspace{0.5ex}

\hspace{.8\funcindent}\begin{boxedminipage}{\funcwidth}

    \raggedright \textbf{hessian}(\textit{f}, \textit{dx}={\tt 1e-05})

    \vspace{-1.5ex}

    \rule{\textwidth}{0.5\fboxrule}
\setlength{\parskip}{2ex}

Creates a function that computes the hessian matrix of a scalar field.

This function takes as a parameter a scalar function and creates a new
function that is able to calculate the second derivative (in case of single
variable functions) or the hessian matrix (in case of multivariable
functions. Please, note that this function takes as a parameter a
\emph{function}, and returns as a result \emph{another function}. Calling the returned
function on a point will give the hessian matrix of the original function
at that point:
%
\begin{quote}{\ttfamily \raggedright \noindent
>{}>{}>~def~f(x):\\
~~~~~~~~return~x\textasciicircum{}4\\
~\\
>{}>{}>~ddf~=~hessian(f)\\
>{}>{}>~ddf(1)\\
12
}
\end{quote}

In the above example, \texttt{ddf} is a generated function which will return the
result of the expression \texttt{12*x**2}, the second derivative of the original
function. In the case \texttt{f} is a multivariable function, it is assumed that
its argument is a line vector.
\setlength{\parskip}{1ex}
      \textbf{Parameters}
      \vspace{-1ex}

      \begin{quote}
        \begin{Ventry}{xx}

          \item[f]


Any function, one- or multivariable. The function must be an scalar
function, though there is no checking at the moment the function is
created. If \texttt{f} is not an scalar function, an exception will be
raised at the moment the returned function is used.
          \item[dx]


Optional argument that gives the precision of the calculation. It is
recommended that \texttt{dx = sqrt(D)}, where \texttt{D} is the machine precision.
It defaults to \texttt{1e-5}, which usually gives a good estimate.
        \end{Ventry}

      \end{quote}

      \textbf{Return Value}
    \vspace{-1ex}

      \begin{quote}

A new function which, upon calling, gives the second derivative or hessian
matrix of the original function on the analised point. The parameter of
the returned function is a real number or a line vector where the hessian
should be calculated.
      \end{quote}

    \end{boxedminipage}


%%%%%%%%%%%%%%%%%%%%%%%%%%%%%%%%%%%%%%%%%%%%%%%%%%%%%%%%%%%%%%%%%%%%%%%%%%%
%%                               Variables                               %%
%%%%%%%%%%%%%%%%%%%%%%%%%%%%%%%%%%%%%%%%%%%%%%%%%%%%%%%%%%%%%%%%%%%%%%%%%%%

  \subsection{Variables}

    \vspace{-1cm}
\hspace{\varindent}\begin{longtable}{|p{\varnamewidth}|p{\vardescrwidth}|l}
\cline{1-2}
\cline{1-2} \centering \textbf{Name} & \centering \textbf{Description}& \\
\cline{1-2}
\endhead\cline{1-2}\multicolumn{3}{r}{\small\textit{continued on next page}}\\\endfoot\cline{1-2}
\endlastfoot\raggedright \_\-\_\-d\-o\-c\-\_\-\_\- & \raggedright \textbf{Value:} 
{\tt \texttt{...}}&\\
\cline{1-2}
\raggedright \_\-\_\-p\-a\-c\-k\-a\-g\-e\-\_\-\_\- & \raggedright \textbf{Value:} 
{\tt \texttt{'}\texttt{peach.optm}\texttt{'}}&\\
\cline{1-2}
\end{longtable}


%%%%%%%%%%%%%%%%%%%%%%%%%%%%%%%%%%%%%%%%%%%%%%%%%%%%%%%%%%%%%%%%%%%%%%%%%%%
%%                           Class Description                           %%
%%%%%%%%%%%%%%%%%%%%%%%%%%%%%%%%%%%%%%%%%%%%%%%%%%%%%%%%%%%%%%%%%%%%%%%%%%%

    \index{peach \textit{(package)}!peach.optm \textit{(package)}!peach.optm.base \textit{(module)}!peach.optm.base.Optimizer \textit{(class)}|(}
\subsection{Class Optimizer}

    \label{peach:optm:base:Optimizer}
\begin{tabular}{cccccc}
% Line for object, linespec=[False]
\multicolumn{2}{r}{\settowidth{\BCL}{object}\multirow{2}{\BCL}{object}}
&&
  \\\cline{3-3}
  &&\multicolumn{1}{c|}{}
&&
  \\
&&\multicolumn{2}{l}{\textbf{peach.optm.base.Optimizer}}
\end{tabular}

\textbf{Known Subclasses:}
peach.optm.quasinewton.BFGS,
    peach.optm.quasinewton.DFP,
    peach.optm.quasinewton.SR1,
    peach.optm.linear.Direct1D,
    peach.optm.linear.Fibonacci,
    peach.optm.linear.GoldenRule,
    peach.optm.linear.Interpolation,
    peach.optm.multivar.Direct,
    peach.optm.multivar.Gradient,
    peach.optm.multivar.MomentumGradient,
    peach.optm.multivar.Newton


Base class for all optimizers.

This class does nothing, and shouldn't be instantiated. Its only purpose is
to serve as a template (or interface) to implemented optimizers. To create
your own optimizer, subclass this.

This class defines 3 methods that should be present in any subclass. They
are defined here:
%
\begin{quote}
%
\begin{description}
\item[{\_\_init\_\_}] \leavevmode 
Initializes the optimizer. There are three usual parameters in this
method, which signature should be:
%
\begin{quote}{\ttfamily \raggedright \noindent
\_\_init\_\_(self,~f,~x0,~...,~emax=1e-8,~imax=1000)
}
\end{quote}
%
\begin{description}
\item[{where:}] \leavevmode %
\begin{itemize}

\item \texttt{f} is the cost function to be minimized;

\item \texttt{x0} is the first estimate of the location of the minimum;

\item \texttt{...} represent additional configuration of the optimizer, and it
is dependent of the technique implemented;

\item \texttt{emax} is the maximum allowed error. The default value above is
only a suggestion;

\item \texttt{imax} is the maximum number of iterations of the method. The
default value above is only a suggestions.

\end{itemize}

\end{description}

\item[{step()}] \leavevmode 
This method should take an estimate and calculate the next, possibly
better, estimate. Notice that the next estimate is strongly dependent of
the method, the optimizer state and configuration, and two calls to this
method with the same estimate might not give the same results. The
method signature is:
%
\begin{quote}{\ttfamily \raggedright \noindent
step(self)
}
\end{quote}

and the implementation should keep track of all the needed parameters.
The method should return a tuple \texttt{(x, e)} with the new estimate of the
solution and the estimate of the error.

\item[{restart()}] \leavevmode 
Implement this method to restart the optimizer. An optimizer might be
restarted for a number of reasons: to escape a local minimum, to try
different estimates and so on. This method should take at least one
argument, \texttt{x0}, a new estimate for the optimizer. Optionally, new
configuration might be given, but, if not, the old ones must be used.

\item[{\_\_call\_\_}] \leavevmode 
This method should take an estimate and iterate the optimizer until one
of the stop criteria is met: either less than the maximum error or more
than the maximum number of iterations. Error is usually calculated as an
estimate using the previous estimate, but any technique might be used.
Use a counter to keep track of the number of iterations. The method
signature is:
%
\begin{quote}{\ttfamily \raggedright \noindent
\_\_call\_\_(self)
}
\end{quote}

and the implementation should keep track of all the needed parameters.
The method should return a tuple \texttt{(x, e)} with the final estimate of
the solution and the estimate of the error.

\end{description}

\end{quote}

%%%%%%%%%%%%%%%%%%%%%%%%%%%%%%%%%%%%%%%%%%%%%%%%%%%%%%%%%%%%%%%%%%%%%%%%%%%
%%                                Methods                                %%
%%%%%%%%%%%%%%%%%%%%%%%%%%%%%%%%%%%%%%%%%%%%%%%%%%%%%%%%%%%%%%%%%%%%%%%%%%%

  \subsubsection{Methods}

    \vspace{0.5ex}

\hspace{.8\funcindent}\begin{boxedminipage}{\funcwidth}

    \raggedright \textbf{\_\_init\_\_}(\textit{self}, \textit{f}={\tt None}, \textit{x0}={\tt None}, \textit{emax}={\tt 1e-08}, \textit{imax}={\tt 1000})

\setlength{\parskip}{2ex}

x.\_\_init\_\_(...) initializes x; see x.\_\_class\_\_.\_\_doc\_\_ for signature
\setlength{\parskip}{1ex}
      Overrides: object.\_\_init\_\_ 	extit{(inherited documentation)}

    \end{boxedminipage}

    \label{peach:optm:base:Optimizer:step}
    \index{peach \textit{(package)}!peach.optm \textit{(package)}!peach.optm.base \textit{(module)}!peach.optm.base.Optimizer \textit{(class)}!peach.optm.base.Optimizer.step \textit{(method)}}

    \vspace{0.5ex}

\hspace{.8\funcindent}\begin{boxedminipage}{\funcwidth}

    \raggedright \textbf{step}(\textit{self}, \textit{x})

\setlength{\parskip}{2ex}
\setlength{\parskip}{1ex}
    \end{boxedminipage}

    \label{peach:optm:base:Optimizer:__call__}
    \index{peach \textit{(package)}!peach.optm \textit{(package)}!peach.optm.base \textit{(module)}!peach.optm.base.Optimizer \textit{(class)}!peach.optm.base.Optimizer.\_\_call\_\_ \textit{(method)}}

    \vspace{0.5ex}

\hspace{.8\funcindent}\begin{boxedminipage}{\funcwidth}

    \raggedright \textbf{\_\_call\_\_}(\textit{self}, \textit{x})

\setlength{\parskip}{2ex}
\setlength{\parskip}{1ex}
    \end{boxedminipage}


\large{\textbf{\textit{Inherited from object}}}

\begin{quote}
\_\_delattr\_\_(), \_\_format\_\_(), \_\_getattribute\_\_(), \_\_hash\_\_(), \_\_new\_\_(), \_\_reduce\_\_(), \_\_reduce\_ex\_\_(), \_\_repr\_\_(), \_\_setattr\_\_(), \_\_sizeof\_\_(), \_\_str\_\_(), \_\_subclasshook\_\_()
\end{quote}

%%%%%%%%%%%%%%%%%%%%%%%%%%%%%%%%%%%%%%%%%%%%%%%%%%%%%%%%%%%%%%%%%%%%%%%%%%%
%%                              Properties                               %%
%%%%%%%%%%%%%%%%%%%%%%%%%%%%%%%%%%%%%%%%%%%%%%%%%%%%%%%%%%%%%%%%%%%%%%%%%%%

  \subsubsection{Properties}

    \vspace{-1cm}
\hspace{\varindent}\begin{longtable}{|p{\varnamewidth}|p{\vardescrwidth}|l}
\cline{1-2}
\cline{1-2} \centering \textbf{Name} & \centering \textbf{Description}& \\
\cline{1-2}
\endhead\cline{1-2}\multicolumn{3}{r}{\small\textit{continued on next page}}\\\endfoot\cline{1-2}
\endlastfoot\multicolumn{2}{|l|}{\textit{Inherited from object}}\\
\multicolumn{2}{|p{\varwidth}|}{\raggedright \_\_class\_\_}\\
\cline{1-2}
\end{longtable}

    \index{peach \textit{(package)}!peach.optm \textit{(package)}!peach.optm.base \textit{(module)}!peach.optm.base.Optimizer \textit{(class)}|)}
    \index{peach \textit{(package)}!peach.optm \textit{(package)}!peach.optm.base \textit{(module)}|)}

%
% API Documentation for Peach - Computational Intelligence for Python
% Module peach.optm.linear
%
% Generated by epydoc 3.0beta1
% [Mon Dec 21 08:51:37 2009]
%

%%%%%%%%%%%%%%%%%%%%%%%%%%%%%%%%%%%%%%%%%%%%%%%%%%%%%%%%%%%%%%%%%%%%%%%%%%%
%%                          Module Description                           %%
%%%%%%%%%%%%%%%%%%%%%%%%%%%%%%%%%%%%%%%%%%%%%%%%%%%%%%%%%%%%%%%%%%%%%%%%%%%

    \index{peach \textit{(package)}!peach.optm \textit{(package)}!peach.optm.linear \textit{(module)}|(}
\section{Module peach.optm.linear}

    \label{peach:optm:linear}

This package implements basic one variable only optimizers.

%%%%%%%%%%%%%%%%%%%%%%%%%%%%%%%%%%%%%%%%%%%%%%%%%%%%%%%%%%%%%%%%%%%%%%%%%%%
%%                               Variables                               %%
%%%%%%%%%%%%%%%%%%%%%%%%%%%%%%%%%%%%%%%%%%%%%%%%%%%%%%%%%%%%%%%%%%%%%%%%%%%

  \subsection{Variables}

\begin{longtable}{|p{.30\textwidth}|p{.62\textwidth}|l}
\cline{1-2}
\cline{1-2} \centering \textbf{Name} & \centering \textbf{Description}& \\
\cline{1-2}
\endhead\cline{1-2}\multicolumn{3}{r}{\small\textit{continued on next page}}\\\endfoot\cline{1-2}
\endlastfoot\raggedright \_\-\_\-d\-o\-c\-\_\-\_\- & \raggedright \textbf{Value:} 
{\tt \texttt{...}}&\\
\cline{1-2}
\end{longtable}


%%%%%%%%%%%%%%%%%%%%%%%%%%%%%%%%%%%%%%%%%%%%%%%%%%%%%%%%%%%%%%%%%%%%%%%%%%%
%%                           Class Description                           %%
%%%%%%%%%%%%%%%%%%%%%%%%%%%%%%%%%%%%%%%%%%%%%%%%%%%%%%%%%%%%%%%%%%%%%%%%%%%

    \index{peach \textit{(package)}!peach.optm \textit{(package)}!peach.optm.linear \textit{(module)}!peach.optm.linear.Direct1D \textit{(class)}|(}
\subsection{Class Direct1D}

    \label{peach:optm:linear:Direct1D}
\begin{tabular}{cccccccc}
% Line for object, linespec=[False, False]
\multicolumn{2}{r}{\settowidth{\BCL}{object}\multirow{2}{\BCL}{object}}
&&
&&
  \\\cline{3-3}
  &&\multicolumn{1}{c|}{}
&&
&&
  \\
% Line for peach.optm.optm.Optimizer, linespec=[False]
\multicolumn{4}{r}{\settowidth{\BCL}{peach.optm.optm.Optimizer}\multirow{2}{\BCL}{peach.optm.optm.Optimizer}}
&&
  \\\cline{5-5}
  &&&&\multicolumn{1}{c|}{}
&&
  \\
&&&&\multicolumn{2}{l}{\textbf{peach.optm.linear.Direct1D}}
\end{tabular}


1-D direct search.

This methods 'oscilates' around the function minimum, reducing the updating
step until it achieves the maximum error or the maximum number of steps.
This is a very inefficient method, and should be used only at times where no
other methods are able to converge (eg., if a function has a lot of
discontinuities).

%%%%%%%%%%%%%%%%%%%%%%%%%%%%%%%%%%%%%%%%%%%%%%%%%%%%%%%%%%%%%%%%%%%%%%%%%%%
%%                                Methods                                %%
%%%%%%%%%%%%%%%%%%%%%%%%%%%%%%%%%%%%%%%%%%%%%%%%%%%%%%%%%%%%%%%%%%%%%%%%%%%

  \subsubsection{Methods}

    \vspace{0.5ex}

    \begin{boxedminipage}{\textwidth}

    \raggedright \textbf{\_\_init\_\_}(\textit{self}, \textit{f}, \textit{dx}=\texttt{0.5}, \textit{emax}=\texttt{1e-08}, \textit{imax}=\texttt{1000})

    \vspace{-1.5ex}

    \rule{\textwidth}{0.5\fboxrule}

Initializes the optimizer.

To create an optimizer of this type, instantiate the class with the
parameters given below:
    \vspace{1ex}

      \textbf{Parameters}
      \begin{quote}
        \begin{Ventry}{xxxx}

          \item[f]


A one variable only function to be optimized. The function should
have only one parameter and return the function value.
          \item[dx]


The initial step of the search. Defaults to 0.5
          \item[emax]


Maximum allowed error. The algorithm stops as soon as the error is
below this level. The error is absolute.
          \item[imax]


Maximum number of iterations, the algorithm stops as soon this
number of iterations are executed, no matter what the error is at
the moment.
        \end{Ventry}

      \end{quote}

    \vspace{1ex}

      Overrides: peach.optm.optm.Optimizer.\_\_init\_\_

    \end{boxedminipage}

    \vspace{0.5ex}

    \begin{boxedminipage}{\textwidth}

    \raggedright \textbf{step}(\textit{self}, \textit{x})

    \vspace{-1.5ex}

    \rule{\textwidth}{0.5\fboxrule}

One step of the search.

In this method, the result of the step is highly dependent of the steps
executed before, as the search step is updated at each call to this
method.
    \vspace{1ex}

      \textbf{Parameters}
      \begin{quote}
        \begin{Ventry}{x}

          \item[x]


The value from where the new estimate should be calculated. This can
of course be the result of a previous iteration of the algorithm.
        \end{Ventry}

      \end{quote}

    \vspace{1ex}

      \textbf{Return Value}
      \begin{quote}

This method returns a tuple \texttt{(x, e)}, where \texttt{x} is the updated
estimate of the minimum, and \texttt{e} is the estimated error.
      \end{quote}

    \vspace{1ex}

      Overrides: peach.optm.optm.Optimizer.step

    \end{boxedminipage}

    \vspace{0.5ex}

    \begin{boxedminipage}{\textwidth}

    \raggedright \textbf{\_\_call\_\_}(\textit{self}, \textit{x})

    \vspace{-1.5ex}

    \rule{\textwidth}{0.5\fboxrule}

Transparently executes the search until the minimum is found. The stop
criteria are the maximum error or the maximum number of iterations,
whichever is reached first. Note that this is a \texttt{{\_}{\_}call{\_}{\_}} method, so
the object is called as a function. This method returns a tuple
\texttt{(x, e)}, with the best estimate of the minimum and the error.
    \vspace{1ex}

      \textbf{Parameters}
      \begin{quote}
        \begin{Ventry}{x}

          \item[x]


The value from where the search must start.
        \end{Ventry}

      \end{quote}

    \vspace{1ex}

      \textbf{Return Value}
      \begin{quote}

This method returns a tuple \texttt{(x, e)}, where \texttt{x} is the best
estimate of the minimum, and \texttt{e} is the estimated error.
      \end{quote}

    \vspace{1ex}

      Overrides: peach.optm.optm.Optimizer.\_\_call\_\_

    \end{boxedminipage}

    \label{object:__delattr__}
    \index{object.\_\_delattr\_\_ \textit{(function)}}

    \vspace{0.5ex}

    \begin{boxedminipage}{\textwidth}

    \raggedright \textbf{\_\_delattr\_\_}(\textit{...})

    \vspace{-1.5ex}

    \rule{\textwidth}{0.5\fboxrule}

x.{\_}{\_}delattr{\_}{\_}('name') {\textless}=={\textgreater} del x.name
    \vspace{1ex}

    \end{boxedminipage}

    \label{object:__getattribute__}
    \index{object.\_\_getattribute\_\_ \textit{(function)}}

    \vspace{0.5ex}

    \begin{boxedminipage}{\textwidth}

    \raggedright \textbf{\_\_getattribute\_\_}(\textit{...})

    \vspace{-1.5ex}

    \rule{\textwidth}{0.5\fboxrule}

x.{\_}{\_}getattribute{\_}{\_}('name') {\textless}=={\textgreater} x.name
    \vspace{1ex}

    \end{boxedminipage}

    \label{object:__hash__}
    \index{object.\_\_hash\_\_ \textit{(function)}}

    \vspace{0.5ex}

    \begin{boxedminipage}{\textwidth}

    \raggedright \textbf{\_\_hash\_\_}(\textit{x})

    \vspace{-1.5ex}

    \rule{\textwidth}{0.5\fboxrule}

hash(x)
    \vspace{1ex}

    \end{boxedminipage}

    \label{object:__new__}
    \index{object.\_\_new\_\_ \textit{(function)}}

    \vspace{0.5ex}

    \begin{boxedminipage}{\textwidth}

    \raggedright \textbf{\_\_new\_\_}(\textit{T}, \textit{S}, \textit{...})

      \textbf{Return Value}
      \begin{quote}
\begin{alltt}
a new object with type S, a subtype of T
\end{alltt}

      \end{quote}

    \vspace{1ex}

    \end{boxedminipage}

    \label{object:__reduce__}
    \index{object.\_\_reduce\_\_ \textit{(function)}}

    \vspace{0.5ex}

    \begin{boxedminipage}{\textwidth}

    \raggedright \textbf{\_\_reduce\_\_}(\textit{...})

    \vspace{-1.5ex}

    \rule{\textwidth}{0.5\fboxrule}

helper for pickle
    \vspace{1ex}

    \end{boxedminipage}

    \label{object:__reduce_ex__}
    \index{object.\_\_reduce\_ex\_\_ \textit{(function)}}

    \vspace{0.5ex}

    \begin{boxedminipage}{\textwidth}

    \raggedright \textbf{\_\_reduce\_ex\_\_}(\textit{...})

    \vspace{-1.5ex}

    \rule{\textwidth}{0.5\fboxrule}

helper for pickle
    \vspace{1ex}

    \end{boxedminipage}

    \label{object:__repr__}
    \index{object.\_\_repr\_\_ \textit{(function)}}

    \vspace{0.5ex}

    \begin{boxedminipage}{\textwidth}

    \raggedright \textbf{\_\_repr\_\_}(\textit{x})

    \vspace{-1.5ex}

    \rule{\textwidth}{0.5\fboxrule}

repr(x)
    \vspace{1ex}

    \end{boxedminipage}

    \label{object:__setattr__}
    \index{object.\_\_setattr\_\_ \textit{(function)}}

    \vspace{0.5ex}

    \begin{boxedminipage}{\textwidth}

    \raggedright \textbf{\_\_setattr\_\_}(\textit{...})

    \vspace{-1.5ex}

    \rule{\textwidth}{0.5\fboxrule}

x.{\_}{\_}setattr{\_}{\_}('name', value) {\textless}=={\textgreater} x.name = value
    \vspace{1ex}

    \end{boxedminipage}

    \label{object:__str__}
    \index{object.\_\_str\_\_ \textit{(function)}}

    \vspace{0.5ex}

    \begin{boxedminipage}{\textwidth}

    \raggedright \textbf{\_\_str\_\_}(\textit{x})

    \vspace{-1.5ex}

    \rule{\textwidth}{0.5\fboxrule}

str(x)
    \vspace{1ex}

    \end{boxedminipage}


%%%%%%%%%%%%%%%%%%%%%%%%%%%%%%%%%%%%%%%%%%%%%%%%%%%%%%%%%%%%%%%%%%%%%%%%%%%
%%                              Properties                               %%
%%%%%%%%%%%%%%%%%%%%%%%%%%%%%%%%%%%%%%%%%%%%%%%%%%%%%%%%%%%%%%%%%%%%%%%%%%%

  \subsubsection{Properties}

\begin{longtable}{|p{.30\textwidth}|p{.62\textwidth}|l}
\cline{1-2}
\cline{1-2} \centering \textbf{Name} & \centering \textbf{Description}& \\
\cline{1-2}
\endhead\cline{1-2}\multicolumn{3}{r}{\small\textit{continued on next page}}\\\endfoot\cline{1-2}
\endlastfoot\raggedright \_\-\_\-c\-l\-a\-s\-s\-\_\-\_\- & \raggedright \textbf{Value:} 
{\tt {\textless}attribute '\_\_class\_\_' of 'object' objects{\textgreater}}&\\
\cline{1-2}
\end{longtable}

    \index{peach \textit{(package)}!peach.optm \textit{(package)}!peach.optm.linear \textit{(module)}!peach.optm.linear.Direct1D \textit{(class)}|)}

%%%%%%%%%%%%%%%%%%%%%%%%%%%%%%%%%%%%%%%%%%%%%%%%%%%%%%%%%%%%%%%%%%%%%%%%%%%
%%                           Class Description                           %%
%%%%%%%%%%%%%%%%%%%%%%%%%%%%%%%%%%%%%%%%%%%%%%%%%%%%%%%%%%%%%%%%%%%%%%%%%%%

    \index{peach \textit{(package)}!peach.optm \textit{(package)}!peach.optm.linear \textit{(module)}!peach.optm.linear.Interpolation \textit{(class)}|(}
\subsection{Class Interpolation}

    \label{peach:optm:linear:Interpolation}
\begin{tabular}{cccccccc}
% Line for object, linespec=[False, False]
\multicolumn{2}{r}{\settowidth{\BCL}{object}\multirow{2}{\BCL}{object}}
&&
&&
  \\\cline{3-3}
  &&\multicolumn{1}{c|}{}
&&
&&
  \\
% Line for peach.optm.optm.Optimizer, linespec=[False]
\multicolumn{4}{r}{\settowidth{\BCL}{peach.optm.optm.Optimizer}\multirow{2}{\BCL}{peach.optm.optm.Optimizer}}
&&
  \\\cline{5-5}
  &&&&\multicolumn{1}{c|}{}
&&
  \\
&&&&\multicolumn{2}{l}{\textbf{peach.optm.linear.Interpolation}}
\end{tabular}


Optimization by quadractic interpolation.

This methods takes three estimates and finds the parabolic function that
fits them, and returns as a new estimate the vertex of the parabola. The
procedure can be repeated until a good approximation is found.

%%%%%%%%%%%%%%%%%%%%%%%%%%%%%%%%%%%%%%%%%%%%%%%%%%%%%%%%%%%%%%%%%%%%%%%%%%%
%%                                Methods                                %%
%%%%%%%%%%%%%%%%%%%%%%%%%%%%%%%%%%%%%%%%%%%%%%%%%%%%%%%%%%%%%%%%%%%%%%%%%%%

  \subsubsection{Methods}

    \vspace{0.5ex}

    \begin{boxedminipage}{\textwidth}

    \raggedright \textbf{\_\_init\_\_}(\textit{self}, \textit{f}, \textit{emax}=\texttt{1e-05}, \textit{imax}=\texttt{1000})

    \vspace{-1.5ex}

    \rule{\textwidth}{0.5\fboxrule}

Initializes the optimizer.

To create an optimizer of this type, instantiate the class with the
parameters given below:
    \vspace{1ex}

      \textbf{Parameters}
      \begin{quote}
        \begin{Ventry}{xxxx}

          \item[f]


A one variable only function to be optimized. The function should
have only one parameter and return the function value.
          \item[dx]


The initial step of the search. Defaults to 0.5
          \item[emax]


Maximum allowed error. The algorithm stops as soon as the error is
below this level. The error is absolute.
          \item[imax]


Maximum number of iterations, the algorithm stops as soon this
number of iterations are executed, no matter what the error is at
the moment.
        \end{Ventry}

      \end{quote}

    \vspace{1ex}

      Overrides: peach.optm.optm.Optimizer.\_\_init\_\_

    \end{boxedminipage}

    \vspace{0.5ex}

    \begin{boxedminipage}{\textwidth}

    \raggedright \textbf{step}(\textit{self}, \textit{x})

    \vspace{-1.5ex}

    \rule{\textwidth}{0.5\fboxrule}

One step of the search.

In this method, the result of the step is dependent only of the given
estimated, so it can be used for different kind of investigations on the
same cost function.
    \vspace{1ex}

      \textbf{Parameters}
      \begin{quote}
        \begin{Ventry}{x}

          \item[x]


A triple \texttt{(x0, x1, x2)}, with \texttt{x0 < x1 < x2} of estimates on the
cost function. From these values the new estimate is calculated.
This can of course be the result of a previous iteration of the
algorithm.
        \end{Ventry}

      \end{quote}

    \vspace{1ex}

      \textbf{Return Value}
      \begin{quote}

This method returns a tuple \texttt{(x, e)}, where \texttt{x} is the updated
triplet of estimates of the minimum, and \texttt{e} is the estimated error.
      \end{quote}

    \vspace{1ex}

      Overrides: peach.optm.optm.Optimizer.step

    \end{boxedminipage}

    \vspace{0.5ex}

    \begin{boxedminipage}{\textwidth}

    \raggedright \textbf{\_\_call\_\_}(\textit{self}, \textit{x})

    \vspace{-1.5ex}

    \rule{\textwidth}{0.5\fboxrule}

Transparently executes the search until the minimum is found. The stop
criteria are the maximum error or the maximum number of iterations,
whichever is reached first. Note that this is a \texttt{{\_}{\_}call{\_}{\_}} method, so
the object is called as a function. This method returns a tuple
\texttt{(x, e)}, with the best estimate of the minimum and the error.
    \vspace{1ex}

      \textbf{Parameters}
      \begin{quote}
        \begin{Ventry}{x}

          \item[x]


The initial triplet of values from where the search must start.
        \end{Ventry}

      \end{quote}

    \vspace{1ex}

      \textbf{Return Value}
      \begin{quote}

This method returns a tuple \texttt{(x, e)}, where \texttt{x} is the best
estimate of the minimum, and \texttt{e} is the estimated error.
      \end{quote}

    \vspace{1ex}

      Overrides: peach.optm.optm.Optimizer.\_\_call\_\_

    \end{boxedminipage}

    \label{object:__delattr__}
    \index{object.\_\_delattr\_\_ \textit{(function)}}

    \vspace{0.5ex}

    \begin{boxedminipage}{\textwidth}

    \raggedright \textbf{\_\_delattr\_\_}(\textit{...})

    \vspace{-1.5ex}

    \rule{\textwidth}{0.5\fboxrule}

x.{\_}{\_}delattr{\_}{\_}('name') {\textless}=={\textgreater} del x.name
    \vspace{1ex}

    \end{boxedminipage}

    \label{object:__getattribute__}
    \index{object.\_\_getattribute\_\_ \textit{(function)}}

    \vspace{0.5ex}

    \begin{boxedminipage}{\textwidth}

    \raggedright \textbf{\_\_getattribute\_\_}(\textit{...})

    \vspace{-1.5ex}

    \rule{\textwidth}{0.5\fboxrule}

x.{\_}{\_}getattribute{\_}{\_}('name') {\textless}=={\textgreater} x.name
    \vspace{1ex}

    \end{boxedminipage}

    \label{object:__hash__}
    \index{object.\_\_hash\_\_ \textit{(function)}}

    \vspace{0.5ex}

    \begin{boxedminipage}{\textwidth}

    \raggedright \textbf{\_\_hash\_\_}(\textit{x})

    \vspace{-1.5ex}

    \rule{\textwidth}{0.5\fboxrule}

hash(x)
    \vspace{1ex}

    \end{boxedminipage}

    \label{object:__new__}
    \index{object.\_\_new\_\_ \textit{(function)}}

    \vspace{0.5ex}

    \begin{boxedminipage}{\textwidth}

    \raggedright \textbf{\_\_new\_\_}(\textit{T}, \textit{S}, \textit{...})

      \textbf{Return Value}
      \begin{quote}
\begin{alltt}
a new object with type S, a subtype of T
\end{alltt}

      \end{quote}

    \vspace{1ex}

    \end{boxedminipage}

    \label{object:__reduce__}
    \index{object.\_\_reduce\_\_ \textit{(function)}}

    \vspace{0.5ex}

    \begin{boxedminipage}{\textwidth}

    \raggedright \textbf{\_\_reduce\_\_}(\textit{...})

    \vspace{-1.5ex}

    \rule{\textwidth}{0.5\fboxrule}

helper for pickle
    \vspace{1ex}

    \end{boxedminipage}

    \label{object:__reduce_ex__}
    \index{object.\_\_reduce\_ex\_\_ \textit{(function)}}

    \vspace{0.5ex}

    \begin{boxedminipage}{\textwidth}

    \raggedright \textbf{\_\_reduce\_ex\_\_}(\textit{...})

    \vspace{-1.5ex}

    \rule{\textwidth}{0.5\fboxrule}

helper for pickle
    \vspace{1ex}

    \end{boxedminipage}

    \label{object:__repr__}
    \index{object.\_\_repr\_\_ \textit{(function)}}

    \vspace{0.5ex}

    \begin{boxedminipage}{\textwidth}

    \raggedright \textbf{\_\_repr\_\_}(\textit{x})

    \vspace{-1.5ex}

    \rule{\textwidth}{0.5\fboxrule}

repr(x)
    \vspace{1ex}

    \end{boxedminipage}

    \label{object:__setattr__}
    \index{object.\_\_setattr\_\_ \textit{(function)}}

    \vspace{0.5ex}

    \begin{boxedminipage}{\textwidth}

    \raggedright \textbf{\_\_setattr\_\_}(\textit{...})

    \vspace{-1.5ex}

    \rule{\textwidth}{0.5\fboxrule}

x.{\_}{\_}setattr{\_}{\_}('name', value) {\textless}=={\textgreater} x.name = value
    \vspace{1ex}

    \end{boxedminipage}

    \label{object:__str__}
    \index{object.\_\_str\_\_ \textit{(function)}}

    \vspace{0.5ex}

    \begin{boxedminipage}{\textwidth}

    \raggedright \textbf{\_\_str\_\_}(\textit{x})

    \vspace{-1.5ex}

    \rule{\textwidth}{0.5\fboxrule}

str(x)
    \vspace{1ex}

    \end{boxedminipage}


%%%%%%%%%%%%%%%%%%%%%%%%%%%%%%%%%%%%%%%%%%%%%%%%%%%%%%%%%%%%%%%%%%%%%%%%%%%
%%                              Properties                               %%
%%%%%%%%%%%%%%%%%%%%%%%%%%%%%%%%%%%%%%%%%%%%%%%%%%%%%%%%%%%%%%%%%%%%%%%%%%%

  \subsubsection{Properties}

\begin{longtable}{|p{.30\textwidth}|p{.62\textwidth}|l}
\cline{1-2}
\cline{1-2} \centering \textbf{Name} & \centering \textbf{Description}& \\
\cline{1-2}
\endhead\cline{1-2}\multicolumn{3}{r}{\small\textit{continued on next page}}\\\endfoot\cline{1-2}
\endlastfoot\raggedright \_\-\_\-c\-l\-a\-s\-s\-\_\-\_\- & \raggedright \textbf{Value:} 
{\tt {\textless}attribute '\_\_class\_\_' of 'object' objects{\textgreater}}&\\
\cline{1-2}
\end{longtable}

    \index{peach \textit{(package)}!peach.optm \textit{(package)}!peach.optm.linear \textit{(module)}!peach.optm.linear.Interpolation \textit{(class)}|)}

%%%%%%%%%%%%%%%%%%%%%%%%%%%%%%%%%%%%%%%%%%%%%%%%%%%%%%%%%%%%%%%%%%%%%%%%%%%
%%                           Class Description                           %%
%%%%%%%%%%%%%%%%%%%%%%%%%%%%%%%%%%%%%%%%%%%%%%%%%%%%%%%%%%%%%%%%%%%%%%%%%%%

    \index{peach \textit{(package)}!peach.optm \textit{(package)}!peach.optm.linear \textit{(module)}!peach.optm.linear.GoldenRule \textit{(class)}|(}
\subsection{Class GoldenRule}

    \label{peach:optm:linear:GoldenRule}
\begin{tabular}{cccccccc}
% Line for object, linespec=[False, False]
\multicolumn{2}{r}{\settowidth{\BCL}{object}\multirow{2}{\BCL}{object}}
&&
&&
  \\\cline{3-3}
  &&\multicolumn{1}{c|}{}
&&
&&
  \\
% Line for peach.optm.optm.Optimizer, linespec=[False]
\multicolumn{4}{r}{\settowidth{\BCL}{peach.optm.optm.Optimizer}\multirow{2}{\BCL}{peach.optm.optm.Optimizer}}
&&
  \\\cline{5-5}
  &&&&\multicolumn{1}{c|}{}
&&
  \\
&&&&\multicolumn{2}{l}{\textbf{peach.optm.linear.GoldenRule}}
\end{tabular}


Optimizer by the Golden Section Rule

This optimizer uses the golden rule to section an interval in search of the
minimum. Using a simple heuristic, the interval is refined until an interval
small enough to satisfy the error requirements is found.

%%%%%%%%%%%%%%%%%%%%%%%%%%%%%%%%%%%%%%%%%%%%%%%%%%%%%%%%%%%%%%%%%%%%%%%%%%%
%%                                Methods                                %%
%%%%%%%%%%%%%%%%%%%%%%%%%%%%%%%%%%%%%%%%%%%%%%%%%%%%%%%%%%%%%%%%%%%%%%%%%%%

  \subsubsection{Methods}

    \vspace{0.5ex}

    \begin{boxedminipage}{\textwidth}

    \raggedright \textbf{\_\_init\_\_}(\textit{self}, \textit{f}, \textit{emax}=\texttt{1e-05}, \textit{imax}=\texttt{1000})

    \vspace{-1.5ex}

    \rule{\textwidth}{0.5\fboxrule}

Initializes the optimizer.

To create an optimizer of this type, instantiate the class with the
parameters given below:
    \vspace{1ex}

      \textbf{Parameters}
      \begin{quote}
        \begin{Ventry}{xxxx}

          \item[f]


A one variable only function to be optimized. The function should
have only one parameter and return the function value.
          \item[dx]


The initial step of the search. Defaults to 0.5
          \item[emax]


Maximum allowed error. The algorithm stops as soon as the error is
below this level. The error is absolute.
          \item[imax]


Maximum number of iterations, the algorithm stops as soon this
number of iterations are executed, no matter what the error is at
the moment.
        \end{Ventry}

      \end{quote}

    \vspace{1ex}

      Overrides: peach.optm.optm.Optimizer.\_\_init\_\_

    \end{boxedminipage}

    \vspace{0.5ex}

    \begin{boxedminipage}{\textwidth}

    \raggedright \textbf{step}(\textit{self}, \textit{x})

    \vspace{-1.5ex}

    \rule{\textwidth}{0.5\fboxrule}

One step of the search.

In this method, the result of the step is dependent only of the given
estimated, so it can be used for different kind of investigations on the
same cost function.
    \vspace{1ex}

      \textbf{Parameters}
      \begin{quote}
        \begin{Ventry}{x}

          \item[x]


A duple \texttt{(x0, x1)}, with \texttt{x0 < x1} of estimates on the cost
function. From these values the new estimate is calculated. This can
of course be the result of a previous iteration of the algorithm.
        \end{Ventry}

      \end{quote}

    \vspace{1ex}

      \textbf{Return Value}
      \begin{quote}

This method returns a tuple \texttt{(x, e)}, where \texttt{x} is the updated
duple of estimates of the minimum, and \texttt{e} is the estimated error.
      \end{quote}

    \vspace{1ex}

      Overrides: peach.optm.optm.Optimizer.step

    \end{boxedminipage}

    \vspace{0.5ex}

    \begin{boxedminipage}{\textwidth}

    \raggedright \textbf{\_\_call\_\_}(\textit{self}, \textit{x})

    \vspace{-1.5ex}

    \rule{\textwidth}{0.5\fboxrule}

Transparently executes the search until the minimum is found. The stop
criteria are the maximum error or the maximum number of iterations,
whichever is reached first. Note that this is a \texttt{{\_}{\_}call{\_}{\_}} method, so
the object is called as a function. This method returns a tuple
\texttt{(x, e)}, with the best estimate of the minimum and the error.
    \vspace{1ex}

      \textbf{Parameters}
      \begin{quote}
        \begin{Ventry}{x}

          \item[x]


The initial duple of values from where the search must start.
        \end{Ventry}

      \end{quote}

    \vspace{1ex}

      \textbf{Return Value}
      \begin{quote}

This method returns a tuple \texttt{(x, e)}, where \texttt{x} is the best
estimate of the minimum, and \texttt{e} is the estimated error.
      \end{quote}

    \vspace{1ex}

      Overrides: peach.optm.optm.Optimizer.\_\_call\_\_

    \end{boxedminipage}

    \label{object:__delattr__}
    \index{object.\_\_delattr\_\_ \textit{(function)}}

    \vspace{0.5ex}

    \begin{boxedminipage}{\textwidth}

    \raggedright \textbf{\_\_delattr\_\_}(\textit{...})

    \vspace{-1.5ex}

    \rule{\textwidth}{0.5\fboxrule}

x.{\_}{\_}delattr{\_}{\_}('name') {\textless}=={\textgreater} del x.name
    \vspace{1ex}

    \end{boxedminipage}

    \label{object:__getattribute__}
    \index{object.\_\_getattribute\_\_ \textit{(function)}}

    \vspace{0.5ex}

    \begin{boxedminipage}{\textwidth}

    \raggedright \textbf{\_\_getattribute\_\_}(\textit{...})

    \vspace{-1.5ex}

    \rule{\textwidth}{0.5\fboxrule}

x.{\_}{\_}getattribute{\_}{\_}('name') {\textless}=={\textgreater} x.name
    \vspace{1ex}

    \end{boxedminipage}

    \label{object:__hash__}
    \index{object.\_\_hash\_\_ \textit{(function)}}

    \vspace{0.5ex}

    \begin{boxedminipage}{\textwidth}

    \raggedright \textbf{\_\_hash\_\_}(\textit{x})

    \vspace{-1.5ex}

    \rule{\textwidth}{0.5\fboxrule}

hash(x)
    \vspace{1ex}

    \end{boxedminipage}

    \label{object:__new__}
    \index{object.\_\_new\_\_ \textit{(function)}}

    \vspace{0.5ex}

    \begin{boxedminipage}{\textwidth}

    \raggedright \textbf{\_\_new\_\_}(\textit{T}, \textit{S}, \textit{...})

      \textbf{Return Value}
      \begin{quote}
\begin{alltt}
a new object with type S, a subtype of T
\end{alltt}

      \end{quote}

    \vspace{1ex}

    \end{boxedminipage}

    \label{object:__reduce__}
    \index{object.\_\_reduce\_\_ \textit{(function)}}

    \vspace{0.5ex}

    \begin{boxedminipage}{\textwidth}

    \raggedright \textbf{\_\_reduce\_\_}(\textit{...})

    \vspace{-1.5ex}

    \rule{\textwidth}{0.5\fboxrule}

helper for pickle
    \vspace{1ex}

    \end{boxedminipage}

    \label{object:__reduce_ex__}
    \index{object.\_\_reduce\_ex\_\_ \textit{(function)}}

    \vspace{0.5ex}

    \begin{boxedminipage}{\textwidth}

    \raggedright \textbf{\_\_reduce\_ex\_\_}(\textit{...})

    \vspace{-1.5ex}

    \rule{\textwidth}{0.5\fboxrule}

helper for pickle
    \vspace{1ex}

    \end{boxedminipage}

    \label{object:__repr__}
    \index{object.\_\_repr\_\_ \textit{(function)}}

    \vspace{0.5ex}

    \begin{boxedminipage}{\textwidth}

    \raggedright \textbf{\_\_repr\_\_}(\textit{x})

    \vspace{-1.5ex}

    \rule{\textwidth}{0.5\fboxrule}

repr(x)
    \vspace{1ex}

    \end{boxedminipage}

    \label{object:__setattr__}
    \index{object.\_\_setattr\_\_ \textit{(function)}}

    \vspace{0.5ex}

    \begin{boxedminipage}{\textwidth}

    \raggedright \textbf{\_\_setattr\_\_}(\textit{...})

    \vspace{-1.5ex}

    \rule{\textwidth}{0.5\fboxrule}

x.{\_}{\_}setattr{\_}{\_}('name', value) {\textless}=={\textgreater} x.name = value
    \vspace{1ex}

    \end{boxedminipage}

    \label{object:__str__}
    \index{object.\_\_str\_\_ \textit{(function)}}

    \vspace{0.5ex}

    \begin{boxedminipage}{\textwidth}

    \raggedright \textbf{\_\_str\_\_}(\textit{x})

    \vspace{-1.5ex}

    \rule{\textwidth}{0.5\fboxrule}

str(x)
    \vspace{1ex}

    \end{boxedminipage}


%%%%%%%%%%%%%%%%%%%%%%%%%%%%%%%%%%%%%%%%%%%%%%%%%%%%%%%%%%%%%%%%%%%%%%%%%%%
%%                              Properties                               %%
%%%%%%%%%%%%%%%%%%%%%%%%%%%%%%%%%%%%%%%%%%%%%%%%%%%%%%%%%%%%%%%%%%%%%%%%%%%

  \subsubsection{Properties}

\begin{longtable}{|p{.30\textwidth}|p{.62\textwidth}|l}
\cline{1-2}
\cline{1-2} \centering \textbf{Name} & \centering \textbf{Description}& \\
\cline{1-2}
\endhead\cline{1-2}\multicolumn{3}{r}{\small\textit{continued on next page}}\\\endfoot\cline{1-2}
\endlastfoot\raggedright \_\-\_\-c\-l\-a\-s\-s\-\_\-\_\- & \raggedright \textbf{Value:} 
{\tt {\textless}attribute '\_\_class\_\_' of 'object' objects{\textgreater}}&\\
\cline{1-2}
\end{longtable}

    \index{peach \textit{(package)}!peach.optm \textit{(package)}!peach.optm.linear \textit{(module)}!peach.optm.linear.GoldenRule \textit{(class)}|)}

%%%%%%%%%%%%%%%%%%%%%%%%%%%%%%%%%%%%%%%%%%%%%%%%%%%%%%%%%%%%%%%%%%%%%%%%%%%
%%                           Class Description                           %%
%%%%%%%%%%%%%%%%%%%%%%%%%%%%%%%%%%%%%%%%%%%%%%%%%%%%%%%%%%%%%%%%%%%%%%%%%%%

    \index{peach \textit{(package)}!peach.optm \textit{(package)}!peach.optm.linear \textit{(module)}!peach.optm.linear.Fibonacci \textit{(class)}|(}
\subsection{Class Fibonacci}

    \label{peach:optm:linear:Fibonacci}
\begin{tabular}{cccccccc}
% Line for object, linespec=[False, False]
\multicolumn{2}{r}{\settowidth{\BCL}{object}\multirow{2}{\BCL}{object}}
&&
&&
  \\\cline{3-3}
  &&\multicolumn{1}{c|}{}
&&
&&
  \\
% Line for peach.optm.optm.Optimizer, linespec=[False]
\multicolumn{4}{r}{\settowidth{\BCL}{peach.optm.optm.Optimizer}\multirow{2}{\BCL}{peach.optm.optm.Optimizer}}
&&
  \\\cline{5-5}
  &&&&\multicolumn{1}{c|}{}
&&
  \\
&&&&\multicolumn{2}{l}{\textbf{peach.optm.linear.Fibonacci}}
\end{tabular}


Optimization by the Golden Rule Section, estimated by Fibonacci numbers.

This optimizer uses the golden rule to section an interval in search of the
minimum. Using a simple heuristic, the interval is refined until an interval
small enough to satisfy the error requirements is found. The golden section
is estimated at each step using Fibonacci numbers. This can be useful in
situations where only integer numbers should be used.

%%%%%%%%%%%%%%%%%%%%%%%%%%%%%%%%%%%%%%%%%%%%%%%%%%%%%%%%%%%%%%%%%%%%%%%%%%%
%%                                Methods                                %%
%%%%%%%%%%%%%%%%%%%%%%%%%%%%%%%%%%%%%%%%%%%%%%%%%%%%%%%%%%%%%%%%%%%%%%%%%%%

  \subsubsection{Methods}

    \vspace{0.5ex}

    \begin{boxedminipage}{\textwidth}

    \raggedright \textbf{\_\_init\_\_}(\textit{self}, \textit{f}, \textit{emax}=\texttt{1e-05}, \textit{imax}=\texttt{1000})

    \vspace{-1.5ex}

    \rule{\textwidth}{0.5\fboxrule}

Initializes the optimizer.

To create an optimizer of this type, instantiate the class with the
parameters given below:
    \vspace{1ex}

      \textbf{Parameters}
      \begin{quote}
        \begin{Ventry}{xxxx}

          \item[f]


A one variable only function to be optimized. The function should
have only one parameter and return the function value.
          \item[dx]


The initial step of the search. Defaults to 0.5
          \item[emax]


Maximum allowed error. The algorithm stops as soon as the error is
below this level. The error is absolute.
          \item[imax]


Maximum number of iterations, the algorithm stops as soon this
number of iterations are executed, no matter what the error is at
the moment.
        \end{Ventry}

      \end{quote}

    \vspace{1ex}

      Overrides: peach.optm.optm.Optimizer.\_\_init\_\_

    \end{boxedminipage}

    \vspace{0.5ex}

    \begin{boxedminipage}{\textwidth}

    \raggedright \textbf{step}(\textit{self}, \textit{x})

    \vspace{-1.5ex}

    \rule{\textwidth}{0.5\fboxrule}

One step of the search.

In this method, the result of the step is highly dependent of the steps
executed before, as the estimate of the golden ratio is updated at each
call to this method.
    \vspace{1ex}

      \textbf{Parameters}
      \begin{quote}
        \begin{Ventry}{x}

          \item[x]


A duple \texttt{(x0, x1)}, with \texttt{x0 < x1} of estimates on the cost
function. From these values the new estimate is calculated. This can
of course be the result of a previous iteration of the algorithm.
        \end{Ventry}

      \end{quote}

    \vspace{1ex}

      \textbf{Return Value}
      \begin{quote}

This method returns a tuple \texttt{(x, e)}, where \texttt{x} is the updated
duple of estimates of the minimum, and \texttt{e} is the estimated error.
      \end{quote}

    \vspace{1ex}

      Overrides: peach.optm.optm.Optimizer.step

    \end{boxedminipage}

    \vspace{0.5ex}

    \begin{boxedminipage}{\textwidth}

    \raggedright \textbf{\_\_call\_\_}(\textit{self}, \textit{x})

    \vspace{-1.5ex}

    \rule{\textwidth}{0.5\fboxrule}

Transparently executes the search until the minimum is found. The stop
criteria are the maximum error or the maximum number of iterations,
whichever is reached first. Note that this is a \texttt{{\_}{\_}call{\_}{\_}} method, so
the object is called as a function. This method returns a tuple
\texttt{(x, e)}, with the best estimate of the minimum and the error.
    \vspace{1ex}

      \textbf{Parameters}
      \begin{quote}
        \begin{Ventry}{x}

          \item[x]


The initial duple of values from where the search must start.
        \end{Ventry}

      \end{quote}

    \vspace{1ex}

      \textbf{Return Value}
      \begin{quote}

This method returns a tuple \texttt{(x, e)}, where \texttt{x} is the best
estimate of the minimum, and \texttt{e} is the estimated error.
      \end{quote}

    \vspace{1ex}

      Overrides: peach.optm.optm.Optimizer.\_\_call\_\_

    \end{boxedminipage}

    \label{object:__delattr__}
    \index{object.\_\_delattr\_\_ \textit{(function)}}

    \vspace{0.5ex}

    \begin{boxedminipage}{\textwidth}

    \raggedright \textbf{\_\_delattr\_\_}(\textit{...})

    \vspace{-1.5ex}

    \rule{\textwidth}{0.5\fboxrule}

x.{\_}{\_}delattr{\_}{\_}('name') {\textless}=={\textgreater} del x.name
    \vspace{1ex}

    \end{boxedminipage}

    \label{object:__getattribute__}
    \index{object.\_\_getattribute\_\_ \textit{(function)}}

    \vspace{0.5ex}

    \begin{boxedminipage}{\textwidth}

    \raggedright \textbf{\_\_getattribute\_\_}(\textit{...})

    \vspace{-1.5ex}

    \rule{\textwidth}{0.5\fboxrule}

x.{\_}{\_}getattribute{\_}{\_}('name') {\textless}=={\textgreater} x.name
    \vspace{1ex}

    \end{boxedminipage}

    \label{object:__hash__}
    \index{object.\_\_hash\_\_ \textit{(function)}}

    \vspace{0.5ex}

    \begin{boxedminipage}{\textwidth}

    \raggedright \textbf{\_\_hash\_\_}(\textit{x})

    \vspace{-1.5ex}

    \rule{\textwidth}{0.5\fboxrule}

hash(x)
    \vspace{1ex}

    \end{boxedminipage}

    \label{object:__new__}
    \index{object.\_\_new\_\_ \textit{(function)}}

    \vspace{0.5ex}

    \begin{boxedminipage}{\textwidth}

    \raggedright \textbf{\_\_new\_\_}(\textit{T}, \textit{S}, \textit{...})

      \textbf{Return Value}
      \begin{quote}
\begin{alltt}
a new object with type S, a subtype of T
\end{alltt}

      \end{quote}

    \vspace{1ex}

    \end{boxedminipage}

    \label{object:__reduce__}
    \index{object.\_\_reduce\_\_ \textit{(function)}}

    \vspace{0.5ex}

    \begin{boxedminipage}{\textwidth}

    \raggedright \textbf{\_\_reduce\_\_}(\textit{...})

    \vspace{-1.5ex}

    \rule{\textwidth}{0.5\fboxrule}

helper for pickle
    \vspace{1ex}

    \end{boxedminipage}

    \label{object:__reduce_ex__}
    \index{object.\_\_reduce\_ex\_\_ \textit{(function)}}

    \vspace{0.5ex}

    \begin{boxedminipage}{\textwidth}

    \raggedright \textbf{\_\_reduce\_ex\_\_}(\textit{...})

    \vspace{-1.5ex}

    \rule{\textwidth}{0.5\fboxrule}

helper for pickle
    \vspace{1ex}

    \end{boxedminipage}

    \label{object:__repr__}
    \index{object.\_\_repr\_\_ \textit{(function)}}

    \vspace{0.5ex}

    \begin{boxedminipage}{\textwidth}

    \raggedright \textbf{\_\_repr\_\_}(\textit{x})

    \vspace{-1.5ex}

    \rule{\textwidth}{0.5\fboxrule}

repr(x)
    \vspace{1ex}

    \end{boxedminipage}

    \label{object:__setattr__}
    \index{object.\_\_setattr\_\_ \textit{(function)}}

    \vspace{0.5ex}

    \begin{boxedminipage}{\textwidth}

    \raggedright \textbf{\_\_setattr\_\_}(\textit{...})

    \vspace{-1.5ex}

    \rule{\textwidth}{0.5\fboxrule}

x.{\_}{\_}setattr{\_}{\_}('name', value) {\textless}=={\textgreater} x.name = value
    \vspace{1ex}

    \end{boxedminipage}

    \label{object:__str__}
    \index{object.\_\_str\_\_ \textit{(function)}}

    \vspace{0.5ex}

    \begin{boxedminipage}{\textwidth}

    \raggedright \textbf{\_\_str\_\_}(\textit{x})

    \vspace{-1.5ex}

    \rule{\textwidth}{0.5\fboxrule}

str(x)
    \vspace{1ex}

    \end{boxedminipage}


%%%%%%%%%%%%%%%%%%%%%%%%%%%%%%%%%%%%%%%%%%%%%%%%%%%%%%%%%%%%%%%%%%%%%%%%%%%
%%                              Properties                               %%
%%%%%%%%%%%%%%%%%%%%%%%%%%%%%%%%%%%%%%%%%%%%%%%%%%%%%%%%%%%%%%%%%%%%%%%%%%%

  \subsubsection{Properties}

\begin{longtable}{|p{.30\textwidth}|p{.62\textwidth}|l}
\cline{1-2}
\cline{1-2} \centering \textbf{Name} & \centering \textbf{Description}& \\
\cline{1-2}
\endhead\cline{1-2}\multicolumn{3}{r}{\small\textit{continued on next page}}\\\endfoot\cline{1-2}
\endlastfoot\raggedright \_\-\_\-c\-l\-a\-s\-s\-\_\-\_\- & \raggedright \textbf{Value:} 
{\tt {\textless}attribute '\_\_class\_\_' of 'object' objects{\textgreater}}&\\
\cline{1-2}
\end{longtable}

    \index{peach \textit{(package)}!peach.optm \textit{(package)}!peach.optm.linear \textit{(module)}!peach.optm.linear.Fibonacci \textit{(class)}|)}
    \index{peach \textit{(package)}!peach.optm \textit{(package)}!peach.optm.linear \textit{(module)}|)}

%
% API Documentation for Peach - Computational Intelligence for Python
% Module peach.optm.multivar
%
% Generated by epydoc 3.0.1
% [Mon Jan 24 15:39:52 2011]
%

%%%%%%%%%%%%%%%%%%%%%%%%%%%%%%%%%%%%%%%%%%%%%%%%%%%%%%%%%%%%%%%%%%%%%%%%%%%
%%                          Module Description                           %%
%%%%%%%%%%%%%%%%%%%%%%%%%%%%%%%%%%%%%%%%%%%%%%%%%%%%%%%%%%%%%%%%%%%%%%%%%%%

    \index{peach \textit{(package)}!peach.optm \textit{(package)}!peach.optm.multivar \textit{(module)}|(}
\section{Module peach.optm.multivar}

    \label{peach:optm:multivar}

This package implements basic multivariable optimizers, including gradient and
Newton searches.

%%%%%%%%%%%%%%%%%%%%%%%%%%%%%%%%%%%%%%%%%%%%%%%%%%%%%%%%%%%%%%%%%%%%%%%%%%%
%%                               Variables                               %%
%%%%%%%%%%%%%%%%%%%%%%%%%%%%%%%%%%%%%%%%%%%%%%%%%%%%%%%%%%%%%%%%%%%%%%%%%%%

  \subsection{Variables}

    \vspace{-1cm}
\hspace{\varindent}\begin{longtable}{|p{\varnamewidth}|p{\vardescrwidth}|l}
\cline{1-2}
\cline{1-2} \centering \textbf{Name} & \centering \textbf{Description}& \\
\cline{1-2}
\endhead\cline{1-2}\multicolumn{3}{r}{\small\textit{continued on next page}}\\\endfoot\cline{1-2}
\endlastfoot\raggedright \_\-\_\-d\-o\-c\-\_\-\_\- & \raggedright \textbf{Value:} 
{\tt \texttt{...}}&\\
\cline{1-2}
\raggedright \_\-\_\-p\-a\-c\-k\-a\-g\-e\-\_\-\_\- & \raggedright \textbf{Value:} 
{\tt \texttt{'}\texttt{peach.optm}\texttt{'}}&\\
\cline{1-2}
\end{longtable}


%%%%%%%%%%%%%%%%%%%%%%%%%%%%%%%%%%%%%%%%%%%%%%%%%%%%%%%%%%%%%%%%%%%%%%%%%%%
%%                           Class Description                           %%
%%%%%%%%%%%%%%%%%%%%%%%%%%%%%%%%%%%%%%%%%%%%%%%%%%%%%%%%%%%%%%%%%%%%%%%%%%%

    \index{peach \textit{(package)}!peach.optm \textit{(package)}!peach.optm.multivar \textit{(module)}!peach.optm.multivar.Direct \textit{(class)}|(}
\subsection{Class Direct}

    \label{peach:optm:multivar:Direct}
\begin{tabular}{cccccccc}
% Line for object, linespec=[False, False]
\multicolumn{2}{r}{\settowidth{\BCL}{object}\multirow{2}{\BCL}{object}}
&&
&&
  \\\cline{3-3}
  &&\multicolumn{1}{c|}{}
&&
&&
  \\
% Line for peach.optm.base.Optimizer, linespec=[False]
\multicolumn{4}{r}{\settowidth{\BCL}{peach.optm.base.Optimizer}\multirow{2}{\BCL}{peach.optm.base.Optimizer}}
&&
  \\\cline{5-5}
  &&&&\multicolumn{1}{c|}{}
&&
  \\
&&&&\multicolumn{2}{l}{\textbf{peach.optm.multivar.Direct}}
\end{tabular}


Multidimensional direct search

This optimization method is a generalization of the 1D method, using
variable swap as search direction. This results in a very simplistic and
inefficient method that should be used only when any other method fails.

%%%%%%%%%%%%%%%%%%%%%%%%%%%%%%%%%%%%%%%%%%%%%%%%%%%%%%%%%%%%%%%%%%%%%%%%%%%
%%                                Methods                                %%
%%%%%%%%%%%%%%%%%%%%%%%%%%%%%%%%%%%%%%%%%%%%%%%%%%%%%%%%%%%%%%%%%%%%%%%%%%%

  \subsubsection{Methods}

    \vspace{0.5ex}

\hspace{.8\funcindent}\begin{boxedminipage}{\funcwidth}

    \raggedright \textbf{\_\_init\_\_}(\textit{self}, \textit{f}, \textit{x0}, \textit{ranges}={\tt None}, \textit{h}={\tt 0.5}, \textit{emax}={\tt 1e-08}, \textit{imax}={\tt 1000})

    \vspace{-1.5ex}

    \rule{\textwidth}{0.5\fboxrule}
\setlength{\parskip}{2ex}

Initializes the optimizer.

To create an optimizer of this type, instantiate the class with the
parameters given below:
\setlength{\parskip}{1ex}
      \textbf{Parameters}
      \vspace{-1ex}

      \begin{quote}
        \begin{Ventry}{xxxxxx}

          \item[f]


A multivariable function to be optimized. The function should have
only one parameter, a multidimensional line-vector, and return the
function value, a scalar.
          \item[x0]


First estimate of the minimum. Estimates can be given in any format,
but internally they are converted to a one-dimension vector, where
each component corresponds to the estimate of that particular
variable. The vector is computed by flattening the array.
          \item[ranges]


A range of values might be passed to the algorithm, but it is not
necessary. If supplied, this parameter should be a list of ranges
for each variable of the objective function. It is specified as a
list of tuples of two values, \texttt{(x0, x1)}, where \texttt{x0} is the
start of the interval, and \texttt{x1} its end. Obviously, \texttt{x0} should
be smaller than \texttt{x1}. It can also be given as a list with a simple
tuple in the same format. In that case, the same range will be
applied for every variable in the optimization.
          \item[h]


The initial step of the search. Defaults to 0.5
          \item[emax]


Maximum allowed error. The algorithm stops as soon as the error is
below this level. The error is absolute.
          \item[imax]


Maximum number of iterations, the algorithm stops as soon this
number of iterations are executed, no matter what the error is at
the moment.
        \end{Ventry}

      \end{quote}

      Overrides: object.\_\_init\_\_

    \end{boxedminipage}

    \label{peach:optm:multivar:Direct:restart}
    \index{peach \textit{(package)}!peach.optm \textit{(package)}!peach.optm.multivar \textit{(module)}!peach.optm.multivar.Direct \textit{(class)}!peach.optm.multivar.Direct.restart \textit{(method)}}

    \vspace{0.5ex}

\hspace{.8\funcindent}\begin{boxedminipage}{\funcwidth}

    \raggedright \textbf{restart}(\textit{self}, \textit{x0}, \textit{h}={\tt 0.5})

    \vspace{-1.5ex}

    \rule{\textwidth}{0.5\fboxrule}
\setlength{\parskip}{2ex}

Resets the optimizer, returning to its original state, and allowing to
use a new first estimate.
\setlength{\parskip}{1ex}
      \textbf{Parameters}
      \vspace{-1ex}

      \begin{quote}
        \begin{Ventry}{xx}

          \item[x0]


New estimate of the minimum. Estimates can be given in any format,
but internally they are converted to a one-dimension vector, where
each component corresponds to the estimate of that particular
variable. The vector is computed by flattening the array.
          \item[h]


The initial step of the search. Defaults to 0.5
        \end{Ventry}

      \end{quote}

    \end{boxedminipage}

    \vspace{0.5ex}

\hspace{.8\funcindent}\begin{boxedminipage}{\funcwidth}

    \raggedright \textbf{step}(\textit{self})

    \vspace{-1.5ex}

    \rule{\textwidth}{0.5\fboxrule}
\setlength{\parskip}{2ex}

One step of the search.

In this method, the result of the step is highly dependent of the steps
executed before, as the search step is updated at each call to this
method.
\setlength{\parskip}{1ex}
      \textbf{Return Value}
    \vspace{-1ex}

      \begin{quote}

This method returns a tuple \texttt{(x, e)}, where \texttt{x} is the updated
estimate of the minimum, and \texttt{e} is the estimated error.
      \end{quote}

      Overrides: peach.optm.base.Optimizer.step

    \end{boxedminipage}

    \vspace{0.5ex}

\hspace{.8\funcindent}\begin{boxedminipage}{\funcwidth}

    \raggedright \textbf{\_\_call\_\_}(\textit{self})

    \vspace{-1.5ex}

    \rule{\textwidth}{0.5\fboxrule}
\setlength{\parskip}{2ex}

Transparently executes the search until the minimum is found. The stop
criteria are the maximum error or the maximum number of iterations,
whichever is reached first. Note that this is a \texttt{\_\_call\_\_} method, so
the object is called as a function. This method returns a tuple
\texttt{(x, e)}, with the best estimate of the minimum and the error.
\setlength{\parskip}{1ex}
      \textbf{Return Value}
    \vspace{-1ex}

      \begin{quote}

This method returns a tuple \texttt{(x, e)}, where \texttt{x} is the best
estimate of the minimum, and \texttt{e} is the estimated error.
      \end{quote}

      Overrides: peach.optm.base.Optimizer.\_\_call\_\_

    \end{boxedminipage}


\large{\textbf{\textit{Inherited from object}}}

\begin{quote}
\_\_delattr\_\_(), \_\_format\_\_(), \_\_getattribute\_\_(), \_\_hash\_\_(), \_\_new\_\_(), \_\_reduce\_\_(), \_\_reduce\_ex\_\_(), \_\_repr\_\_(), \_\_setattr\_\_(), \_\_sizeof\_\_(), \_\_str\_\_(), \_\_subclasshook\_\_()
\end{quote}

%%%%%%%%%%%%%%%%%%%%%%%%%%%%%%%%%%%%%%%%%%%%%%%%%%%%%%%%%%%%%%%%%%%%%%%%%%%
%%                              Properties                               %%
%%%%%%%%%%%%%%%%%%%%%%%%%%%%%%%%%%%%%%%%%%%%%%%%%%%%%%%%%%%%%%%%%%%%%%%%%%%

  \subsubsection{Properties}

    \vspace{-1cm}
\hspace{\varindent}\begin{longtable}{|p{\varnamewidth}|p{\vardescrwidth}|l}
\cline{1-2}
\cline{1-2} \centering \textbf{Name} & \centering \textbf{Description}& \\
\cline{1-2}
\endhead\cline{1-2}\multicolumn{3}{r}{\small\textit{continued on next page}}\\\endfoot\cline{1-2}
\endlastfoot\raggedright x\- & &\\
\cline{1-2}
\multicolumn{2}{|l|}{\textit{Inherited from object}}\\
\multicolumn{2}{|p{\varwidth}|}{\raggedright \_\_class\_\_}\\
\cline{1-2}
\end{longtable}


%%%%%%%%%%%%%%%%%%%%%%%%%%%%%%%%%%%%%%%%%%%%%%%%%%%%%%%%%%%%%%%%%%%%%%%%%%%
%%                          Instance Variables                           %%
%%%%%%%%%%%%%%%%%%%%%%%%%%%%%%%%%%%%%%%%%%%%%%%%%%%%%%%%%%%%%%%%%%%%%%%%%%%

  \subsubsection{Instance Variables}

    \vspace{-1cm}
\hspace{\varindent}\begin{longtable}{|p{\varnamewidth}|p{\vardescrwidth}|l}
\cline{1-2}
\cline{1-2} \centering \textbf{Name} & \centering \textbf{Description}& \\
\cline{1-2}
\endhead\cline{1-2}\multicolumn{3}{r}{\small\textit{continued on next page}}\\\endfoot\cline{1-2}
\endlastfoot\raggedright r\-a\-n\-g\-e\-s\- & Holds the ranges for every variable. Although it is a writable
property, care should be taken in changing parameters before ending the
convergence.&\\
\cline{1-2}
\end{longtable}

    \index{peach \textit{(package)}!peach.optm \textit{(package)}!peach.optm.multivar \textit{(module)}!peach.optm.multivar.Direct \textit{(class)}|)}

%%%%%%%%%%%%%%%%%%%%%%%%%%%%%%%%%%%%%%%%%%%%%%%%%%%%%%%%%%%%%%%%%%%%%%%%%%%
%%                           Class Description                           %%
%%%%%%%%%%%%%%%%%%%%%%%%%%%%%%%%%%%%%%%%%%%%%%%%%%%%%%%%%%%%%%%%%%%%%%%%%%%

    \index{peach \textit{(package)}!peach.optm \textit{(package)}!peach.optm.multivar \textit{(module)}!peach.optm.multivar.Gradient \textit{(class)}|(}
\subsection{Class Gradient}

    \label{peach:optm:multivar:Gradient}
\begin{tabular}{cccccccc}
% Line for object, linespec=[False, False]
\multicolumn{2}{r}{\settowidth{\BCL}{object}\multirow{2}{\BCL}{object}}
&&
&&
  \\\cline{3-3}
  &&\multicolumn{1}{c|}{}
&&
&&
  \\
% Line for peach.optm.base.Optimizer, linespec=[False]
\multicolumn{4}{r}{\settowidth{\BCL}{peach.optm.base.Optimizer}\multirow{2}{\BCL}{peach.optm.base.Optimizer}}
&&
  \\\cline{5-5}
  &&&&\multicolumn{1}{c|}{}
&&
  \\
&&&&\multicolumn{2}{l}{\textbf{peach.optm.multivar.Gradient}}
\end{tabular}


Gradient search

This method uses the fact that the gradient of a function points to the
direction of largest increase in the function (in general called \emph{uphill}
direction). So, the contrary direction (\emph{downhill}) is used as search
direction.

%%%%%%%%%%%%%%%%%%%%%%%%%%%%%%%%%%%%%%%%%%%%%%%%%%%%%%%%%%%%%%%%%%%%%%%%%%%
%%                                Methods                                %%
%%%%%%%%%%%%%%%%%%%%%%%%%%%%%%%%%%%%%%%%%%%%%%%%%%%%%%%%%%%%%%%%%%%%%%%%%%%

  \subsubsection{Methods}

    \vspace{0.5ex}

\hspace{.8\funcindent}\begin{boxedminipage}{\funcwidth}

    \raggedright \textbf{\_\_init\_\_}(\textit{self}, \textit{f}, \textit{x0}, \textit{ranges}={\tt None}, \textit{df}={\tt None}, \textit{h}={\tt 0.1}, \textit{emax}={\tt 1e-05}, \textit{imax}={\tt 1000})

    \vspace{-1.5ex}

    \rule{\textwidth}{0.5\fboxrule}
\setlength{\parskip}{2ex}

Initializes the optimizer.

To create an optimizer of this type, instantiate the class with the
parameters given below:
\setlength{\parskip}{1ex}
      \textbf{Parameters}
      \vspace{-1ex}

      \begin{quote}
        \begin{Ventry}{xxxxxx}

          \item[f]


A multivariable function to be optimized. The function should have
only one parameter, a multidimensional line-vector, and return the
function value, a scalar.
          \item[x0]


First estimate of the minimum. Estimates can be given in any format,
but internally they are converted to a one-dimension vector, where
each component corresponds to the estimate of that particular
variable. The vector is computed by flattening the array.
          \item[ranges]


A range of values might be passed to the algorithm, but it is not
necessary. If supplied, this parameter should be a list of ranges
for each variable of the objective function. It is specified as a
list of tuples of two values, \texttt{(x0, x1)}, where \texttt{x0} is the
start of the interval, and \texttt{x1} its end. Obviously, \texttt{x0} should
be smaller than \texttt{x1}. It can also be given as a list with a simple
tuple in the same format. In that case, the same range will be
applied for every variable in the optimization.
          \item[df]


A function to calculate the gradient vector of the cost function
\texttt{f}. Defaults to \texttt{None}, if no gradient is supplied, then it is
estimated from the cost function using Euler equations.
          \item[h]


Convergence step. This method does not takes into consideration the
possibility of varying the convergence step, to avoid Stiefel cages.
          \item[emax]


Maximum allowed error. The algorithm stops as soon as the error is
below this level. The error is absolute.
          \item[imax]


Maximum number of iterations, the algorithm stops as soon this
number of iterations are executed, no matter what the error is at
the moment.
        \end{Ventry}

      \end{quote}

      Overrides: object.\_\_init\_\_

    \end{boxedminipage}

    \label{peach:optm:multivar:Gradient:restart}
    \index{peach \textit{(package)}!peach.optm \textit{(package)}!peach.optm.multivar \textit{(module)}!peach.optm.multivar.Gradient \textit{(class)}!peach.optm.multivar.Gradient.restart \textit{(method)}}

    \vspace{0.5ex}

\hspace{.8\funcindent}\begin{boxedminipage}{\funcwidth}

    \raggedright \textbf{restart}(\textit{self}, \textit{x0}, \textit{h}={\tt None})

    \vspace{-1.5ex}

    \rule{\textwidth}{0.5\fboxrule}
\setlength{\parskip}{2ex}

Resets the optimizer, returning to its original state, and allowing to
use a new first estimate.
\setlength{\parskip}{1ex}
      \textbf{Parameters}
      \vspace{-1ex}

      \begin{quote}
        \begin{Ventry}{xx}

          \item[x0]


New estimate of the minimum. Estimates can be given in any format,
but internally they are converted to a one-dimension vector, where
each component corresponds to the estimate of that particular
variable. The vector is computed by flattening the array.
          \item[h]


Convergence step. This method does not takes into consideration the
possibility of varying the convergence step, to avoid Stiefel cages.
        \end{Ventry}

      \end{quote}

    \end{boxedminipage}

    \vspace{0.5ex}

\hspace{.8\funcindent}\begin{boxedminipage}{\funcwidth}

    \raggedright \textbf{step}(\textit{self})

    \vspace{-1.5ex}

    \rule{\textwidth}{0.5\fboxrule}
\setlength{\parskip}{2ex}

One step of the search.

In this method, the result of the step is dependent only of the given
estimated, so it can be used for different kind of investigations on the
same cost function.
\setlength{\parskip}{1ex}
      \textbf{Return Value}
    \vspace{-1ex}

      \begin{quote}

This method returns a tuple \texttt{(x, e)}, where \texttt{x} is the updated
estimate of the minimum, and \texttt{e} is the estimated error.
      \end{quote}

      Overrides: peach.optm.base.Optimizer.step

    \end{boxedminipage}

    \vspace{0.5ex}

\hspace{.8\funcindent}\begin{boxedminipage}{\funcwidth}

    \raggedright \textbf{\_\_call\_\_}(\textit{self})

    \vspace{-1.5ex}

    \rule{\textwidth}{0.5\fboxrule}
\setlength{\parskip}{2ex}

Transparently executes the search until the minimum is found. The stop
criteria are the maximum error or the maximum number of iterations,
whichever is reached first. Note that this is a \texttt{\_\_call\_\_} method, so
the object is called as a function. This method returns a tuple
\texttt{(x, e)}, with the best estimate of the minimum and the error.
\setlength{\parskip}{1ex}
      \textbf{Return Value}
    \vspace{-1ex}

      \begin{quote}

This method returns a tuple \texttt{(x, e)}, where \texttt{x} is the best
estimate of the minimum, and \texttt{e} is the estimated error.
      \end{quote}

      Overrides: peach.optm.base.Optimizer.\_\_call\_\_

    \end{boxedminipage}


\large{\textbf{\textit{Inherited from object}}}

\begin{quote}
\_\_delattr\_\_(), \_\_format\_\_(), \_\_getattribute\_\_(), \_\_hash\_\_(), \_\_new\_\_(), \_\_reduce\_\_(), \_\_reduce\_ex\_\_(), \_\_repr\_\_(), \_\_setattr\_\_(), \_\_sizeof\_\_(), \_\_str\_\_(), \_\_subclasshook\_\_()
\end{quote}

%%%%%%%%%%%%%%%%%%%%%%%%%%%%%%%%%%%%%%%%%%%%%%%%%%%%%%%%%%%%%%%%%%%%%%%%%%%
%%                              Properties                               %%
%%%%%%%%%%%%%%%%%%%%%%%%%%%%%%%%%%%%%%%%%%%%%%%%%%%%%%%%%%%%%%%%%%%%%%%%%%%

  \subsubsection{Properties}

    \vspace{-1cm}
\hspace{\varindent}\begin{longtable}{|p{\varnamewidth}|p{\vardescrwidth}|l}
\cline{1-2}
\cline{1-2} \centering \textbf{Name} & \centering \textbf{Description}& \\
\cline{1-2}
\endhead\cline{1-2}\multicolumn{3}{r}{\small\textit{continued on next page}}\\\endfoot\cline{1-2}
\endlastfoot\raggedright x\- & &\\
\cline{1-2}
\multicolumn{2}{|l|}{\textit{Inherited from object}}\\
\multicolumn{2}{|p{\varwidth}|}{\raggedright \_\_class\_\_}\\
\cline{1-2}
\end{longtable}


%%%%%%%%%%%%%%%%%%%%%%%%%%%%%%%%%%%%%%%%%%%%%%%%%%%%%%%%%%%%%%%%%%%%%%%%%%%
%%                          Instance Variables                           %%
%%%%%%%%%%%%%%%%%%%%%%%%%%%%%%%%%%%%%%%%%%%%%%%%%%%%%%%%%%%%%%%%%%%%%%%%%%%

  \subsubsection{Instance Variables}

    \vspace{-1cm}
\hspace{\varindent}\begin{longtable}{|p{\varnamewidth}|p{\vardescrwidth}|l}
\cline{1-2}
\cline{1-2} \centering \textbf{Name} & \centering \textbf{Description}& \\
\cline{1-2}
\endhead\cline{1-2}\multicolumn{3}{r}{\small\textit{continued on next page}}\\\endfoot\cline{1-2}
\endlastfoot\raggedright r\-a\-n\-g\-e\-s\- & Holds the ranges for every variable. Although it is a writable
property, care should be taken in changing parameters before ending the
convergence.&\\
\cline{1-2}
\end{longtable}

    \index{peach \textit{(package)}!peach.optm \textit{(package)}!peach.optm.multivar \textit{(module)}!peach.optm.multivar.Gradient \textit{(class)}|)}

%%%%%%%%%%%%%%%%%%%%%%%%%%%%%%%%%%%%%%%%%%%%%%%%%%%%%%%%%%%%%%%%%%%%%%%%%%%
%%                           Class Description                           %%
%%%%%%%%%%%%%%%%%%%%%%%%%%%%%%%%%%%%%%%%%%%%%%%%%%%%%%%%%%%%%%%%%%%%%%%%%%%

    \index{peach \textit{(package)}!peach.optm \textit{(package)}!peach.optm.multivar \textit{(module)}!peach.optm.multivar.MomentumGradient \textit{(class)}|(}
\subsection{Class MomentumGradient}

    \label{peach:optm:multivar:MomentumGradient}
\begin{tabular}{cccccccc}
% Line for object, linespec=[False, False]
\multicolumn{2}{r}{\settowidth{\BCL}{object}\multirow{2}{\BCL}{object}}
&&
&&
  \\\cline{3-3}
  &&\multicolumn{1}{c|}{}
&&
&&
  \\
% Line for peach.optm.base.Optimizer, linespec=[False]
\multicolumn{4}{r}{\settowidth{\BCL}{peach.optm.base.Optimizer}\multirow{2}{\BCL}{peach.optm.base.Optimizer}}
&&
  \\\cline{5-5}
  &&&&\multicolumn{1}{c|}{}
&&
  \\
&&&&\multicolumn{2}{l}{\textbf{peach.optm.multivar.MomentumGradient}}
\end{tabular}


Gradient search with momentum

This method uses the fact that the gradient of a function points to the
direction of largest increase in the function (in general called \emph{uphill}
direction). So, the contrary direction (\emph{downhill}) is used as search
direction. A momentum term is added to avoid local minima.

%%%%%%%%%%%%%%%%%%%%%%%%%%%%%%%%%%%%%%%%%%%%%%%%%%%%%%%%%%%%%%%%%%%%%%%%%%%
%%                                Methods                                %%
%%%%%%%%%%%%%%%%%%%%%%%%%%%%%%%%%%%%%%%%%%%%%%%%%%%%%%%%%%%%%%%%%%%%%%%%%%%

  \subsubsection{Methods}

    \vspace{0.5ex}

\hspace{.8\funcindent}\begin{boxedminipage}{\funcwidth}

    \raggedright \textbf{\_\_init\_\_}(\textit{self}, \textit{f}, \textit{x0}, \textit{ranges}={\tt None}, \textit{df}={\tt None}, \textit{h}={\tt 0.1}, \textit{a}={\tt 0.1}, \textit{emax}={\tt 1e-05}, \textit{imax}={\tt 1000})

    \vspace{-1.5ex}

    \rule{\textwidth}{0.5\fboxrule}
\setlength{\parskip}{2ex}

Initializes the optimizer.

To create an optimizer of this type, instantiate the class with the
parameters given below:
\setlength{\parskip}{1ex}
      \textbf{Parameters}
      \vspace{-1ex}

      \begin{quote}
        \begin{Ventry}{xxxxxx}

          \item[f]


A multivariable function to be optimized. The function should have
only one parameter, a multidimensional line-vector, and return the
function value, a scalar.
          \item[x0]


First estimate of the minimum. Estimates can be given in any format,
but internally they are converted to a one-dimension vector, where
each component corresponds to the estimate of that particular
variable. The vector is computed by flattening the array.
          \item[ranges]


A range of values might be passed to the algorithm, but it is not
necessary. If supplied, this parameter should be a list of ranges
for each variable of the objective function. It is specified as a
list of tuples of two values, \texttt{(x0, x1)}, where \texttt{x0} is the
start of the interval, and \texttt{x1} its end. Obviously, \texttt{x0} should
be smaller than \texttt{x1}. It can also be given as a list with a simple
tuple in the same format. In that case, the same range will be
applied for every variable in the optimization.
          \item[df]


A function to calculate the gradient vector of the cost function
\texttt{f}. Defaults to \texttt{None}, if no gradient is supplied, then it is
estimated from the cost function using Euler equations.
          \item[h]


Convergence step. This method does not takes into consideration the
possibility of varying the convergence step, to avoid Stiefel cages.
Defaults to 0.1.
          \item[a]


Momentum term. This term is a measure of the memory of the optmizer.
The bigger it is, the more the past values influence in the outcome
of the optimization. Defaults to 0.1
          \item[emax]


Maximum allowed error. The algorithm stops as soon as the error is
below this level. The error is absolute.
          \item[imax]


Maximum number of iterations, the algorithm stops as soon this
number of iterations are executed, no matter what the error is at
the moment.
        \end{Ventry}

      \end{quote}

      Overrides: object.\_\_init\_\_

    \end{boxedminipage}

    \label{peach:optm:multivar:MomentumGradient:restart}
    \index{peach \textit{(package)}!peach.optm \textit{(package)}!peach.optm.multivar \textit{(module)}!peach.optm.multivar.MomentumGradient \textit{(class)}!peach.optm.multivar.MomentumGradient.restart \textit{(method)}}

    \vspace{0.5ex}

\hspace{.8\funcindent}\begin{boxedminipage}{\funcwidth}

    \raggedright \textbf{restart}(\textit{self}, \textit{x0}, \textit{h}={\tt None}, \textit{a}={\tt None})

    \vspace{-1.5ex}

    \rule{\textwidth}{0.5\fboxrule}
\setlength{\parskip}{2ex}

Resets the optimizer, returning to its original state, and allowing to
use a new first estimate.
\setlength{\parskip}{1ex}
      \textbf{Parameters}
      \vspace{-1ex}

      \begin{quote}
        \begin{Ventry}{xx}

          \item[x0]


New estimate of the minimum. Estimates can be given in any format,
but internally they are converted to a one-dimension vector, where
each component corresponds to the estimate of that particular
variable. The vector is computed by flattening the array.
          \item[h]


Convergence step. This method does not takes into consideration the
possibility of varying the convergence step, to avoid Stiefel cages.
If not given in this method, the old value is used.
          \item[a]


Momentum term. This term is a measure of the memory of the optmizer.
The bigger it is, the more the past values influence in the outcome
of the optimization. If not given in this method, the old value is
used.
        \end{Ventry}

      \end{quote}

    \end{boxedminipage}

    \vspace{0.5ex}

\hspace{.8\funcindent}\begin{boxedminipage}{\funcwidth}

    \raggedright \textbf{step}(\textit{self})

    \vspace{-1.5ex}

    \rule{\textwidth}{0.5\fboxrule}
\setlength{\parskip}{2ex}

One step of the search.

In this method, the result of the step is dependent only of the given
estimated, so it can be used for different kind of investigations on the
same cost function.
\setlength{\parskip}{1ex}
      \textbf{Return Value}
    \vspace{-1ex}

      \begin{quote}

This method returns a tuple \texttt{(x, e)}, where \texttt{x} is the updated
estimate of the minimum, and \texttt{e} is the estimated error.
      \end{quote}

      Overrides: peach.optm.base.Optimizer.step

    \end{boxedminipage}

    \vspace{0.5ex}

\hspace{.8\funcindent}\begin{boxedminipage}{\funcwidth}

    \raggedright \textbf{\_\_call\_\_}(\textit{self})

    \vspace{-1.5ex}

    \rule{\textwidth}{0.5\fboxrule}
\setlength{\parskip}{2ex}

Transparently executes the search until the minimum is found. The stop
criteria are the maximum error or the maximum number of iterations,
whichever is reached first. Note that this is a \texttt{\_\_call\_\_} method, so
the object is called as a function. This method returns a tuple
\texttt{(x, e)}, with the best estimate of the minimum and the error.
\setlength{\parskip}{1ex}
      \textbf{Return Value}
    \vspace{-1ex}

      \begin{quote}

This method returns a tuple \texttt{(x, e)}, where \texttt{x} is the best
estimate of the minimum, and \texttt{e} is the estimated error.
      \end{quote}

      Overrides: peach.optm.base.Optimizer.\_\_call\_\_

    \end{boxedminipage}


\large{\textbf{\textit{Inherited from object}}}

\begin{quote}
\_\_delattr\_\_(), \_\_format\_\_(), \_\_getattribute\_\_(), \_\_hash\_\_(), \_\_new\_\_(), \_\_reduce\_\_(), \_\_reduce\_ex\_\_(), \_\_repr\_\_(), \_\_setattr\_\_(), \_\_sizeof\_\_(), \_\_str\_\_(), \_\_subclasshook\_\_()
\end{quote}

%%%%%%%%%%%%%%%%%%%%%%%%%%%%%%%%%%%%%%%%%%%%%%%%%%%%%%%%%%%%%%%%%%%%%%%%%%%
%%                              Properties                               %%
%%%%%%%%%%%%%%%%%%%%%%%%%%%%%%%%%%%%%%%%%%%%%%%%%%%%%%%%%%%%%%%%%%%%%%%%%%%

  \subsubsection{Properties}

    \vspace{-1cm}
\hspace{\varindent}\begin{longtable}{|p{\varnamewidth}|p{\vardescrwidth}|l}
\cline{1-2}
\cline{1-2} \centering \textbf{Name} & \centering \textbf{Description}& \\
\cline{1-2}
\endhead\cline{1-2}\multicolumn{3}{r}{\small\textit{continued on next page}}\\\endfoot\cline{1-2}
\endlastfoot\raggedright x\- & &\\
\cline{1-2}
\multicolumn{2}{|l|}{\textit{Inherited from object}}\\
\multicolumn{2}{|p{\varwidth}|}{\raggedright \_\_class\_\_}\\
\cline{1-2}
\end{longtable}


%%%%%%%%%%%%%%%%%%%%%%%%%%%%%%%%%%%%%%%%%%%%%%%%%%%%%%%%%%%%%%%%%%%%%%%%%%%
%%                          Instance Variables                           %%
%%%%%%%%%%%%%%%%%%%%%%%%%%%%%%%%%%%%%%%%%%%%%%%%%%%%%%%%%%%%%%%%%%%%%%%%%%%

  \subsubsection{Instance Variables}

    \vspace{-1cm}
\hspace{\varindent}\begin{longtable}{|p{\varnamewidth}|p{\vardescrwidth}|l}
\cline{1-2}
\cline{1-2} \centering \textbf{Name} & \centering \textbf{Description}& \\
\cline{1-2}
\endhead\cline{1-2}\multicolumn{3}{r}{\small\textit{continued on next page}}\\\endfoot\cline{1-2}
\endlastfoot\raggedright r\-a\-n\-g\-e\-s\- & Holds the ranges for every variable. Although it is a writable
property, care should be taken in changing parameters before ending the
convergence.&\\
\cline{1-2}
\end{longtable}

    \index{peach \textit{(package)}!peach.optm \textit{(package)}!peach.optm.multivar \textit{(module)}!peach.optm.multivar.MomentumGradient \textit{(class)}|)}

%%%%%%%%%%%%%%%%%%%%%%%%%%%%%%%%%%%%%%%%%%%%%%%%%%%%%%%%%%%%%%%%%%%%%%%%%%%
%%                           Class Description                           %%
%%%%%%%%%%%%%%%%%%%%%%%%%%%%%%%%%%%%%%%%%%%%%%%%%%%%%%%%%%%%%%%%%%%%%%%%%%%

    \index{peach \textit{(package)}!peach.optm \textit{(package)}!peach.optm.multivar \textit{(module)}!peach.optm.multivar.Newton \textit{(class)}|(}
\subsection{Class Newton}

    \label{peach:optm:multivar:Newton}
\begin{tabular}{cccccccc}
% Line for object, linespec=[False, False]
\multicolumn{2}{r}{\settowidth{\BCL}{object}\multirow{2}{\BCL}{object}}
&&
&&
  \\\cline{3-3}
  &&\multicolumn{1}{c|}{}
&&
&&
  \\
% Line for peach.optm.base.Optimizer, linespec=[False]
\multicolumn{4}{r}{\settowidth{\BCL}{peach.optm.base.Optimizer}\multirow{2}{\BCL}{peach.optm.base.Optimizer}}
&&
  \\\cline{5-5}
  &&&&\multicolumn{1}{c|}{}
&&
  \\
&&&&\multicolumn{2}{l}{\textbf{peach.optm.multivar.Newton}}
\end{tabular}


Newton search

This is a very effective method to find minimum points in functions. In a
very basic fashion, this method corresponds to using Newton root finding
method on f'(x). Converges \emph{very} fast if the cost function is quadratic
of similar to it.

%%%%%%%%%%%%%%%%%%%%%%%%%%%%%%%%%%%%%%%%%%%%%%%%%%%%%%%%%%%%%%%%%%%%%%%%%%%
%%                                Methods                                %%
%%%%%%%%%%%%%%%%%%%%%%%%%%%%%%%%%%%%%%%%%%%%%%%%%%%%%%%%%%%%%%%%%%%%%%%%%%%

  \subsubsection{Methods}

    \vspace{0.5ex}

\hspace{.8\funcindent}\begin{boxedminipage}{\funcwidth}

    \raggedright \textbf{\_\_init\_\_}(\textit{self}, \textit{f}, \textit{x0}, \textit{ranges}={\tt None}, \textit{df}={\tt None}, \textit{hf}={\tt None}, \textit{h}={\tt 0.1}, \textit{emax}={\tt 1e-05}, \textit{imax}={\tt 1000})

    \vspace{-1.5ex}

    \rule{\textwidth}{0.5\fboxrule}
\setlength{\parskip}{2ex}

Initializes the optimizer.

To create an optimizer of this type, instantiate the class with the
parameters given below:
\setlength{\parskip}{1ex}
      \textbf{Parameters}
      \vspace{-1ex}

      \begin{quote}
        \begin{Ventry}{xxxxxx}

          \item[f]


A multivariable function to be optimized. The function should have
only one parameter, a multidimensional line-vector, and return the
function value, a scalar.
          \item[x0]


First estimate of the minimum. Estimates can be given in any format,
but internally they are converted to a one-dimension vector, where
each component corresponds to the estimate of that particular
variable. The vector is computed by flattening the array.
          \item[ranges]


A range of values might be passed to the algorithm, but it is not
necessary. If supplied, this parameter should be a list of ranges
for each variable of the objective function. It is specified as a
list of tuples of two values, \texttt{(x0, x1)}, where \texttt{x0} is the
start of the interval, and \texttt{x1} its end. Obviously, \texttt{x0} should
be smaller than \texttt{x1}. It can also be given as a list with a simple
tuple in the same format. In that case, the same range will be
applied for every variable in the optimization.
          \item[df]


A function to calculate the gradient vector of the cost function
\texttt{f}. Defaults to \texttt{None}, if no gradient is supplied, then it is
estimated from the cost function using Euler equations.
          \item[hf]


A function to calculate the hessian matrix of the cost function
\texttt{f}. Defaults to \texttt{None}, if no hessian is supplied, then it is
estimated from the cost function using Euler equations.
          \item[h]


Convergence step. This method does not takes into consideration the
possibility of varying the convergence step, to avoid Stiefel cages.
          \item[emax]


Maximum allowed error. The algorithm stops as soon as the error is
below this level. The error is absolute.
          \item[imax]


Maximum number of iterations, the algorithm stops as soon this
number of iterations are executed, no matter what the error is at
the moment.
        \end{Ventry}

      \end{quote}

      Overrides: object.\_\_init\_\_

    \end{boxedminipage}

    \label{peach:optm:multivar:Newton:restart}
    \index{peach \textit{(package)}!peach.optm \textit{(package)}!peach.optm.multivar \textit{(module)}!peach.optm.multivar.Newton \textit{(class)}!peach.optm.multivar.Newton.restart \textit{(method)}}

    \vspace{0.5ex}

\hspace{.8\funcindent}\begin{boxedminipage}{\funcwidth}

    \raggedright \textbf{restart}(\textit{self}, \textit{x0}, \textit{h}={\tt None})

    \vspace{-1.5ex}

    \rule{\textwidth}{0.5\fboxrule}
\setlength{\parskip}{2ex}

Resets the optimizer, returning to its original state, and allowing to
use a new first estimate.
\setlength{\parskip}{1ex}
      \textbf{Parameters}
      \vspace{-1ex}

      \begin{quote}
        \begin{Ventry}{xx}

          \item[x0]


New estimate of the minimum. Estimates can be given in any format,
but internally they are converted to a one-dimension vector, where
each component corresponds to the estimate of that particular
variable. The vector is computed by flattening the array.
          \item[h]


Convergence step. This method does not takes into consideration the
possibility of varying the convergence step, to avoid Stiefel cages.
        \end{Ventry}

      \end{quote}

    \end{boxedminipage}

    \vspace{0.5ex}

\hspace{.8\funcindent}\begin{boxedminipage}{\funcwidth}

    \raggedright \textbf{step}(\textit{self})

    \vspace{-1.5ex}

    \rule{\textwidth}{0.5\fboxrule}
\setlength{\parskip}{2ex}

One step of the search.

In this method, the result of the step is dependent only of the given
estimated, so it can be used for different kind of investigations on the
same cost function.
\setlength{\parskip}{1ex}
      \textbf{Return Value}
    \vspace{-1ex}

      \begin{quote}

This method returns a tuple \texttt{(x, e)}, where \texttt{x} is the updated
estimate of the minimum, and \texttt{e} is the estimated error.
      \end{quote}

      Overrides: peach.optm.base.Optimizer.step

    \end{boxedminipage}

    \vspace{0.5ex}

\hspace{.8\funcindent}\begin{boxedminipage}{\funcwidth}

    \raggedright \textbf{\_\_call\_\_}(\textit{self})

    \vspace{-1.5ex}

    \rule{\textwidth}{0.5\fboxrule}
\setlength{\parskip}{2ex}

Transparently executes the search until the minimum is found. The stop
criteria are the maximum error or the maximum number of iterations,
whichever is reached first. Note that this is a \texttt{\_\_call\_\_} method, so
the object is called as a function. This method returns a tuple
\texttt{(x, e)}, with the best estimate of the minimum and the error.
\setlength{\parskip}{1ex}
      \textbf{Return Value}
    \vspace{-1ex}

      \begin{quote}

This method returns a tuple \texttt{(x, e)}, where \texttt{x} is the best
estimate of the minimum, and \texttt{e} is the estimated error.
      \end{quote}

      Overrides: peach.optm.base.Optimizer.\_\_call\_\_

    \end{boxedminipage}


\large{\textbf{\textit{Inherited from object}}}

\begin{quote}
\_\_delattr\_\_(), \_\_format\_\_(), \_\_getattribute\_\_(), \_\_hash\_\_(), \_\_new\_\_(), \_\_reduce\_\_(), \_\_reduce\_ex\_\_(), \_\_repr\_\_(), \_\_setattr\_\_(), \_\_sizeof\_\_(), \_\_str\_\_(), \_\_subclasshook\_\_()
\end{quote}

%%%%%%%%%%%%%%%%%%%%%%%%%%%%%%%%%%%%%%%%%%%%%%%%%%%%%%%%%%%%%%%%%%%%%%%%%%%
%%                              Properties                               %%
%%%%%%%%%%%%%%%%%%%%%%%%%%%%%%%%%%%%%%%%%%%%%%%%%%%%%%%%%%%%%%%%%%%%%%%%%%%

  \subsubsection{Properties}

    \vspace{-1cm}
\hspace{\varindent}\begin{longtable}{|p{\varnamewidth}|p{\vardescrwidth}|l}
\cline{1-2}
\cline{1-2} \centering \textbf{Name} & \centering \textbf{Description}& \\
\cline{1-2}
\endhead\cline{1-2}\multicolumn{3}{r}{\small\textit{continued on next page}}\\\endfoot\cline{1-2}
\endlastfoot\raggedright x\- & &\\
\cline{1-2}
\multicolumn{2}{|l|}{\textit{Inherited from object}}\\
\multicolumn{2}{|p{\varwidth}|}{\raggedright \_\_class\_\_}\\
\cline{1-2}
\end{longtable}


%%%%%%%%%%%%%%%%%%%%%%%%%%%%%%%%%%%%%%%%%%%%%%%%%%%%%%%%%%%%%%%%%%%%%%%%%%%
%%                          Instance Variables                           %%
%%%%%%%%%%%%%%%%%%%%%%%%%%%%%%%%%%%%%%%%%%%%%%%%%%%%%%%%%%%%%%%%%%%%%%%%%%%

  \subsubsection{Instance Variables}

    \vspace{-1cm}
\hspace{\varindent}\begin{longtable}{|p{\varnamewidth}|p{\vardescrwidth}|l}
\cline{1-2}
\cline{1-2} \centering \textbf{Name} & \centering \textbf{Description}& \\
\cline{1-2}
\endhead\cline{1-2}\multicolumn{3}{r}{\small\textit{continued on next page}}\\\endfoot\cline{1-2}
\endlastfoot\raggedright r\-a\-n\-g\-e\-s\- & Holds the ranges for every variable. Although it is a writable
property, care should be taken in changing parameters before ending the
convergence.&\\
\cline{1-2}
\end{longtable}

    \index{peach \textit{(package)}!peach.optm \textit{(package)}!peach.optm.multivar \textit{(module)}!peach.optm.multivar.Newton \textit{(class)}|)}
    \index{peach \textit{(package)}!peach.optm \textit{(package)}!peach.optm.multivar \textit{(module)}|)}

%
% API Documentation for Peach - Computational Intelligence for Python
% Module peach.optm.quasinewton
%
% Generated by epydoc 3.0.1
% [Mon Jan 24 15:39:52 2011]
%

%%%%%%%%%%%%%%%%%%%%%%%%%%%%%%%%%%%%%%%%%%%%%%%%%%%%%%%%%%%%%%%%%%%%%%%%%%%
%%                          Module Description                           %%
%%%%%%%%%%%%%%%%%%%%%%%%%%%%%%%%%%%%%%%%%%%%%%%%%%%%%%%%%%%%%%%%%%%%%%%%%%%

    \index{peach \textit{(package)}!peach.optm \textit{(package)}!peach.optm.quasinewton \textit{(module)}|(}
\section{Module peach.optm.quasinewton}

    \label{peach:optm:quasinewton}

This package implements basic quasi-Newton optimizers. Newton optimizer is very
efficient, except that inverse matrices need to be calculated at each
convergence step. These methods try to estimate the hessian inverse iteratively,
thus increasing performance.

%%%%%%%%%%%%%%%%%%%%%%%%%%%%%%%%%%%%%%%%%%%%%%%%%%%%%%%%%%%%%%%%%%%%%%%%%%%
%%                               Variables                               %%
%%%%%%%%%%%%%%%%%%%%%%%%%%%%%%%%%%%%%%%%%%%%%%%%%%%%%%%%%%%%%%%%%%%%%%%%%%%

  \subsection{Variables}

    \vspace{-1cm}
\hspace{\varindent}\begin{longtable}{|p{\varnamewidth}|p{\vardescrwidth}|l}
\cline{1-2}
\cline{1-2} \centering \textbf{Name} & \centering \textbf{Description}& \\
\cline{1-2}
\endhead\cline{1-2}\multicolumn{3}{r}{\small\textit{continued on next page}}\\\endfoot\cline{1-2}
\endlastfoot\raggedright \_\-\_\-d\-o\-c\-\_\-\_\- & \raggedright \textbf{Value:} 
{\tt \texttt{...}}&\\
\cline{1-2}
\raggedright \_\-\_\-p\-a\-c\-k\-a\-g\-e\-\_\-\_\- & \raggedright \textbf{Value:} 
{\tt \texttt{'}\texttt{peach.optm}\texttt{'}}&\\
\cline{1-2}
\end{longtable}


%%%%%%%%%%%%%%%%%%%%%%%%%%%%%%%%%%%%%%%%%%%%%%%%%%%%%%%%%%%%%%%%%%%%%%%%%%%
%%                           Class Description                           %%
%%%%%%%%%%%%%%%%%%%%%%%%%%%%%%%%%%%%%%%%%%%%%%%%%%%%%%%%%%%%%%%%%%%%%%%%%%%

    \index{peach \textit{(package)}!peach.optm \textit{(package)}!peach.optm.quasinewton \textit{(module)}!peach.optm.quasinewton.DFP \textit{(class)}|(}
\subsection{Class DFP}

    \label{peach:optm:quasinewton:DFP}
\begin{tabular}{cccccccc}
% Line for object, linespec=[False, False]
\multicolumn{2}{r}{\settowidth{\BCL}{object}\multirow{2}{\BCL}{object}}
&&
&&
  \\\cline{3-3}
  &&\multicolumn{1}{c|}{}
&&
&&
  \\
% Line for peach.optm.base.Optimizer, linespec=[False]
\multicolumn{4}{r}{\settowidth{\BCL}{peach.optm.base.Optimizer}\multirow{2}{\BCL}{peach.optm.base.Optimizer}}
&&
  \\\cline{5-5}
  &&&&\multicolumn{1}{c|}{}
&&
  \\
&&&&\multicolumn{2}{l}{\textbf{peach.optm.quasinewton.DFP}}
\end{tabular}


DFP (\emph{Davidon-Fletcher-Powell}) search

%%%%%%%%%%%%%%%%%%%%%%%%%%%%%%%%%%%%%%%%%%%%%%%%%%%%%%%%%%%%%%%%%%%%%%%%%%%
%%                                Methods                                %%
%%%%%%%%%%%%%%%%%%%%%%%%%%%%%%%%%%%%%%%%%%%%%%%%%%%%%%%%%%%%%%%%%%%%%%%%%%%

  \subsubsection{Methods}

    \vspace{0.5ex}

\hspace{.8\funcindent}\begin{boxedminipage}{\funcwidth}

    \raggedright \textbf{\_\_init\_\_}(\textit{self}, \textit{f}, \textit{x0}, \textit{ranges}={\tt None}, \textit{df}={\tt None}, \textit{h}={\tt 0.1}, \textit{emax}={\tt 1e-08}, \textit{imax}={\tt 1000})

    \vspace{-1.5ex}

    \rule{\textwidth}{0.5\fboxrule}
\setlength{\parskip}{2ex}

Initializes the optimizer.

To create an optimizer of this type, instantiate the class with the
parameters given below:
\setlength{\parskip}{1ex}
      \textbf{Parameters}
      \vspace{-1ex}

      \begin{quote}
        \begin{Ventry}{xxxxxx}

          \item[f]


A multivariable function to be optimized. The function should have
only one parameter, a multidimensional line-vector, and return the
function value, a scalar.
          \item[x0]


First estimate of the minimum. Estimates can be given in any format,
but internally they are converted to a one-dimension vector, where
each component corresponds to the estimate of that particular
variable. The vector is computed by flattening the array.
          \item[ranges]


A range of values might be passed to the algorithm, but it is not
necessary. If supplied, this parameter should be a list of ranges
for each variable of the objective function. It is specified as a
list of tuples of two values, \texttt{(x0, x1)}, where \texttt{x0} is the
start of the interval, and \texttt{x1} its end. Obviously, \texttt{x0} should
be smaller than \texttt{x1}. It can also be given as a list with a simple
tuple in the same format. In that case, the same range will be
applied for every variable in the optimization.
          \item[df]


A function to calculate the gradient vector of the cost function
\texttt{f}. Defaults to \texttt{None}, if no gradient is supplied, then it is
estimated from the cost function using Euler equations.
          \item[h]


Convergence step. This method does not takes into consideration the
possibility of varying the convergence step, to avoid Stiefel cages.
          \item[emax]


Maximum allowed error. The algorithm stops as soon as the error is
below this level. The error is absolute.
          \item[imax]


Maximum number of iterations, the algorithm stops as soon this
number of iterations are executed, no matter what the error is at
the moment.
        \end{Ventry}

      \end{quote}

      Overrides: object.\_\_init\_\_

    \end{boxedminipage}

    \label{peach:optm:quasinewton:DFP:restart}
    \index{peach \textit{(package)}!peach.optm \textit{(package)}!peach.optm.quasinewton \textit{(module)}!peach.optm.quasinewton.DFP \textit{(class)}!peach.optm.quasinewton.DFP.restart \textit{(method)}}

    \vspace{0.5ex}

\hspace{.8\funcindent}\begin{boxedminipage}{\funcwidth}

    \raggedright \textbf{restart}(\textit{self}, \textit{x0}, \textit{h}={\tt None})

    \vspace{-1.5ex}

    \rule{\textwidth}{0.5\fboxrule}
\setlength{\parskip}{2ex}

Resets the optimizer, returning to its original state, and allowing to
use a new first estimate.
\setlength{\parskip}{1ex}
      \textbf{Parameters}
      \vspace{-1ex}

      \begin{quote}
        \begin{Ventry}{xx}

          \item[x0]


New estimate of the minimum. Estimates can be given in any format,
but internally they are converted to a one-dimension vector, where
each component corresponds to the estimate of that particular
variable. The vector is computed by flattening the array.
          \item[h]


Convergence step. This method does not takes into consideration the
possibility of varying the convergence step, to avoid Stiefel cages.
        \end{Ventry}

      \end{quote}

    \end{boxedminipage}

    \vspace{0.5ex}

\hspace{.8\funcindent}\begin{boxedminipage}{\funcwidth}

    \raggedright \textbf{step}(\textit{self})

    \vspace{-1.5ex}

    \rule{\textwidth}{0.5\fboxrule}
\setlength{\parskip}{2ex}

One step of the search.

In this method, the result of the step is dependent of parameters
calculated before (namely, the estimate of the inverse hessian), so it
is not recomended that different investigations are used with the same
optimizer in the same cost function.
\setlength{\parskip}{1ex}
      \textbf{Return Value}
    \vspace{-1ex}

      \begin{quote}

This method returns a tuple \texttt{(x, e)}, where \texttt{x} is the updated
estimate of the minimum, and \texttt{e} is the estimated error.
      \end{quote}

      Overrides: peach.optm.base.Optimizer.step

    \end{boxedminipage}

    \vspace{0.5ex}

\hspace{.8\funcindent}\begin{boxedminipage}{\funcwidth}

    \raggedright \textbf{\_\_call\_\_}(\textit{self})

    \vspace{-1.5ex}

    \rule{\textwidth}{0.5\fboxrule}
\setlength{\parskip}{2ex}

Transparently executes the search until the minimum is found. The stop
criteria are the maximum error or the maximum number of iterations,
whichever is reached first. Note that this is a \texttt{\_\_call\_\_} method, so
the object is called as a function. This method returns a tuple
\texttt{(x, e)}, with the best estimate of the minimum and the error.
\setlength{\parskip}{1ex}
      \textbf{Return Value}
    \vspace{-1ex}

      \begin{quote}

This method returns a tuple \texttt{(x, e)}, where \texttt{x} is the best
estimate of the minimum, and \texttt{e} is the estimated error.
      \end{quote}

      Overrides: peach.optm.base.Optimizer.\_\_call\_\_

    \end{boxedminipage}


\large{\textbf{\textit{Inherited from object}}}

\begin{quote}
\_\_delattr\_\_(), \_\_format\_\_(), \_\_getattribute\_\_(), \_\_hash\_\_(), \_\_new\_\_(), \_\_reduce\_\_(), \_\_reduce\_ex\_\_(), \_\_repr\_\_(), \_\_setattr\_\_(), \_\_sizeof\_\_(), \_\_str\_\_(), \_\_subclasshook\_\_()
\end{quote}

%%%%%%%%%%%%%%%%%%%%%%%%%%%%%%%%%%%%%%%%%%%%%%%%%%%%%%%%%%%%%%%%%%%%%%%%%%%
%%                              Properties                               %%
%%%%%%%%%%%%%%%%%%%%%%%%%%%%%%%%%%%%%%%%%%%%%%%%%%%%%%%%%%%%%%%%%%%%%%%%%%%

  \subsubsection{Properties}

    \vspace{-1cm}
\hspace{\varindent}\begin{longtable}{|p{\varnamewidth}|p{\vardescrwidth}|l}
\cline{1-2}
\cline{1-2} \centering \textbf{Name} & \centering \textbf{Description}& \\
\cline{1-2}
\endhead\cline{1-2}\multicolumn{3}{r}{\small\textit{continued on next page}}\\\endfoot\cline{1-2}
\endlastfoot\raggedright x\- & &\\
\cline{1-2}
\multicolumn{2}{|l|}{\textit{Inherited from object}}\\
\multicolumn{2}{|p{\varwidth}|}{\raggedright \_\_class\_\_}\\
\cline{1-2}
\end{longtable}


%%%%%%%%%%%%%%%%%%%%%%%%%%%%%%%%%%%%%%%%%%%%%%%%%%%%%%%%%%%%%%%%%%%%%%%%%%%
%%                          Instance Variables                           %%
%%%%%%%%%%%%%%%%%%%%%%%%%%%%%%%%%%%%%%%%%%%%%%%%%%%%%%%%%%%%%%%%%%%%%%%%%%%

  \subsubsection{Instance Variables}

    \vspace{-1cm}
\hspace{\varindent}\begin{longtable}{|p{\varnamewidth}|p{\vardescrwidth}|l}
\cline{1-2}
\cline{1-2} \centering \textbf{Name} & \centering \textbf{Description}& \\
\cline{1-2}
\endhead\cline{1-2}\multicolumn{3}{r}{\small\textit{continued on next page}}\\\endfoot\cline{1-2}
\endlastfoot\raggedright r\-a\-n\-g\-e\-s\- & Holds the ranges for every variable. Although it is a writable
property, care should be taken in changing parameters before ending the
convergence.&\\
\cline{1-2}
\end{longtable}

    \index{peach \textit{(package)}!peach.optm \textit{(package)}!peach.optm.quasinewton \textit{(module)}!peach.optm.quasinewton.DFP \textit{(class)}|)}

%%%%%%%%%%%%%%%%%%%%%%%%%%%%%%%%%%%%%%%%%%%%%%%%%%%%%%%%%%%%%%%%%%%%%%%%%%%
%%                           Class Description                           %%
%%%%%%%%%%%%%%%%%%%%%%%%%%%%%%%%%%%%%%%%%%%%%%%%%%%%%%%%%%%%%%%%%%%%%%%%%%%

    \index{peach \textit{(package)}!peach.optm \textit{(package)}!peach.optm.quasinewton \textit{(module)}!peach.optm.quasinewton.BFGS \textit{(class)}|(}
\subsection{Class BFGS}

    \label{peach:optm:quasinewton:BFGS}
\begin{tabular}{cccccccc}
% Line for object, linespec=[False, False]
\multicolumn{2}{r}{\settowidth{\BCL}{object}\multirow{2}{\BCL}{object}}
&&
&&
  \\\cline{3-3}
  &&\multicolumn{1}{c|}{}
&&
&&
  \\
% Line for peach.optm.base.Optimizer, linespec=[False]
\multicolumn{4}{r}{\settowidth{\BCL}{peach.optm.base.Optimizer}\multirow{2}{\BCL}{peach.optm.base.Optimizer}}
&&
  \\\cline{5-5}
  &&&&\multicolumn{1}{c|}{}
&&
  \\
&&&&\multicolumn{2}{l}{\textbf{peach.optm.quasinewton.BFGS}}
\end{tabular}


BFGS (\emph{Broyden-Fletcher-Goldfarb-Shanno}) search

%%%%%%%%%%%%%%%%%%%%%%%%%%%%%%%%%%%%%%%%%%%%%%%%%%%%%%%%%%%%%%%%%%%%%%%%%%%
%%                                Methods                                %%
%%%%%%%%%%%%%%%%%%%%%%%%%%%%%%%%%%%%%%%%%%%%%%%%%%%%%%%%%%%%%%%%%%%%%%%%%%%

  \subsubsection{Methods}

    \vspace{0.5ex}

\hspace{.8\funcindent}\begin{boxedminipage}{\funcwidth}

    \raggedright \textbf{\_\_init\_\_}(\textit{self}, \textit{f}, \textit{x0}, \textit{ranges}={\tt None}, \textit{df}={\tt None}, \textit{h}={\tt 0.1}, \textit{emax}={\tt 1e-05}, \textit{imax}={\tt 1000})

    \vspace{-1.5ex}

    \rule{\textwidth}{0.5\fboxrule}
\setlength{\parskip}{2ex}

Initializes the optimizer.

To create an optimizer of this type, instantiate the class with the
parameters given below:
\setlength{\parskip}{1ex}
      \textbf{Parameters}
      \vspace{-1ex}

      \begin{quote}
        \begin{Ventry}{xxxxxx}

          \item[f]


A multivariable function to be optimized. The function should have
only one parameter, a multidimensional line-vector, and return the
function value, a scalar.
          \item[x0]


First estimate of the minimum. Estimates can be given in any format,
but internally they are converted to a one-dimension vector, where
each component corresponds to the estimate of that particular
variable. The vector is computed by flattening the array.
          \item[ranges]


A range of values might be passed to the algorithm, but it is not
necessary. If supplied, this parameter should be a list of ranges
for each variable of the objective function. It is specified as a
list of tuples of two values, \texttt{(x0, x1)}, where \texttt{x0} is the
start of the interval, and \texttt{x1} its end. Obviously, \texttt{x0} should
be smaller than \texttt{x1}. It can also be given as a list with a simple
tuple in the same format. In that case, the same range will be
applied for every variable in the optimization.
          \item[df]


A function to calculate the gradient vector of the cost function
\texttt{f}. Defaults to \texttt{None}, if no gradient is supplied, then it is
estimated from the cost function using Euler equations.
          \item[h]


Convergence step. This method does not takes into consideration the
possibility of varying the convergence step, to avoid Stiefel cages.
          \item[emax]


Maximum allowed error. The algorithm stops as soon as the error is
below this level. The error is absolute.
          \item[imax]


Maximum number of iterations, the algorithm stops as soon this
number of iterations are executed, no matter what the error is at
the moment.
        \end{Ventry}

      \end{quote}

      Overrides: object.\_\_init\_\_

    \end{boxedminipage}

    \label{peach:optm:quasinewton:BFGS:restart}
    \index{peach \textit{(package)}!peach.optm \textit{(package)}!peach.optm.quasinewton \textit{(module)}!peach.optm.quasinewton.BFGS \textit{(class)}!peach.optm.quasinewton.BFGS.restart \textit{(method)}}

    \vspace{0.5ex}

\hspace{.8\funcindent}\begin{boxedminipage}{\funcwidth}

    \raggedright \textbf{restart}(\textit{self}, \textit{x0}, \textit{h}={\tt None})

    \vspace{-1.5ex}

    \rule{\textwidth}{0.5\fboxrule}
\setlength{\parskip}{2ex}

Resets the optimizer, returning to its original state, and allowing to
use a new first estimate.
\setlength{\parskip}{1ex}
      \textbf{Parameters}
      \vspace{-1ex}

      \begin{quote}
        \begin{Ventry}{xx}

          \item[x0]


New estimate of the minimum. Estimates can be given in any format,
but internally they are converted to a one-dimension vector, where
each component corresponds to the estimate of that particular
variable. The vector is computed by flattening the array.
          \item[h]


Convergence step. This method does not takes into consideration the
possibility of varying the convergence step, to avoid Stiefel cages.
        \end{Ventry}

      \end{quote}

    \end{boxedminipage}

    \vspace{0.5ex}

\hspace{.8\funcindent}\begin{boxedminipage}{\funcwidth}

    \raggedright \textbf{step}(\textit{self})

    \vspace{-1.5ex}

    \rule{\textwidth}{0.5\fboxrule}
\setlength{\parskip}{2ex}

One step of the search.

In this method, the result of the step is dependent of parameters
calculated before (namely, the estimate of the inverse hessian), so it
is not recomended that different investigations are used with the same
optimizer in the same cost function.
\setlength{\parskip}{1ex}
      \textbf{Return Value}
    \vspace{-1ex}

      \begin{quote}

This method returns a tuple \texttt{(x, e)}, where \texttt{x} is the updated
estimate of the minimum, and \texttt{e} is the estimated error.
      \end{quote}

      Overrides: peach.optm.base.Optimizer.step

    \end{boxedminipage}

    \vspace{0.5ex}

\hspace{.8\funcindent}\begin{boxedminipage}{\funcwidth}

    \raggedright \textbf{\_\_call\_\_}(\textit{self})

    \vspace{-1.5ex}

    \rule{\textwidth}{0.5\fboxrule}
\setlength{\parskip}{2ex}

Transparently executes the search until the minimum is found. The stop
criteria are the maximum error or the maximum number of iterations,
whichever is reached first. Note that this is a \texttt{\_\_call\_\_} method, so
the object is called as a function. This method returns a tuple
\texttt{(x, e)}, with the best estimate of the minimum and the error.
\setlength{\parskip}{1ex}
      \textbf{Return Value}
    \vspace{-1ex}

      \begin{quote}

This method returns a tuple \texttt{(x, e)}, where \texttt{x} is the best
estimate of the minimum, and \texttt{e} is the estimated error.
      \end{quote}

      Overrides: peach.optm.base.Optimizer.\_\_call\_\_

    \end{boxedminipage}


\large{\textbf{\textit{Inherited from object}}}

\begin{quote}
\_\_delattr\_\_(), \_\_format\_\_(), \_\_getattribute\_\_(), \_\_hash\_\_(), \_\_new\_\_(), \_\_reduce\_\_(), \_\_reduce\_ex\_\_(), \_\_repr\_\_(), \_\_setattr\_\_(), \_\_sizeof\_\_(), \_\_str\_\_(), \_\_subclasshook\_\_()
\end{quote}

%%%%%%%%%%%%%%%%%%%%%%%%%%%%%%%%%%%%%%%%%%%%%%%%%%%%%%%%%%%%%%%%%%%%%%%%%%%
%%                              Properties                               %%
%%%%%%%%%%%%%%%%%%%%%%%%%%%%%%%%%%%%%%%%%%%%%%%%%%%%%%%%%%%%%%%%%%%%%%%%%%%

  \subsubsection{Properties}

    \vspace{-1cm}
\hspace{\varindent}\begin{longtable}{|p{\varnamewidth}|p{\vardescrwidth}|l}
\cline{1-2}
\cline{1-2} \centering \textbf{Name} & \centering \textbf{Description}& \\
\cline{1-2}
\endhead\cline{1-2}\multicolumn{3}{r}{\small\textit{continued on next page}}\\\endfoot\cline{1-2}
\endlastfoot\multicolumn{2}{|l|}{\textit{Inherited from object}}\\
\multicolumn{2}{|p{\varwidth}|}{\raggedright \_\_class\_\_}\\
\cline{1-2}
\end{longtable}


%%%%%%%%%%%%%%%%%%%%%%%%%%%%%%%%%%%%%%%%%%%%%%%%%%%%%%%%%%%%%%%%%%%%%%%%%%%
%%                          Instance Variables                           %%
%%%%%%%%%%%%%%%%%%%%%%%%%%%%%%%%%%%%%%%%%%%%%%%%%%%%%%%%%%%%%%%%%%%%%%%%%%%

  \subsubsection{Instance Variables}

    \vspace{-1cm}
\hspace{\varindent}\begin{longtable}{|p{\varnamewidth}|p{\vardescrwidth}|l}
\cline{1-2}
\cline{1-2} \centering \textbf{Name} & \centering \textbf{Description}& \\
\cline{1-2}
\endhead\cline{1-2}\multicolumn{3}{r}{\small\textit{continued on next page}}\\\endfoot\cline{1-2}
\endlastfoot\raggedright r\-a\-n\-g\-e\-s\- & Holds the ranges for every variable. Although it is a writable
property, care should be taken in changing parameters before ending the
convergence.&\\
\cline{1-2}
\end{longtable}

    \index{peach \textit{(package)}!peach.optm \textit{(package)}!peach.optm.quasinewton \textit{(module)}!peach.optm.quasinewton.BFGS \textit{(class)}|)}

%%%%%%%%%%%%%%%%%%%%%%%%%%%%%%%%%%%%%%%%%%%%%%%%%%%%%%%%%%%%%%%%%%%%%%%%%%%
%%                           Class Description                           %%
%%%%%%%%%%%%%%%%%%%%%%%%%%%%%%%%%%%%%%%%%%%%%%%%%%%%%%%%%%%%%%%%%%%%%%%%%%%

    \index{peach \textit{(package)}!peach.optm \textit{(package)}!peach.optm.quasinewton \textit{(module)}!peach.optm.quasinewton.SR1 \textit{(class)}|(}
\subsection{Class SR1}

    \label{peach:optm:quasinewton:SR1}
\begin{tabular}{cccccccc}
% Line for object, linespec=[False, False]
\multicolumn{2}{r}{\settowidth{\BCL}{object}\multirow{2}{\BCL}{object}}
&&
&&
  \\\cline{3-3}
  &&\multicolumn{1}{c|}{}
&&
&&
  \\
% Line for peach.optm.base.Optimizer, linespec=[False]
\multicolumn{4}{r}{\settowidth{\BCL}{peach.optm.base.Optimizer}\multirow{2}{\BCL}{peach.optm.base.Optimizer}}
&&
  \\\cline{5-5}
  &&&&\multicolumn{1}{c|}{}
&&
  \\
&&&&\multicolumn{2}{l}{\textbf{peach.optm.quasinewton.SR1}}
\end{tabular}


SR1 (\emph{Symmetric Rank 1} ) search method

%%%%%%%%%%%%%%%%%%%%%%%%%%%%%%%%%%%%%%%%%%%%%%%%%%%%%%%%%%%%%%%%%%%%%%%%%%%
%%                                Methods                                %%
%%%%%%%%%%%%%%%%%%%%%%%%%%%%%%%%%%%%%%%%%%%%%%%%%%%%%%%%%%%%%%%%%%%%%%%%%%%

  \subsubsection{Methods}

    \vspace{0.5ex}

\hspace{.8\funcindent}\begin{boxedminipage}{\funcwidth}

    \raggedright \textbf{\_\_init\_\_}(\textit{self}, \textit{f}, \textit{x0}, \textit{ranges}={\tt None}, \textit{df}={\tt None}, \textit{h}={\tt 0.1}, \textit{emax}={\tt 1e-05}, \textit{imax}={\tt 1000})

    \vspace{-1.5ex}

    \rule{\textwidth}{0.5\fboxrule}
\setlength{\parskip}{2ex}

Initializes the optimizer.

To create an optimizer of this type, instantiate the class with the
parameters given below:
\setlength{\parskip}{1ex}
      \textbf{Parameters}
      \vspace{-1ex}

      \begin{quote}
        \begin{Ventry}{xxxxxx}

          \item[f]


A multivariable function to be optimized. The function should have
only one parameter, a multidimensional line-vector, and return the
function value, a scalar.
          \item[x0]


First estimate of the minimum. Estimates can be given in any format,
but internally they are converted to a one-dimension vector, where
each component corresponds to the estimate of that particular
variable. The vector is computed by flattening the array.
          \item[ranges]


A range of values might be passed to the algorithm, but it is not
necessary. If supplied, this parameter should be a list of ranges
for each variable of the objective function. It is specified as a
list of tuples of two values, \texttt{(x0, x1)}, where \texttt{x0} is the
start of the interval, and \texttt{x1} its end. Obviously, \texttt{x0} should
be smaller than \texttt{x1}. It can also be given as a list with a simple
tuple in the same format. In that case, the same range will be
applied for every variable in the optimization.
          \item[df]


A function to calculate the gradient vector of the cost function
\texttt{f}. Defaults to \texttt{None}, if no gradient is supplied, then it is
estimated from the cost function using Euler equations.
          \item[h]


Convergence step. This method does not takes into consideration the
possibility of varying the convergence step, to avoid Stiefel cages.
          \item[emax]


Maximum allowed error. The algorithm stops as soon as the error is
below this level. The error is absolute.
          \item[imax]


Maximum number of iterations, the algorithm stops as soon this
number of iterations are executed, no matter what the error is at
the moment.
        \end{Ventry}

      \end{quote}

      Overrides: object.\_\_init\_\_

    \end{boxedminipage}

    \label{peach:optm:quasinewton:SR1:restart}
    \index{peach \textit{(package)}!peach.optm \textit{(package)}!peach.optm.quasinewton \textit{(module)}!peach.optm.quasinewton.SR1 \textit{(class)}!peach.optm.quasinewton.SR1.restart \textit{(method)}}

    \vspace{0.5ex}

\hspace{.8\funcindent}\begin{boxedminipage}{\funcwidth}

    \raggedright \textbf{restart}(\textit{self}, \textit{x0}, \textit{h}={\tt None})

    \vspace{-1.5ex}

    \rule{\textwidth}{0.5\fboxrule}
\setlength{\parskip}{2ex}

Resets the optimizer, returning to its original state, and allowing to
use a new first estimate.
\setlength{\parskip}{1ex}
      \textbf{Parameters}
      \vspace{-1ex}

      \begin{quote}
        \begin{Ventry}{xx}

          \item[x0]


New estimate of the minimum. Estimates can be given in any format,
but internally they are converted to a one-dimension vector, where
each component corresponds to the estimate of that particular
variable. The vector is computed by flattening the array.
          \item[h]


Convergence step. This method does not takes into consideration the
possibility of varying the convergence step, to avoid Stiefel cages.
        \end{Ventry}

      \end{quote}

    \end{boxedminipage}

    \vspace{0.5ex}

\hspace{.8\funcindent}\begin{boxedminipage}{\funcwidth}

    \raggedright \textbf{step}(\textit{self})

    \vspace{-1.5ex}

    \rule{\textwidth}{0.5\fboxrule}
\setlength{\parskip}{2ex}

One step of the search.

In this method, the result of the step is dependent of parameters
calculated before (namely, the estimate of the inverse hessian), so it
is not recomended that different investigations are used with the same
optimizer in the same cost function.
\setlength{\parskip}{1ex}
      \textbf{Return Value}
    \vspace{-1ex}

      \begin{quote}

This method returns a tuple \texttt{(x, e)}, where \texttt{x} is the updated
estimate of the minimum, and \texttt{e} is the estimated error.
      \end{quote}

      Overrides: peach.optm.base.Optimizer.step

    \end{boxedminipage}

    \vspace{0.5ex}

\hspace{.8\funcindent}\begin{boxedminipage}{\funcwidth}

    \raggedright \textbf{\_\_call\_\_}(\textit{self})

    \vspace{-1.5ex}

    \rule{\textwidth}{0.5\fboxrule}
\setlength{\parskip}{2ex}

Transparently executes the search until the minimum is found. The stop
criteria are the maximum error or the maximum number of iterations,
whichever is reached first. Note that this is a \texttt{\_\_call\_\_} method, so
the object is called as a function. This method returns a tuple
\texttt{(x, e)}, with the best estimate of the minimum and the error.
\setlength{\parskip}{1ex}
      \textbf{Return Value}
    \vspace{-1ex}

      \begin{quote}

This method returns a tuple \texttt{(x, e)}, where \texttt{x} is the best
estimate of the minimum, and \texttt{e} is the estimated error.
      \end{quote}

      Overrides: peach.optm.base.Optimizer.\_\_call\_\_

    \end{boxedminipage}


\large{\textbf{\textit{Inherited from object}}}

\begin{quote}
\_\_delattr\_\_(), \_\_format\_\_(), \_\_getattribute\_\_(), \_\_hash\_\_(), \_\_new\_\_(), \_\_reduce\_\_(), \_\_reduce\_ex\_\_(), \_\_repr\_\_(), \_\_setattr\_\_(), \_\_sizeof\_\_(), \_\_str\_\_(), \_\_subclasshook\_\_()
\end{quote}

%%%%%%%%%%%%%%%%%%%%%%%%%%%%%%%%%%%%%%%%%%%%%%%%%%%%%%%%%%%%%%%%%%%%%%%%%%%
%%                              Properties                               %%
%%%%%%%%%%%%%%%%%%%%%%%%%%%%%%%%%%%%%%%%%%%%%%%%%%%%%%%%%%%%%%%%%%%%%%%%%%%

  \subsubsection{Properties}

    \vspace{-1cm}
\hspace{\varindent}\begin{longtable}{|p{\varnamewidth}|p{\vardescrwidth}|l}
\cline{1-2}
\cline{1-2} \centering \textbf{Name} & \centering \textbf{Description}& \\
\cline{1-2}
\endhead\cline{1-2}\multicolumn{3}{r}{\small\textit{continued on next page}}\\\endfoot\cline{1-2}
\endlastfoot\raggedright x\- & &\\
\cline{1-2}
\multicolumn{2}{|l|}{\textit{Inherited from object}}\\
\multicolumn{2}{|p{\varwidth}|}{\raggedright \_\_class\_\_}\\
\cline{1-2}
\end{longtable}


%%%%%%%%%%%%%%%%%%%%%%%%%%%%%%%%%%%%%%%%%%%%%%%%%%%%%%%%%%%%%%%%%%%%%%%%%%%
%%                          Instance Variables                           %%
%%%%%%%%%%%%%%%%%%%%%%%%%%%%%%%%%%%%%%%%%%%%%%%%%%%%%%%%%%%%%%%%%%%%%%%%%%%

  \subsubsection{Instance Variables}

    \vspace{-1cm}
\hspace{\varindent}\begin{longtable}{|p{\varnamewidth}|p{\vardescrwidth}|l}
\cline{1-2}
\cline{1-2} \centering \textbf{Name} & \centering \textbf{Description}& \\
\cline{1-2}
\endhead\cline{1-2}\multicolumn{3}{r}{\small\textit{continued on next page}}\\\endfoot\cline{1-2}
\endlastfoot\raggedright r\-a\-n\-g\-e\-s\- & Holds the ranges for every variable. Although it is a writable
property, care should be taken in changing parameters before ending the
convergence.&\\
\cline{1-2}
\end{longtable}

    \index{peach \textit{(package)}!peach.optm \textit{(package)}!peach.optm.quasinewton \textit{(module)}!peach.optm.quasinewton.SR1 \textit{(class)}|)}
    \index{peach \textit{(package)}!peach.optm \textit{(package)}!peach.optm.quasinewton \textit{(module)}|)}

%
% API Documentation for Peach - Computational Intelligence for Python
% Module peach.optm.stochastic
%
% Generated by epydoc 3.0.1
% [Mon Jan 24 15:39:52 2011]
%

%%%%%%%%%%%%%%%%%%%%%%%%%%%%%%%%%%%%%%%%%%%%%%%%%%%%%%%%%%%%%%%%%%%%%%%%%%%
%%                          Module Description                           %%
%%%%%%%%%%%%%%%%%%%%%%%%%%%%%%%%%%%%%%%%%%%%%%%%%%%%%%%%%%%%%%%%%%%%%%%%%%%

    \index{peach \textit{(package)}!peach.optm \textit{(package)}!peach.optm.stochastic \textit{(module)}|(}
\section{Module peach.optm.stochastic}

    \label{peach:optm:stochastic}

%%%%%%%%%%%%%%%%%%%%%%%%%%%%%%%%%%%%%%%%%%%%%%%%%%%%%%%%%%%%%%%%%%%%%%%%%%%
%%                               Variables                               %%
%%%%%%%%%%%%%%%%%%%%%%%%%%%%%%%%%%%%%%%%%%%%%%%%%%%%%%%%%%%%%%%%%%%%%%%%%%%

  \subsection{Variables}

    \vspace{-1cm}
\hspace{\varindent}\begin{longtable}{|p{\varnamewidth}|p{\vardescrwidth}|l}
\cline{1-2}
\cline{1-2} \centering \textbf{Name} & \centering \textbf{Description}& \\
\cline{1-2}
\endhead\cline{1-2}\multicolumn{3}{r}{\small\textit{continued on next page}}\\\endfoot\cline{1-2}
\endlastfoot\raggedright \_\-\_\-d\-o\-c\-\_\-\_\- & \raggedright \textbf{Value:} 
{\tt \texttt{...}}&\\
\cline{1-2}
\end{longtable}


%%%%%%%%%%%%%%%%%%%%%%%%%%%%%%%%%%%%%%%%%%%%%%%%%%%%%%%%%%%%%%%%%%%%%%%%%%%
%%                           Class Description                           %%
%%%%%%%%%%%%%%%%%%%%%%%%%%%%%%%%%%%%%%%%%%%%%%%%%%%%%%%%%%%%%%%%%%%%%%%%%%%

    \index{peach \textit{(package)}!peach.optm \textit{(package)}!peach.optm.stochastic \textit{(module)}!peach.optm.stochastic.CrossEntropy \textit{(class)}|(}
\subsection{Class CrossEntropy}

    \label{peach:optm:stochastic:CrossEntropy}
\begin{tabular}{cccccc}
% Line for peach.optm.Optimizer, linespec=[False]
\multicolumn{2}{r}{\settowidth{\BCL}{peach.optm.Optimizer}\multirow{2}{\BCL}{peach.optm.Optimizer}}
&&
  \\\cline{3-3}
  &&\multicolumn{1}{c|}{}
&&
  \\
&&\multicolumn{2}{l}{\textbf{peach.optm.stochastic.CrossEntropy}}
\end{tabular}


Multidimensional search based on cross-entropy technique.

In cross-entropy, a set of N possible solutions is randomly generated at
each interaction. To converge the solutions, the best M solutions are
selected and its statistics are calculated. A new set of solutions are
randomly generated from these statistics.

%%%%%%%%%%%%%%%%%%%%%%%%%%%%%%%%%%%%%%%%%%%%%%%%%%%%%%%%%%%%%%%%%%%%%%%%%%%
%%                                Methods                                %%
%%%%%%%%%%%%%%%%%%%%%%%%%%%%%%%%%%%%%%%%%%%%%%%%%%%%%%%%%%%%%%%%%%%%%%%%%%%

  \subsubsection{Methods}

    \label{peach:optm:stochastic:CrossEntropy:__init__}
    \index{peach \textit{(package)}!peach.optm \textit{(package)}!peach.optm.stochastic \textit{(module)}!peach.optm.stochastic.CrossEntropy \textit{(class)}!peach.optm.stochastic.CrossEntropy.\_\_init\_\_ \textit{(method)}}

    \vspace{0.5ex}

\hspace{.8\funcindent}\begin{boxedminipage}{\funcwidth}

    \raggedright \textbf{\_\_init\_\_}(\textit{self}, \textit{f}, \textit{M}={\tt 30}, \textit{N}={\tt 60}, \textit{emax}={\tt 1e-8}, \textit{imax}={\tt 1000})

    \vspace{-1.5ex}

    \rule{\textwidth}{0.5\fboxrule}
\setlength{\parskip}{2ex}

Initializes the optimizer.

To create an optimizer of this type, instantiate the class with the
parameters given below:
\setlength{\parskip}{1ex}
      \textbf{Parameters}
      \vspace{-1ex}

      \begin{quote}
        \begin{Ventry}{xxxx}

          \item[f]


A multivariable function to be optimized. The function should have
only one parameter, a multidimensional line-vector, and return the
function value, a scalar.
          \item[M]


Size of the solution set used to calculate the statistics to
generate the next set of solutions
          \item[N]


Total size of the solution set.
          \item[emax]


Maximum allowed error. The algorithm stops as soon as the error is
below this level. The error is absolute.
          \item[imax]


Maximum number of iterations, the algorithm stops as soon this
number of iterations are executed, no matter what the error is at
the moment.
        \end{Ventry}

      \end{quote}

    \end{boxedminipage}

    \label{peach:optm:stochastic:CrossEntropy:step}
    \index{peach \textit{(package)}!peach.optm \textit{(package)}!peach.optm.stochastic \textit{(module)}!peach.optm.stochastic.CrossEntropy \textit{(class)}!peach.optm.stochastic.CrossEntropy.step \textit{(method)}}

    \vspace{0.5ex}

\hspace{.8\funcindent}\begin{boxedminipage}{\funcwidth}

    \raggedright \textbf{step}(\textit{self})

    \vspace{-1.5ex}

    \rule{\textwidth}{0.5\fboxrule}
\setlength{\parskip}{2ex}

One step of the search (\emph{NOT IMPLEMENTED YET})

In this method, the solution set is searched for the M best solutions.
Mean and variance of these solutions is calculated, and these values are
used to randomly generate, from a gaussian distribution, a set of N new
solutions.
\setlength{\parskip}{1ex}
    \end{boxedminipage}

    \index{peach \textit{(package)}!peach.optm \textit{(package)}!peach.optm.stochastic \textit{(module)}!peach.optm.stochastic.CrossEntropy \textit{(class)}|)}
    \index{peach \textit{(package)}!peach.optm \textit{(package)}!peach.optm.stochastic \textit{(module)}|)}

%
% API Documentation for Peach - Computational Intelligence for Python
% Package peach.pso
%
% Generated by epydoc 3.0.1
% [Mon Jan 24 15:39:52 2011]
%

%%%%%%%%%%%%%%%%%%%%%%%%%%%%%%%%%%%%%%%%%%%%%%%%%%%%%%%%%%%%%%%%%%%%%%%%%%%
%%                          Module Description                           %%
%%%%%%%%%%%%%%%%%%%%%%%%%%%%%%%%%%%%%%%%%%%%%%%%%%%%%%%%%%%%%%%%%%%%%%%%%%%

    \index{peach \textit{(package)}!peach.pso \textit{(package)}|(}
\section{Package peach.pso}

    \label{peach:pso}

Basic Particle Swarm Optimization (PSO)

This sub-package implements traditional particle swarm optimizers as described
in literature. It consists of a very simple algorithm emulating the behaviour
of a flock of birds (though in a very simplified way). A population of particles
is created, each particle with its corresponding velocity. They fly towards the
particle local best and the swarm global best, thus exploring the whole domain.

For consistency purposes, the particles are represented internally as a list of
vectors. The particles can be acessed externally by using the \texttt{{[} {]}} interface.
See the rest of the documentation for more information.

%%%%%%%%%%%%%%%%%%%%%%%%%%%%%%%%%%%%%%%%%%%%%%%%%%%%%%%%%%%%%%%%%%%%%%%%%%%
%%                                Modules                                %%
%%%%%%%%%%%%%%%%%%%%%%%%%%%%%%%%%%%%%%%%%%%%%%%%%%%%%%%%%%%%%%%%%%%%%%%%%%%

\subsection{Modules}

\begin{itemize}
\setlength{\parskip}{0ex}
\item \textbf{acc}: 
Functions to update the velocity (ie, accelerate) of the particles in a swarm.


  \textit{(Section \ref{peach:pso:acc}, p.~\pageref{peach:pso:acc})}

\item \textbf{base}: 
This package implements the simple continuous version of the particle swarm
optimizer. In this implementation, it is possible to specify, besides the
objective function and the first estimates, the ranges of search, which will
influence the max velocity of the particles, and the population size. Other
parameters are available too, please refer to the rest of this documentation for
further details.


  \textit{(Section \ref{peach:pso:base}, p.~\pageref{peach:pso:base})}

\end{itemize}

    \index{peach \textit{(package)}!peach.pso \textit{(package)}|)}

%
% API Documentation for Peach - Computational Intelligence for Python
% Module peach.pso.acc
%
% Generated by epydoc 3.0.1
% [Mon Jan 24 15:39:52 2011]
%

%%%%%%%%%%%%%%%%%%%%%%%%%%%%%%%%%%%%%%%%%%%%%%%%%%%%%%%%%%%%%%%%%%%%%%%%%%%
%%                          Module Description                           %%
%%%%%%%%%%%%%%%%%%%%%%%%%%%%%%%%%%%%%%%%%%%%%%%%%%%%%%%%%%%%%%%%%%%%%%%%%%%

    \index{peach \textit{(package)}!peach.pso \textit{(package)}!peach.pso.acc \textit{(module)}|(}
\section{Module peach.pso.acc}

    \label{peach:pso:acc}

Functions to update the velocity (ie, accelerate) of the particles in a swarm.

Acceleration of a particle is an important concept in the theory of particle
swarm optimizers. By choosing an adequate acceleration, particle velocity is
changed so that they can search the domain of definition of the objective
function such that there is a greater probability that a global minimum is
found. Since particle swarm optimizers are derived from genetic algorithms, it
can be said that this is what creates diversity in a swarm, such that the space
is more thoroughly searched.

%%%%%%%%%%%%%%%%%%%%%%%%%%%%%%%%%%%%%%%%%%%%%%%%%%%%%%%%%%%%%%%%%%%%%%%%%%%
%%                               Variables                               %%
%%%%%%%%%%%%%%%%%%%%%%%%%%%%%%%%%%%%%%%%%%%%%%%%%%%%%%%%%%%%%%%%%%%%%%%%%%%

  \subsection{Variables}

    \vspace{-1cm}
\hspace{\varindent}\begin{longtable}{|p{\varnamewidth}|p{\vardescrwidth}|l}
\cline{1-2}
\cline{1-2} \centering \textbf{Name} & \centering \textbf{Description}& \\
\cline{1-2}
\endhead\cline{1-2}\multicolumn{3}{r}{\small\textit{continued on next page}}\\\endfoot\cline{1-2}
\endlastfoot\raggedright \_\-\_\-d\-o\-c\-\_\-\_\- & \raggedright \textbf{Value:} 
{\tt \texttt{...}}&\\
\cline{1-2}
\raggedright \_\-\_\-p\-a\-c\-k\-a\-g\-e\-\_\-\_\- & \raggedright \textbf{Value:} 
{\tt \texttt{'}\texttt{peach.pso}\texttt{'}}&\\
\cline{1-2}
\end{longtable}


%%%%%%%%%%%%%%%%%%%%%%%%%%%%%%%%%%%%%%%%%%%%%%%%%%%%%%%%%%%%%%%%%%%%%%%%%%%
%%                           Class Description                           %%
%%%%%%%%%%%%%%%%%%%%%%%%%%%%%%%%%%%%%%%%%%%%%%%%%%%%%%%%%%%%%%%%%%%%%%%%%%%

    \index{peach \textit{(package)}!peach.pso \textit{(package)}!peach.pso.acc \textit{(module)}!peach.pso.acc.Accelerator \textit{(class)}|(}
\subsection{Class Accelerator}

    \label{peach:pso:acc:Accelerator}
\begin{tabular}{cccccc}
% Line for object, linespec=[False]
\multicolumn{2}{r}{\settowidth{\BCL}{object}\multirow{2}{\BCL}{object}}
&&
  \\\cline{3-3}
  &&\multicolumn{1}{c|}{}
&&
  \\
&&\multicolumn{2}{l}{\textbf{peach.pso.acc.Accelerator}}
\end{tabular}

\textbf{Known Subclasses:} peach.pso.acc.StandardPSO


Base class for accelerators.

This class should be derived to implement a function which computes the
acceleration of a vector of particles in a swarm. Every accelerator function
should implement at least two methods, defined below:
%
\begin{quote}
%
\begin{description}
\item[{\_\_init\_\_(self, %
\raisebox{1em}{\hypertarget{id2}{}}\hyperlink{id1}{\textbf{\color{red}*}}cnf, %
\raisebox{1em}{\hypertarget{id4}{}}\hyperlink{id3}{\textbf{\color{red}**}}kw)}] \leavevmode 
Initializes the object. There are no mandatory arguments, but any
parameters can be used here to configure the operator. For example, a
class can define a variance for randomly chose the acceleration -{}- this
should be defined here:
%
\begin{quote}{\ttfamily \raggedright \noindent
\_\_init\_\_(self,~variance=1.0)
}
\end{quote}

A default value should always be offered, if possible.

\item[{\_\_call\_\_(self, v):}] \leavevmode 
The \texttt{\_\_call\_\_} interface should be programmed to actually compute the
new velocity of a vector of particles. This method should receive a
velocity in \texttt{v} and use whatever parameters from the instantiation to
compute the new velocities. Notice that this function should operate
over a vector of velocities, not on a single velocity. This class,
however, can be instantiated with a single function that is adapted to
perform over a vector.

\end{description}

\end{quote}

%%%%%%%%%%%%%%%%%%%%%%%%%%%%%%%%%%%%%%%%%%%%%%%%%%%%%%%%%%%%%%%%%%%%%%%%%%%
%%                                Methods                                %%
%%%%%%%%%%%%%%%%%%%%%%%%%%%%%%%%%%%%%%%%%%%%%%%%%%%%%%%%%%%%%%%%%%%%%%%%%%%

  \subsubsection{Methods}

    \vspace{0.5ex}

\hspace{.8\funcindent}\begin{boxedminipage}{\funcwidth}

    \raggedright \textbf{\_\_init\_\_}(\textit{self}, \textit{f})

    \vspace{-1.5ex}

    \rule{\textwidth}{0.5\fboxrule}
\setlength{\parskip}{2ex}

Initializes an accelerator object.

This method initializes an accelerator. It receives as argument a simple
function that is adapted to operate over a vector of velocities.
\setlength{\parskip}{1ex}
      \textbf{Parameters}
      \vspace{-1ex}

      \begin{quote}
        \begin{Ventry}{x}

          \item[f]


The function to be used as acceleration. This function can be simple
function that receives a \texttt{n}-dimensional vector representing the
velocity of a single particle, where \texttt{n} is the dimensionality of
the objective function. The object then wraps the function such that
it can receive a list of velocities and applies the acceleration on
every one of them.
        \end{Ventry}

      \end{quote}

      Overrides: object.\_\_init\_\_

    \end{boxedminipage}

    \label{peach:pso:acc:Accelerator:__call__}
    \index{peach \textit{(package)}!peach.pso \textit{(package)}!peach.pso.acc \textit{(module)}!peach.pso.acc.Accelerator \textit{(class)}!peach.pso.acc.Accelerator.\_\_call\_\_ \textit{(method)}}

    \vspace{0.5ex}

\hspace{.8\funcindent}\begin{boxedminipage}{\funcwidth}

    \raggedright \textbf{\_\_call\_\_}(\textit{self}, \textit{v})

    \vspace{-1.5ex}

    \rule{\textwidth}{0.5\fboxrule}
\setlength{\parskip}{2ex}

Computes new velocities for every particle.

This method should be overloaded in implementations of different
accelerators. This method receives the velocities as a list or a vector
of the velocities (a \texttt{n}-dimensional vector in each line) or each
particle in a swarm and computes, for each one of them, a new velocity.
\setlength{\parskip}{1ex}
      \textbf{Parameters}
      \vspace{-1ex}

      \begin{quote}
        \begin{Ventry}{x}

          \item[v]


A list or a vector of velocities, where each velocity is one line of
the vector or one element of the list.
        \end{Ventry}

      \end{quote}

      \textbf{Return Value}
    \vspace{-1ex}

      \begin{quote}

A vector of the same size as the argument with the updated velocities.
The returned vector is returned as a bidimensional array.
      \end{quote}

    \end{boxedminipage}


\large{\textbf{\textit{Inherited from object}}}

\begin{quote}
\_\_delattr\_\_(), \_\_format\_\_(), \_\_getattribute\_\_(), \_\_hash\_\_(), \_\_new\_\_(), \_\_reduce\_\_(), \_\_reduce\_ex\_\_(), \_\_repr\_\_(), \_\_setattr\_\_(), \_\_sizeof\_\_(), \_\_str\_\_(), \_\_subclasshook\_\_()
\end{quote}

%%%%%%%%%%%%%%%%%%%%%%%%%%%%%%%%%%%%%%%%%%%%%%%%%%%%%%%%%%%%%%%%%%%%%%%%%%%
%%                              Properties                               %%
%%%%%%%%%%%%%%%%%%%%%%%%%%%%%%%%%%%%%%%%%%%%%%%%%%%%%%%%%%%%%%%%%%%%%%%%%%%

  \subsubsection{Properties}

    \vspace{-1cm}
\hspace{\varindent}\begin{longtable}{|p{\varnamewidth}|p{\vardescrwidth}|l}
\cline{1-2}
\cline{1-2} \centering \textbf{Name} & \centering \textbf{Description}& \\
\cline{1-2}
\endhead\cline{1-2}\multicolumn{3}{r}{\small\textit{continued on next page}}\\\endfoot\cline{1-2}
\endlastfoot\multicolumn{2}{|l|}{\textit{Inherited from object}}\\
\multicolumn{2}{|p{\varwidth}|}{\raggedright \_\_class\_\_}\\
\cline{1-2}
\end{longtable}

    \index{peach \textit{(package)}!peach.pso \textit{(package)}!peach.pso.acc \textit{(module)}!peach.pso.acc.Accelerator \textit{(class)}|)}

%%%%%%%%%%%%%%%%%%%%%%%%%%%%%%%%%%%%%%%%%%%%%%%%%%%%%%%%%%%%%%%%%%%%%%%%%%%
%%                           Class Description                           %%
%%%%%%%%%%%%%%%%%%%%%%%%%%%%%%%%%%%%%%%%%%%%%%%%%%%%%%%%%%%%%%%%%%%%%%%%%%%

    \index{peach \textit{(package)}!peach.pso \textit{(package)}!peach.pso.acc \textit{(module)}!peach.pso.acc.StandardPSO \textit{(class)}|(}
\subsection{Class StandardPSO}

    \label{peach:pso:acc:StandardPSO}
\begin{tabular}{cccccccc}
% Line for object, linespec=[False, False]
\multicolumn{2}{r}{\settowidth{\BCL}{object}\multirow{2}{\BCL}{object}}
&&
&&
  \\\cline{3-3}
  &&\multicolumn{1}{c|}{}
&&
&&
  \\
% Line for peach.pso.acc.Accelerator, linespec=[False]
\multicolumn{4}{r}{\settowidth{\BCL}{peach.pso.acc.Accelerator}\multirow{2}{\BCL}{peach.pso.acc.Accelerator}}
&&
  \\\cline{5-5}
  &&&&\multicolumn{1}{c|}{}
&&
  \\
&&&&\multicolumn{2}{l}{\textbf{peach.pso.acc.StandardPSO}}
\end{tabular}


Standard PSO Accelerator

This class implements a method for changing the velocities of particles in
a particle swarm. The standard way is to retain information on local bests
and the global bests, and update the velocity based on that.

%%%%%%%%%%%%%%%%%%%%%%%%%%%%%%%%%%%%%%%%%%%%%%%%%%%%%%%%%%%%%%%%%%%%%%%%%%%
%%                                Methods                                %%
%%%%%%%%%%%%%%%%%%%%%%%%%%%%%%%%%%%%%%%%%%%%%%%%%%%%%%%%%%%%%%%%%%%%%%%%%%%

  \subsubsection{Methods}

    \vspace{0.5ex}

\hspace{.8\funcindent}\begin{boxedminipage}{\funcwidth}

    \raggedright \textbf{\_\_init\_\_}(\textit{self}, \textit{ps}, \textit{vmax}={\tt None}, \textit{cp}={\tt 2.05}, \textit{cg}={\tt 2.05})

    \vspace{-1.5ex}

    \rule{\textwidth}{0.5\fboxrule}
\setlength{\parskip}{2ex}

Initializes the accelerator.
\setlength{\parskip}{1ex}
      \textbf{Parameters}
      \vspace{-1ex}

      \begin{quote}
        \begin{Ventry}{xx}

          \item[ps]


A reference to the Particle Swarm that should be updated. This
class, in instantiation, will assume that the position of the
particles in the moment of creation are the local best. The
objective function is computed for all particles, and the values
saved for reference in the future. Also, at the same time, the
global best is computed.
          \item[cp]


The velocity adjustment constant associated with the particle best
values. Defaults to 2.05.
          \item[cg]


The velocity adjustment constant associated with the global best
values. Defaults to 2.05. The defaults in the \texttt{cp} and \texttt{cg}
parameters are such that the inertia weight in the constrition
method satisfies \texttt{cp + cg > 4}. Please, look in the bibliography
for more information.
        \end{Ventry}

      \end{quote}

      Overrides: object.\_\_init\_\_

    \end{boxedminipage}

    \vspace{0.5ex}

\hspace{.8\funcindent}\begin{boxedminipage}{\funcwidth}

    \raggedright \textbf{\_\_call\_\_}(\textit{self}, \textit{v})

    \vspace{-1.5ex}

    \rule{\textwidth}{0.5\fboxrule}
\setlength{\parskip}{2ex}

Computes the new velocities for every particle in the swarm. This method
receives the velocities as a list or a vector of the velocities (a
\texttt{n}-dimensional vector in each line) or each particle in a swarm and
computes, for each one of them, a new velocity.
\setlength{\parskip}{1ex}
      \textbf{Parameters}
      \vspace{-1ex}

      \begin{quote}
        \begin{Ventry}{x}

          \item[v]


A list or a vector of velocities, where each velocity is one line of
the vector or one element of the list.
        \end{Ventry}

      \end{quote}

      \textbf{Return Value}
    \vspace{-1ex}

      \begin{quote}

A vector of the same size as the argument with the updated velocities.
The returned vector is returned as a bidimensional array.
      \end{quote}

      Overrides: peach.pso.acc.Accelerator.\_\_call\_\_

    \end{boxedminipage}


\large{\textbf{\textit{Inherited from object}}}

\begin{quote}
\_\_delattr\_\_(), \_\_format\_\_(), \_\_getattribute\_\_(), \_\_hash\_\_(), \_\_new\_\_(), \_\_reduce\_\_(), \_\_reduce\_ex\_\_(), \_\_repr\_\_(), \_\_setattr\_\_(), \_\_sizeof\_\_(), \_\_str\_\_(), \_\_subclasshook\_\_()
\end{quote}

%%%%%%%%%%%%%%%%%%%%%%%%%%%%%%%%%%%%%%%%%%%%%%%%%%%%%%%%%%%%%%%%%%%%%%%%%%%
%%                              Properties                               %%
%%%%%%%%%%%%%%%%%%%%%%%%%%%%%%%%%%%%%%%%%%%%%%%%%%%%%%%%%%%%%%%%%%%%%%%%%%%

  \subsubsection{Properties}

    \vspace{-1cm}
\hspace{\varindent}\begin{longtable}{|p{\varnamewidth}|p{\vardescrwidth}|l}
\cline{1-2}
\cline{1-2} \centering \textbf{Name} & \centering \textbf{Description}& \\
\cline{1-2}
\endhead\cline{1-2}\multicolumn{3}{r}{\small\textit{continued on next page}}\\\endfoot\cline{1-2}
\endlastfoot\multicolumn{2}{|l|}{\textit{Inherited from object}}\\
\multicolumn{2}{|p{\varwidth}|}{\raggedright \_\_class\_\_}\\
\cline{1-2}
\end{longtable}


%%%%%%%%%%%%%%%%%%%%%%%%%%%%%%%%%%%%%%%%%%%%%%%%%%%%%%%%%%%%%%%%%%%%%%%%%%%
%%                          Instance Variables                           %%
%%%%%%%%%%%%%%%%%%%%%%%%%%%%%%%%%%%%%%%%%%%%%%%%%%%%%%%%%%%%%%%%%%%%%%%%%%%

  \subsubsection{Instance Variables}

    \vspace{-1cm}
\hspace{\varindent}\begin{longtable}{|p{\varnamewidth}|p{\vardescrwidth}|l}
\cline{1-2}
\cline{1-2} \centering \textbf{Name} & \centering \textbf{Description}& \\
\cline{1-2}
\endhead\cline{1-2}\multicolumn{3}{r}{\small\textit{continued on next page}}\\\endfoot\cline{1-2}
\endlastfoot\raggedright c\-p\- & Velocity adjustment constant associated with the particle best values.&\\
\cline{1-2}
\raggedright c\-g\- & Velocity adjustment constant associated with the global best values.&\\
\cline{1-2}
\end{longtable}

    \index{peach \textit{(package)}!peach.pso \textit{(package)}!peach.pso.acc \textit{(module)}!peach.pso.acc.StandardPSO \textit{(class)}|)}
    \index{peach \textit{(package)}!peach.pso \textit{(package)}!peach.pso.acc \textit{(module)}|)}

%
% API Documentation for Peach - Computational Intelligence for Python
% Module peach.pso.base
%
% Generated by epydoc 3.0.1
% [Thu Jul 28 16:37:50 2011]
%

%%%%%%%%%%%%%%%%%%%%%%%%%%%%%%%%%%%%%%%%%%%%%%%%%%%%%%%%%%%%%%%%%%%%%%%%%%%
%%                          Module Description                           %%
%%%%%%%%%%%%%%%%%%%%%%%%%%%%%%%%%%%%%%%%%%%%%%%%%%%%%%%%%%%%%%%%%%%%%%%%%%%

    \index{peach \textit{(package)}!peach.pso \textit{(package)}!peach.pso.base \textit{(module)}|(}
\section{Module peach.pso.base}

    \label{peach:pso:base}

This package implements the simple continuous version of the particle swarm
optimizer. In this implementation, it is possible to specify, besides the
objective function and the first estimates, the ranges of search, which will
influence the max velocity of the particles, and the population size. Other
parameters are available too, please refer to the rest of this documentation for
further details.

%%%%%%%%%%%%%%%%%%%%%%%%%%%%%%%%%%%%%%%%%%%%%%%%%%%%%%%%%%%%%%%%%%%%%%%%%%%
%%                               Variables                               %%
%%%%%%%%%%%%%%%%%%%%%%%%%%%%%%%%%%%%%%%%%%%%%%%%%%%%%%%%%%%%%%%%%%%%%%%%%%%

  \subsection{Variables}

    \vspace{-1cm}
\hspace{\varindent}\begin{longtable}{|p{\varnamewidth}|p{\vardescrwidth}|l}
\cline{1-2}
\cline{1-2} \centering \textbf{Name} & \centering \textbf{Description}& \\
\cline{1-2}
\endhead\cline{1-2}\multicolumn{3}{r}{\small\textit{continued on next page}}\\\endfoot\cline{1-2}
\endlastfoot\raggedright \_\-\_\-d\-o\-c\-\_\-\_\- & \raggedright \textbf{Value:} 
{\tt \texttt{...}}&\\
\cline{1-2}
\raggedright \_\-\_\-p\-a\-c\-k\-a\-g\-e\-\_\-\_\- & \raggedright \textbf{Value:} 
{\tt \texttt{'}\texttt{peach.pso}\texttt{'}}&\\
\cline{1-2}
\raggedright a\-b\-s\- & \raggedright \textbf{Value:} 
{\tt {\textless}ufunc 'absolute'{\textgreater}}&\\
\cline{1-2}
\raggedright s\-i\-g\-n\- & \raggedright \textbf{Value:} 
{\tt {\textless}ufunc 'sign'{\textgreater}}&\\
\cline{1-2}
\raggedright s\-q\-r\-t\- & \raggedright \textbf{Value:} 
{\tt {\textless}ufunc 'sqrt'{\textgreater}}&\\
\cline{1-2}
\end{longtable}


%%%%%%%%%%%%%%%%%%%%%%%%%%%%%%%%%%%%%%%%%%%%%%%%%%%%%%%%%%%%%%%%%%%%%%%%%%%
%%                           Class Description                           %%
%%%%%%%%%%%%%%%%%%%%%%%%%%%%%%%%%%%%%%%%%%%%%%%%%%%%%%%%%%%%%%%%%%%%%%%%%%%

    \index{peach \textit{(package)}!peach.pso \textit{(package)}!peach.pso.base \textit{(module)}!peach.pso.base.ParticleSwarmOptimizer \textit{(class)}|(}
\subsection{Class ParticleSwarmOptimizer}

    \label{peach:pso:base:ParticleSwarmOptimizer}
\begin{tabular}{cccccccc}
% Line for object, linespec=[False, False]
\multicolumn{2}{r}{\settowidth{\BCL}{object}\multirow{2}{\BCL}{object}}
&&
&&
  \\\cline{3-3}
  &&\multicolumn{1}{c|}{}
&&
&&
  \\
% Line for list, linespec=[False]
\multicolumn{4}{r}{\settowidth{\BCL}{list}\multirow{2}{\BCL}{list}}
&&
  \\\cline{5-5}
  &&&&\multicolumn{1}{c|}{}
&&
  \\
&&&&\multicolumn{2}{l}{\textbf{peach.pso.base.ParticleSwarmOptimizer}}
\end{tabular}

\textbf{Known Subclasses:} peach.pso.base.PSO


A standard Particle Swarm Optimizer

This class implements a particle swarm optimization (PSO) procedure. A
swarm is a list of estimates, and should answer to every \texttt{list} method. A
population of particles is created to travel through the search domain with
a certain velocity. At each point, the objective function is evaluated for
each particle, and the positions are adjusted correspondingly. The velocity
is then modified (ie, the particles are accelerated) towards its 'personal'
best (the best value found by that particle at the moment) and a global best
(the best value found overall at the moment).

%%%%%%%%%%%%%%%%%%%%%%%%%%%%%%%%%%%%%%%%%%%%%%%%%%%%%%%%%%%%%%%%%%%%%%%%%%%
%%                                Methods                                %%
%%%%%%%%%%%%%%%%%%%%%%%%%%%%%%%%%%%%%%%%%%%%%%%%%%%%%%%%%%%%%%%%%%%%%%%%%%%

  \subsubsection{Methods}

    \vspace{0.5ex}

\hspace{.8\funcindent}\begin{boxedminipage}{\funcwidth}

    \raggedright \textbf{\_\_init\_\_}(\textit{self}, \textit{f}, \textit{x0}, \textit{ranges}={\tt None}, \textit{accelerator}={\tt {\textless}class 'peach.pso.acc.StandardPSO'{\textgreater}}, \textit{emax}={\tt 1e-05}, \textit{imax}={\tt 1000})

    \vspace{-1.5ex}

    \rule{\textwidth}{0.5\fboxrule}
\setlength{\parskip}{2ex}

Initializes the optimizer.
\setlength{\parskip}{1ex}
      \textbf{Parameters}
      \vspace{-1ex}

      \begin{quote}
        \begin{Ventry}{xxxxxxxxxxx}

          \item[f]


A multivariable function to be evaluated. It must receive only one
parameter, a multidimensional line-vector with the same dimensions
of the range list (see below) and return a real value, a scalar.
          \item[x0]


A population of first estimates. This is a list, array or tuple of
one-dimension arrays, each one corresponding to an estimate of the
position of the minimum. The population size of the algorithm will
be the same as the number of estimates in this list. Each component
of the vectors in this list are one of the variables in the function
to be optimized.
          \item[ranges]


A range of values might be passed to the algorithm, but it is not
necessary. If this parameter is not supplied, then the ranges will
be computed from the estimates, but be aware that this might not
represent the complete search space. If supplied, this parameter
should be a list of ranges for each variable of the objective
function. It is specified as a list of tuples of two values,
\texttt{(x0, x1)}, where \texttt{x0} is the start of the interval, and \texttt{x1}
its end. Obviously, \texttt{x0} should be smaller than \texttt{x1}. It can
also be given as a list with a simple tuple in the same format. In
that case, the same range will be applied for every variable in the
optimization.
          \item[accelerator]


An acceleration method, please consult the documentation on \texttt{acc}
module. Defaults to StandardPSO, that is, velocities change based on
local and global bests.
          \item[emax]


Maximum allowed error. The algorithm stops as soon as the error is
below this level. The error is absolute.
          \item[imax]


Maximum number of iterations, the algorithm stops as soon this
number of iterations are executed, no matter what the error is at
the moment.
        \end{Ventry}

      \end{quote}

      \textbf{Return Value}
    \vspace{-1ex}

      \begin{quote}

new empty list
      \end{quote}

      Overrides: object.\_\_init\_\_

    \end{boxedminipage}

    \label{peach:pso:base:ParticleSwarmOptimizer:restart}
    \index{peach \textit{(package)}!peach.pso \textit{(package)}!peach.pso.base \textit{(module)}!peach.pso.base.ParticleSwarmOptimizer \textit{(class)}!peach.pso.base.ParticleSwarmOptimizer.restart \textit{(method)}}

    \vspace{0.5ex}

\hspace{.8\funcindent}\begin{boxedminipage}{\funcwidth}

    \raggedright \textbf{restart}(\textit{self}, \textit{x0})

    \vspace{-1.5ex}

    \rule{\textwidth}{0.5\fboxrule}
\setlength{\parskip}{2ex}

Resets the optimizer, allowing the use of a new set of estimates. This
can be used to avoid stagnation
\setlength{\parskip}{1ex}
      \textbf{Parameters}
      \vspace{-1ex}

      \begin{quote}
        \begin{Ventry}{xx}

          \item[x0]


A new set of estimates. It doesn't need to have the same size of the
original swarm, but it must be a list of estimates in the same
format as in the object instantiation. Please, see the documentation
on the instantiation of the class. New velocities will be computed.
        \end{Ventry}

      \end{quote}

    \end{boxedminipage}

    \label{peach:pso:base:ParticleSwarmOptimizer:step}
    \index{peach \textit{(package)}!peach.pso \textit{(package)}!peach.pso.base \textit{(module)}!peach.pso.base.ParticleSwarmOptimizer \textit{(class)}!peach.pso.base.ParticleSwarmOptimizer.step \textit{(method)}}

    \vspace{0.5ex}

\hspace{.8\funcindent}\begin{boxedminipage}{\funcwidth}

    \raggedright \textbf{step}(\textit{self})

    \vspace{-1.5ex}

    \rule{\textwidth}{0.5\fboxrule}
\setlength{\parskip}{2ex}

Computes the new positions of the particles, a step of the algorithm.

This method updates the velocity given the constants associated with the
particle and global bests; and then updates the positions accordingly.

This method has no parameters and returns no values. The particles
positions can be consulted with the \texttt{{[}{]}} interface (as a swarm of
particles is a list of estimates), \texttt{best} property, to find the global
best, and \texttt{fbest} property to find the minimum (see above).
\setlength{\parskip}{1ex}
    \end{boxedminipage}

    \label{peach:pso:base:ParticleSwarmOptimizer:__call__}
    \index{peach \textit{(package)}!peach.pso \textit{(package)}!peach.pso.base \textit{(module)}!peach.pso.base.ParticleSwarmOptimizer \textit{(class)}!peach.pso.base.ParticleSwarmOptimizer.\_\_call\_\_ \textit{(method)}}

    \vspace{0.5ex}

\hspace{.8\funcindent}\begin{boxedminipage}{\funcwidth}

    \raggedright \textbf{\_\_call\_\_}(\textit{self})

    \vspace{-1.5ex}

    \rule{\textwidth}{0.5\fboxrule}
\setlength{\parskip}{2ex}

Transparently executes the search until the minimum is found. The stop
criteria are the maximum error or the maximum number of iterations,
whichever is reached first. Note that this is a \texttt{\_\_call\_\_} method, so
the object is called as a function. This method returns a tuple
\texttt{(x, e)}, with the best estimate of the minimum and the error.
\setlength{\parskip}{1ex}
      \textbf{Return Value}
    \vspace{-1ex}

      \begin{quote}

This method returns a tuple \texttt{(x, e)}, where \texttt{x} is the best
estimate of the minimum, and \texttt{e} is the estimated error.
      \end{quote}

    \end{boxedminipage}


\large{\textbf{\textit{Inherited from list}}}

\begin{quote}
\_\_add\_\_(), \_\_contains\_\_(), \_\_delitem\_\_(), \_\_delslice\_\_(), \_\_eq\_\_(), \_\_ge\_\_(), \_\_getattribute\_\_(), \_\_getitem\_\_(), \_\_getslice\_\_(), \_\_gt\_\_(), \_\_iadd\_\_(), \_\_imul\_\_(), \_\_iter\_\_(), \_\_le\_\_(), \_\_len\_\_(), \_\_lt\_\_(), \_\_mul\_\_(), \_\_ne\_\_(), \_\_new\_\_(), \_\_repr\_\_(), \_\_reversed\_\_(), \_\_rmul\_\_(), \_\_setitem\_\_(), \_\_setslice\_\_(), \_\_sizeof\_\_(), append(), count(), extend(), index(), insert(), pop(), remove(), reverse(), sort()
\end{quote}

\large{\textbf{\textit{Inherited from object}}}

\begin{quote}
\_\_delattr\_\_(), \_\_format\_\_(), \_\_reduce\_\_(), \_\_reduce\_ex\_\_(), \_\_setattr\_\_(), \_\_str\_\_(), \_\_subclasshook\_\_()
\end{quote}

%%%%%%%%%%%%%%%%%%%%%%%%%%%%%%%%%%%%%%%%%%%%%%%%%%%%%%%%%%%%%%%%%%%%%%%%%%%
%%                              Properties                               %%
%%%%%%%%%%%%%%%%%%%%%%%%%%%%%%%%%%%%%%%%%%%%%%%%%%%%%%%%%%%%%%%%%%%%%%%%%%%

  \subsubsection{Properties}

    \vspace{-1cm}
\hspace{\varindent}\begin{longtable}{|p{\varnamewidth}|p{\vardescrwidth}|l}
\cline{1-2}
\cline{1-2} \centering \textbf{Name} & \centering \textbf{Description}& \\
\cline{1-2}
\endhead\cline{1-2}\multicolumn{3}{r}{\small\textit{continued on next page}}\\\endfoot\cline{1-2}
\endlastfoot\raggedright f\-x\- & &\\
\cline{1-2}
\raggedright b\-e\-s\-t\- & &\\
\cline{1-2}
\raggedright f\-b\-e\-s\-t\- & &\\
\cline{1-2}
\multicolumn{2}{|l|}{\textit{Inherited from object}}\\
\multicolumn{2}{|p{\varwidth}|}{\raggedright \_\_class\_\_}\\
\cline{1-2}
\end{longtable}


%%%%%%%%%%%%%%%%%%%%%%%%%%%%%%%%%%%%%%%%%%%%%%%%%%%%%%%%%%%%%%%%%%%%%%%%%%%
%%                            Class Variables                            %%
%%%%%%%%%%%%%%%%%%%%%%%%%%%%%%%%%%%%%%%%%%%%%%%%%%%%%%%%%%%%%%%%%%%%%%%%%%%

  \subsubsection{Class Variables}

    \vspace{-1cm}
\hspace{\varindent}\begin{longtable}{|p{\varnamewidth}|p{\vardescrwidth}|l}
\cline{1-2}
\cline{1-2} \centering \textbf{Name} & \centering \textbf{Description}& \\
\cline{1-2}
\endhead\cline{1-2}\multicolumn{3}{r}{\small\textit{continued on next page}}\\\endfoot\cline{1-2}
\endlastfoot\multicolumn{2}{|l|}{\textit{Inherited from list}}\\
\multicolumn{2}{|p{\varwidth}|}{\raggedright \_\_hash\_\_}\\
\cline{1-2}
\end{longtable}


%%%%%%%%%%%%%%%%%%%%%%%%%%%%%%%%%%%%%%%%%%%%%%%%%%%%%%%%%%%%%%%%%%%%%%%%%%%
%%                          Instance Variables                           %%
%%%%%%%%%%%%%%%%%%%%%%%%%%%%%%%%%%%%%%%%%%%%%%%%%%%%%%%%%%%%%%%%%%%%%%%%%%%

  \subsubsection{Instance Variables}

    \vspace{-1cm}
\hspace{\varindent}\begin{longtable}{|p{\varnamewidth}|p{\vardescrwidth}|l}
\cline{1-2}
\cline{1-2} \centering \textbf{Name} & \centering \textbf{Description}& \\
\cline{1-2}
\endhead\cline{1-2}\multicolumn{3}{r}{\small\textit{continued on next page}}\\\endfoot\cline{1-2}
\endlastfoot\raggedright r\-a\-n\-g\-e\-s\- & Holds the ranges for every variable. Although it is a writable
property, care should be taken in changing parameters before ending the
convergence.&\\
\cline{1-2}
\end{longtable}

    \index{peach \textit{(package)}!peach.pso \textit{(package)}!peach.pso.base \textit{(module)}!peach.pso.base.ParticleSwarmOptimizer \textit{(class)}|)}

%%%%%%%%%%%%%%%%%%%%%%%%%%%%%%%%%%%%%%%%%%%%%%%%%%%%%%%%%%%%%%%%%%%%%%%%%%%
%%                           Class Description                           %%
%%%%%%%%%%%%%%%%%%%%%%%%%%%%%%%%%%%%%%%%%%%%%%%%%%%%%%%%%%%%%%%%%%%%%%%%%%%

    \index{peach \textit{(package)}!peach.pso \textit{(package)}!peach.pso.base \textit{(module)}!peach.pso.base.PSO \textit{(class)}|(}
\subsection{Class PSO}

    \label{peach:pso:base:PSO}
\begin{tabular}{cccccccccc}
% Line for object, linespec=[False, False, False]
\multicolumn{2}{r}{\settowidth{\BCL}{object}\multirow{2}{\BCL}{object}}
&&
&&
&&
  \\\cline{3-3}
  &&\multicolumn{1}{c|}{}
&&
&&
&&
  \\
% Line for list, linespec=[False, False]
\multicolumn{4}{r}{\settowidth{\BCL}{list}\multirow{2}{\BCL}{list}}
&&
&&
  \\\cline{5-5}
  &&&&\multicolumn{1}{c|}{}
&&
&&
  \\
% Line for peach.pso.base.ParticleSwarmOptimizer, linespec=[False]
\multicolumn{6}{r}{\settowidth{\BCL}{peach.pso.base.ParticleSwarmOptimizer}\multirow{2}{\BCL}{peach.pso.base.ParticleSwarmOptimizer}}
&&
  \\\cline{7-7}
  &&&&&&\multicolumn{1}{c|}{}
&&
  \\
&&&&&&\multicolumn{2}{l}{\textbf{peach.pso.base.PSO}}
\end{tabular}


PSO is an alias to \texttt{ParticleSwarmOptimizer}

%%%%%%%%%%%%%%%%%%%%%%%%%%%%%%%%%%%%%%%%%%%%%%%%%%%%%%%%%%%%%%%%%%%%%%%%%%%
%%                                Methods                                %%
%%%%%%%%%%%%%%%%%%%%%%%%%%%%%%%%%%%%%%%%%%%%%%%%%%%%%%%%%%%%%%%%%%%%%%%%%%%

  \subsubsection{Methods}


\large{\textbf{\textit{Inherited from peach.pso.base.ParticleSwarmOptimizer\textit{(Section \ref{peach:pso:base:ParticleSwarmOptimizer})}}}}

\begin{quote}
\_\_call\_\_(), \_\_init\_\_(), restart(), step()
\end{quote}

\large{\textbf{\textit{Inherited from list}}}

\begin{quote}
\_\_add\_\_(), \_\_contains\_\_(), \_\_delitem\_\_(), \_\_delslice\_\_(), \_\_eq\_\_(), \_\_ge\_\_(), \_\_getattribute\_\_(), \_\_getitem\_\_(), \_\_getslice\_\_(), \_\_gt\_\_(), \_\_iadd\_\_(), \_\_imul\_\_(), \_\_iter\_\_(), \_\_le\_\_(), \_\_len\_\_(), \_\_lt\_\_(), \_\_mul\_\_(), \_\_ne\_\_(), \_\_new\_\_(), \_\_repr\_\_(), \_\_reversed\_\_(), \_\_rmul\_\_(), \_\_setitem\_\_(), \_\_setslice\_\_(), \_\_sizeof\_\_(), append(), count(), extend(), index(), insert(), pop(), remove(), reverse(), sort()
\end{quote}

\large{\textbf{\textit{Inherited from object}}}

\begin{quote}
\_\_delattr\_\_(), \_\_format\_\_(), \_\_reduce\_\_(), \_\_reduce\_ex\_\_(), \_\_setattr\_\_(), \_\_str\_\_(), \_\_subclasshook\_\_()
\end{quote}

%%%%%%%%%%%%%%%%%%%%%%%%%%%%%%%%%%%%%%%%%%%%%%%%%%%%%%%%%%%%%%%%%%%%%%%%%%%
%%                              Properties                               %%
%%%%%%%%%%%%%%%%%%%%%%%%%%%%%%%%%%%%%%%%%%%%%%%%%%%%%%%%%%%%%%%%%%%%%%%%%%%

  \subsubsection{Properties}

    \vspace{-1cm}
\hspace{\varindent}\begin{longtable}{|p{\varnamewidth}|p{\vardescrwidth}|l}
\cline{1-2}
\cline{1-2} \centering \textbf{Name} & \centering \textbf{Description}& \\
\cline{1-2}
\endhead\cline{1-2}\multicolumn{3}{r}{\small\textit{continued on next page}}\\\endfoot\cline{1-2}
\endlastfoot\multicolumn{2}{|l|}{\textit{Inherited from peach.pso.base.ParticleSwarmOptimizer \textit{(Section \ref{peach:pso:base:ParticleSwarmOptimizer})}}}\\
\multicolumn{2}{|p{\varwidth}|}{\raggedright best, fbest, fx}\\
\cline{1-2}
\multicolumn{2}{|l|}{\textit{Inherited from object}}\\
\multicolumn{2}{|p{\varwidth}|}{\raggedright \_\_class\_\_}\\
\cline{1-2}
\end{longtable}


%%%%%%%%%%%%%%%%%%%%%%%%%%%%%%%%%%%%%%%%%%%%%%%%%%%%%%%%%%%%%%%%%%%%%%%%%%%
%%                            Class Variables                            %%
%%%%%%%%%%%%%%%%%%%%%%%%%%%%%%%%%%%%%%%%%%%%%%%%%%%%%%%%%%%%%%%%%%%%%%%%%%%

  \subsubsection{Class Variables}

    \vspace{-1cm}
\hspace{\varindent}\begin{longtable}{|p{\varnamewidth}|p{\vardescrwidth}|l}
\cline{1-2}
\cline{1-2} \centering \textbf{Name} & \centering \textbf{Description}& \\
\cline{1-2}
\endhead\cline{1-2}\multicolumn{3}{r}{\small\textit{continued on next page}}\\\endfoot\cline{1-2}
\endlastfoot\multicolumn{2}{|l|}{\textit{Inherited from list}}\\
\multicolumn{2}{|p{\varwidth}|}{\raggedright \_\_hash\_\_}\\
\cline{1-2}
\end{longtable}


%%%%%%%%%%%%%%%%%%%%%%%%%%%%%%%%%%%%%%%%%%%%%%%%%%%%%%%%%%%%%%%%%%%%%%%%%%%
%%                          Instance Variables                           %%
%%%%%%%%%%%%%%%%%%%%%%%%%%%%%%%%%%%%%%%%%%%%%%%%%%%%%%%%%%%%%%%%%%%%%%%%%%%

  \subsubsection{Instance Variables}

    \vspace{-1cm}
\hspace{\varindent}\begin{longtable}{|p{\varnamewidth}|p{\vardescrwidth}|l}
\cline{1-2}
\cline{1-2} \centering \textbf{Name} & \centering \textbf{Description}& \\
\cline{1-2}
\endhead\cline{1-2}\multicolumn{3}{r}{\small\textit{continued on next page}}\\\endfoot\cline{1-2}
\endlastfoot\multicolumn{2}{|l|}{\textit{Inherited from peach.pso.base.ParticleSwarmOptimizer \textit{(Section \ref{peach:pso:base:ParticleSwarmOptimizer})}}}\\
\multicolumn{2}{|p{\varwidth}|}{\raggedright ranges}\\
\cline{1-2}
\end{longtable}

    \index{peach \textit{(package)}!peach.pso \textit{(package)}!peach.pso.base \textit{(module)}!peach.pso.base.PSO \textit{(class)}|)}
    \index{peach \textit{(package)}!peach.pso \textit{(package)}!peach.pso.base \textit{(module)}|)}

%
% API Documentation for Peach - Computational Intelligence for Python
% Package peach.sa
%
% Generated by epydoc 3.0.1
% [Thu Jul 28 16:37:50 2011]
%

%%%%%%%%%%%%%%%%%%%%%%%%%%%%%%%%%%%%%%%%%%%%%%%%%%%%%%%%%%%%%%%%%%%%%%%%%%%
%%                          Module Description                           %%
%%%%%%%%%%%%%%%%%%%%%%%%%%%%%%%%%%%%%%%%%%%%%%%%%%%%%%%%%%%%%%%%%%%%%%%%%%%

    \index{peach \textit{(package)}!peach.sa \textit{(package)}|(}
\section{Package peach.sa}

    \label{peach:sa}

This package implements optimization by simulated annealing. Consult:
%
\begin{quote}
%
\begin{description}
\item[{base}] \leavevmode 
Implementation of the basic simulated annealing algorithms;

\item[{neighbor}] \leavevmode 
Some methods for determining the neighbor of the present estimate;

\end{description}

\end{quote}

Simulated Annealing is a meta-heuristic designed for optimization of functions.
It tries to mimic the way that atoms settle in crystal structures of metals. By
slowly cooling the metal, atoms settle in a position of low energy -{}- thus, it
is a natural optimization method.

Two kinds of optimizer are implemented here. The continuous version of the
algorithm can be used for optimization of continuous objective functions; the
discrete (or binary) one, can be used in combinatorial optimization problems.

%%%%%%%%%%%%%%%%%%%%%%%%%%%%%%%%%%%%%%%%%%%%%%%%%%%%%%%%%%%%%%%%%%%%%%%%%%%
%%                                Modules                                %%
%%%%%%%%%%%%%%%%%%%%%%%%%%%%%%%%%%%%%%%%%%%%%%%%%%%%%%%%%%%%%%%%%%%%%%%%%%%

\subsection{Modules}

\begin{itemize}
\setlength{\parskip}{0ex}
\item \textbf{base}: 
This package implements two versions of simulated annealing optimization. One
works with numeric data, and the other with a codified bit string. This last
method can be used in discrete optimization problems.


  \textit{(Section \ref{peach:sa:base}, p.~\pageref{peach:sa:base})}

\item \textbf{neighbor}: 
This module implements a general class to compute neighbors for continuous and
binary simulated annealing algorithms. The continuous neighbor functions return
an array with a neighbor of a given estimate; the binary neighbor functions
return a \texttt{bitarray} object.


  \textit{(Section \ref{peach:sa:neighbor}, p.~\pageref{peach:sa:neighbor})}

\end{itemize}

    \index{peach \textit{(package)}!peach.sa \textit{(package)}|)}

%
% API Documentation for Peach - Computational Intelligence for Python
% Module peach.sa.base
%
% Generated by epydoc 3.0.1
% [Thu Jul 28 16:37:50 2011]
%

%%%%%%%%%%%%%%%%%%%%%%%%%%%%%%%%%%%%%%%%%%%%%%%%%%%%%%%%%%%%%%%%%%%%%%%%%%%
%%                          Module Description                           %%
%%%%%%%%%%%%%%%%%%%%%%%%%%%%%%%%%%%%%%%%%%%%%%%%%%%%%%%%%%%%%%%%%%%%%%%%%%%

    \index{peach \textit{(package)}!peach.sa \textit{(package)}!peach.sa.base \textit{(module)}|(}
\section{Module peach.sa.base}

    \label{peach:sa:base}

This package implements two versions of simulated annealing optimization. One
works with numeric data, and the other with a codified bit string. This last
method can be used in discrete optimization problems.

%%%%%%%%%%%%%%%%%%%%%%%%%%%%%%%%%%%%%%%%%%%%%%%%%%%%%%%%%%%%%%%%%%%%%%%%%%%
%%                               Functions                               %%
%%%%%%%%%%%%%%%%%%%%%%%%%%%%%%%%%%%%%%%%%%%%%%%%%%%%%%%%%%%%%%%%%%%%%%%%%%%

  \subsection{Functions}

    \label{peach:sa:base:standard_normal}
    \index{peach \textit{(package)}!peach.sa \textit{(package)}!peach.sa.base \textit{(module)}!peach.sa.base.standard\_normal \textit{(function)}}

    \vspace{0.5ex}

\hspace{.8\funcindent}\begin{boxedminipage}{\funcwidth}

    \raggedright \textbf{standard\_normal}(\textit{size}={\tt None})

    \vspace{-1.5ex}

    \rule{\textwidth}{0.5\fboxrule}
\setlength{\parskip}{2ex}

Returns samples from a Standard Normal distribution (mean=0, stdev=1).


%___________________________________________________________________________

\paragraph*{Parameters%
  \phantomsection%
  \addcontentsline{toc}{paragraph}{Parameters}%
  \label{parameters}%
}
%
\begin{description}
\item[{size}] \leavevmode (\textbf{int, shape tuple, optional})

Returns the number of samples required to satisfy the \texttt{size} parameter.
If not given or 'None' indicates to return one sample.

\end{description}


%___________________________________________________________________________

\paragraph*{Returns%
  \phantomsection%
  \addcontentsline{toc}{paragraph}{Returns}%
  \label{returns}%
}
%
\begin{description}
\item[{out}] \leavevmode (\textbf{float, ndarray})

Samples the Standard Normal distribution with a shape satisfying the
\texttt{size} parameter.

\end{description}
\setlength{\parskip}{1ex}
    \end{boxedminipage}


%%%%%%%%%%%%%%%%%%%%%%%%%%%%%%%%%%%%%%%%%%%%%%%%%%%%%%%%%%%%%%%%%%%%%%%%%%%
%%                               Variables                               %%
%%%%%%%%%%%%%%%%%%%%%%%%%%%%%%%%%%%%%%%%%%%%%%%%%%%%%%%%%%%%%%%%%%%%%%%%%%%

  \subsection{Variables}

    \vspace{-1cm}
\hspace{\varindent}\begin{longtable}{|p{\varnamewidth}|p{\vardescrwidth}|l}
\cline{1-2}
\cline{1-2} \centering \textbf{Name} & \centering \textbf{Description}& \\
\cline{1-2}
\endhead\cline{1-2}\multicolumn{3}{r}{\small\textit{continued on next page}}\\\endfoot\cline{1-2}
\endlastfoot\raggedright \_\-\_\-d\-o\-c\-\_\-\_\- & \raggedright \textbf{Value:} 
{\tt \texttt{...}}&\\
\cline{1-2}
\raggedright \_\-\_\-p\-a\-c\-k\-a\-g\-e\-\_\-\_\- & \raggedright \textbf{Value:} 
{\tt \texttt{'}\texttt{peach.sa}\texttt{'}}&\\
\cline{1-2}
\end{longtable}


%%%%%%%%%%%%%%%%%%%%%%%%%%%%%%%%%%%%%%%%%%%%%%%%%%%%%%%%%%%%%%%%%%%%%%%%%%%
%%                           Class Description                           %%
%%%%%%%%%%%%%%%%%%%%%%%%%%%%%%%%%%%%%%%%%%%%%%%%%%%%%%%%%%%%%%%%%%%%%%%%%%%

    \index{peach \textit{(package)}!peach.sa \textit{(package)}!peach.sa.base \textit{(module)}!peach.sa.base.ContinuousSA \textit{(class)}|(}
\subsection{Class ContinuousSA}

    \label{peach:sa:base:ContinuousSA}
\begin{tabular}{cccccc}
% Line for object, linespec=[False]
\multicolumn{2}{r}{\settowidth{\BCL}{object}\multirow{2}{\BCL}{object}}
&&
  \\\cline{3-3}
  &&\multicolumn{1}{c|}{}
&&
  \\
&&\multicolumn{2}{l}{\textbf{peach.sa.base.ContinuousSA}}
\end{tabular}


Simulated Annealing continuous optimization.

This is a simulated annealing optimizer implemented to work with vectors of
continuous variables (obviouslly, implemented as floating point numbers). In
general, simulated annealing methods searches for neighbors of one estimate,
which makes a lot more sense in discrete problems. While in this class the
method is implemented in a different way (to deal with continuous
variables), the principle is pretty much the same -{}- the neighbor is found
based on a gaussian neighborhood.

A simulated annealing algorithm adapted to deal with continuous variables
has an enhancement that can be used: a gradient vector can be given and, in
case the neighbor is not accepted, the estimate is updated in the downhill
direction.

%%%%%%%%%%%%%%%%%%%%%%%%%%%%%%%%%%%%%%%%%%%%%%%%%%%%%%%%%%%%%%%%%%%%%%%%%%%
%%                                Methods                                %%
%%%%%%%%%%%%%%%%%%%%%%%%%%%%%%%%%%%%%%%%%%%%%%%%%%%%%%%%%%%%%%%%%%%%%%%%%%%

  \subsubsection{Methods}

    \vspace{0.5ex}

\hspace{.8\funcindent}\begin{boxedminipage}{\funcwidth}

    \raggedright \textbf{\_\_init\_\_}(\textit{self}, \textit{f}, \textit{x0}, \textit{ranges}={\tt None}, \textit{neighbor}={\tt {\textless}class 'peach.sa.neighbor.GaussianNeighbor'{\textgreater}}, \textit{optm}={\tt None}, \textit{T0}={\tt 1000.0}, \textit{rt}={\tt 0.95}, \textit{emax}={\tt 1e-08}, \textit{imax}={\tt 1000})

    \vspace{-1.5ex}

    \rule{\textwidth}{0.5\fboxrule}
\setlength{\parskip}{2ex}

Initializes the optimizer.

To create an optimizer of this type, instantiate the class with the
parameters given below:
\setlength{\parskip}{1ex}
      \textbf{Parameters}
      \vspace{-1ex}

      \begin{quote}
        \begin{Ventry}{xxxxxxxx}

          \item[f]


A multivariable function to be optimized. The function should have
only one parameter, a multidimensional line-vector, and return the
function value, a scalar.
          \item[x0]


First estimate of the minimum. Estimates can be given in any format,
but internally they are converted to a one-dimension vector, where
each component corresponds to the estimate of that particular
variable. The vector is computed by flattening the array.
          \item[ranges]


A range of values might be passed to the algorithm, but it is not
necessary. If supplied, this parameter should be a list of ranges
for each variable of the objective function. It is specified as a
list of tuples of two values, \texttt{(x0, x1)}, where \texttt{x0} is the
start of the interval, and \texttt{x1} its end. Obviously, \texttt{x0} should
be smaller than \texttt{x1}. It can also be given as a list with a simple
tuple in the same format. In that case, the same range will be
applied for every variable in the optimization.
          \item[neighbor]


Neighbor function. This is a function used to compute the neighbor
of the present estimate. You can use the ones defined in the
\texttt{neighbor} module, or you can implement your own. In any case, the
\texttt{neighbor} parameter must be an instance of \texttt{ContinuousNeighbor}
or of a subclass. Please, see the documentation on the \texttt{neighbor}
module for more information. The default is \texttt{GaussianNeighbor},
which computes the new estimate based on a gaussian distribution
around the present estimate.
          \item[optm]


A standard optimizer such as gradient or Newton. This is used in
case the estimate is not accepted by the algorithm -{}- in this case,
a new estimate is computed in a standard way, providing a little
improvement in any case. It defaults to None; in that case, no
standard optimizatiion will be used. Notice that, if you want to use
a standard optimizer, you must create it before you instantiate this
class. By doing it this way, you can configure the optimizer in any
way you want. Please, consult the documentation in \texttt{Gradient},
\texttt{Newton} and others.
          \item[T0]


Initial temperature of the system. The temperature is, of course, an
analogy. Defaults to 1000.
          \item[rt]


Temperature decreasing rate. The temperature must slowly decrease in
simulated annealing algorithms. In this implementation, this is
controlled by this parameter. At each step, the temperature is
multiplied by this value, so it is necessary that \texttt{0 < rt < 1}.
Defaults to 0.95, smaller values make the temperature decay faster,
while larger values make the temperature decay slower.
          \item[h]


Convergence step. In the case that the neighbor estimate is not
accepted, a simple gradient step is executed. This parameter is the
convergence step to the gradient step.
          \item[emax]


Maximum allowed error. The algorithm stops as soon as the error is
below this level. The error is absolute.
          \item[imax]


Maximum number of iterations, the algorithm stops as soon this
number of iterations are executed, no matter what the error is at
the moment.
        \end{Ventry}

      \end{quote}

      Overrides: object.\_\_init\_\_

    \end{boxedminipage}

    \label{peach:sa:base:ContinuousSA:restart}
    \index{peach \textit{(package)}!peach.sa \textit{(package)}!peach.sa.base \textit{(module)}!peach.sa.base.ContinuousSA \textit{(class)}!peach.sa.base.ContinuousSA.restart \textit{(method)}}

    \vspace{0.5ex}

\hspace{.8\funcindent}\begin{boxedminipage}{\funcwidth}

    \raggedright \textbf{restart}(\textit{self}, \textit{x0}, \textit{T0}={\tt 1000.0}, \textit{rt}={\tt 0.95}, \textit{h}={\tt 0.5})

    \vspace{-1.5ex}

    \rule{\textwidth}{0.5\fboxrule}
\setlength{\parskip}{2ex}

Resets the optimizer, returning to its original state, and allowing to
use a new first estimate. Restartings are essential to the working of
simulated annealing algorithms, to allow them to leave local minima.
\setlength{\parskip}{1ex}
      \textbf{Parameters}
      \vspace{-1ex}

      \begin{quote}
        \begin{Ventry}{xx}

          \item[x0]


New estimate of the minimum. Estimates can be given in any format,
but internally they are converted to a one-dimension vector, where
each component corresponds to the estimate of that particular
variable. The vector is computed by flattening the array.
          \item[T0]


Initial temperature of the system. The temperature is, of course, an
analogy. Defaults to 1000.
          \item[rt]


Temperature decreasing rate. The temperature must slowly decrease in
simulated annealing algorithms. In this implementation, this is
controlled by this parameter. At each step, the temperature is
multiplied by this value, so it is necessary that \texttt{0 < rt < 1}.
Defaults to 0.95, smaller values make the temperature decay faster,
while larger values make the temperature decay slower.
          \item[h]


The initial step of the search. Defaults to 0.5
        \end{Ventry}

      \end{quote}

    \end{boxedminipage}

    \label{peach:sa:base:ContinuousSA:step}
    \index{peach \textit{(package)}!peach.sa \textit{(package)}!peach.sa.base \textit{(module)}!peach.sa.base.ContinuousSA \textit{(class)}!peach.sa.base.ContinuousSA.step \textit{(method)}}

    \vspace{0.5ex}

\hspace{.8\funcindent}\begin{boxedminipage}{\funcwidth}

    \raggedright \textbf{step}(\textit{self})

    \vspace{-1.5ex}

    \rule{\textwidth}{0.5\fboxrule}
\setlength{\parskip}{2ex}

One step of the search.

In this method, a neighbor of the given estimate is chosen at random,
using a gaussian neighborhood. It is accepted as a new estimate if it
performs better in the cost function \emph{or} if the temperature is high
enough. In case it is not accepted, a gradient step is executed.
\setlength{\parskip}{1ex}
      \textbf{Return Value}
    \vspace{-1ex}

      \begin{quote}

This method returns a tuple \texttt{(x, e)}, where \texttt{x} is the updated
estimate of the minimum, and \texttt{e} is the estimated error.
      \end{quote}

    \end{boxedminipage}

    \label{peach:sa:base:ContinuousSA:__call__}
    \index{peach \textit{(package)}!peach.sa \textit{(package)}!peach.sa.base \textit{(module)}!peach.sa.base.ContinuousSA \textit{(class)}!peach.sa.base.ContinuousSA.\_\_call\_\_ \textit{(method)}}

    \vspace{0.5ex}

\hspace{.8\funcindent}\begin{boxedminipage}{\funcwidth}

    \raggedright \textbf{\_\_call\_\_}(\textit{self})

    \vspace{-1.5ex}

    \rule{\textwidth}{0.5\fboxrule}
\setlength{\parskip}{2ex}

Transparently executes the search until the minimum is found. The stop
criteria are the maximum error or the maximum number of iterations,
whichever is reached first. Note that this is a \texttt{\_\_call\_\_} method, so
the object is called as a function. This method returns a tuple
\texttt{(x, e)}, with the best estimate of the minimum and the error.
\setlength{\parskip}{1ex}
      \textbf{Return Value}
    \vspace{-1ex}

      \begin{quote}

This method returns a tuple \texttt{(x, e)}, where \texttt{x} is the best
estimate of the minimum, and \texttt{e} is the estimated error.
      \end{quote}

    \end{boxedminipage}


\large{\textbf{\textit{Inherited from object}}}

\begin{quote}
\_\_delattr\_\_(), \_\_format\_\_(), \_\_getattribute\_\_(), \_\_hash\_\_(), \_\_new\_\_(), \_\_reduce\_\_(), \_\_reduce\_ex\_\_(), \_\_repr\_\_(), \_\_setattr\_\_(), \_\_sizeof\_\_(), \_\_str\_\_(), \_\_subclasshook\_\_()
\end{quote}

%%%%%%%%%%%%%%%%%%%%%%%%%%%%%%%%%%%%%%%%%%%%%%%%%%%%%%%%%%%%%%%%%%%%%%%%%%%
%%                              Properties                               %%
%%%%%%%%%%%%%%%%%%%%%%%%%%%%%%%%%%%%%%%%%%%%%%%%%%%%%%%%%%%%%%%%%%%%%%%%%%%

  \subsubsection{Properties}

    \vspace{-1cm}
\hspace{\varindent}\begin{longtable}{|p{\varnamewidth}|p{\vardescrwidth}|l}
\cline{1-2}
\cline{1-2} \centering \textbf{Name} & \centering \textbf{Description}& \\
\cline{1-2}
\endhead\cline{1-2}\multicolumn{3}{r}{\small\textit{continued on next page}}\\\endfoot\cline{1-2}
\endlastfoot\raggedright x\- & &\\
\cline{1-2}
\raggedright f\-x\- & &\\
\cline{1-2}
\multicolumn{2}{|l|}{\textit{Inherited from object}}\\
\multicolumn{2}{|p{\varwidth}|}{\raggedright \_\_class\_\_}\\
\cline{1-2}
\end{longtable}


%%%%%%%%%%%%%%%%%%%%%%%%%%%%%%%%%%%%%%%%%%%%%%%%%%%%%%%%%%%%%%%%%%%%%%%%%%%
%%                          Instance Variables                           %%
%%%%%%%%%%%%%%%%%%%%%%%%%%%%%%%%%%%%%%%%%%%%%%%%%%%%%%%%%%%%%%%%%%%%%%%%%%%

  \subsubsection{Instance Variables}

    \vspace{-1cm}
\hspace{\varindent}\begin{longtable}{|p{\varnamewidth}|p{\vardescrwidth}|l}
\cline{1-2}
\cline{1-2} \centering \textbf{Name} & \centering \textbf{Description}& \\
\cline{1-2}
\endhead\cline{1-2}\multicolumn{3}{r}{\small\textit{continued on next page}}\\\endfoot\cline{1-2}
\endlastfoot\raggedright r\-a\-n\-g\-e\-s\- & Holds the ranges for every variable. Although it is a writable
property, care should be taken in changing parameters before ending the
convergence.&\\
\cline{1-2}
\end{longtable}

    \index{peach \textit{(package)}!peach.sa \textit{(package)}!peach.sa.base \textit{(module)}!peach.sa.base.ContinuousSA \textit{(class)}|)}

%%%%%%%%%%%%%%%%%%%%%%%%%%%%%%%%%%%%%%%%%%%%%%%%%%%%%%%%%%%%%%%%%%%%%%%%%%%
%%                           Class Description                           %%
%%%%%%%%%%%%%%%%%%%%%%%%%%%%%%%%%%%%%%%%%%%%%%%%%%%%%%%%%%%%%%%%%%%%%%%%%%%

    \index{peach \textit{(package)}!peach.sa \textit{(package)}!peach.sa.base \textit{(module)}!peach.sa.base.BinarySA \textit{(class)}|(}
\subsection{Class BinarySA}

    \label{peach:sa:base:BinarySA}
\begin{tabular}{cccccc}
% Line for object, linespec=[False]
\multicolumn{2}{r}{\settowidth{\BCL}{object}\multirow{2}{\BCL}{object}}
&&
  \\\cline{3-3}
  &&\multicolumn{1}{c|}{}
&&
  \\
&&\multicolumn{2}{l}{\textbf{peach.sa.base.BinarySA}}
\end{tabular}


Simulated Annealing binary optimization.

This is a simulated annealing optimizer implemented to work with vectors of
bits, which can be floating point or integer numbers, characters or anything
allowed by the \texttt{struct} module of the Python standard library. The
neighborhood of an estimate is calculated by an appropriate method given in
the class instantiation. Given the nature of this implementation, no
alternate convergence can be used in the case of rejection of an estimate.

%%%%%%%%%%%%%%%%%%%%%%%%%%%%%%%%%%%%%%%%%%%%%%%%%%%%%%%%%%%%%%%%%%%%%%%%%%%
%%                                Methods                                %%
%%%%%%%%%%%%%%%%%%%%%%%%%%%%%%%%%%%%%%%%%%%%%%%%%%%%%%%%%%%%%%%%%%%%%%%%%%%

  \subsubsection{Methods}

    \vspace{0.5ex}

\hspace{.8\funcindent}\begin{boxedminipage}{\funcwidth}

    \raggedright \textbf{\_\_init\_\_}(\textit{self}, \textit{f}, \textit{x0}, \textit{ranges}={\tt \texttt{[}\texttt{]}}, \textit{fmt}={\tt None}, \textit{neighbor}={\tt {\textless}class 'peach.sa.neighbor.InvertBitsNeighbor'{\textgreater}}, \textit{T0}={\tt 1000.0}, \textit{rt}={\tt 0.95}, \textit{emax}={\tt 1e-08}, \textit{imax}={\tt 1000})

    \vspace{-1.5ex}

    \rule{\textwidth}{0.5\fboxrule}
\setlength{\parskip}{2ex}

Initializes the optimizer.

To create an optimizer of this type, instantiate the class with the
parameters given below:
\setlength{\parskip}{1ex}
      \textbf{Parameters}
      \vspace{-1ex}

      \begin{quote}
        \begin{Ventry}{xxxxxxxx}

          \item[f]


A multivariable function to be optimized. The function should have
only one parameter, a multidimensional line-vector, and return the
function value, a scalar.
          \item[x0]


First estimate of the minimum. Estimates can be given in any format,
but internally they are converted to a one-dimension vector, where
each component corresponds to the estimate of that particular
variable. The vector is computed by flattening the array.
          \item[ranges]


Ranges of values allowed for each component of the input vector. If
given, ranges are checked and a new estimate is generated in case
any of the components fall beyond the value. \texttt{range} can be a
tuple containing the inferior and superior limits of the interval;
in that case, the same range is used for every variable in the input
vector. \texttt{range} can also be a list of tuples of the same format,
inferior and superior limits; in that case, the first tuple is
assumed as the range allowed for the first variable, the second
tuple is assumed as the range allowed for the second variable and so
on.
          \item[fmt]


A \texttt{struct}-module string with the format of the data used. Please,
consult the \texttt{struct} documentation, since what is explained there
is exactly what is used here. For example, if you are going to use
the optimizer to deal with three-dimensional vectors of continuous
variables, the format would be something like:
%
\begin{quote}{\ttfamily \raggedright \noindent
fmt~=~'fff'
}
\end{quote}

Default value is an empty string. Notice that this is implemented as
a \texttt{bitarray}, so this module must be present.

It is strongly recommended that integer numbers are used! Floating
point numbers can be simulated with long integers. The reason for
this is that random bit sequences can have no representation as
floating point numbers, and that can make the algorithm not perform
adequatelly.

The default value for this parameter is \texttt{None}, meaning that a
default format is not supplied. If a format is not supplied, then
the estimate will be passed as a bitarray to the objective function.
This means that your function must take care to decode the bit
stream to extract meaning from it.
          \item[neighbor]


Neighbor function. This is a function used to compute the neighbor
of the present estimate. You can use the ones defined in the
\texttt{neighbor} module, or you can implement your own. In any case, the
\texttt{neighbor} parameter must be an instance of \texttt{BinaryNeighbor} or
of a subclass. Please, see the documentation on the \texttt{neighbor}
module for more information. The default is \texttt{InvertBitsNeighbor},
which computes the new estimate by inverting some bits in the
present estimate.
          \item[T0]


Initial temperature of the system. The temperature is, of course, an
analogy. Defaults to 1000.
          \item[rt]


Temperature decreasing rate. The temperature must slowly decrease in
simulated annealing algorithms. In this implementation, this is
controlled by this parameter. At each step, the temperature is
multiplied by this value, so it is necessary that \texttt{0 < rt < 1}.
Defaults to 0.95, smaller values make the temperature decay faster,
while larger values make the temperature decay slower.
          \item[emax]


Maximum allowed error. The algorithm stops as soon as the error is
below this level. The error is absolute.
          \item[imax]


Maximum number of iterations, the algorithm stops as soon this
number of iterations are executed, no matter what the error is at
the moment.
        \end{Ventry}

      \end{quote}

      Overrides: object.\_\_init\_\_

    \end{boxedminipage}

    \label{peach:sa:base:BinarySA:restart}
    \index{peach \textit{(package)}!peach.sa \textit{(package)}!peach.sa.base \textit{(module)}!peach.sa.base.BinarySA \textit{(class)}!peach.sa.base.BinarySA.restart \textit{(method)}}

    \vspace{0.5ex}

\hspace{.8\funcindent}\begin{boxedminipage}{\funcwidth}

    \raggedright \textbf{restart}(\textit{self}, \textit{x0}, \textit{ranges}={\tt None}, \textit{T0}={\tt 1000.0}, \textit{rt}={\tt 0.95}, \textit{h}={\tt 0.5})

    \vspace{-1.5ex}

    \rule{\textwidth}{0.5\fboxrule}
\setlength{\parskip}{2ex}

Resets the optimizer, returning to its original state, and allowing to
use a new first estimate. Restartings are essential to the working of
simulated annealing algorithms, to allow them to leave local minima.
\setlength{\parskip}{1ex}
      \textbf{Parameters}
      \vspace{-1ex}

      \begin{quote}
        \begin{Ventry}{xxxxxx}

          \item[x0]


New estimate of the minimum. Estimates can be given in any format,
but internally they are converted to a one-dimension vector, where
each component corresponds to the estimate of that particular
variable. The vector is computed by flattening the array.
          \item[ranges]


Ranges of values allowed for each component of the input vector. If
given, ranges are checked and a new estimate is generated in case
any of the components fall beyond the value. \texttt{range} can be a
tuple containing the inferior and superior limits of the interval;
in that case, the same range is used for every variable in the input
vector. \texttt{range} can also be a list of tuples of the same format,
inferior and superior limits; in that case, the first tuple is
assumed as the range allowed for the first variable, the second
tuple is assumed as the range allowed for the second variable and so
on.
          \item[T0]


Initial temperature of the system. The temperature is, of course, an
analogy. Defaults to 1000.
          \item[rt]


Temperature decreasing rate. The temperature must slowly decrease in
simulated annealing algorithms. In this implementation, this is
controlled by this parameter. At each step, the temperature is
multiplied by this value, so it is necessary that \texttt{0 < rt < 1}.
Defaults to 0.95, smaller values make the temperature decay faster,
while larger values make the temperature decay slower.
        \end{Ventry}

      \end{quote}

    \end{boxedminipage}

    \label{peach:sa:base:BinarySA:step}
    \index{peach \textit{(package)}!peach.sa \textit{(package)}!peach.sa.base \textit{(module)}!peach.sa.base.BinarySA \textit{(class)}!peach.sa.base.BinarySA.step \textit{(method)}}

    \vspace{0.5ex}

\hspace{.8\funcindent}\begin{boxedminipage}{\funcwidth}

    \raggedright \textbf{step}(\textit{self})

    \vspace{-1.5ex}

    \rule{\textwidth}{0.5\fboxrule}
\setlength{\parskip}{2ex}

One step of the search.

In this method, a neighbor of the given estimate is obtained from the
present estimate by choosing \texttt{nb} bits and inverting them. It is
accepted as a new estimate if it performs better in the cost function
\emph{or} if the temperature is high enough. In case it is not accepted, the
previous estimate is mantained.
\setlength{\parskip}{1ex}
      \textbf{Return Value}
    \vspace{-1ex}

      \begin{quote}

This method returns a tuple \texttt{(x, e)}, where \texttt{x} is the updated
estimate of the minimum, and \texttt{e} is the estimated error.
      \end{quote}

    \end{boxedminipage}

    \label{peach:sa:base:BinarySA:__call__}
    \index{peach \textit{(package)}!peach.sa \textit{(package)}!peach.sa.base \textit{(module)}!peach.sa.base.BinarySA \textit{(class)}!peach.sa.base.BinarySA.\_\_call\_\_ \textit{(method)}}

    \vspace{0.5ex}

\hspace{.8\funcindent}\begin{boxedminipage}{\funcwidth}

    \raggedright \textbf{\_\_call\_\_}(\textit{self})

    \vspace{-1.5ex}

    \rule{\textwidth}{0.5\fboxrule}
\setlength{\parskip}{2ex}

Transparently executes the search until the minimum is found. The stop
criteria are the maximum error or the maximum number of iterations,
whichever is reached first. Note that this is a \texttt{\_\_call\_\_} method, so
the object is called as a function. This method returns a tuple
\texttt{(x, e)}, with the best estimate of the minimum and the error.
\setlength{\parskip}{1ex}
      \textbf{Return Value}
    \vspace{-1ex}

      \begin{quote}

This method returns a tuple \texttt{(x, e)}, where \texttt{x} is the best
estimate of the minimum, and \texttt{e} is the estimated error.
      \end{quote}

    \end{boxedminipage}


\large{\textbf{\textit{Inherited from object}}}

\begin{quote}
\_\_delattr\_\_(), \_\_format\_\_(), \_\_getattribute\_\_(), \_\_hash\_\_(), \_\_new\_\_(), \_\_reduce\_\_(), \_\_reduce\_ex\_\_(), \_\_repr\_\_(), \_\_setattr\_\_(), \_\_sizeof\_\_(), \_\_str\_\_(), \_\_subclasshook\_\_()
\end{quote}

%%%%%%%%%%%%%%%%%%%%%%%%%%%%%%%%%%%%%%%%%%%%%%%%%%%%%%%%%%%%%%%%%%%%%%%%%%%
%%                              Properties                               %%
%%%%%%%%%%%%%%%%%%%%%%%%%%%%%%%%%%%%%%%%%%%%%%%%%%%%%%%%%%%%%%%%%%%%%%%%%%%

  \subsubsection{Properties}

    \vspace{-1cm}
\hspace{\varindent}\begin{longtable}{|p{\varnamewidth}|p{\vardescrwidth}|l}
\cline{1-2}
\cline{1-2} \centering \textbf{Name} & \centering \textbf{Description}& \\
\cline{1-2}
\endhead\cline{1-2}\multicolumn{3}{r}{\small\textit{continued on next page}}\\\endfoot\cline{1-2}
\endlastfoot\raggedright x\- & \raggedright Getter for the estimate. The estimate is decoded as the format supplied.
If no format was supplied, then the estimate is returned as a bitarray.&\\
\cline{1-2}
\raggedright b\-e\-s\-t\- & \raggedright Getter for the best value so far. Returns a tuple containing both the
best estimate and its value.&\\
\cline{1-2}
\multicolumn{2}{|l|}{\textit{Inherited from object}}\\
\multicolumn{2}{|p{\varwidth}|}{\raggedright \_\_class\_\_}\\
\cline{1-2}
\end{longtable}

    \index{peach \textit{(package)}!peach.sa \textit{(package)}!peach.sa.base \textit{(module)}!peach.sa.base.BinarySA \textit{(class)}|)}
    \index{peach \textit{(package)}!peach.sa \textit{(package)}!peach.sa.base \textit{(module)}|)}

%
% API Documentation for Peach - Computational Intelligence for Python
% Module peach.sa.neighbor
%
% Generated by epydoc 3.0.1
% [Mon Jan 24 15:39:53 2011]
%

%%%%%%%%%%%%%%%%%%%%%%%%%%%%%%%%%%%%%%%%%%%%%%%%%%%%%%%%%%%%%%%%%%%%%%%%%%%
%%                          Module Description                           %%
%%%%%%%%%%%%%%%%%%%%%%%%%%%%%%%%%%%%%%%%%%%%%%%%%%%%%%%%%%%%%%%%%%%%%%%%%%%

    \index{peach \textit{(package)}!peach.sa \textit{(package)}!peach.sa.neighbor \textit{(module)}|(}
\section{Module peach.sa.neighbor}

    \label{peach:sa:neighbor}

This module implements a general class to compute neighbors for continuous and
binary simulated annealing algorithms. The continuous neighbor functions return
an array with a neighbor of a given estimate; the binary neighbor functions
return a \texttt{bitarray} object.

%%%%%%%%%%%%%%%%%%%%%%%%%%%%%%%%%%%%%%%%%%%%%%%%%%%%%%%%%%%%%%%%%%%%%%%%%%%
%%                               Variables                               %%
%%%%%%%%%%%%%%%%%%%%%%%%%%%%%%%%%%%%%%%%%%%%%%%%%%%%%%%%%%%%%%%%%%%%%%%%%%%

  \subsection{Variables}

    \vspace{-1cm}
\hspace{\varindent}\begin{longtable}{|p{\varnamewidth}|p{\vardescrwidth}|l}
\cline{1-2}
\cline{1-2} \centering \textbf{Name} & \centering \textbf{Description}& \\
\cline{1-2}
\endhead\cline{1-2}\multicolumn{3}{r}{\small\textit{continued on next page}}\\\endfoot\cline{1-2}
\endlastfoot\raggedright \_\-\_\-d\-o\-c\-\_\-\_\- & \raggedright \textbf{Value:} 
{\tt \texttt{...}}&\\
\cline{1-2}
\raggedright \_\-\_\-p\-a\-c\-k\-a\-g\-e\-\_\-\_\- & \raggedright \textbf{Value:} 
{\tt \texttt{'}\texttt{peach.sa}\texttt{'}}&\\
\cline{1-2}
\end{longtable}


%%%%%%%%%%%%%%%%%%%%%%%%%%%%%%%%%%%%%%%%%%%%%%%%%%%%%%%%%%%%%%%%%%%%%%%%%%%
%%                           Class Description                           %%
%%%%%%%%%%%%%%%%%%%%%%%%%%%%%%%%%%%%%%%%%%%%%%%%%%%%%%%%%%%%%%%%%%%%%%%%%%%

    \index{peach \textit{(package)}!peach.sa \textit{(package)}!peach.sa.neighbor \textit{(module)}!peach.sa.neighbor.ContinuousNeighbor \textit{(class)}|(}
\subsection{Class ContinuousNeighbor}

    \label{peach:sa:neighbor:ContinuousNeighbor}
\begin{tabular}{cccccc}
% Line for object, linespec=[False]
\multicolumn{2}{r}{\settowidth{\BCL}{object}\multirow{2}{\BCL}{object}}
&&
  \\\cline{3-3}
  &&\multicolumn{1}{c|}{}
&&
  \\
&&\multicolumn{2}{l}{\textbf{peach.sa.neighbor.ContinuousNeighbor}}
\end{tabular}

\textbf{Known Subclasses:}
peach.sa.neighbor.GaussianNeighbor,
    peach.sa.neighbor.UniformNeighbor


Base class for continuous neighbor functions

This class should be derived to implement a function which computes the
neighbor of a given estimate. Every neighbor function should implement at
least two methods, defined below:
%
\begin{quote}
%
\begin{description}
\item[{\_\_init\_\_(self, %
\raisebox{1em}{\hypertarget{id2}{}}\hyperlink{id1}{\textbf{\color{red}*}}cnf, %
\raisebox{1em}{\hypertarget{id4}{}}\hyperlink{id3}{\textbf{\color{red}**}}kw)}] \leavevmode 
Initializes the object. There are no mandatory arguments, but any
parameters can be used here to configure the operator. For example, a
class can define a variance for randomly chose the neighbor -{}- this
should be defined here:
%
\begin{quote}{\ttfamily \raggedright \noindent
\_\_init\_\_(self,~variance=1.0)
}
\end{quote}

A default value should always be offered, if possible.

\item[{\_\_call\_\_(self, x):}] \leavevmode 
The \texttt{\_\_call\_\_} interface should be programmed to actually compute the
value of the neighbor. This method should receive an estimate in \texttt{x}
and use whatever parameters from the instantiation to compute the new
estimate. It should return the new estimate.

\end{description}

\end{quote}

Please, note that the SA implementations relies on this behaviour: it will
pass an estimate to your \texttt{\_\_call\_\_} method and expects to received the
result back.

This class can be used also to transform a simple function in a neighbor
function. In this case, the outside function must compute in an appropriate
way the new estimate.

%%%%%%%%%%%%%%%%%%%%%%%%%%%%%%%%%%%%%%%%%%%%%%%%%%%%%%%%%%%%%%%%%%%%%%%%%%%
%%                                Methods                                %%
%%%%%%%%%%%%%%%%%%%%%%%%%%%%%%%%%%%%%%%%%%%%%%%%%%%%%%%%%%%%%%%%%%%%%%%%%%%

  \subsubsection{Methods}

    \vspace{0.5ex}

\hspace{.8\funcindent}\begin{boxedminipage}{\funcwidth}

    \raggedright \textbf{\_\_init\_\_}(\textit{self}, \textit{f})

    \vspace{-1.5ex}

    \rule{\textwidth}{0.5\fboxrule}
\setlength{\parskip}{2ex}

Creates a neighbor function from a function.
\setlength{\parskip}{1ex}
      \textbf{Parameters}
      \vspace{-1ex}

      \begin{quote}
        \begin{Ventry}{x}

          \item[f]


The function to be transformed. This function must receive an array
of any size and shape as an estimate, and return an estimate of the
same size and shape as a result. A function that operates only over
a single number can be used -{}- in this case, the function operation
will propagate over all components of the estimate.
        \end{Ventry}

      \end{quote}

      Overrides: object.\_\_init\_\_

    \end{boxedminipage}

    \label{peach:sa:neighbor:ContinuousNeighbor:__call__}
    \index{peach \textit{(package)}!peach.sa \textit{(package)}!peach.sa.neighbor \textit{(module)}!peach.sa.neighbor.ContinuousNeighbor \textit{(class)}!peach.sa.neighbor.ContinuousNeighbor.\_\_call\_\_ \textit{(method)}}

    \vspace{0.5ex}

\hspace{.8\funcindent}\begin{boxedminipage}{\funcwidth}

    \raggedright \textbf{\_\_call\_\_}(\textit{self}, \textit{x})

    \vspace{-1.5ex}

    \rule{\textwidth}{0.5\fboxrule}
\setlength{\parskip}{2ex}

Computes the neighbor of the given estimate.
\setlength{\parskip}{1ex}
      \textbf{Parameters}
      \vspace{-1ex}

      \begin{quote}
        \begin{Ventry}{x}

          \item[x]


The estimate to which the neighbor must be computed.
        \end{Ventry}

      \end{quote}

    \end{boxedminipage}


\large{\textbf{\textit{Inherited from object}}}

\begin{quote}
\_\_delattr\_\_(), \_\_format\_\_(), \_\_getattribute\_\_(), \_\_hash\_\_(), \_\_new\_\_(), \_\_reduce\_\_(), \_\_reduce\_ex\_\_(), \_\_repr\_\_(), \_\_setattr\_\_(), \_\_sizeof\_\_(), \_\_str\_\_(), \_\_subclasshook\_\_()
\end{quote}

%%%%%%%%%%%%%%%%%%%%%%%%%%%%%%%%%%%%%%%%%%%%%%%%%%%%%%%%%%%%%%%%%%%%%%%%%%%
%%                              Properties                               %%
%%%%%%%%%%%%%%%%%%%%%%%%%%%%%%%%%%%%%%%%%%%%%%%%%%%%%%%%%%%%%%%%%%%%%%%%%%%

  \subsubsection{Properties}

    \vspace{-1cm}
\hspace{\varindent}\begin{longtable}{|p{\varnamewidth}|p{\vardescrwidth}|l}
\cline{1-2}
\cline{1-2} \centering \textbf{Name} & \centering \textbf{Description}& \\
\cline{1-2}
\endhead\cline{1-2}\multicolumn{3}{r}{\small\textit{continued on next page}}\\\endfoot\cline{1-2}
\endlastfoot\multicolumn{2}{|l|}{\textit{Inherited from object}}\\
\multicolumn{2}{|p{\varwidth}|}{\raggedright \_\_class\_\_}\\
\cline{1-2}
\end{longtable}

    \index{peach \textit{(package)}!peach.sa \textit{(package)}!peach.sa.neighbor \textit{(module)}!peach.sa.neighbor.ContinuousNeighbor \textit{(class)}|)}

%%%%%%%%%%%%%%%%%%%%%%%%%%%%%%%%%%%%%%%%%%%%%%%%%%%%%%%%%%%%%%%%%%%%%%%%%%%
%%                           Class Description                           %%
%%%%%%%%%%%%%%%%%%%%%%%%%%%%%%%%%%%%%%%%%%%%%%%%%%%%%%%%%%%%%%%%%%%%%%%%%%%

    \index{peach \textit{(package)}!peach.sa \textit{(package)}!peach.sa.neighbor \textit{(module)}!peach.sa.neighbor.GaussianNeighbor \textit{(class)}|(}
\subsection{Class GaussianNeighbor}

    \label{peach:sa:neighbor:GaussianNeighbor}
\begin{tabular}{cccccccc}
% Line for object, linespec=[False, False]
\multicolumn{2}{r}{\settowidth{\BCL}{object}\multirow{2}{\BCL}{object}}
&&
&&
  \\\cline{3-3}
  &&\multicolumn{1}{c|}{}
&&
&&
  \\
% Line for peach.sa.neighbor.ContinuousNeighbor, linespec=[False]
\multicolumn{4}{r}{\settowidth{\BCL}{peach.sa.neighbor.ContinuousNeighbor}\multirow{2}{\BCL}{peach.sa.neighbor.ContinuousNeighbor}}
&&
  \\\cline{5-5}
  &&&&\multicolumn{1}{c|}{}
&&
  \\
&&&&\multicolumn{2}{l}{\textbf{peach.sa.neighbor.GaussianNeighbor}}
\end{tabular}


A new estimate based on a gaussian distribution

This class creates a function that computes the neighbor of an estimate by
adding a gaussian distributed randomly choosen vector with the same shape
and size of the estimate.

%%%%%%%%%%%%%%%%%%%%%%%%%%%%%%%%%%%%%%%%%%%%%%%%%%%%%%%%%%%%%%%%%%%%%%%%%%%
%%                                Methods                                %%
%%%%%%%%%%%%%%%%%%%%%%%%%%%%%%%%%%%%%%%%%%%%%%%%%%%%%%%%%%%%%%%%%%%%%%%%%%%

  \subsubsection{Methods}

    \vspace{0.5ex}

\hspace{.8\funcindent}\begin{boxedminipage}{\funcwidth}

    \raggedright \textbf{\_\_init\_\_}(\textit{self}, \textit{variance}={\tt 0.05})

    \vspace{-1.5ex}

    \rule{\textwidth}{0.5\fboxrule}
\setlength{\parskip}{2ex}

Initializes the neighbor operator
\setlength{\parskip}{1ex}
      \textbf{Parameters}
      \vspace{-1ex}

      \begin{quote}
        \begin{Ventry}{xxxxxxxx}

          \item[variance]


This is the variance of the gaussian distribution used to randomize
the estimate. This can be given as a single value or as an array. In
the first case, the same value will be used for all the components
of the estimate; in the second case, \texttt{variance} should be an array
with the same number of components of the estimate, and each
component in this array is the variance of the corresponding
component in the estimate array.
        \end{Ventry}

      \end{quote}

      Overrides: object.\_\_init\_\_

    \end{boxedminipage}

    \vspace{0.5ex}

\hspace{.8\funcindent}\begin{boxedminipage}{\funcwidth}

    \raggedright \textbf{\_\_call\_\_}(\textit{self}, \textit{x})

    \vspace{-1.5ex}

    \rule{\textwidth}{0.5\fboxrule}
\setlength{\parskip}{2ex}

Computes the neighbor of the given estimate.
\setlength{\parskip}{1ex}
      \textbf{Parameters}
      \vspace{-1ex}

      \begin{quote}
        \begin{Ventry}{x}

          \item[x]


The estimate to which the neighbor must be computed.
        \end{Ventry}

      \end{quote}

      Overrides: peach.sa.neighbor.ContinuousNeighbor.\_\_call\_\_

    \end{boxedminipage}


\large{\textbf{\textit{Inherited from object}}}

\begin{quote}
\_\_delattr\_\_(), \_\_format\_\_(), \_\_getattribute\_\_(), \_\_hash\_\_(), \_\_new\_\_(), \_\_reduce\_\_(), \_\_reduce\_ex\_\_(), \_\_repr\_\_(), \_\_setattr\_\_(), \_\_sizeof\_\_(), \_\_str\_\_(), \_\_subclasshook\_\_()
\end{quote}

%%%%%%%%%%%%%%%%%%%%%%%%%%%%%%%%%%%%%%%%%%%%%%%%%%%%%%%%%%%%%%%%%%%%%%%%%%%
%%                              Properties                               %%
%%%%%%%%%%%%%%%%%%%%%%%%%%%%%%%%%%%%%%%%%%%%%%%%%%%%%%%%%%%%%%%%%%%%%%%%%%%

  \subsubsection{Properties}

    \vspace{-1cm}
\hspace{\varindent}\begin{longtable}{|p{\varnamewidth}|p{\vardescrwidth}|l}
\cline{1-2}
\cline{1-2} \centering \textbf{Name} & \centering \textbf{Description}& \\
\cline{1-2}
\endhead\cline{1-2}\multicolumn{3}{r}{\small\textit{continued on next page}}\\\endfoot\cline{1-2}
\endlastfoot\multicolumn{2}{|l|}{\textit{Inherited from object}}\\
\multicolumn{2}{|p{\varwidth}|}{\raggedright \_\_class\_\_}\\
\cline{1-2}
\end{longtable}


%%%%%%%%%%%%%%%%%%%%%%%%%%%%%%%%%%%%%%%%%%%%%%%%%%%%%%%%%%%%%%%%%%%%%%%%%%%
%%                          Instance Variables                           %%
%%%%%%%%%%%%%%%%%%%%%%%%%%%%%%%%%%%%%%%%%%%%%%%%%%%%%%%%%%%%%%%%%%%%%%%%%%%

  \subsubsection{Instance Variables}

    \vspace{-1cm}
\hspace{\varindent}\begin{longtable}{|p{\varnamewidth}|p{\vardescrwidth}|l}
\cline{1-2}
\cline{1-2} \centering \textbf{Name} & \centering \textbf{Description}& \\
\cline{1-2}
\endhead\cline{1-2}\multicolumn{3}{r}{\small\textit{continued on next page}}\\\endfoot\cline{1-2}
\endlastfoot\raggedright v\-a\-r\-i\-a\-n\-c\-e\- & Variance of the gaussian distribution.&\\
\cline{1-2}
\end{longtable}

    \index{peach \textit{(package)}!peach.sa \textit{(package)}!peach.sa.neighbor \textit{(module)}!peach.sa.neighbor.GaussianNeighbor \textit{(class)}|)}

%%%%%%%%%%%%%%%%%%%%%%%%%%%%%%%%%%%%%%%%%%%%%%%%%%%%%%%%%%%%%%%%%%%%%%%%%%%
%%                           Class Description                           %%
%%%%%%%%%%%%%%%%%%%%%%%%%%%%%%%%%%%%%%%%%%%%%%%%%%%%%%%%%%%%%%%%%%%%%%%%%%%

    \index{peach \textit{(package)}!peach.sa \textit{(package)}!peach.sa.neighbor \textit{(module)}!peach.sa.neighbor.UniformNeighbor \textit{(class)}|(}
\subsection{Class UniformNeighbor}

    \label{peach:sa:neighbor:UniformNeighbor}
\begin{tabular}{cccccccc}
% Line for object, linespec=[False, False]
\multicolumn{2}{r}{\settowidth{\BCL}{object}\multirow{2}{\BCL}{object}}
&&
&&
  \\\cline{3-3}
  &&\multicolumn{1}{c|}{}
&&
&&
  \\
% Line for peach.sa.neighbor.ContinuousNeighbor, linespec=[False]
\multicolumn{4}{r}{\settowidth{\BCL}{peach.sa.neighbor.ContinuousNeighbor}\multirow{2}{\BCL}{peach.sa.neighbor.ContinuousNeighbor}}
&&
  \\\cline{5-5}
  &&&&\multicolumn{1}{c|}{}
&&
  \\
&&&&\multicolumn{2}{l}{\textbf{peach.sa.neighbor.UniformNeighbor}}
\end{tabular}


A new estimate based on a uniform distribution

This class creates a function that computes the neighbor of an estimate by
adding a uniform distributed randomly choosen vector with the same shape
and size of the estimate.

%%%%%%%%%%%%%%%%%%%%%%%%%%%%%%%%%%%%%%%%%%%%%%%%%%%%%%%%%%%%%%%%%%%%%%%%%%%
%%                                Methods                                %%
%%%%%%%%%%%%%%%%%%%%%%%%%%%%%%%%%%%%%%%%%%%%%%%%%%%%%%%%%%%%%%%%%%%%%%%%%%%

  \subsubsection{Methods}

    \vspace{0.5ex}

\hspace{.8\funcindent}\begin{boxedminipage}{\funcwidth}

    \raggedright \textbf{\_\_init\_\_}(\textit{self}, \textit{xl}={\tt -1.0}, \textit{xh}={\tt 1.0})

    \vspace{-1.5ex}

    \rule{\textwidth}{0.5\fboxrule}
\setlength{\parskip}{2ex}

Initializes the neighbor operator
\setlength{\parskip}{1ex}
      \textbf{Parameters}
      \vspace{-1ex}

      \begin{quote}
        \begin{Ventry}{xx}

          \item[xl]


The lower limit of the distribution;
          \item[xh]


The upper limit of the distribution. Both values can be given as a
single value or as an array. In the first case, the same value will
be used for all the components of the estimate; in the second case,
they should be an array with the same number of components of the
estimate, and each component in this array is the variance of the
corresponding component in the estimate array.
        \end{Ventry}

      \end{quote}

      Overrides: object.\_\_init\_\_

    \end{boxedminipage}

    \vspace{0.5ex}

\hspace{.8\funcindent}\begin{boxedminipage}{\funcwidth}

    \raggedright \textbf{\_\_call\_\_}(\textit{self}, \textit{x})

    \vspace{-1.5ex}

    \rule{\textwidth}{0.5\fboxrule}
\setlength{\parskip}{2ex}

Computes the neighbor of the given estimate.
\setlength{\parskip}{1ex}
      \textbf{Parameters}
      \vspace{-1ex}

      \begin{quote}
        \begin{Ventry}{x}

          \item[x]


The estimate to which the neighbor must be computed.
        \end{Ventry}

      \end{quote}

      Overrides: peach.sa.neighbor.ContinuousNeighbor.\_\_call\_\_

    \end{boxedminipage}


\large{\textbf{\textit{Inherited from object}}}

\begin{quote}
\_\_delattr\_\_(), \_\_format\_\_(), \_\_getattribute\_\_(), \_\_hash\_\_(), \_\_new\_\_(), \_\_reduce\_\_(), \_\_reduce\_ex\_\_(), \_\_repr\_\_(), \_\_setattr\_\_(), \_\_sizeof\_\_(), \_\_str\_\_(), \_\_subclasshook\_\_()
\end{quote}

%%%%%%%%%%%%%%%%%%%%%%%%%%%%%%%%%%%%%%%%%%%%%%%%%%%%%%%%%%%%%%%%%%%%%%%%%%%
%%                              Properties                               %%
%%%%%%%%%%%%%%%%%%%%%%%%%%%%%%%%%%%%%%%%%%%%%%%%%%%%%%%%%%%%%%%%%%%%%%%%%%%

  \subsubsection{Properties}

    \vspace{-1cm}
\hspace{\varindent}\begin{longtable}{|p{\varnamewidth}|p{\vardescrwidth}|l}
\cline{1-2}
\cline{1-2} \centering \textbf{Name} & \centering \textbf{Description}& \\
\cline{1-2}
\endhead\cline{1-2}\multicolumn{3}{r}{\small\textit{continued on next page}}\\\endfoot\cline{1-2}
\endlastfoot\multicolumn{2}{|l|}{\textit{Inherited from object}}\\
\multicolumn{2}{|p{\varwidth}|}{\raggedright \_\_class\_\_}\\
\cline{1-2}
\end{longtable}


%%%%%%%%%%%%%%%%%%%%%%%%%%%%%%%%%%%%%%%%%%%%%%%%%%%%%%%%%%%%%%%%%%%%%%%%%%%
%%                          Instance Variables                           %%
%%%%%%%%%%%%%%%%%%%%%%%%%%%%%%%%%%%%%%%%%%%%%%%%%%%%%%%%%%%%%%%%%%%%%%%%%%%

  \subsubsection{Instance Variables}

    \vspace{-1cm}
\hspace{\varindent}\begin{longtable}{|p{\varnamewidth}|p{\vardescrwidth}|l}
\cline{1-2}
\cline{1-2} \centering \textbf{Name} & \centering \textbf{Description}& \\
\cline{1-2}
\endhead\cline{1-2}\multicolumn{3}{r}{\small\textit{continued on next page}}\\\endfoot\cline{1-2}
\endlastfoot\raggedright x\-l\- & Lower limit of the uniform distribution.&\\
\cline{1-2}
\raggedright x\-h\- & Upper limit of the uniform distribution.&\\
\cline{1-2}
\end{longtable}

    \index{peach \textit{(package)}!peach.sa \textit{(package)}!peach.sa.neighbor \textit{(module)}!peach.sa.neighbor.UniformNeighbor \textit{(class)}|)}

%%%%%%%%%%%%%%%%%%%%%%%%%%%%%%%%%%%%%%%%%%%%%%%%%%%%%%%%%%%%%%%%%%%%%%%%%%%
%%                           Class Description                           %%
%%%%%%%%%%%%%%%%%%%%%%%%%%%%%%%%%%%%%%%%%%%%%%%%%%%%%%%%%%%%%%%%%%%%%%%%%%%

    \index{peach \textit{(package)}!peach.sa \textit{(package)}!peach.sa.neighbor \textit{(module)}!peach.sa.neighbor.BinaryNeighbor \textit{(class)}|(}
\subsection{Class BinaryNeighbor}

    \label{peach:sa:neighbor:BinaryNeighbor}
\begin{tabular}{cccccc}
% Line for object, linespec=[False]
\multicolumn{2}{r}{\settowidth{\BCL}{object}\multirow{2}{\BCL}{object}}
&&
  \\\cline{3-3}
  &&\multicolumn{1}{c|}{}
&&
  \\
&&\multicolumn{2}{l}{\textbf{peach.sa.neighbor.BinaryNeighbor}}
\end{tabular}

\textbf{Known Subclasses:} peach.sa.neighbor.InvertBitsNeighbor


Base class for binary neighbor functions

This class should be derived to implement a function which computes the
neighbor of a given estimate. Every neighbor functions should implement at
least two methods, defined below:
%
\begin{quote}
%
\begin{description}
\item[{\_\_init\_\_(self, %
\raisebox{1em}{\hypertarget{id2}{}}\hyperlink{id1}{\textbf{\color{red}*}}cnf, %
\raisebox{1em}{\hypertarget{id4}{}}\hyperlink{id3}{\textbf{\color{red}**}}kw)}] \leavevmode 
Initializes the object. There are no mandatory arguments, but any
parameters can be used here to configure the operator. For example, a
class can define a bit change rate -{}- this should be defined here:
%
\begin{quote}{\ttfamily \raggedright \noindent
\_\_init\_\_(self,~rate=0.01)
}
\end{quote}

A default value should always be offered, if possible.

\item[{\_\_call\_\_(self, x):}] \leavevmode 
The \texttt{\_\_call\_\_} interface should be programmed to actually compute the
value of the neighbor. This method should receive an estimate in \texttt{x}
and use whatever parameters from the instantiation to compute the new
estimate. It should return the new estimate.

\end{description}

\end{quote}

Please, note that the SA implementations relies on this behaviour: it will
pass an estimate to your \texttt{\_\_call\_\_} method and expects to received the
result back. Notice, however, that the SA implementation does not expect
that the result is sane, ie, that it is in conformity with the
representation used in the algorithm. A sanity check is done inside the
binary SA class. Please, consult the documentation on \texttt{BinarySA} for
further details.

This class can be used also to transform a simple function in a neighbor
function. In this case, the outside function must compute in an appropriate
way the new estimate.

%%%%%%%%%%%%%%%%%%%%%%%%%%%%%%%%%%%%%%%%%%%%%%%%%%%%%%%%%%%%%%%%%%%%%%%%%%%
%%                                Methods                                %%
%%%%%%%%%%%%%%%%%%%%%%%%%%%%%%%%%%%%%%%%%%%%%%%%%%%%%%%%%%%%%%%%%%%%%%%%%%%

  \subsubsection{Methods}

    \vspace{0.5ex}

\hspace{.8\funcindent}\begin{boxedminipage}{\funcwidth}

    \raggedright \textbf{\_\_init\_\_}(\textit{self}, \textit{f})

    \vspace{-1.5ex}

    \rule{\textwidth}{0.5\fboxrule}
\setlength{\parskip}{2ex}

Creates a neighbor function from a function.
\setlength{\parskip}{1ex}
      \textbf{Parameters}
      \vspace{-1ex}

      \begin{quote}
        \begin{Ventry}{x}

          \item[f]


The function to be transformed. This function must receive a
bitarray of any length as an estimate, and return a new bitarray of
the same length as a result.
        \end{Ventry}

      \end{quote}

      Overrides: object.\_\_init\_\_

    \end{boxedminipage}

    \label{peach:sa:neighbor:BinaryNeighbor:__call__}
    \index{peach \textit{(package)}!peach.sa \textit{(package)}!peach.sa.neighbor \textit{(module)}!peach.sa.neighbor.BinaryNeighbor \textit{(class)}!peach.sa.neighbor.BinaryNeighbor.\_\_call\_\_ \textit{(method)}}

    \vspace{0.5ex}

\hspace{.8\funcindent}\begin{boxedminipage}{\funcwidth}

    \raggedright \textbf{\_\_call\_\_}(\textit{self}, \textit{x})

    \vspace{-1.5ex}

    \rule{\textwidth}{0.5\fboxrule}
\setlength{\parskip}{2ex}

Computes the neighbor of the given estimate.
\setlength{\parskip}{1ex}
      \textbf{Parameters}
      \vspace{-1ex}

      \begin{quote}
        \begin{Ventry}{x}

          \item[x]


The estimate to which the neighbor must be computed.
        \end{Ventry}

      \end{quote}

    \end{boxedminipage}


\large{\textbf{\textit{Inherited from object}}}

\begin{quote}
\_\_delattr\_\_(), \_\_format\_\_(), \_\_getattribute\_\_(), \_\_hash\_\_(), \_\_new\_\_(), \_\_reduce\_\_(), \_\_reduce\_ex\_\_(), \_\_repr\_\_(), \_\_setattr\_\_(), \_\_sizeof\_\_(), \_\_str\_\_(), \_\_subclasshook\_\_()
\end{quote}

%%%%%%%%%%%%%%%%%%%%%%%%%%%%%%%%%%%%%%%%%%%%%%%%%%%%%%%%%%%%%%%%%%%%%%%%%%%
%%                              Properties                               %%
%%%%%%%%%%%%%%%%%%%%%%%%%%%%%%%%%%%%%%%%%%%%%%%%%%%%%%%%%%%%%%%%%%%%%%%%%%%

  \subsubsection{Properties}

    \vspace{-1cm}
\hspace{\varindent}\begin{longtable}{|p{\varnamewidth}|p{\vardescrwidth}|l}
\cline{1-2}
\cline{1-2} \centering \textbf{Name} & \centering \textbf{Description}& \\
\cline{1-2}
\endhead\cline{1-2}\multicolumn{3}{r}{\small\textit{continued on next page}}\\\endfoot\cline{1-2}
\endlastfoot\multicolumn{2}{|l|}{\textit{Inherited from object}}\\
\multicolumn{2}{|p{\varwidth}|}{\raggedright \_\_class\_\_}\\
\cline{1-2}
\end{longtable}

    \index{peach \textit{(package)}!peach.sa \textit{(package)}!peach.sa.neighbor \textit{(module)}!peach.sa.neighbor.BinaryNeighbor \textit{(class)}|)}

%%%%%%%%%%%%%%%%%%%%%%%%%%%%%%%%%%%%%%%%%%%%%%%%%%%%%%%%%%%%%%%%%%%%%%%%%%%
%%                           Class Description                           %%
%%%%%%%%%%%%%%%%%%%%%%%%%%%%%%%%%%%%%%%%%%%%%%%%%%%%%%%%%%%%%%%%%%%%%%%%%%%

    \index{peach \textit{(package)}!peach.sa \textit{(package)}!peach.sa.neighbor \textit{(module)}!peach.sa.neighbor.InvertBitsNeighbor \textit{(class)}|(}
\subsection{Class InvertBitsNeighbor}

    \label{peach:sa:neighbor:InvertBitsNeighbor}
\begin{tabular}{cccccccc}
% Line for object, linespec=[False, False]
\multicolumn{2}{r}{\settowidth{\BCL}{object}\multirow{2}{\BCL}{object}}
&&
&&
  \\\cline{3-3}
  &&\multicolumn{1}{c|}{}
&&
&&
  \\
% Line for peach.sa.neighbor.BinaryNeighbor, linespec=[False]
\multicolumn{4}{r}{\settowidth{\BCL}{peach.sa.neighbor.BinaryNeighbor}\multirow{2}{\BCL}{peach.sa.neighbor.BinaryNeighbor}}
&&
  \\\cline{5-5}
  &&&&\multicolumn{1}{c|}{}
&&
  \\
&&&&\multicolumn{2}{l}{\textbf{peach.sa.neighbor.InvertBitsNeighbor}}
\end{tabular}


A simple neighborhood based on the change of a few bits.

This neighbor will be computed by randomly choosing a bit in the bitarray
representing the estimate and change a number of bits in the bitarray and
inverting their value.

%%%%%%%%%%%%%%%%%%%%%%%%%%%%%%%%%%%%%%%%%%%%%%%%%%%%%%%%%%%%%%%%%%%%%%%%%%%
%%                                Methods                                %%
%%%%%%%%%%%%%%%%%%%%%%%%%%%%%%%%%%%%%%%%%%%%%%%%%%%%%%%%%%%%%%%%%%%%%%%%%%%

  \subsubsection{Methods}

    \vspace{0.5ex}

\hspace{.8\funcindent}\begin{boxedminipage}{\funcwidth}

    \raggedright \textbf{\_\_init\_\_}(\textit{self}, \textit{nb}={\tt 2})

    \vspace{-1.5ex}

    \rule{\textwidth}{0.5\fboxrule}
\setlength{\parskip}{2ex}

Initializes the operator.
\setlength{\parskip}{1ex}
      \textbf{Parameters}
      \vspace{-1ex}

      \begin{quote}
        \begin{Ventry}{xx}

          \item[nb]


The number of bits to be randomly choosen to be inverted in the
calculation of the neighbor. Be very careful while choosing this
parameter. While very large optimizations can benefit from a big
value here, it is not recommended that more than one bit per
variable is inverted at each step -{}- otherwise, the neighbor might
fall very far from the present estimate, which can make the
algorithm not work accordingly. This defaults to 2, that is, at each
step, only one bit will be inverted at most.
        \end{Ventry}

      \end{quote}

      Overrides: object.\_\_init\_\_

    \end{boxedminipage}

    \vspace{0.5ex}

\hspace{.8\funcindent}\begin{boxedminipage}{\funcwidth}

    \raggedright \textbf{\_\_call\_\_}(\textit{self}, \textit{x})

    \vspace{-1.5ex}

    \rule{\textwidth}{0.5\fboxrule}
\setlength{\parskip}{2ex}

Computes the neighbor of the given estimate.
\setlength{\parskip}{1ex}
      \textbf{Parameters}
      \vspace{-1ex}

      \begin{quote}
        \begin{Ventry}{x}

          \item[x]


The estimate to which the neighbor must be computed.
        \end{Ventry}

      \end{quote}

      Overrides: peach.sa.neighbor.BinaryNeighbor.\_\_call\_\_

    \end{boxedminipage}


\large{\textbf{\textit{Inherited from object}}}

\begin{quote}
\_\_delattr\_\_(), \_\_format\_\_(), \_\_getattribute\_\_(), \_\_hash\_\_(), \_\_new\_\_(), \_\_reduce\_\_(), \_\_reduce\_ex\_\_(), \_\_repr\_\_(), \_\_setattr\_\_(), \_\_sizeof\_\_(), \_\_str\_\_(), \_\_subclasshook\_\_()
\end{quote}

%%%%%%%%%%%%%%%%%%%%%%%%%%%%%%%%%%%%%%%%%%%%%%%%%%%%%%%%%%%%%%%%%%%%%%%%%%%
%%                              Properties                               %%
%%%%%%%%%%%%%%%%%%%%%%%%%%%%%%%%%%%%%%%%%%%%%%%%%%%%%%%%%%%%%%%%%%%%%%%%%%%

  \subsubsection{Properties}

    \vspace{-1cm}
\hspace{\varindent}\begin{longtable}{|p{\varnamewidth}|p{\vardescrwidth}|l}
\cline{1-2}
\cline{1-2} \centering \textbf{Name} & \centering \textbf{Description}& \\
\cline{1-2}
\endhead\cline{1-2}\multicolumn{3}{r}{\small\textit{continued on next page}}\\\endfoot\cline{1-2}
\endlastfoot\multicolumn{2}{|l|}{\textit{Inherited from object}}\\
\multicolumn{2}{|p{\varwidth}|}{\raggedright \_\_class\_\_}\\
\cline{1-2}
\end{longtable}

    \index{peach \textit{(package)}!peach.sa \textit{(package)}!peach.sa.neighbor \textit{(module)}!peach.sa.neighbor.InvertBitsNeighbor \textit{(class)}|)}
    \index{peach \textit{(package)}!peach.sa \textit{(package)}!peach.sa.neighbor \textit{(module)}|)}


%%%%%%%%%%%%%%%%%%%%%%%%%%%%%%%%%%%%%%%%%%%%%%%%%%%%%%%%%%%%%%%%%%%%%%%%%%%
%%                                 Index                                 %%
%%%%%%%%%%%%%%%%%%%%%%%%%%%%%%%%%%%%%%%%%%%%%%%%%%%%%%%%%%%%%%%%%%%%%%%%%%%

\printindex


%%%%%%%%%%%%%%%%%%%%%%%%%%%%%%%%%%%%%%%%%%%%%%%%%%%%%%%%%%%%%%%%%%%%%%%%%%%
%%                                Footer                                 %%
%%%%%%%%%%%%%%%%%%%%%%%%%%%%%%%%%%%%%%%%%%%%%%%%%%%%%%%%%%%%%%%%%%%%%%%%%%%

\end{document}

