%
% API Documentation for Peach - Computational Intelligence for Python
% Package peach.nn
%
% Generated by epydoc 3.0beta1
% [Mon Dec 21 08:51:37 2009]
%

%%%%%%%%%%%%%%%%%%%%%%%%%%%%%%%%%%%%%%%%%%%%%%%%%%%%%%%%%%%%%%%%%%%%%%%%%%%
%%                          Module Description                           %%
%%%%%%%%%%%%%%%%%%%%%%%%%%%%%%%%%%%%%%%%%%%%%%%%%%%%%%%%%%%%%%%%%%%%%%%%%%%

    \index{peach \textit{(package)}!peach.nn \textit{(package)}|(}
\section{Package peach.nn}

    \label{peach:nn}

This package implements support for neural networks. Consult:
\begin{quote}
\begin{description}
%[visit_definition_list_item]
\item[{af}] %[visit_definition]

A list of activation functions for use with neurons and a base class to
implement different activation functions;

%[depart_definition]
%[depart_definition_list_item]
%[visit_definition_list_item]
\item[{base}] %[visit_definition]

Basic definitions of the objects used with neural networks;

%[depart_definition]
%[depart_definition_list_item]
%[visit_definition_list_item]
\item[{lrule}] %[visit_definition]

Learning rules;

%[depart_definition]
%[depart_definition_list_item]
%[visit_definition_list_item]
\item[{nn}] %[visit_definition]

Implementation of different classes of neural networks;

%[depart_definition]
%[depart_definition_list_item]
\end{description}
\end{quote}

%%%%%%%%%%%%%%%%%%%%%%%%%%%%%%%%%%%%%%%%%%%%%%%%%%%%%%%%%%%%%%%%%%%%%%%%%%%
%%                                Modules                                %%
%%%%%%%%%%%%%%%%%%%%%%%%%%%%%%%%%%%%%%%%%%%%%%%%%%%%%%%%%%%%%%%%%%%%%%%%%%%

\subsection{Modules}

\begin{itemize}
\setlength{\parskip}{0ex}
\item \textbf{af}: 
Base activation functions and base class


  \textit{(Section \ref{peach:nn:af}, p.~\pageref{peach:nn:af})}

\item \textbf{base}: 
Basic definitions for layers of neurons.


  \textit{(Section \ref{peach:nn:base}, p.~\pageref{peach:nn:base})}

\item \textbf{lrules}: 
Learning rules for neural networks and base classes for custom learning.


  \textit{(Section \ref{peach:nn:lrules}, p.~\pageref{peach:nn:lrules})}

\item \textbf{nn}: 
Basic topologies of neural networks.


  \textit{(Section \ref{peach:nn:nn}, p.~\pageref{peach:nn:nn})}

\end{itemize}


%%%%%%%%%%%%%%%%%%%%%%%%%%%%%%%%%%%%%%%%%%%%%%%%%%%%%%%%%%%%%%%%%%%%%%%%%%%
%%                               Variables                               %%
%%%%%%%%%%%%%%%%%%%%%%%%%%%%%%%%%%%%%%%%%%%%%%%%%%%%%%%%%%%%%%%%%%%%%%%%%%%

  \subsection{Variables}

\begin{longtable}{|p{.30\textwidth}|p{.62\textwidth}|l}
\cline{1-2}
\cline{1-2} \centering \textbf{Name} & \centering \textbf{Description}& \\
\cline{1-2}
\endhead\cline{1-2}\multicolumn{3}{r}{\small\textit{continued on next page}}\\\endfoot\cline{1-2}
\endlastfoot\raggedright \_\-\_\-d\-o\-c\-\_\-\_\- & \raggedright \textbf{Value:} 
{\tt \texttt{...}}&\\
\cline{1-2}
\end{longtable}

    \index{peach \textit{(package)}!peach.nn \textit{(package)}|)}
