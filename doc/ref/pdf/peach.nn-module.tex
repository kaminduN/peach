%
% API Documentation for Peach - Computational Intelligence for Python
% Package peach.nn
%
% Generated by epydoc 3.0.1
% [Fri Feb  4 17:21:20 2011]
%

%%%%%%%%%%%%%%%%%%%%%%%%%%%%%%%%%%%%%%%%%%%%%%%%%%%%%%%%%%%%%%%%%%%%%%%%%%%
%%                          Module Description                           %%
%%%%%%%%%%%%%%%%%%%%%%%%%%%%%%%%%%%%%%%%%%%%%%%%%%%%%%%%%%%%%%%%%%%%%%%%%%%

    \index{peach \textit{(package)}!peach.nn \textit{(package)}|(}
\section{Package peach.nn}

    \label{peach:nn}

This package implements support for neural networks. Consult:
%
\begin{quote}
%
\begin{description}
\item[{base}] \leavevmode 
Basic definitions of the objects used with neural networks;

\item[{af}] \leavevmode 
A list of activation functions for use with neurons and a base class to
implement different activation functions;

\item[{lrule}] \leavevmode 
Learning rules;

\item[{nnet}] \leavevmode 
Implementation of different classes of neural networks;

\item[{mem}] \leavevmode 
Associative memories and Hopfield model;

\end{description}

\end{quote}

%%%%%%%%%%%%%%%%%%%%%%%%%%%%%%%%%%%%%%%%%%%%%%%%%%%%%%%%%%%%%%%%%%%%%%%%%%%
%%                                Modules                                %%
%%%%%%%%%%%%%%%%%%%%%%%%%%%%%%%%%%%%%%%%%%%%%%%%%%%%%%%%%%%%%%%%%%%%%%%%%%%

\subsection{Modules}

\begin{itemize}
\setlength{\parskip}{0ex}
\item \textbf{af}: 
Base activation functions and base class


  \textit{(Section \ref{peach:nn:af}, p.~\pageref{peach:nn:af})}

\item \textbf{base}: 
Basic definitions for layers of neurons.


  \textit{(Section \ref{peach:nn:base}, p.~\pageref{peach:nn:base})}

\item \textbf{lrules}: 
Learning rules for neural networks and base classes for custom learning.


  \textit{(Section \ref{peach:nn:lrules}, p.~\pageref{peach:nn:lrules})}

\item \textbf{mem}: 
Associative memories and Hopfield network model.


  \textit{(Section \ref{peach:nn:mem}, p.~\pageref{peach:nn:mem})}

\item \textbf{nnet}: 
Basic topologies of neural networks.


  \textit{(Section \ref{peach:nn:nnet}, p.~\pageref{peach:nn:nnet})}

\end{itemize}


%%%%%%%%%%%%%%%%%%%%%%%%%%%%%%%%%%%%%%%%%%%%%%%%%%%%%%%%%%%%%%%%%%%%%%%%%%%
%%                               Functions                               %%
%%%%%%%%%%%%%%%%%%%%%%%%%%%%%%%%%%%%%%%%%%%%%%%%%%%%%%%%%%%%%%%%%%%%%%%%%%%

  \subsection{Functions}

    \label{peach:nn:nnet:randn}
    \index{peach \textit{(package)}!peach.nn \textit{(package)}!peach.nn.nnet \textit{(module)}!peach.nn.nnet.randn \textit{(function)}}

    \vspace{0.5ex}

\hspace{.8\funcindent}\begin{boxedminipage}{\funcwidth}

    \raggedright \textbf{randn}(\textit{d0}, \textit{d1}, \textit{dn}, \textit{...})

    \vspace{-1.5ex}

    \rule{\textwidth}{0.5\fboxrule}
\setlength{\parskip}{2ex}

Returns zero-mean, unit-variance Gaussian random numbers in an
array of shape (d0, d1, ..., dn).
%
\begin{description}
\item[{Note:  This is a convenience function. If you want an}] \leavevmode 
interface that takes a tuple as the first argument
use numpy.random.standard\_normal(shape\_tuple).

\end{description}
\setlength{\parskip}{1ex}
    \end{boxedminipage}


%%%%%%%%%%%%%%%%%%%%%%%%%%%%%%%%%%%%%%%%%%%%%%%%%%%%%%%%%%%%%%%%%%%%%%%%%%%
%%                               Variables                               %%
%%%%%%%%%%%%%%%%%%%%%%%%%%%%%%%%%%%%%%%%%%%%%%%%%%%%%%%%%%%%%%%%%%%%%%%%%%%

  \subsection{Variables}

    \vspace{-1cm}
\hspace{\varindent}\begin{longtable}{|p{\varnamewidth}|p{\vardescrwidth}|l}
\cline{1-2}
\cline{1-2} \centering \textbf{Name} & \centering \textbf{Description}& \\
\cline{1-2}
\endhead\cline{1-2}\multicolumn{3}{r}{\small\textit{continued on next page}}\\\endfoot\cline{1-2}
\endlastfoot\raggedright \_\-\_\-d\-o\-c\-\_\-\_\- & \raggedright \textbf{Value:} 
{\tt \texttt{...}}&\\
\cline{1-2}
\raggedright \_\-\_\-p\-a\-c\-k\-a\-g\-e\-\_\-\_\- & \raggedright \textbf{Value:} 
{\tt \texttt{'}\texttt{peach.nn}\texttt{'}}&\\
\cline{1-2}
\raggedright a\-b\-s\- & \raggedright \textbf{Value:} 
{\tt {\textless}ufunc 'absolute'{\textgreater}}&\\
\cline{1-2}
\raggedright a\-r\-c\-t\-a\-n\- & \raggedright \textbf{Value:} 
{\tt {\textless}ufunc 'arctan'{\textgreater}}&\\
\cline{1-2}
\raggedright c\-o\-s\-h\- & \raggedright \textbf{Value:} 
{\tt {\textless}ufunc 'cosh'{\textgreater}}&\\
\cline{1-2}
\raggedright e\-x\-p\- & \raggedright \textbf{Value:} 
{\tt {\textless}ufunc 'exp'{\textgreater}}&\\
\cline{1-2}
\raggedright p\-i\- & \raggedright \textbf{Value:} 
{\tt 3.14159265359}&\\
\cline{1-2}
\raggedright s\-i\-g\-n\- & \raggedright \textbf{Value:} 
{\tt {\textless}ufunc 'sign'{\textgreater}}&\\
\cline{1-2}
\raggedright s\-q\-r\-t\- & \raggedright \textbf{Value:} 
{\tt {\textless}ufunc 'sqrt'{\textgreater}}&\\
\cline{1-2}
\raggedright t\-a\-n\-h\- & \raggedright \textbf{Value:} 
{\tt {\textless}ufunc 'tanh'{\textgreater}}&\\
\cline{1-2}
\end{longtable}

    \index{peach \textit{(package)}!peach.nn \textit{(package)}|)}
